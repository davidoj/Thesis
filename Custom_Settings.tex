%!TEX root = main.tex


% ********************************************************************
% Useful commands
% ********************************************************************
\newcommand{\ie}{i.\,e.~}
\newcommand{\Ie}{I.\,e.~}
\newcommand{\eg}{e.\,g.~}
\newcommand{\Eg}{E.\,g.~}

%----------------------------------------------------------------------------------------

% Define some commands to keep the formatting separated from the content 
\newcommand{\keyword}[1]{\textbf{#1}}
\newcommand{\tabhead}[1]{\textbf{#1}}
\newcommand{\code}[1]{\texttt{#1}}
\newcommand{\file}[1]{\texttt{\bfseries#1}}
\newcommand{\option}[1]{\texttt{\itshape#1}}


%----------------------------------------------------------------------------------------

%----------------------------------------------------------------------------------------
%	Optional Packages
%----------------------------------------------------------------------------------------
\usepackage{graphicx}
\usepackage[figuresright]{rotating}

\usepackage[utf8]{inputenc} % Required for inputting international characters
\usepackage[T1]{fontenc} % Output font encoding for international characters
\usepackage{mathpazo} % Use the Palatino font by default
%\usepackage{tgpagella} %The TEX Gyre Pagella family of fonts is based on the URW Palladio family, but heavily extended (https://www.tug.org/FontCatalogue/texgyrepagella/)

\usepackage{fontawesome} % Use for loading symbol for URLs
\usepackage{textcomp} % use for degree symbol (\textdegree)

%\usepackage[document]{ragged2e}

\usepackage[section]{minted} %works well in Overleaf but offline may need Python package to be installed and shell escape setting applied
\usemintedstyle{friendly}
\usemintedstyle{borland}

\usepackage[autostyle=true]{csquotes} % Required to generate language-dependent quotes in the bibliography
\usepackage[font={itshape,raggedright},begintext=``,endtext="]{quoting}
\usepackage{microtype}
\usepackage[framemethod=TikZ]{mdframed}
\usepackage{amssymb}
\usepackage{multirow}
\usepackage{tabulary}
\usepackage{glossaries}
\usepackage{textcomp}

%New colors defined below
\definecolor{LightGray}{gray}{0.9}
\definecolor{codegreen}{rgb}{0,0.6,0}
\definecolor{codegray}{rgb}{0.5,0.5,0.5}
\definecolor{codepurple}{rgb}{0.58,0,0.82}
\definecolor{backcolour}{rgb}{0.95,0.95,0.92}
\definecolor{mintedbackground}{rgb}{0.95,0.95,0.95}


\usepackage{hyperref}       % hyperlinks
\usepackage{url}            % simple URL typesetting
\usepackage{booktabs}       % professional-quality tables
\usepackage{amsfonts}       % blackboard math symbols
\usepackage{nicefrac}       % compact symbols for 1/2, etc.
\usepackage{microtype}      % microtypography
% \usepackage{bibleref}

\pdfcompresslevel=9 %best compression
\pdfadjustspacing=1 %for small font expansion

% My packages
\usepackage{tikzit}
\input{diagrams.tikzstyles}
\usepackage[mathscr]{euscript}
\usepackage {tikz}
\usetikzlibrary {positioning}
\usetikzlibrary{shapes.misc}
\usetikzlibrary{shapes.geometric}
\usetikzlibrary{calc}
\usetikzlibrary{arrows.meta}
\usetikzlibrary{intersections}
\usepackage{amsthm}
\usepackage{amsmath}
\usepackage{amssymb}
\usepackage{dsfont}
\usepackage{stmaryrd }
\usepackage{csquotes}
\usepackage{wasysym}
\usepackage[]{todonotes}
\usepackage[shortlabels]{enumitem}
\usepackage{bm}
\usepackage{isomath}
\usepackage{mathtools}
\usepackage{algpseudocode}
\usepackage{algorithm}
\usepackage{multirow}

\usepackage{mathtools}
\mathtoolsset{showonlyrefs}


\hyphenation{un-con-found-ed-ness-like}
\hyphenation{un-con-found-ed-ness}
%----------------------------------------------------------------------------------------
%	MATHS SETTINGS
%----------------------------------------------------------------------------------------


\makeatletter
\newcommand{\newreptheorem}[2]
  {\newtheorem*{rep@#1}{\rep@title}\newenvironment{rep#1}[1]
  {\def\rep@title{#2 \ref*{##1}}\begin{rep@#1}}{\end{rep@#1}}}
\makeatother

\theoremstyle{plain}
\newtheorem{theorem}{Theorem}[section]
\newtheorem{corollary}[theorem]{Corollary}
\newtheorem{lemma}[theorem]{Lemma}
\newtheorem{proposition}[theorem]{Proposition}
\newreptheorem{theorem}{Theorem}
\newreptheorem{lemma}{Lemma}
\newreptheorem{definition}{Definition}

\newtheorem{innercustomthm}{Theorem}
\newenvironment{customthm}[1]
  {\renewcommand\theinnercustomthm{#1}\innercustomthm}
  {\endinnercustomthm}

\theoremstyle{definition}
\newtheorem{definition}[theorem]{Definition}
\newtheorem{example}[theorem]{Example}
\newtheorem{notation}[theorem]{Nxample}


\DeclareMathAlphabet{\mathsfit}{T1}{\sfdefault}{\mddefault}{\sldefault}

\newcommand{\CI}{\mathrel{\text{\scalebox{1.07}{$\perp\mkern-10mu\perp$}}}}
\newcommand{\CII}{\mathrel{\text{\scalebox{1.07}{$\perp\mkern-10mu\perp\mkern-10mu\perp$}}}}
\newcommand{\RV}[1]{\ensuremath{\mathsf{#1}}}
\newcommand{\node}[1]{\ensuremath{\mathsfit{#1}}}
\newcommand{\graph}[1]{\ensuremath{\mathsfbfit{#1}}}
\newcommand{\URV}[1]{\ensuremath{\underline{\RV{#1}}}}
\newcommand{\PA}[2]{\ensuremath{\text{Pa}_{#1}(#2)}}
\newcommand{\ND}[2]{\ensuremath{\text{ND}_{#1}(#2)}}
\newcommand{\CH}[2]{\ensuremath{\text{Ch}_{#1}(#2)}}
\newcommand{\DE}[2]{\ensuremath{\text{De}_{#1}(#2)}}
\newcommand{\ID}[1]{\ensuremath{\text{Id}_{#1}}}
\newcommand{\utimes}{\ensuremath{\underline{\otimes}}}
\newcommand{\prob}[1]{\ensuremath{\mathbb{#1}}}
\newcommand{\disint}[1]{\ensuremath{\overline{\prob{#1}}}}
\newcommand{\kernel}[1]{\ensuremath{\mathbb{#1}}}
\newcommand{\model}[1]{\ensuremath{\mathbb{#1}}}
\newcommand{\diagram}[1]{\ensuremath{\mathscr{#1}}}
\newcommand{\sigalg}[1]{\ensuremath{\mathcal{#1}}}
\newcommand{\vecRV}[1]{\ensuremath{\mathsfbfit{#1}}}
\newcommand{\vecVal}[1]{\ensuremath{\mathbf{#1}}}
\newcommand{\prodSet}[1]{\ensuremath{\mathbf{#1}}}
\newcommand{\indx}[1]{\ensuremath{\mathcal{#1}}}
\newcommand{\nod}[1]{\ensuremath{\mathsfit{#1}}}
\newcommand{\kto}{\ensuremath{\rightarrowtriangle}}
\newcommand{\proc}[1]{\ensuremath{\mathscr{#1}}}
\newcommand{\yields}{\ensuremath{\bowtie}}
\newcommand{\submodel}{\ensuremath{\sqsubset}}
\newcommand{\seedo}[5]{\ensuremath{\model{#1}^{\RV{#3}|\RV{#2}\square\RV{#5}|\RV{#4}}}}
\newcommand{\rseedo}[6]{\ensuremath{\model{#1}^{\RV{#3}|\RV{#2}\framebox{#6}\RV{#5}|\RV{#4}}}}
\newcommand{\set}{\ensuremath{\bowtie}}
\newcommand{\cprod}{\ensuremath{\odot}}
\newcommand{\bigcprod}{\ensuremath{\bigodot}}
\newcommand{\combprod}{\ensuremath{\underline{\cprod}}}
\newcommand{\combbreak}{\ensuremath{\wr}}
\newcommand{\combgap}{\ensuremath{\shortleftarrow}}
\newcommand{\bigcombprod}{\ensuremath{\underline{\bigcprod}}}
\newcommand{\varlessthan}{\ensuremath{\preccurlyeq}}
\algnewcommand\algorithmicassert{\texttt{assert}}
\algnewcommand\Assert[1]{\State \algorithmicassert(#1)}%



\providecommand\longrightarrowRHD{\relbar\joinrel\relbar\joinrel\mathrel\RHD}
\providecommand\longleftarrowRHD{\mathrel\LHD\joinrel\relbar\joinrel\relbar}

\makeatletter
\newcommand*\bigcdot{\mathpalette\bigcdot@{.5}}
\newcommand*\bigcdot@[2]{\mathbin{\vcenter{\hbox{\scalebox{#2}{$\m@th#1\bullet$}}}}}
\makeatother

\tikzset{
    triangle/.style = {regular polygon, regular polygon sides=3 },
    node rotated/.style = {rotate=90},
    border rotated/.style = {shape border rotate=90},
    dist/.style = {triangle,draw,border rotated, inner sep=0pt},
    smalldist/.style = {triangle,draw,border rotated},
    kernel/.style={rectangle,draw,inner sep = 2pt},
    expectation/.style = {triangle,draw,inner sep=0pt,shape border rotate=270},
    copymap/.style = {circle,fill,inner sep=1pt}}

\newcommand\DCI{
    \begin{tikzpicture}[scale=0.35]
    \draw[->] (1,0) -- (0,0);
    \draw (0.6,0) -- (0.6,0.75);
    \draw (0.4,0) -- (0.4,0.75);
    \end{tikzpicture}
}

\newcommand\splitter[1]{%
\begin{tikzpicture}[scale=#1]
\draw (0,-1) -- (0,0);
\draw (0,0) to [bend right] (1,1);
\draw (0,0) to [bend left] (-1,1);
\end{tikzpicture}
}

\newcommand\stopper[1]{%
\begin{tikzpicture}[scale=#1]
\draw[-{Rays [n=8]}] (0,-1) -- (0,0);
\end{tikzpicture}
}

\newcommand\swap[1]{%
\begin{tikzpicture}[scale=#1]
\draw (0,0) to [out=90, in=270] (0.5,1);
\draw (0.5,0) to [out=90,in=270] (0,1);
\end{tikzpicture}
}

\newcommand\source[1]{%
\begin{tikzpicture}[scale=#1]
\path (0,0) node[prob,fill=gray] (P) {};
\draw (P) -- ($(P.east) + (1,0)$);
\end{tikzpicture}
}

\DeclareMathOperator*{\argmax}{arg\,max}
\DeclareMathOperator*{\argmin}{arg\,min}
\DeclareMathOperator*{\arginf}{arg\,inf}
\DeclareMathOperator*{\argsup}{arg\,sup}

\newcommand{\cheng}[1]{ {\color{purple}[{\bf Cheng:~{#1}}]} }


%----------------------------------------------------------------------------------------
%	BOX SETTINGS
%----------------------------------------------------------------------------------------
% from https://texblog.org/2015/09/30/fancy-boxes-for-theorem-lemma-and-proof-with-mdframed/

%Proof
\newcounter{prf}[section]\setcounter{prf}{0}
\renewcommand{\theprf}{\arabic{chapter}.\arabic{section}.\arabic{prf}}
\newenvironment{prf}[2][]{%
\refstepcounter{prf}%
\ifstrempty{#1}%
{\mdfsetup{%
frametitle={%
\tikz[baseline=(current bounding box.east),outer sep=0pt]
\node[anchor=east,rectangle,fill=red!20]
{\strut Proof~\theprf};}}
}%
{\mdfsetup{%
frametitle={%
\tikz[baseline=(current bounding box.east),outer sep=0pt]
\node[anchor=east,rectangle,fill=red!20]
{\strut Box~\theprf:~#1};}}%
}%
\mdfsetup{innertopmargin=10pt,linecolor=red!20,%
linewidth=2pt,topline=true,%
frametitleaboveskip=\dimexpr-\ht\strutbox\relax
}
\begin{mdframed}[]\relax%
\label{#2}}{\end{mdframed}}
%%%%%%%%%%%%%%%%%%%%%%%%%%%%%%


%----------------------------------------------------------------------------------------
%	MARGIN SETTINGS
%----------------------------------------------------------------------------------------

\geometry{
	paper=a4paper, % Change to letterpaper for US letter
	inner=2.5cm, % Inner margin
	outer=2.5cm, % Outer margin
	bindingoffset=.5cm, % Binding offset
	top=1.5cm, % Top margin
	bottom=1.5cm, % Bottom margin
	%showframe, % Uncomment to show how the type block is set on the page
}