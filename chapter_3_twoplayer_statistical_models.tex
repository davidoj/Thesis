
%!TEX root = main.tex

\chapter{Two player statistical models and see-do models}\label{ch:2p_statmodels}

\todo[inline]{These are ``todo'' notes. All such notes that involve theoretical development are also collected in an unordered list of outstanding theoretical questions}

In this chapter I introduce two types of model. Models of the first type are called \emph{two player statistical models} and the second type are a special class of the first called \emph{see-do models}. Fundamentally, each of these is just a particular kind of stochastic function. The reason we are interested in these kinds of stochastic functions is that is that almost all causal models are instances of see-do models. Before introducing two player models and discussing what makes them causal, it is worth briefly considering models in statistics and machine learning generally.

A \emph{world model} is something I will informally define as a family of ``descriptions'' indexed by hypotheses $\{R_h|h\in H\}$. The set $H$ represents hypotheses or proposals for how the world ought to be described, and each proposal $h\in H$ entails some description of the world $\prob{R}_h$. Some examples of world models:

\begin{itemize}
    \item A linear regressor may take some data $\mathbf{x}$ and $\mathbf{y}$ and returns a parameter $\beta\in B$ with the property that $(\mathbf{y}-\mathbf{x}^T \beta)^2$ is small. A normal way to interpret the parameter $\beta$ is to consider it to be a proposal about how some phenomenon of interest should be described, with this description explicitly given by the function $f:x\mapsto \beta x$.
    \item A neural network used in classification may take data $\mathbf{x}$ and labels $\mathbf{y}$ and returns parameters $\mathbf{w}\in W$ with the property that $-\mathbf{y} \log \mathbf{x}+(1-y)\log(1-\mathbf{x})$ is small. Each $\mathbf{w}$ is a proposal for how to classify data and the classification rule associated with each $\mathbf{w}$ is a function $x\mapsto f(\mathbf{w},x)$.
    \item A crude description of a general election pre-poll result can be given by the ``true fraction'' $\theta$ of voters for each candidate and, under some unreasonably strong sampling assumptions, and the results of the survey for each $\theta$ can be described by $\prod_N\prob{P}_\theta^\RV{X}$ where $N$ is the number of voters surveyed and $\RV{X}$ is the vote choice of each.
\end{itemize}

In the first two examples the ``description'' that goes with each hypothesis is a function, while in the third example the descriptions are probability measures. In almost all practical cases, these descriptions of the world do not tell us exactly how the world will turn out under each hypothesis, but at best offer us a prediction that is as good as we can hope for. Probability is the tool that is very widely used to formalise such ``descriptions with uncertainty''. Say I have two different linear regressors: one which minimises squared error on the training data and one that always returns $\beta=10$. I want to ask which one produces descriptions that are more fit for my purpose. It is pointless to ask which one is correct because, in general, I cannot know that either will offer a description that is even approximately correct. However, I can consider a second level world model $\{\prob{P}^{\RV{X}\RV{Y}}_\alpha|\alpha\in A\}$ in which the phenomenon of interest is described by a family of probability measures, and then I can ask, given an $\alpha$, which $\beta$ is my regressor likely to return and how closely will $x\mapsto \beta x$ be to $\mathbb{E}_{\prob{P}_\alpha}[\RV{Y}|\RV{X}]$ for each likely choice. Generally, if I need to model a world with uncertainty I will need a world model that is an indexed family of probability measures.

A world model that consists of a family of probability measures $\{\prob{P}_h|h\in H\}$ is a \emph{statistical model} or \emph{statistical experiment}. Because I almost always need to Statistical models can be found everywhere in theoretical statistics and machine learning \cite{fisher_statistical_1992,le_cam_comparison_1996,freedman_asymptotic_1963,de_finetti_foresight_1992,vapnik_nature_2013,wald_statistical_1950}. A key point about statistical models -- even if I can only state it somewhat vaguely -- is that the truth of any hypothesis $h\in H$ has no dependence on what I might want to be true. As a user of statistical models, I have no authority to choose a hypothesis -- this is Nature's choice alone. 

I can sometimes make choices that will affect the way that the future turns out. I might have some set $D$ of choices I can make, and for each $d\in D$ I require a description of the results of my choice. Just as the results of hypotheses are often uncertain, so are the results of choices. I might be motivated to choose a probability measure $\prob{P}_d$ to describe them, maybe because it is common to do so or because I find arguments for subjective expected utility theory compelling \citep{steele_decision_2020}. A family of probability measures indexed by a set of choices $\{\prob{P}_d|d\in D\}$ will be called a \emph{consequence model}.

Statistical models and consequence models are both families of probability measures indexed by arbitrary sets, which we have called hypotheses $H$ and choices $D$ respectively. These sets are distinguished by how they are interpreted when a two-player statistical model is used in the course of solving some kind of problem. The difference can be informally summarised in this manner: I do not get to tell Nature what choice $h\in H$ she makes, and Nature does not get to tell me what choice $d\in D$ I make. It will often be the case that I have multiple choices that can affect how the world turns out \emph{and} I have multiple hypotheses about how each choice will affect the world. In this case, I will have a \emph{two-player statistical model} $\{\prob{P}_{h,d}|h\in H,d\in D\}$. 

So far I have explained the distinction between ``player 1'' and ``player 2'' in vague metaphorical terms. If I am using a two-player statistical model in the context of a well defined problem such as ``given data, what choice should I make?'' then we can say precisely what $H$ and $D$ are and what role each plays in the problem. However, the field of causal inference includes other types of problem such as counterfactual problems which involve a choice set $D$ that plays a different role to the choice set in decision problems. Thus, while I will argue that causal models are two-player statistical models, and the second player is what distinguishes them from ordinary statistical models, the same kind of model can be used with different interpretations of what the second player's choices represent. This will be explored in more detail in the coming chapters.

\todo[inline]{Note to proof readers: I moved the discussion of decomposability to the next chapter so I can introduce it alongside the result that uses it}

\section{Two player statistical models and see-do models}

Two player statistical models were introduced as doubly indexed sets of probability measures $\{\prob{P}_{h,d}|h\in H,d\in D\}$. If each $\prob{P}_{h,d}\in \Delta(\sigalg{E})$ for some measurable space $(E,\sigalg{E})$, the indexed set is equivalent to a function $H\times D\to \Delta(\sigalg{E})$. In the following work, we will make two simplifying assumptions:

\begin{enumerate}
    \item A two player statistical model can be represented by a \emph{Markov kernel} $\kernel{T}:H\times D\to \Delta(\sigalg{E})$
    \item The kernel space $(\kernel{T},(H\times D,\sigalg{H}\otimes\sigalg{D}),(E,\sigalg{E}))$ admits disintegrations $\kernel{T}^{\RV{Y}|\RV{XDH}}$ for arbitrary random variables $\RV{X},\RV{Y}$ on $H\times D\times E$ and domain variable $\RV{D}\utimes\RV{H}$
\end{enumerate}

The first condition amounts to the additional requirement that $(h,d)\mapsto \kernel{T}_{h,d}(A)$ is measurable for every $A\in \sigalg{H}\otimes\sigalg{D}\otimes\sigalg{E}$, and sufficient for the second condition is that $D\times H$ is countable and $X\times Y$ standard measurable (though this is not necessary, see Theorem \ref{th:existence_continous}).

\begin{definition}[Two player statistical model]\label{def:2p_stat}
A \emph{two-player statistical model} $(\kernel{T},\RV{H},\RV{D},\RV{O})$ is a Markov kernel $\kernel{T}:H\times D\to \Delta(\sigalg{O})$ such that, for any random variables $\RV{X}: H\times D\times O\to X$ and $\RV{Y}:H\times D\times O\to Y$, a disintegration $\kernel{K}^{\RV{Y}|\RV{XDH}}:X\times D\times H\to \Delta(\sigalg{Y})$ exists along with three distinguished random variables: the \emph{hypothesis} $\RV{H}:H\times D\times O\to H$ given by $(h,d,o)\mapsto h$ (forgetting the choice and outcome) and the \emph{choice} $\RV{D}:H\times D\times O\to D$ given by $(h,d,o)\mapsto d$ (forgetting the hypothesis and outcome) and the \emph{outcome} $\RV{O}:H\times D\times O\to O$ given by $(h,d,o)\mapsto o$ (forgetting the choice and hypothesis).
\end{definition}

Decision problems involving often involve some data $\RV{X}$ is observed, then a choice is made, then the consequences $\RV{Y}$ are observed. In such a model, the observed data $\RV{X}$ cannot be affected by the choice. These models will be called \emph{see-do} models to capture the assumption that there is an order in which seeing and doing happen.

\begin{definition}[See-Do model]\label{def:seedo}
A \emph{see-do model} $(\kernel{T},\RV{H},\RV{D},\RV{X},\RV{Y})$ is a two-player statistical model $(\kernel{T},\RV{H},\RV{D},\RV{O})$ with two additional distinguished random variables: the \emph{observation} $\RV{X}: H\times D\times O\to X$ and the \emph{consequence} $\RV{Y}:H\times D\times O\to Y$ such that the outcome is the coupled product of the observation and the consequence $\RV{O}=\RV{X}\utimes\RV{Y}$. A see-do model must observe the conditional independence $\RV{X}\CI_\kernel{T} \RV{D}|\RV{H}$, i.e. the observation is independent of the choice conditional on the hypothesis.

Because $\RV{O}=\RV{X}\utimes \RV{Y}$, we do not need to explicitly define $\RV{O}$ when specifying a see-do model.
\end{definition}


\section{Frequentist random variables and Bayesian forecasts}

We've chosen to represent the ``description of the world'' using probability. This is an overwhelmingly common choice for describing things with uncertainty, but it is worth asking precisely what is being described here. It's well-known that probability is suitable for representing a number of different things. Two common choices are:

\begin{enumerate}
    \item The long run convergence of relative frequencies of sequences or ensembles of observations of certain types of systems (\emph{frequentist probability})
    \item Forecasts of observations that will take place in the future, or, more generally, forecasts of things which we are uncertain about for any reason (\emph{Bayesian forecasts})
\end{enumerate}

The first view is a very common interpretation of probability as it is used in statistical models. One can view each hypotheses in a statistical model as representing the proposition that the system will tend to produce long sequences of observations with relative frequencies consistent with the associated probability measure. For example, given a possibly loaded die, we might entertain hypotheses a) it is a system that produces a 6 $\frac{1}{6}$ of the time, b) it is a system that produces a 6 $\frac{1}{4}$ of the time, and many other hypotheses besides. 

On the other hand, if I view a sequence of random variables as a sequence of \emph{Bayesian forecasts} then I do not strictly need to entertain a set of hypotheses regarding the long run relative frequencies of observations. If $\RV{X}_1,\RV{X}_2,\RV{X}_3,...$ are rolls of a possibly loaded die, then I can make forecasts from a single probability distribution over $\RV{X}_{[n]}$ (sometimes called the ``posterior distribution''). Furthermore, this distribution need not have any particular frequentist properties: I can forecast the probability of a $6$ is, for each roll, $1,0,1,1,0,0,0,0,1,1,1,1,1,1,1,1,1$ with no dependence on previusly seen rolls, such that relative frequencies never converge. Provided my forecast is a valid probability measure, it may seem unwise but it is still a well formulated Bayesian forecast.

Consider a see-do model in which $\RV{Y}\CI\RV{D}|\RV{H}$ - that is, fixing each hypothesis, the consequence $\RV{Y}$ is independent of the choice $\RV{D}$. This may be a model of a situation in which we consider a number of hypotheses possible, and for each hypothesis the consequences we will observe will actually be independent of the choices we make (where ``actually independent'' means something like ``independent in the frequentist limit'', acknowledging that this isn't a fully satisfactory definition, see for example \citet{hajek_interpretations_2019}).  However, it could also model a situation in which we consider a number of hypotheses viable and within each hypothesis we are indifferent between different relationships $\RV{D}$ and $\RV{Y}$ might have, without expressing any view on whether the values obtained from repeated experiments would be independent in the frequentist sense (in any case, such repetitions may be impossible).

While forecasts are more commonly associated with a single probability measure and frequentist models with sets of probability measures indexed by hypotheses (also called parameters or states), this does not have to be the case. For example, \citet{walley_statistical_1991} develops a theory of reasoning under uncertainty he calls \emph{imprecise probability} that makes use of sets of probability measures.

Properties of see-do models like $\RV{Y}\CI\RV{D}|\RV{H}$ may sometimes be objective claims about relationships that may or may not be observed under future experimentation. At the same time, the names given to the random variables -- $\RV{Y}$ is the \emph{consequence} and $\RV{D}$ is the \emph{choice} -- suggest that these variables may often refer to things that haven't happened yet. In fact, this is explicit in the following chapter where we consider using see-do models to make decisions -- if I am considering which decision I should make, then by definition I have not made that decision yet. Variables that refer to things that haven't happened yet are, as mentioned above, forecasts.

So it seems that we sometimes want to view see-do models as Bayesian forecasts of the consequences of choices, and sometimes we also want properties of see-do models to describe ``objective facts'' of some sort. In Chapter \ref{ch:ints_counterfactuals} we will discuss how Causal Bayesian Networks are a particular kind of see-do model, so any discussion of what we want to use see-do models to represent also has some bearing on what we want to use Causal Bayesian Networks to represent. Furthermore, in the first chapter of \citet{pearl_causality:_2009} we find examples of both of the views we have considered so far. Firstly, in introducing probability Pearl writes

\begin{quote}
We will adhere to the Bayesian interpretation of probability, according to which probabilities encode degrees of belief about events in the world and data are used to strengthen, update, or weaken those degrees of belief. In this formalism, degrees of belief are assigned to propositions (sentences that take on true or false values) in some language, and those degrees of belief are combined and manipulated according to the rules of probability calculus.
\end{quote}

While later he writes

\begin{quote}
[...] causal relationships are ontological, describing objective physical constraints in our world, whereas probabilistic relationships are epistemic, reflecting what we know or believe about the world.
\end{quote}

Causal Bayesian Networks are see-do models, so if Causal Bayesian Networks describe causal relationships and causal relationships are objective facts then see-do models must describe objective facts. What exactly does it mean for see-do models to describe objective facts?

In the world of ``ordinary'' one-player statistical models, De Finetti's Representation Theorem establishes that if we are using probabilities to represent forecasts and we are willing to assume that these forecasts are \emph{exchangeable} -- that is, the probability distribution we choose to represent our forecast of future events is exactly the same as the probability distribution we would choose to represent our forecasts of any permutation of these events -- then it is possible to define a random variable representing the \emph{hypothesis} such that the probability distribution representing our forecast can be written as the product of a \emph{prior distribution} over hypotheses and a statistical model that maps hypotheses to independent and identically distributed sequences of observations, which is a sequence of precisely the type that converges in the frequentist sense.

Thus, one way that statistical models could describe objective facts is if forecasts can be assumed to be exchangeable, in which case the forecast can be factored into a prior and a statistical model describing  independent and identically distributed sequences. As \citet{walley_statistical_1991} cautions, exchangeability is quite a strong assumption -- for example, it entails that one will never believe, no matter how much evidence is apparent, that a sequence is cyclic rather than converging to a stable relative frequency. Acknowledging there is a lot more to explore in this area, in Theorem \ref{th:rep_dex_sdf} we extend De Finetti's representation theorem to see-do models and show that, considering \emph{see-do forecasts}, if both observations and consequences satisfy a kind of exchangeability then the forecast is the product of a prior and a see-do model which, given a hypothesis, describes a sequence of independent and identical consequence maps.

\begin{definition}[Forecasts, see-do forecast]\label{def:do_forecast}
A \emph{do forecast} $(\kernel{F},\RV{D},\RV{O})$ is a Markov kernel $\kernel{F}:D\to \Delta(\sigalg{O})$ for some set of choices $D$ and outcomes $O$. The choice variable $\RV{D}:D\times O\to D$ is the map $(d,e)\mapsto d$ that forgest the outcome and and the outcome variable $\RV{O}:D\times O\to O$ is the map $(d,o)\to o$ that forgets the choice.

A \emph{see-do forecast} is a forecast $(\kernel{F},\RV{D},\RV{X},\RV{Y})$ with an \emph{observation variable} $\RV{X}:D\times O\to X$ and a \emph{consequence variable} $\RV{Y}:D\times O\to Y$ such that $\RV{O}=\RV{X}\utimes\RV{Y}$ and $\RV{X}\CI\RV{D}$.

A \emph{forecast} $(\kernel{F},\RV{O})$ is a probability measure $\prob{F}\in\Delta(\sigalg{O})$ and an outcome variable $\RV{O}:O\to O$.
\end{definition}

We can go from a see-do model to a see-do forecast by adding a prior to the model.

\begin{theorem}
Given a two player statistical model $(\kernel{K},\RV{H},\RV{D},\RV{O})$ and any prior $\mu\in \Delta(\sigalg{H})$, defining $\kernel{L}:=(\mu\otimes \mathrm{Id}_D)\kernel{K}$ we have $(\kernel{L},\RV{D},\RV{O})$ is a do-forecast. 

If $(\kernel{K},\RV{H},\RV{D},\RV{X},\RV{Y})$ is a see-do model then, defining $\kernel{L}$ as before, $(\kernel{L},\RV{D},\RV{X},\RV{Y})$ is a see-do forecast.
\end{theorem}


\begin{proof}
The first part is trivial: $(\mu\otimes \mathrm{Id}_D)\kernel{K}$ is a Markov kernel $D\to \Delta(\sigalg{O})$ by construction, and $\RV{D}$ and $\RV{O}$ are choice and outcome variables by definition of the original two player statistical model.

For the second part $\RV{X}\CI_{\kernel{L}}\RV{D}$ is required. By Theorem \ref{th:iterated_disint} we have

\begin{align}
    \kernel{K} &= \kernel{K}^{\RV{X}\RV{Y}|\RV{HD}}\\
    &= \begin{tikzpicture}
        \path (0,0) node (H) {$\RV{H}$}
        + (0,-1.15) node (D) {$\RV{D}$}
        ++ (0.5,0) node[copymap] (copy0) {}
        ++ (0.7,0) node[kernel] (X) {$\kernel{K}^{\RV{X}|\RV{H}}$}
        ++ (0.5,0) node[copymap] (copy1) {}
        +  (0.8,-1) node[kernel] (Y) {$\kernel{K}^{\RV{Y}|\RV{XHD}}$}
        ++ (2,0) node (Xr) {$\RV{X}^{sd}$}
        +  (0,-1) node (Yr) {$\RV{Y}^{sd}$};
        \draw (H) -- (X) -- (Xr);
        \draw (copy0) to [out=-90,in=180] ($(Y.west) + (0,0)$) 
              (copy1) to [out=-90,in=180] ($(Y.west) + (0,0.15)$)
              (D) -- ($(Y.west) + (0,-0.15)$);
        \draw (Y) -- (Yr);
    \end{tikzpicture}
\end{align}

It follows that

\begin{align}
    \kernel{L}^{\RV{X}\RV{Y}|\RV{D}} &= \kernel{L}\\
    &= (\mu\otimes \mathrm{Id}_D)\kernel{K}\\
    &=  \begin{tikzpicture}
        \path (0,0) node[dist] (H) {$\mu$}
        + (0,-1.15) node (D) {$\RV{D}$}
        ++ (0.5,0) node[copymap] (copy0) {}
        ++ (0.7,0) node[kernel] (X) {$\kernel{K}^{\RV{X}|\RV{H}}$}
        ++ (0.5,0) node[copymap] (copy1) {}
        +  (0.8,-1) node[kernel] (Y) {$\kernel{K}^{\RV{Y}|\RV{XHD}}$}
        ++ (2,0) node (Xr) {$\RV{X}$}
        +  (0,-1) node (Yr) {$\RV{Y}$};
        \draw (H) -- (X) -- (Xr);
        \draw (copy0) to [out=-90,in=180] ($(Y.west) + (0,0)$) 
              (copy1) to [out=-90,in=180] ($(Y.west) + (0,0.15)$)
              (D) -- ($(Y.west) + (0,-0.15)$);
        \draw (Y) -- (Yr);
    \end{tikzpicture}
\end{align}

Then

\begin{align}
    \kernel{L}^{\RV{X}|\RV{D}} &= \begin{tikzpicture}
        \path (0,0) node[dist] (H) {$\mu$}
        + (0,-1.15) node (D) {$\RV{D}$}
        ++ (0.5,0) node[copymap] (copy0) {}
        ++ (0.7,0) node[kernel] (X) {$\kernel{K}^{\RV{X}|\RV{H}}$}
        ++ (0.5,0) node[copymap] (copy1) {}
        +  (0.8,-1) node[kernel] (Y) {$\kernel{K}^{\RV{Y}|\RV{XHD}}$}
        ++ (2,0) node (Xr) {$\RV{X}$}
        +  (0,-1) node (Yr) {};
        \draw (H) -- (X) -- (Xr);
        \draw (copy0) to [out=-90,in=180] ($(Y.west) + (0,0)$) 
              (copy1) to [out=-90,in=180] ($(Y.west) + (0,0.15)$)
              (D) -- ($(Y.west) + (0,-0.15)$);
        \draw[-{Rays[n=8]}] (Y) -- (Yr);
    \end{tikzpicture}\\
    &= \begin{tikzpicture}
        \path (0,0) node[dist] (H) {$\mu$}
        + (0,-1.) node (D) {$\RV{D}$}
        ++ (1.2,0) node[kernel] (X) {$\kernel{K}^{\RV{X}|\RV{H}}$}
        ++ (2.5,0) node (Xr) {$\RV{X}$}
        +  (0,-1) node (Yr) {};
        \draw (H) -- (X) -- (Xr);
        \draw[-{Rays[n=8]}] (D) -- (Yr);
    \end{tikzpicture}
\end{align}

And so $\RV{X}\CI_{\kernel{L}}\RV{D}$.
\end{proof}

In addition, any see-do forecast can be interpreted as a see-do model with a single hypothesis. Recalling the discussion of the indiscrete space $\{*\}$ in \ref{sec:string_diagram_elements}, we can identify a Markov kernel $\kernel{F}:D\to \Delta(\sigalg{E})$ with a Markov kernel $\kernel{T}:\{*\}\times D\to \Delta(\sigalg{E})$ where $\kernel{T}_{*,d}=\kernel{F}_d$ for all $d\in D$. Defining the hypothesis $\RV{H}:\{*\}\times D\times E\to H$ given by the constant fuction $(*,d,e)\mapsto *$, we can create from any see-do forecast $(\kernel{F},\RV{D},\RV{X},\RV{Y})$ a see-do model $(\RV{T},\RV{H},\RV{D},\RV{X},\RV{Y})$ (the required conditional independence is observed by construction in the single hypothesis $*$). However, this single hypothesis model is typically not a \emph{frequentist model}.

\citet{de_finetti_foresight_1992} has shown how frequentist models in particular can be recovered from exchangeable forecasts. Informally speaking, if and only if a forecast $(\prob{P},\RV{O})$ has the property that distribution of a sequence of random variables $\prob{P}^{\RV{X}_1\RV{X}_2\RV{X}_3}$ is identical to the distribution of any permutation of the sequence $\prob{P}^{\RV{X}_2\RV{X}_1\RV{X}_3}$ (an assumption known as \emph{exchangeability}), and this sequence can be extended infinitely, then there exists a hypothesis class $(H,\sigalg{H})$, a Markov kernel $\kernel{Q}:H\to \Delta(\sigalg{O})$ and a \emph{prior} $\mu\in \Delta(\sigalg{H})$ such that

\begin{align}
    \prob{P}^{\RV{X}_1\RV{X}_2\RV{X}_3} = \begin{tikzpicture}
        \path (0,0) node[dist] (P) {$\mu$}
        ++ (0.7,0) node[copymap] (copy0) {}
        ++ (0.5,0.5) node[kernel] (Q1) {$\kernel{Q}$}
        +  (0,-0.5) node[kernel] (Q2) {$\kernel{Q}$}
        +  (0,-1) node[kernel] (Q3) {$\kernel{Q}$}
        ++ (1,0) node (X1) {$\RV{X}_1$}
        + (0,-0.5) node (X2) {$\RV{X}_2$}
        + (0,-1) node (X3) {$\RV{X}_3$};
        \draw (P) -- (copy0);
        \draw (copy0) to [out=45,in=180] (Q1) (copy0) to [out=0, in=180] (Q2) (copy0) to [out=-45,in=180] (Q3);
        \draw (Q1) -- (X1) (Q2) -- (X2) (Q3) -- (X3);
    \end{tikzpicture}
\end{align}

Defining the hypothesis $\RV{H}:O\mapsto H$ such that $\prob{P}^{\RV{H}}=\mu$ and $\prob{P}^{\RV{O}|\RV{H}}=\kernel{Q}$, $(\kernel{Q},\RV{H},\RV{O})$ is a statistical model.

In the following section, we extend this result to the case of see-do forecasts. We first consider a model that is exchangeable in the observations only, and then introduce the notion of functional exchangeability which is a generalisation of exchangeability to Markov kernels. Finally, we prove a representation theorem for see-do forecasts that are both exchangeable in observations and functionally exchangeable in consequences.

\begin{definition}[Permutations and swaps]\label{def:permut_swap}
A \emph{finite permutation} $\rho'$ on $B\subseteq\mathbb{N}$ is a map $B\to B$ such that there is some finite $A\subset B$ for which $\rho'|_A:A\to A$ is an invertible function and $\rho'|_{B\setminus A} = \mathrm{Id}_{B\setminus A}$.

Given measureable space $(E,\sigalg{E})$ and a set of random variables $\{\RV{X}_i|i\in B\}$ the swap function $\rho^{\RV{X}}:E\to E$ associated with a finite permutation $\rho:B\to B$ and the random variables $\{\RV{X}_i\}_B$ is a $\sigalg{E}$ measurable function which has the property $\RV{X}_i\circ \rho^{\RV{X}}=\RV{X}_{\rho'(i)} $ for all $i\in B$, and for any $\RV{Y}:E\to Y$ with $\RV{Y}(\RV{X}^{-1}(A))=Y$ for any $A\in \sigalg{E}$, $Y\circ\rho^{\RV{X}}=\RV{Y}$. This swap function also has an associated Markov kernel $R:=\kernel{F}_{\rho^{\RV{X}}}$.

For example, if $E = Y\times X_1^{|B|}$ and $\RV{X}_i:E\to X_1$ projects the $i$-th ``x'' element of the space $(y,x_1,...,x_i,...)\mapsto x_i$, then for some finite permuation $\rho$ the associated swap is the the fuction $\rho^{\RV{X}}:(y,x_1,...,x_i,...)\mapsto (y,x_{\rho'(1)},...,x_{\rho'(i)},...)$.
\end{definition}

\begin{align}
    R \utimes_{i\in B} \kernel{F}_{\RV{X}_i} &= \utimes_{i\in B} R \kernel{F}_{\RV{X}_{i}}\label{eq:determ_commute}\\
                                               &= \utimes_{i\in B} \kernel{F}_{\RV{X}_{\rho(i)}}\label{eq:function_composition}
\end{align}

Where line \ref{eq:determ_commute} follows from the fact that deterministic kernels commute with the split map (\ref{eq:composition}), and line \ref{eq:function_composition} follows from the fact that for two functional kernels 

\begin{align}
    (\kernel{F}_{f}\kernel{F}_g)_x(A) &= \int_X (\kernel{F}_g)_y(A) d(\kernel{F}_f)_x(y)\\
                                      &= \int_X \delta_{g(y)}(A) d\delta_{f(x)}(y)\\
                                      &= \delta_{g(f(x))} (A)\\
                                      &= (\kernel{F}_{g\circ f})_x(A) 
\end{align}

\todo[inline]{lemmafy, move to chapter 2}

\begin{definition}[Partial frequencies]\label{def:partial_freq}
Given a standard measurable space $(E,\sigalg{E})$ along with random variables $\RV{X}_i:E\to X_1$ for each $i\in \mathbb{N}$, for $A\in \sigalg{X}_1$ and $m\leq n\in \mathbb{N}$ define the size $n$, $m$-tuple \emph{partial frequency of} $A$ with respect to $\RV{X}:=\utimes_{i\in\mathbb{N}}\RV{X}_i$ to be $\RV{Z}^{m,n}_A:=\frac{(n-m)!}{n!}\sum_{I\subset [n]} \prod_{i\in I} \mathds{1}_A\circ \RV{X}_i$ where $I$ ranges over all $m$-sized ordered subsets of $n$.

Define the $m$-tuple \emph{relative frequency of } $A$ with respect to $\RV{X}$ to be $\RV{Z}^{m,\infty}_A:= \lim_{n\to \infty}\frac{(n-m)!}{n!}\sum_{I\in [n]} \prod_{i\in I} \mathds{1}_A\circ \RV{X}_i$.

Given a countable set $\sigalg{G}$ generating $\sigalg{X}_1$ (i.e. $\sigma(\sigalg{G})=\sigalg{X}_1)$, define the $m$-tuple \emph{relative frequency of} $\RV{X}$ to be $\RV{Z}^{m,\infty}:=\utimes_{A\in G} \RV{Z}^{m,\infty}_A$, if such a limit exists for all $A$.

For the special case of $m=1$, let $\RV{Z}:=\RV{Z}^{1,\infty}$
\end{definition}

\begin{definition}[Exchangeable $\sigma$-algebra]\label{def:exchange_sig_alb}
Given a set of random variables $\RV{X}_i:E\to X_1$ for each $i\in \mathbb{N}$, with $X=X_1^{\mathbb{N}}$, a \emph{n-place symmetric function} $f:X\to W$ is a function for which $f = f\circ \rho$ for any permuation $\rho:\mathbb{N}\to\mathbb{N}$ such that $i>n\implies\rho(i)=i$. 

The \emph{n-place exchangeable $\sigma$-algebra} (with respect to the random variable $\RV{X}_i$), $\sigalg{H}^n$, is the $\sigma$-algebra generated by all n-place symmetric functions, and $\sigma{H}=\cap_{n\in \mathbb{N}} \sigalg{H}^n$. 

For standard measurable $(E,\sigalg{E})$ and $n\in \mathbb{N}$, a size $n$ swap function $\rho^{\RV{X}n}:E\to E$ is a swap function associated with a permutation $\rho^{\prime n}$ with the property $i>n\implies \rho^{\prime n}(i)=i$. An $n$-symmetric set $S\subset E$ has the property $\rho^{\RV{X}n}(S)=S$ for all size $n$ swap functions $\rho^{\RV{X}n}$. Define the symmetric sets $\sigalg{S}^n$ $n$-symmetric sets, and $\sigalg{S}=\cap_{n\in\mathbb{N}} \sigalg{S}^n$. Given random variables $\RV{X}_i:E\to X_1$ for each $i\in\mathbb{N}$, the size $n$ \emph{exchangeable $\sigma$-algebra} with respect to $\RV{X}:=\utimes_{i\in\mathbb{N}}\RV{X}_i$, denoted $\sigalg{H}^n\subset\sigma(\RV{X})$, is the $\sigma$-algebra generated by $\{\RV{X}^{-1}(A)|A\in\sigalg{X}\}\cap S^n$.

\end{definition}

\begin{lemma}
The size $n$ exchangeable sigma algebra $\sigalg{H}^n$ on $(E,\sigalg{E})$ with respect to $\RV{X}:=\utimes_{i\in\mathbb{N}}\RV{X}_i$ has the property $\rho^{\RV{X}n}(A)=A$ for all $A\in \sigalg{H}^n$, and all size $n$ swap functions $\rho^{\RV{X}n}$.
\end{lemma}

\begin{proof}
Let $W_f$ be the codomain of a function $f$, and $\sigalg{W}_f$ its $\sigma$-algebra. $\sigalg{H}^n$ is generated by $\sigalg{G}^n=\{f^{-1}(A)|A\in \sigalg{W}_f,f\text{ n-place symmetric}\}$. By the definition of n-place symmetric functions, any set of the form $f^{-1}(A)=(f\circ\rho^{\RV{X}})^{-1}(A) = (\rho^{\RV{X}}){-1}(f^{-1}(A))$. Because every $n$-place permutation $\rho$ has an inverse $\rho^{-1}$ that is also an $n$-place permutation, all sets in $B\in\sigalg{G}$ have the property $\rho^{\RV{X}}(B)=B$ for all $n$-place permuations $\rho$.

Define $\sigalg{S}^n$ to be all subsets of $E$ such that for $B\in \sigalg{S}^n$, $n$-place permuations $\rho$, $\rho^{\RV{X}}(B)=B$. $\sigalg{S}^n$ is a $\sigma$-algebra, and it contains $\sigalg{G}^n$, so it also contains $\sigalg{H}^n$.

Take $A\in\sigalg{S}^n$. By assumption, for any $\omega\in A$, $\rho^{\RV{X}}(\omega)\in A$ for all size $n$ swap functions $\rho^{\RV{X}}$. Consider $\omega\not\in A$, and suppose there is some swap function $\rho^{\RV{X}}$ such that $\rho^{\RV{X}}(\omega)\in A$. By definition, the permutation $\rho$ has an inverse $\rho^{-1}$ which is also a size $n$ permutation. By construction, $(\rho^{-1})^{\RV{X}}$ is also the inverse of $\rho^{\RV{X}}$. Thus $(\rho^{-1})^{\RV{X}}(\omega)\not\in A$ and so $\omega\not\in A$, a contradiction. Thus $E\setminus A\in \sigalg{G}^n$.

For any invertible function $f:E\to E$, $f(E)=E$. Thus $E\in \sigalg{G}^n$.

Finally, for a countable collection $A_1,A_2,...$ and any size $n$ swap function $\rho^{\RV{X}}$, $\rho^{\RV{X}}(\cup_{i=1}^{\infty} A_i) = \cup_{i=1}^\infty \rho^{\RV{X}}(A_i) = \cup_{i=1}^\infty A_i$. Thus $\sigalg{S}^n$ is a $\sigma$-algebra, and by the monotone class theorem it contains $\RV{H}^n$.
\end{proof}

\begin{definition}[Exchangeability]
Given a see-do forecast $(\kernel{T},\RV{D},\RV{X},\RV{Y})$ with the property that $\RV{X}:=\utimes_{i\in A} \RV{X}_i$ for some set of random variables $\{\RV{X}_i|i\in A\}$ all taking values in $X_1$ where $A\subseteq \mathbb{N}$. 

If for every finite $B\subset A$ and every permutation $\rho':B\to B$ of $B$ we have $\kernel{T}\rho=\kernel{T}$, where $\rho$ is the swap associated with $\rho'$ and $\{\RV{X}_i|i\in B\}$, then $(\kernel{T},\RV{D},\RV{X},\RV{Y})$ is \emph{exchangeable} with respect to $\{\RV{X}_i|i\in A\}$.

If $A$ is an infinite set then $\kernel{T}$ is \emph{infinitely exchangeable}, and if $\kernel{T}=\kernel{S}(\mathrm{Id}_{X}\otimes \stopper{0.15}\otimes \mathrm{Id}_Y)$ for some infinitely exchangeable $(\kernel{S},\RV{D},\RV{X}',\RV{Y}')$, then $\kernel{T}$ is infinitely exchangeably extendable.
\end{definition}

Note that $\kernel{T}R^{\RV{X}_i|\RV{D}} = \kernel{T}^{\RV{X}_{\rho(i)}|\RV{D}}$.

This implies the usual notion of exchangeability if we take $Y=\{*\}$ (that is, if we assume the consequences are trivial), as by assumption $\RV{X}$ is independent of $\RV{D}$.

\begin{lemma}[Infinitely exchangeably extendable forecasts]\label{lem:partial_representation}
Given a forecast $(\prob{P},\RV{X})$ where $\RV{X}=\utimes_{i\in A} \RV{X}_i$ for some $\{\RV{X}_i|i\in \mathbb{N}\}$ and $X$ is standard measurable, if $\prob{P}$ is exchangeable with respect to $\RV{X}$ then there exists a function $f:X\to H$ such that, defining $\RV{Z}:f\circ \RV{X}$:
    \begin{itemize}
        \item $\RV{X}_i\CI\RV{X}_{A\setminus\{i\}}|\RV{Z}$ for all $i\in A$
        \item $\prob{P}^{\RV{X}_i|\RV{Z}}=\prob{P}^{\RV{X}_j|\RV{Z}}$ for all $i,j\in A$
    \end{itemize}
Call $\RV{Z}$ the \emph{hypothesis}.
\end{lemma}

\begin{proof}
Without loss of generality, assume $X_1=[0,1]$, $\sigalg{X}=\mathcal{B}([0,1])$ and $\RV{X}=[0,1]^{\mathbb{N}}$.

Let $\mathbb{Q}$ be the rationals between $[0,1]$ and define $\RV{Z}_q^n:D\times X\times Y \to [0,1]$ by $\omega \mapsto \frac{1}{n}\sum_{i}^n \mathds{1}_{[0,q)}(\RV{X}_i(\omega))$ Let $\sigalg{Z}^n_q=\sigma(\RV{Z}_q^n)$, i.e. $\RV{Z}^n$ is a 1-tuple partial frequency as in Definition \ref{def:partial_freq}.

$\RV{Z}^n\circ\rho^{\RV{X}n}=\RV{Z}^n$ for any size $n$ swap function $\rho^{\RV{X}n}$, so $\RV{Z}^n$ is $\sigalg{H}^n$-measurable.

Let $\rho'_{ij}:\mathbb{N}\to\mathbb{N}$ swaps indices $i$ and $j$ for some $i,j\in[n]$ and otherwise acts as the identity. $\rho_{ij}:D\times X\times Y\to \Delta(\sigalg{D}\times \sigalg{X}\times \sigalg{Y})$ is the swap kernel associated with $\rho'_{ij}$ and $\{\RV{X}_i|i\in \mathbb{N}\}$, and $\rho^{\RV{X}}_{ij}$ the function associated with $\rho_{ij}$. For any $m$, $n$, $A\in \sigalg{H}^n$, $d\in D$: 

\begin{align}
    \int_{A} \RV{Z}_q^{n} (\omega) d\prob{P}(\omega) &= \int_{A} \frac{1}{n}\sum_{i}^{n} (\mathds{1}_{[0,q)}\circ \RV{X}_i)(\omega)d\prob{P}(\omega)\\
    &= \frac{1}{n}\sum_{i}^{n} \int_{(\rho^{\RV{X}}_{ij})^{-1}\rho^{\RV{X}}_{ij}(A)} (\mathds{1}_{[0,q)}\circ \RV{X}_i)(\omega)d\prob{P}\rho_{ij}(\omega)\label{eq:permutation_invertible}\\
    &= \frac{1}{n}\sum_{i}^{n} \int_{\rho^{\RV{X}}_{ij}(A)} (\mathds{1}_{[0,q)}\circ \RV{X}_{i}\circ \rho_{ij}) (\omega)d\prob{P}(\omega)\label{eq:using_push2}\\
    &= \frac{1}{n}\sum_{i}^{n} \int_{A} (\mathds{1}_{[0,q)}\circ \RV{X}_{j})(\omega)d\prob{P}(\omega)\label{eq:closure_under_permutation}\\
    &= \int_{A} (\mathds{1}_{[0,q)} \circ \RV{X}_j)(\omega)d\prob{P}(\omega) \label{eq:cond_expectation_first}
\end{align}

Where line \ref{eq:permutation_invertible} follows from exchangeability of $\kernel{T}$ and invertibility of $\rho_{ij}$. Line \ref{eq:using_push2} follows from the fact that $\prob{P}\rho_{ij}$ is the pushforward measure of $\prob{P}$ with respect to $\rho^{\RV{X}}_{ij}$ and \ref{eq:closure_under_permutation} uses the fact that $\rho(A) = A$ for all $A\in \RV{Z}^n_q$ and all permutations $\rho$.

From Equation \ref{eq:cond_expectation_first}, we have for all $n$, $A\in \sigalg{H}^{n+1}$

\begin{align}
    \int_{A} \RV{Z}_q^{n+1} (\omega) d\prob{P}(\omega) &= \int_{A} \RV{Z}_q^{n} (\omega) d\prob{P}(\omega)
\end{align}

Because $\RV{Z}_q^{n+1}$ is $\sigalg{H}^{n+1}$ measurable, $\RV{Z}_q^{n+1} = \mathbb{E}[\RV{Z}_q^{n}|\sigalg{H}^{n+1}]$.

Thus the sequence $[\RV{Z}^1_q,\RV{Z}^2_q,...]$ is a backwards martingale with respect to the reversed filtration $\sigalg{H}^1\supset\sigalg{H}^2\supset...\supset \sigalg{H}^3$.

Furthermore, for all $n\in \mathbb{N}$, $\sup_\omega|\RV{Z}^n(\omega)|\leq 1$ so the sequence is also uniformly integrable. Thus it goes almost surely to the limit $\RV{Z}_q$, and for all $A\in \sigalg{H}$

\begin{align}
    \lim_{n\to\infty} \int_A \RV{Z}^n_q(\omega) d\prob{P}(\omega) &= \int_A \RV{Z}_q(\omega) d\prob{P}(\omega)
\end{align}

Finally, because for all $n\in \mathbb{N}$, all $j\in[n]$ and all $A\in \sigalg{H}^n$ we also have

\begin{align}
    \int_A \mathds{1}_{[0,q)}(\RV{X}_j(\omega))d\prob{P}(\omega) &= \int_A \RV{Z}_q^n(\omega) d\prob{P}(\omega)
\end{align}

it follows that for all $A\in \sigalg{H}$, $j\in \mathbb{N}$

\begin{align}
    \int_A \RV{Z}_q(\omega) d\prob{P}(\omega) = \int_A \mathds{1}_{[0,q)}(\RV{X}_j(\omega))d\prob{P}(\omega)\label{eq:cond_expectation}
\end{align}

[\citet{cinlar_probability_2011} Thm 4.7.]. Thus $\RV{Z}_q = \mathbb{E}[\mathds{1}_{[0,q)}\circ\RV{X}_j|\sigalg{H}]$ for all $j\in \mathbb{N}$. This implies $\RV{Z}_q$ is a version of $\mathbb{E}[\mathds{1}_{[0,q)}\circ \RV{X}_j|\sigalg{H}]$.

Define $\RV{Z} = \utimes_q\in \mathbb{Q} \RV{Z}_q$. As $\sigma(\RV{Z})\subset\sigalg{H}$, Equation \ref{eq:cond_expectation} establishes in addition that $\RV{Z}_q$ is a version of $\mathbb{E}[\mathds{1}_{[0,q)}\circ \RV{X}_j|\sigma(Z)]$. 

By the definition of conditional expectation, for any version of $\prob{P}_{\RV{Z}(\omega)}^{\RV{X}_j|\RV{Z}\RV{D}}([0,q))$ we have

\begin{align}
    \prob{P}_{\RV{Z}(\omega)}^{\RV{X}_j|\RV{Z}}([0,q)) &= \mathbb{E}[\mathds{1}_{[0,q)}\circ \RV{X}_j|\sigma(\RV{Z})](\omega)\\
\end{align}

$\prob{P}$-almost surely.

Furthermore, the measure $\prob{P}_{h}^{\RV{X}_j|\RV{Z}}$ is uniquely defined by its value on $[0,q)$ for all $q\in \mathbb{Q}$. Thus for all $i,j\in \mathbb{N}$ we have

\begin{align}
    \prob{P}^{\RV{X}_j|\RV{H}} = \prob{P}^{\RV{X}_i|\RV{H}}
\end{align}

Completing the proof of property 1.

Next, we will show $\RV{X}_i\CI_\kernel{T} \RV{X}_{\mathbb{N}\setminus \{i\}} | \RV{H}$.

Let $\RV{Z}^{m,n}_q$ be a partial frequency as in Definition \ref{def:partial_freq} where $q$ stands for the set $[0,q)$. $\RV{Z}^{m,n}_q\circ \rho^{\RV{X}n}=\RV{Z}^{m,n}_q$ for all size $n$ swap functions so $\RV{Z}^{m,n}_q$  is $\sigalg{H}^n$ measurable.

Let $J\subset[n]$ be some set of $m$ elements from $n$ and $\rho'_{IJ}$ be a permutation that sends the elements of $I\subset[n]$ to $J$.

\begin{align}
    \int_{A} \RV{Z}_q^{m,n} (\omega) d\prob{P}(\omega) &= \int_{A} \frac{(n-m)!}{n!}\sum_{I\subset[n]}\prod_{i\in I} (\mathds{1}_{[0,q)}\circ \RV{X}_i)(\omega)d\prob{P}(\omega)\\
    &= \frac{(n-m)!}{n!}\sum_{I\subset[n]} \int_{(\rho^{\RV{X}}_{IJ})^{-1}\rho^{\RV{X}}_{IJ}(A)} \prod_{i\in I}(\mathds{1}_{[0,q)}\circ \RV{X}_i)(\omega)d\prob{P}\rho_{IJ}(\omega)\\
    &= \frac{(n-m)!}{n!}\sum_{I\subset[n]} \int_{\rho^{\RV{X}}_{IJ}(A)} \prod_{i\in I}(\mathds{1}_{[0,q)}\circ \RV{X}_{i}\circ \rho_{IJ}) (\omega)d\prob{P}(\omega)\label{eq:using_pushforward}\\
    &= \frac{(n-m)!}{n!}\sum_{I\subset[n]} \int_{A} \prod_{j\in J}(\mathds{1}_{[0,q)}\circ \RV{X}_{j})(\omega)d\prob{P}(\omega)\label{eq:closure_under_permutation2}\\
    &= \int_{A} \prod_{j\in J}(\mathds{1}_{[0,q)} \circ \RV{X}_j)(\omega)d\prob{P}(\omega) \label{eq:cond_expectation_first2}
\end{align}

From Equation \ref{eq:cond_expectation_first2}, we have for all $n$, $m<n$, $A\in \sigalg{H}^{n+1}$

\begin{align}
    \int_{A} \RV{Z}_q^{m,n+1} (\omega) d\prob{P}(\omega) &= \int_{A} \RV{Z}_q^{m,n} (\omega) d\prob{P}(\omega)
\end{align}

Because $\RV{Z}_q^{m,n+1}$ is $\sigalg{H}^{n+1}$ measurable, $\RV{Z}_q^{m,n+1} = \mathbb{E}[\RV{Z}_q^{m,n}|\sigalg{H}^{n+1}]$.

Thus the sequence $[\RV{Z}^{m,1}_q,\RV{Z}^{m,2}_q,...]$ is a backwards martingale with respect to the reversed filtration $\sigalg{H}^1\supset\sigalg{H}^2\supset...\supset \sigalg{H}^3$.

Furthermore, for all $n\in \mathbb{N}$, $\sup_\omega|\RV{Z}^{m,n}_q(\omega)|\leq 1$ so the sequence is also uniformly integrable. Thus it goes almost surely to a limit $\RV{Z}^{m,\infty}_q$, and for all $A\in \sigalg{H}$

\begin{align}
    \lim_{n\to\infty} \int_A \RV{Z}^{m,n}_q(\omega) d\prob{P}(\omega) &= \int_A \RV{Z}^{m,\infty}_q(\omega) d\prob{P}(\omega)\\
    \implies \RV{Z}^{m,n}_q &= \mathbb{E}[\prod_{j\in J} \mathds{1}_{[0,q)}\circ \RV{X}_j|\sigalg{H}]
\end{align}

Let $[n]^m_\mathrm{rep}$ be the set of all $m$ length sequences of elements of $[n]$ with repeats. Note that $\lim_{n\to\infty}\frac{(n-m)!}{n!}|[n]^m_\mathrm{rep}| = \lim_{n\to\infty}\frac{n^m(n-m)!}{n!}-1 = 0$

\begin{align}
    \mathbb{E}[\prod_{j\in J} \mathds{1}_{[0,q)}\circ \RV{X}_j|\sigalg{H}](\omega) &= \lim_{n\to\infty}\frac{(n-m)!}{n!}\sum_{I\subset[n]}\prod_{i\in I} (\mathds{1}_{[0,q)}\circ \RV{X}_i)\\
                                                                      &= \lim_{n\to\infty}\frac{1}{n^m}\sum_{I\subset[n]}\prod_{i\in I}(\mathds{1}_{[0,q)}\circ \RV{X}_i)\\
                                                                      &= \lim_{n\to\infty}\frac{1}{n^m}\left[\sum_{i_1\in [n]}...\sum_{i_m\in [n]} (1_{[0,q)}\circ \RV{X}_{i_k})-\sum_{J\subset[n]^m_\mathrm{rep}}\prod_{i\in I} (\mathds{1}_{[0,q)}\circ \RV{X}_i)\right]\\ 
                                                                      &= \prod_{k\in [m]} \lim_{n\to\infty} \frac{1}{n}\sum_{i_k\in[n]}(1_{[0,q)}\circ\RV{X}_{ik})\\
                                                                      &= \prod_{k\in [m]} \mathbb{E}[\mathds{1}_{[0,q)}\circ \RV{X}_j|\sigalg{H}]\\
                                                                      &= \prod_{k\in J} \mathbb{E}[\mathds{1}_{[0,q)}\circ \RV{X}_j|\sigalg{H}]\\
                                                                      &= \prod_{k\in J} \RV{Z}_q\label{eq:h_measurable}
\end{align}

Because $\mathbb{E}[\prod_{j\in J} \mathds{1}_{[0,q)}\circ \RV{X}_j|\sigalg{H}]$ is $\RV{Z}$-measurable we also have

\begin{align}
    \mathbb{E}[\prod_{j\in J} \mathds{1}_{[0,q)}\circ \RV{X}_j|\sigma(\RV{Z})] &= \prod_{k\in J} \RV{Z}_q
\end{align}

By the definition of conditional expectation, for all $J\subset \mathbb{N}$

\begin{align}
    \prob{P}_{\RV{H}(\omega)}^{\RV{X}_J|\RV{H}}(\times_{j\in J}[0,q)) &= \mathbb{E}[\prod_{j\in J} \mathds{1}_{[0,q)}\circ \RV{X}_j|\sigalg{H}](\omega)\\
                                                                      &= \prod_{k\in J} \prob{P}_{\RV{H}(\omega)}^{\RV{X}_j|\RV{H}}([0,q))
\end{align}

And thus $\RV{X}_i\CI\RV{X}_{\mathbb{N}\setminus\{i\}}|\RV{Z}$ for all $i\in\mathbb{N}$.
\end{proof}

\begin{lemma}[Independent and identically distributed random variables]\label{lem:iid_rvs}
Suppose we have a forecast $(\prob{P},\RV{X})$ where $\RV{X}=\utimes_{i\in A} \RV{X}_i$ for some $\{\RV{X}_i|i\in \mathbb{N}\}$ and $X$ is standard measurable, and $\prob{P}$ is exchangeable with respect to $\RV{X}$. Furthermore, suppose we have $\RV{V}=\utimes_{i\in\mathbb{N}} f\circ \RV{X}_i$ for some measurable $f:X_1\to V_1$ such that $\RV{V}$ is independent and identically distributes - that is, $\prob{P}^{\RV{V}} = \otimes_{i\in\mathbb{N}} \prob{P}^{\RV{V}_1}$. Then, letting $\RV{Z}$ be the hypothesis from Lemma \ref{lem:partial_representation}, $\RV{V}\CI\RV{Z}$.
\end{lemma}

\begin{proof}

We have by Lemma \ref{lem:partial_representation} that fo rany measurable $g:Z_1\to \mathbb{R}$,
\begin{align}
    \mathbb{E}[g(\RV{V}_1)|\sigma(\RV{Z})] = \lim_{n\to\infty} \frac{1}{n}\sum_i^n g(\RV{Z}_i)
\end{align}

However, by the strong law of large numbers (\citet{cinlar_probability_2011}, pg 119), because the $\RV{V}_i$ are independent and identically distributed

\begin{align}
    \mathbb{E}[g(\RV{V}_1)] &\overset{a.s.}{=} \lim_{n\to\infty}\frac{1}{n}\sum_i^n g(\RV{Z}_i)\\
                            &= \mathbb{E}[g(\RV{V}_1)|\sigma(\RV{Z})]
\end{align}

Thus $\RV{V}\CI\RV{Z}$.

\end{proof}

\begin{lemma}[Representation of infinitely exchangeably extendable see-do forecasts]\label{lem:rep_seedo_obs}
Given a see-do forecast $(\kernel{T},\RV{D},\RV{X},\RV{Y})$ where $\RV{X}=\utimes_{i\in \mathbb{N}} \RV{X}_i$ for some $\{\RV{X}_i|i\in \mathbb{N}\}$ and $X\times Y$ is standard measurable, if $(\kernel{T},\RV{D},\RV{X},\RV{Y})$ is infinitely exchangeable with respect to $\{\RV{X}_i|i\in \mathbb{N}\}$ then there exists a function $f:X\to Z$ such that, defining $\RV{Z}:f\circ \RV{X}$:
\begin{itemize}
    \item $\RV{X}_i\CI\RV{X}_{\mathbb{N}\setminus\{i\}}|\RV{Z}$ for all $i\in A$
    \item $\kernel{T}^{\RV{X}_i|\RV{Z}}=\kernel{T}^{\RV{X}_j|\RV{Z}}$ for all $i,j\in A$
    \item $\RV{Y}\CI\RV{X}|\RV{Z}\utimes \RV{D}$
\end{itemize}
\end{lemma}

\begin{proof}

Without loss of generality, assume $X_1=Y=[0,1]$.

$\kernel{T}$ is a see-do forecast, so $\RV{X}\CI_{\kernel{T}}\RV{D}$. Thus there exists a marginal $\kernel{T}^{\RV{X}}$ independent of $\RV{D}$.

From Lemma \ref{lem:partial_representation}, $\kernel{T}^{\RV{X}_i|\RV{Z}}=\kernel{T}^{\RV{X}_j|\RV{Z}}$ for all $i,j\in \mathbb{N}$ and $\RV{X}_i\CI\RV{X}_{\mathbb{N}\setminus\{i\}} |\RV{Z}$.

We will show $\RV{Y}\CI\RV{X}|\RV{Z}\utimes\RV{D}$. 

For any swap function $\rho^{\RV{X}}$ there is, by definition, a permutation of indices $\rho'$ such that $\RV{X}\circ\rho^{\RV{X}}(\omega) = [\RV{X}_{\rho'(1)}(\omega),\RV{X}_{\rho'(2)}(\omega),...]$. Define $\rho'':[0,1]^{\mathbb{N}}\to[0,1]^{\mathbb{N}}$ to be the bijective map that performs the permutation in the codomain of $\RV{X}$, i.e. $\RV{X}\circ\rho^{\RV{X}} = \rho''\circ\RV{X}$.

Consider some $B\in \sigalg{B}([0,1])^\mathbb{N}$ and its preimage $\RV{X}^{-1}(B) = \{\omega|\RV{X}(\omega)\in B\}$, and some finite swap function $\rho^{\RV{X}}$. Then there exists $\rho''(B)\in [0,1]^{\mathbb{N}}$ such that $(\RV{X}\circ\rho^{\RV{X}})^{-1}(\rho''(B)) = \{\omega|\RV{X}(\rho^{\RV{X}}(\omega))\in \rho''(B)\} = \{\omega|\RV{X}\in B\}$. Thus $\sigma(\RV{X})=\sigma(\RV{X}\circ\rho^{\RV{X}})$ for any finite swap function $\rho^{\RV{X}}$.

Because $E=X\times Y$ and $\RV{X}$ forgets the $Y$-component of any $(x,y)\in E$, for any $A,B\in\sigalg{B}([0,1])$ there is some $C\subset X$ such that $\mathds{1}_{A}\circ\RV{Y}(\RV{X}^{-1}(B)) = \mathds{1}_{A}\circ\RV{Y} (C\times Y) = [0,1]$. Thus for any finite swap function $\rho^{\RV{X}}$, $\mathds{1}_{A}\circ\RV{Y}\circ \rho^{\RV{X}} = \mathds{1}_{A}\circ\RV{Y}$.

For $A\in \sigma(\RV{X})$:

\begin{align}
    \int_{A} \mathbb{E}[\mathds{1}_A\circ\RV{Y}|\sigma(\RV{X})] \circ\rho^{\RV{X}} d\kernel{T}_d &= \int_{\rho^{\RV{X}}(A)} \mathbb{E}[\mathds{1}_A\circ\RV{Y}|\sigma(\RV{X})] d(\kernel{T}R^{-1})_d\\
                                                                         &= \int_{\rho^{\RV{X}}(A)} \mathbb{E}[\mathds{1}_A\circ\RV{Y}|\sigma(\RV{X})] d\kernel{T}_d \\
                                                                         &= \int_{\rho^{\RV{X}}(A)} \mathds{1}_A\circ\RV{Y}d\kernel{T}_d\\
                                                                         &= \int_A \mathds{1}_A\circ\RV{Y} d\kernel{T}_d
\end{align}

It follows that $\mathbb{E}[\mathds{1}_A\circ\RV{Y}|\sigma(\RV{X})] \circ\rho^{\RV{X}}$ is a version of $\mathbb{E}[\mathds{1}_A\circ\RV{Y}|\sigma(\RV{X})]$. Because there are only a countable number of finite swap functions, it also follows that a version of $\mathbb{E}[\mathds{1}_A\circ\RV{Y}|\sigma(\RV{X})]$ that is $\sigalg{H}$-measurable exists (i.e. for which $\mathbb{E}[\mathds{1}_A\circ\RV{Y}|\sigma(\RV{X})]\circ\rho^{\RV{X}} = \mathbb{E}[\mathds{1}_A\circ\RV{Y}|\sigma(\RV{X})]$, for all $\rho^{\RV{X}}$).

Consider also for some swap function $\rho^{\RV{X}}$, $B\in \sigalg{B}([0,1])$:

\begin{align}
    \int_{\RV{X}_i^{-1}(B)} \mathbb{E}[\RV{V}|\sigma(\RV{X}_j)] \circ \rho^{\RV{X}}_{ji} d\kernel{T}_d &= \int_{\rho^{\RV{X}}_{ji}(\RV{X}_i^{-1}(B))} \mathbb{E}[\RV{V}|\sigma(\RV{X}_j)] d(\kernel{T}R_{ij})_d\\
                                                                               &= \int_{\RV{X}^{-1}_{j}(B)} \mathbb{E}[\RV{V}|\sigma(\RV{X}_j)] d\kernel{T}_d\\
                                                                               &= \int_{\RV{X}_j^{-1}(B)} \RV{V} d\kernel{T}_d\\
                                                                               &= \int_{\rho_{ji}^*(\RV{X}_j^{-1}(B))}\RV{V}\circ \rho^{\RV{X}}_{ji} d(\kernel{T}R_{ji})_d\\
                                                                               &= \int_{\RV{X}_i^{-1}(B)} \RV{V}d\kernel{T}_d
\end{align}

Thus $\mathbb{E}[\RV{V}|\sigma(\RV{X}_j)]\circ\rho^{\RV{X}}_{ji}$ is a version of $\mathbb{E}[\RV{V}|\sigma({\RV{X}_i})]$.

Define $\RV{W}^n_q := \frac{1}{n}\sum_{i\in [n]} \mathbb{E}[\mathds{1}_{[0,q]}\RV{Y}|\sigma(\RV{X}_i)]$. Note that for any size $n$ swap function $\rho^{\RV{X}}$, $\RV{W}^n_q\circ\rho^{\RV{X}}=\RV{W}_q^n$, therefore $\RV{W}_q^n$ is $\sigalg{H}^n$-measurable. 

Consider $\omega,\omega'$ such that $\RV{W}_q^n(\omega)\neq \RV{W}_q^n(\omega')$. Then there exists no size $n$ swap function $\rho^{\RV{X}}$ such that $\RV{X}_{[n]}(\omega)=\RV{X}_{[n]}(\rho^{\RV{X}}(\omega'))$. Without loss of generality, suppose $\RV{X}_1(\omega)\leq\RV{X}_2(\omega)\leq...\leq\RV{X}_n(\omega)$ and $\RV{X}_1(\omega')\leq\RV{X}_2(\omega')\leq...\leq\RV{X}_n(\omega')$. Then there is some first index $j$ for which $\RV{X}_{j}(\omega)> \RV{X}_{j}(\omega')$, and some and some rational $r$ such that $\RV{X}_j(\omega)>r>\RV{X}_j(\omega')$. Then $\sum_{i}^n \mathds{1}_{[0,q)}(\RV{X}_i(\omega)) > \sum_i^n \mathds{1}_{[0,q)}(\RV{X}_i(\omega'))$ and so $\RV{Z}^n(\omega)\neq\RV{Z}^n(\omega')$ also.

Thus $\RV{W}_q^n$ is $\sigma(\RV{Z}^n)$ measurable. Define $\sigalg{I}^n:=\vee_{n\to\infty} \sigma(\RV{Z}^n)$. Then $\sigalg{I}^1\supset\sigalg{I}^2\supset...\supset\sigma(\RV{Z})=\cap_{i}^\infty \sigalg{I}^i$. In addition, but the $\sigalg{H}^n$-measurability of $\RV{Z}^{>n}$, $\sigalg{I}^n\subset\sigalg{H}^n$ and $\sigalg{I}\subset\sigalg{H}$.

For $A\in\sigalg{I}^n$, $d\in D$:

\begin{align}
    \int_A \RV{W}_q^n d\kernel{T}_d &= \frac{1}{n}\sum_{i\in[n]} \int_A \mathbb{E}[\mathds{1}_{[0,q)}\circ\RV{Y}|\sigma(\RV{X}_i)] d\kernel{T}_d\\
                                  &= \frac{1}{n}\sum_{i\in[n]} \int_{\rho_{ij}^*(A)} \mathbb{E}[\mathds{1}_{[0,q)}\circ\RV{Y}|\sigma(\RV{X}_i)]\circ\rho_{ij}^*d(\kernel{T}R_{ij})_d\\
                                  &= \int_{A} \mathbb{E}[\mathds{1}_{[0,q)}\circ \RV{Y} \sigma(\RV{X}_j)]d\kernel{T}_d\\
                                  &= \int_{A} \mathds{1}_{[0,q)}\circ \RV{Y} d\kernel{T}_d
\end{align}

Where the last line follows from the fact that $\sigalg{I}^n\subset \sigma(\RV{X}_j)$. Therefore $\RV{W}^n_q$ is a version of $\mathbb{E}[\mathds{1}_{[0,q)}\circ|\sigalg{I}^n]$. Furthermore $\RV{W}^1_q,...,\RV{W}^n_q,..$ is a backwards martingale with respect to $\sigalg{I}^1,....,sigalg{I}^n,..\sigalg{I}$ and so it goes to a limit $\RV{W}_q=\mathbb{E}[\mathds{1}_{[0,q)}\circ \RV{V} |\sigma(\RV{Z})]$.

So $\RV{W}_q$ is a $\sigma(\RV{Z})$-measurable version of $\mathbb{E}[\mathds{1}_{[0,q]}\circ\RV{Y}|\sigalg{H}]$ which is itself a version of $\mathbb{E}[\mathds{1}_{[0,q]}\circ\RV{Y}|\sigma(\RV{X})]$. As before, for any $d\in D$, $\omega\in E$ we have

\begin{align}
    \kernel{T}_{\RV{D}(\omega),\RV{X}(\omega),\RV{Z}(\omega)}^{\RV{Y}|\RV{X}\RV{Z}\RV{D}}([0,q)) &= \mathbb{E}[\mathds{1}_{[0,q]}\circ\RV{Y}|\sigalg{H}](\omega)\\
                                                                                                 &= \kernel{T}_{\RV{D}(\omega),\RV{Z}(\omega)}^{\RV{Y}|\RV{Z}\RV{D}}([0,q))
\end{align}

This shows that $\RV{Y}\CI_{\kernel{T}} \RV{X}|\RV{D}\RV{Z}$, as desired.

\end{proof}

Using Lemma \ref{lem:rep_seedo_obs} it is possible to prove a version of De Finetti's representation theorem, which is a well known result \citep{de_finetti_foresight_1992,hewitt_symmetric_1955}. We are interested in establishing an analogous result for consequence maps. This requires the assumption of \emph{functional exchangeability}.

Functional exchangeability is a generalisation of exchangeablility to Markov kernels. It captures the intuition that if we swap the order of the outputs (say, $\RV{Y}_1,\RV{Y}_2$ is swapped to $\RV{Y}_2, \RV{Y}_1$), we need to make analagous exchange of choices ($\RV{D}_1$, $\RV{D}_2$ becomes $\RV{D}_2$, $\RV{D}_1$) in order to maintain the correspondence of choices and outputs.

\begin{definition}[Functional Exchangeability]
\todo[inline]{Maybe include a diagram here? I think the pictorial representation helps with intuition, though it's hard to state as rigorously with pictures}
Given a see-do forecast $(\kernel{T},\RV{D},\RV{X},\RV{Y})$ where $\RV{D}=\utimes_{i\in A}\RV{D}_i$ and $\RV{Y}=\utimes_{i\in A} \RV{X}_i$ for some random variables $\{\RV{D}_i\}_A, \{\RV{Y}_i\}_A$, for any perumtation $\rho:A\to A$ define the observation and choice swap function $\rho^{\RV{D}\RV{Y}}$ and the choice swap function $\rho^{\RV{D}}$ as in \ref{def:permut_swap}. Then $\kernel{T}$ is functionally exchangeable with respect to $\RV{D}$ and $\RV{X}$ if $\kernel{T}=R^{\RV{D}}\kernel{T}R^{\RV{D}\RV{Y}}$.

$\kernel{T}$ is infinitely functionally exchangeably extendable if here exists a do forecast $(\kernel{T}',\RV{D}',\RV{X}')$ non-interfering and functionally exchangeable with respect to $\RV{D}'=\utimes_{i\in\mathbb{N}}\RV{D}'_i$ and $\RV{Y}'=\utimes_{i\in\mathbb{N}}\RV{Y}'_i$ such that for all $B\subset A$

\begin{align}
    \kernel{T}^{\prime \RV{Y}'_B|\RV{D}'_B}=\kernel{T}^{\RV{Y}_B|\RV{D}_B} 
\end{align}

A see-do forecast that is infinitely exchangeably extendable with respect to $\RV{X}$ and infinitely functionally exchangeably extendable with respect to $\RV{D}$, $\RV{Y}$ is \emph{doubly exchangeable} (we omit ``infinitely extendable'' for the sake of brevity).

\end{definition}

The following lemma interprets an exchangeable probability measure as an exchangeable see-do forecast with observations and consequences interchanged. This is so we can re-use Lemma \ref{lem:rep_seedo_obs} without separately proving an almost identical result for probability measures.

\begin{lemma}[Functionally exchangeable see-do models with exchangeable choices induce exchangeable see do models]\label{lem:f-ex2ex}
Given a see-do forecast $(\kernel{T},\RV{D},\RV{X},\RV{Y})$ functionally exchangeable with respect to $\RV{Y}=\utimes_{i\in A} \RV{Y}_i$ and $\RV{D}=\utimes_{i\in A}\RV{D}_i$ and some $\prob{P}^{\RV{D}}$ exchangeable with respect to $\RV{D}$, then $\prob{P}\kernel{T}\in \Delta(\sigalg{D}\otimes\sigalg{X}\otimes\sigalg{Y})$ is exchangeable with respect to $\RV{G}:=\utimes_i\in A \RV{Y}_i\utimes\RV{D}_i$. 

Furthermore, defining the trivial choice $\RV{C}:\{*\}\times X\times Y\to *$, $(\prob{P}\kernel{T},\RV{C},\RV{G},\RV{X})$ is a see-do forecast exchangeable with respect to $\RV{G}$.
\end{lemma}

\begin{proof}

For arbitrary $\rho:A\to A$, associated swap kernel $R^{\RV{D}}$ and $R^{\RV{X}}$:

\begin{align}
    \prob{P}\kernel{T} R^{\RV{D}\RV{X}} &= (\prob{P} R^{\RV{D}}) \kernel{T} R^{\RV{D}\RV{X}}\\
                                  &= \prob{P}(R^{\RV{D}}\kernel{T}R^{\RV{D}\RV{X}})\\
                                  &= \prob{P}\kernel{T}
\end{align}

This is sufficient for exchangeability of $(\prob{P}\kernel{T},\RV{C},\RV{G},\RV{O})$ with respect to $\RV{G}$.
\end{proof}

\begin{theorem}[Representation of doubly exchangeable see-do forecasts]\label{th:rep_dex_sdf}
Given a see-do forecast $(\kernel{T},\RV{D},\RV{X},\RV{Y})$ with denumerable $D$ and standard measurable $X$, $Y$, the following statements are equivalent (given finite $A\subset\mathbb{N}$, $B\subset\mathbb{N}$ not necessarily finite):

\begin{enumerate}
    \item $(\kernel{T},\RV{D},\RV{X},\RV{Y})$ is doubly exchangeable with respect to $\RV{D}=\utimes_{i\in A}\RV{D}_i$, $\RV{X}=\utimes_{i\in B} \RV{X}_i$, $\RV{Y}=\utimes_{i\in A} \RV{Y}_i$
    \item There exists a set $H$, $\kernel{T}^{\RV{H}}\in \Delta(\sigalg{H})$ and Markov kernels $\kernel{T}^{\RV{X}_1|\RV{H}}$ and $\kernel{T}^{\RV{Y}_1|\RV{H}\RV{D}_1}$ such that
    \begin{align}
        \kernel{T}^{\RV{X}\RV{Y}|\RV{D}} = \begin{tikzpicture}
            \path (0,0) node[dist,inner sep=-2pt] (H) {$\kernel{T}^{\RV{H}}$}
            + (0,-1.4) node (D) {$\RV{D}_1$}
            + (0,-1.9) node (D2) {$\RV{D}_2$}
            + (0,-2.4) node (D3) {$...$}
            + (0,-2.7) node (D4) {$\RV{D}_{|A|}$}
            ++ (0.5,0) node[copymap] (copy0) {}
            ++ (0.6,0) node[copymap,label={$B$}] (copy1) {}
            ++ (1.2,0.5) node[kernel] (XH) {$\kernel{T}^{\RV{X}_1|\RV{H}}$}
            + (0,-0.5) node[kernel] (XH2) {$\kernel{T}^{\RV{X}_1|\RV{H}}$}
            + (0,-1.75) node[kernel] (YD) {$\kernel{T}^{\RV{Y}_1|\RV{H}\RV{D}_1}$}
            + (-.9,-1.6) node[copymap,label={$A$}] (copy2) {}
            + (0,-2.25) node[kernel] (YD2) {$\kernel{T}^{\RV{Y}_1|\RV{H}\RV{D}_1}$}
            + (0,-2.95) node[kernel] (YD3) {$\kernel{T}^{\RV{Y}_1|\RV{H}\RV{D}_1}$}
            ++ (1.2,-0.25) node (X1) {$\RV{X}$}
            + (0,-1.5) node (Y) {$\RV{Y}$};
            \draw (H) -- (copy1) (copy1) to [out=35,in=180] (XH) (copy1) to [out=-35,in=180] (XH2);
            \draw ($(XH.east)$) to [out=-15,in=180] ($(X1.west) + (-0.2,0.25)$) ($(XH2.east)$) to [out=15,in=180] ($(X1.west)+(-0.2,-0.25)$);
            \draw ($(X1.west) + (-0.2,0)$) to (X1);
            \draw (copy0) to [out=-90,in=180] (copy2) -- ($(YD.west) + (0,0.15)$);
            \draw (D) to [out=0,in=180] ($(YD.west)+(0,-0.15)$);
            \draw (D2) to [out=0,in=180] ($(YD2.west)+(0,-0.15)$);
            \draw (D4) to [out=0,in=180] ($(YD3.west)+(0,-0.15)$);
            \draw (copy2) to [out=-45,in=180] ($(YD2.west) + (0,0.15)$);
            \draw (copy2) to [out=-65,in=180] ($(YD3.west) + (0,0.15)$);
            \draw ($(copy1.west)+(-0.1,0.8)$) rectangle ($(X1.west) + (-0.2,-0.55)$);
            \draw ($(copy2.west)+(-0.1,0.5)$) rectangle ($(Y.west) + (-0.2,-1.75)$);
            \draw ($(Y.west) + (-0.2,0)$) -- (Y);
        \end{tikzpicture}
    \end{align}
    \item There exists a set $H$, $\mu\in \Delta(\sigalg{H})$ and Markov kernels $\kernel{T}^{\RV{H}}$, $\kernel{T}^{\RV{X}_1|\RV{H}}$ and $\kernel{T}^{\RV{Y}_1|\RV{H}\RV{D}_1}$ such that for all $\mathbf{d}_A\in D$, $\{J_i\in \sigalg{X}_1|i\in B\}$, $\{K_i\in \sigalg{Y}_1|i\in A\}$:
    \begin{align}
        \kernel{T}^{\RV{X}\RV{Y}|\RV{D}}_{\mathbf{d}_A}((\times_{i\in B}J_i)\times (\times_{j\in A} K_j)) = \int_{H} \prod_{i\in B} \kernel{T}^{\RV{X}_1|\RV{H}}_h(J_i)\prod_{i\in A}\kernel{T}_{h,d_i}^{\RV{Y}_1|\RV{H}\RV{D}_1}(K_i)d\kernel{T}^{\RV{H}}(h)
    \end{align}

\end{enumerate}
\end{theorem}

\begin{proof}
$(2)$ and $(3)$ are string and integral notation for the same statement. 

$(1)\implies (2)$:

Define $\prob{P}\in \Delta(\sigalg{D_1}^{\mathbb{N}})$ such that $\prob{P}=\otimes_{i\in\mathbb{N}} \prob{P}^{\RV{D}_1}$ for some strictly positive $\prob{P}^{\RV{D}_1}$ (recall that $D$ and hence $D_1$ is denumerable). $\prob{P}$ is exchangeable and independent and identically distributed. Consider some infinite doubly exchangeable extension $\kernel{T}'$ of $\kernel{T}$. Then $\prob{P}\kernel{T}'$ is exchangeable with respect to $\RV{DY}':=\utimes_{i\in \mathbb{N}}\RV{D}'_i\utimes\RV{Y}'_i$ (Lemma \ref{lem:f-ex2ex}) and exchangeable with respect to $\RV{X}'=\utimes_{i\in\mathbb{N}} \RV{X}'_i$ as $\RV{X}$ is independent of $\RV{D}$.


By Lemma \ref{lem:rep_seedo_obs} we have $\RV{Z}':= f\circ\RV{X}'$ for some $f$ such that 
\begin{enumerate}
    \item $\RV{X}'_i\CI_{\prob{P}\kernel{T}'}\RV{X}'_{\mathbb{N}\setminus\{i\}}|\RV{Z}'$ for all $i\in A$
    \item $(\prob{P}\kernel{T}')^{\RV{X}'_i|\RV{Z}''}=(\prob{P}\kernel{T}'')^{\RV{X}'_j|\RV{Z}''}$ for all $i,j\in A$
    \item $\RV{D}\utimes\RV{Y}'\CI_{\prob{P}\kernel{T}'}\RV{X}'|\RV{Z}'$
\end{enumerate}

Applying Lemma  \ref{lem:rep_seedo_obs} to $\RV{DY}'$, and noting that $\RV{D}'\utimes\RV{Y}'$ is an invertible function of $\RV{DY}'$, we have $\RV{W}'=g\circ\RV{DY}'$ for some $g$ such that

\begin{enumerate}
    \setcounter{enumi}{3}
    \item for all $i\in \mathbb{N}$, $\RV{D}_i\utimes\RV{Y}'_i\CI_{\prob{P}\kernel{T}'}\RV{D}'_{\mathbb{N}\setminus\{i\}}\RV{Y}'_{\mathbb{N}\setminus\{i\}}|\RV{W}'$
    \item $(\prob{P}\kernel{T}')^{\RV{Y}'_i|\RV{D}'_i}=(\prob{P}\kernel{T}')^{\RV{Y}'_j|\RV{D}'_j}$ for all $i,j\in \mathbb{N}$ 
    \item $\RV{X}'\CI_{\prob{P}\kernel{T}'}\RV{D}'\utimes\RV{Y}'|\RV{W}'$
\end{enumerate}

Because $\RV{W}'\utimes\RV{D}'$ is a function of $\RV{DY}'$, we also have $\RV{X}'\CI_{\prob{P}\kernel{T}'} \RV{W}'\utimes\RV{D}'|\RV{Z}'$ by property 6.

$\RV{D}'$ is also a function of $\RV{DY}'$ and $\RV{Z}'$ is a function of $\RV{X}'$ so $\RV{D}'\CI_{\prob{P}\kernel{T}'}\RV{Z}'|\RV{W}'$, also by property 6. Because $\prob{P}$ is independent and identically distributed, $\RV{D}'\CI_{\prob{P}\kernel{T}'} \RV{W}'$, so $\RV{D}'\CI_{\prob{P}\kernel{T}'}\RV{Z}'\utimes \RV{W}'$

Becuase $\RV{Z}'$ is a function of $\RV{X}'$, $\RV{Y}'\CI_{\prob{P}\kernel{T}'}\RV{Z}'|\RV{D}'\utimes\RV{X}'\utimes\RV{W}'$. 

Applying weak union and symmetry to property 6 we have $\RV{Y}'\CI \RV{X}'|\RV{W}'\utimes\RV{D}'$, which combined with the above gives $\RV{Y}'\CI \RV{X}'\utimes \RV{Z}' |\RV{W}'\utimes\RV{D}'$ by contraction.

In summary, we will use the following conditional independences:
\begin{itemize}
    \item $\RV{X}'\CI_{\prob{P}\kernel{T}'} \RV{W}'\utimes \RV{D}'|\RV{Z}'$
    \item $\RV{D}'\CI_{\prob{P}\kernel{T}'}\RV{Z}'\utimes \RV{W}'$
    \item $\RV{Y}'\CI_{\prob{P}\kernel{T}'}\RV{Z}'|\RV{D}'\utimes\RV{X}'\utimes\RV{W}'$
    \item $\RV{Y}'\CI \RV{X}'\utimes \RV{Z}'|\RV{W}'\utimes\RV{D}'$
\end{itemize}

Let $\prob{U}:=\prob{P}\kernel{T}'$. By Lemma \ref{lem:representation_of_kernels}, and 
\begin{align}
    (\prob{P}\kernel{T}')^{\RV{X}'\RV{Y}'\RV{D}'} &= \begin{tikzpicture}
        \path (0,0) node[dist,inner sep=-2pt] (Z) {$\prob{U}^{\RV{Z}'}$}
        ++ (0.5,0) node[copymap] (copy0) {}
        ++ (0.7,0) node[kernel] (W) {$\prob{U}^{\RV{W}'|\RV{Z}'}$}
        ++ (0.9,0) node[copymap] (copy1) {}
        ++ (1.0,0.) node[kernel] (D) {$\prob{U}^{\RV{D'}|\RV{W'Z'}}$}
        ++ (0.9,0) node[copymap] (copy2) {}
        ++ (1.3,0.5)  node[kernel] (X) {$\prob{U}^{\RV{X}'|\RV{Z'W'D'}}$}
        ++ (0.9,0) node[copymap] (copy3) {}
        ++ (1.3,-0.5) node[kernel] (Y) {$\prob{U}^{\RV{Y'}|\RV{X'W'Z'D'}}$}
        ++ (1.5,0) node (Yrv) {$\RV{Y}'$}
        +  (0,0.5) node (Xrv) {$\RV{X}'$}
        +  (0,-0.5) node (Drv) {$\RV{D}'$};
        \draw (Z) -- (W);
        \draw (copy0) to [out=-65,in=180] ($(D.west) + (0,-0.15)$);
        \draw (copy0) to [out=65,in=180] ($(X.west) + (0,-0.15)$);
        \draw (copy1) to [out=65,in=180] ($(X.west) + (0,0)$);
        \draw (copy1) to [out=25,in=180] ($(D.west) + (0,0.15)$);
        \draw (copy2) to [out=25,in=180] ($(X.west) + (0,0.15)$);
        \draw (copy0) to [out=-65,in=180] ($(Y.west) + (0,-0.18)$);
        \draw (copy1) to [out=-65,in=180] ($(Y.west) + (0,-0.09)$);
        \draw (copy2) to [out=-65,in=180] ($(Y.west) + (0,0.09)$);
        \draw (copy3) to [out=-65,in=180] ($(Y.west) + (0,0.18)$);
        \draw (Z) -- (copy0) (W) -- (copy1) (D) -- (copy2);
        \draw (X) -- (Xrv) (Y) -- (Yrv);
        \draw (copy2) to [out=-90,in=180] (Drv);
    \end{tikzpicture}\\
     &= \begin{tikzpicture}
        \path (0,0) node[dist,inner sep=-2pt] (Z) {$\prob{U}^{\RV{Z}'}$}
        ++ (0.5,0) node[copymap] (copy0) {}
        ++ (0.7,0) node[kernel] (W) {$\prob{U}^{\RV{W}'|\RV{Z}'}$}
        ++ (0.9,0) node[copymap] (copy1) {}
        ++ (1.0,0.) node[dist,inner sep=-2pt] (D) {$\prob{U}^{\RV{D'}}$}
        ++ (0.9,0) node[copymap] (copy2) {}
        ++ (1.3,0.5)  node[kernel] (X) {$\prob{U}^{\RV{X}'|\RV{Z'}}$}
        ++ (0.9,0) node[copymap] (copy3) {}
        ++ (1.3,-0.5) node[kernel] (Y) {$\prob{U}^{\RV{Y'}|\RV{W'D'}}$}
        ++ (1.5,0) node (Yrv) {$\RV{Y}'$}
        +  (0,0.5) node (Xrv) {$\RV{X}'$}
        +  (0,-0.5) node (Drv) {$\RV{D}'$};
        \draw (Z) -- (W);
        \draw[-{Rays[n=8]}] (copy0) to [out=-65,in=180] ($(D.west) + (-0.1,-0.15)$);
        \draw (copy0) to [out=65,in=180] ($(X.west) + (0,-0.15)$);
        \draw[-{Rays[n=8]}] (copy1) to [out=65,in=180] ($(X.west) + (-0.4,0)$);
        \draw[-{Rays[n=8]}] (copy1) to [out=25,in=180] ($(D.west) + (-0.1,0.15)$);
        \draw[-{Rays[n=8]}] (copy2) to [out=25,in=180] ($(X.west) + (-0.4,0.15)$);
        \draw[-{Rays[n=8]}] (copy0) to [out=-65,in=180] ($(Y.west) + (-0.4,-0.18)$);
        \draw (copy1) to [out=-65,in=180] ($(Y.west) + (0,-0.09)$);
        \draw (copy2) to [out=-65,in=180] ($(Y.west) + (0,0.09)$);
        \draw[-{Rays[n=8]}] (copy3) to [out=-65,in=180] ($(Y.west) + (-0.4,0.18)$);
        \draw (Z) -- (copy0) (W) -- (copy1) (D) -- (copy2);
        \draw (X) -- (Xrv) (Y) -- (Yrv);
        \draw (copy2) to [out=-90,in=180] (Drv);
    \end{tikzpicture}\\
     &= \begin{tikzpicture}
        \path (0,0) node[dist,inner sep=-2pt] (Z) {$\prob{U}^{\RV{Z}'}$}
        + (0,-1) node[dist,inner sep=-2pt] (D) {$\prob{U}^{\RV{D'}}$}
        + (0.9,-1) node[copymap] (copy2) {}
        ++ (0.5,0) node[copymap] (copy0) {}
        ++ (0.7,-0.5) node[kernel] (W) {$\prob{U}^{\RV{W}'|\RV{Z}'}$}
        ++ (1.7,0.5)  node[kernel] (X) {$\prob{U}^{\RV{X}'|\RV{Z'}}$}
        +  (0,-1) node[kernel] (Y) {$\prob{U}^{\RV{Y'}|\RV{W'D'}}$}
        ++ (1.5,-1) node (Yrv) {$\RV{Y}'$}
        +  (0,1) node (Xrv) {$\RV{X}'$}
        +  (0,-0.5) node (Drv) {$\RV{D}'$};
        \draw (copy0) to [out=-60,in=180] ($(W.west) + (0,0)$);
        \draw (copy0) to [out=0,in=180] ($(X.west) + (0,0)$);
        \draw (W) to [out=0,in=180] ($(Y.west) + (0,0.15)$);
        \draw (copy2) to [out=0,in=180] ($(Y.west) + (0,-0.15)$);
        \draw (Z) -- (copy0) (D) -- (copy2);
        \draw (X) -- (Xrv) (Y) -- (Yrv);
        \draw (copy2) to [out=-60,in=180] (Drv);
    \end{tikzpicture}
\end{align}

By mutual indepenence of $\RV{Y}'_i\utimes\RV{D}'_i$'s', we have in particular $\RV{X}'_A\utimes \RV{Y}'_A\utimes \RV{D}'_A\CI_{\prob{U}} \RV{X}'_{A^C}\utimes \RV{Y}'_{A^C}\utimes \RV{D}'_{A^C}$. Therefore $\prob{U}^{\RV{X}'_A\RV{Y}'_A\RV{D}'_A}=(\prob{P}^{\RV{D}_A}\kernel{T})$. Furthermore, $\prob{P}^{\RV{D}_A}$ is positive by assumption, so by Lemma \ref{lem:agree_disint}:

\begin{align}
    \kernel{T} &= \prob{U}^{\RV{X}'_A\RV{Y}'_A|\RV{D}'_A}\\
               &= \begin{tikzpicture}
        \path (0,0) node[dist,inner sep=-2pt] (Z) {$\prob{U}^{\RV{Z}'}$}
        + (0,-1) node (D) {$\RV{D}$}
        + (0.9,-1) node[copymap] (copy2) {}
        ++ (0.5,0) node[copymap] (copy0) {}
        ++ (0.7,-0.5) node[kernel] (W) {$\prob{U}^{\RV{W}'|\RV{Z}'}$}
        ++ (1.7,0.5)  node[kernel] (X) {$\prob{U}^{\RV{X}'|\RV{Z'}}$}
        +  (0,-1) node[kernel] (Y) {$\prob{U}^{\RV{Y'}|\RV{W'D'}}$}
        ++ (1.5,-1) node (Yrv) {$\RV{Y}$}
        +  (0,1) node (Xrv) {$\RV{X}$};
        \draw (copy0) to [out=-60,in=180] ($(W.west) + (0,0)$);
        \draw (copy0) to [out=0,in=180] ($(X.west) + (0,0)$);
        \draw (W) to [out=0,in=180] ($(Y.west) + (0,0.15)$);
        \draw (copy2) to [out=0,in=180] ($(Y.west) + (0,-0.15)$);
        \draw (Z) -- (copy0) (D) -- (copy2);
        \draw (X) -- (Xrv) (Y) -- (Yrv);
    \end{tikzpicture}\\
    &\overset{def}{=} \begin{tikzpicture}
        \path (0,0) node[dist,inner sep=-2pt] (Z) {$\prob{T}^{\RV{H}}$}
        + (0,-1) node (D) {$\RV{D}$}
        ++ (0.5,0) node[copymap] (copy0) {}
        ++ (1.7,0)  node[kernel] (X) {$\kernel{T}^{\RV{X}|\RV{H}}$}
        +  (0,-1) node[kernel] (Y) {$\kernel{T}^{\RV{Y}|\RV{H D}}$}
        ++ (1.5,-1) node (Yrv) {$\RV{Y}$}
        +  (0,1) node (Xrv) {$\RV{X}$};
        \draw (copy0) to [out=0,in=180] ($(X.west) + (0,0)$);
        \draw (copy0) to [out=-25,in=180] ($(Y.west) + (0,0.15)$);
        \draw (D) to [out=0,in=180] ($(Y.west) + (0,-0.15)$);
        \draw (Z) -- (copy0);
        \draw (X) -- (Xrv) (Y) -- (Yrv);
    \end{tikzpicture}
\end{align}

We still need to show

\begin{align}
    \kernel{T}^{\RV{X}} &= \prob{U}^{\RV{X}'_A}\\
    &\overset{?}{=} \begin{tikzpicture}
            \path (0,0) node[dist,inner sep=-2pt] (H) {$\prob{U}^{\RV{H}'}$}
            ++ (0.5,0) node[copymap] (copy0) {}
            ++ (0.6,0) node[copymap,label={$A$}] (copy1) {}
            ++ (1.2,0.5) node[kernel] (XH) {$\prob{U}^{\RV{X}'_1|\RV{H}''}$}
            + (0,-0.5) node[kernel] (XH2) {$\prob{U}^{\RV{X}'_1|\RV{H}''}$}
            ++ (1.2,-0.25) node (X1) {$\RV{X}$};
            \draw (H) -- (copy1) (copy1) to [out=35,in=180] (XH) (copy1) to [out=-35,in=180] (XH2);
            \draw ($(XH.east)$) to [out=-15,in=180] ($(X1.west) + (-0.2,0.25)$) ($(XH2.east)$) to [out=15,in=180] ($(X1.west)+(-0.2,-0.25)$);
            \draw ($(X1.west) + (-0.2,0)$) to (X1);
            \draw ($(copy1.west)+(-0.1,0.8)$) rectangle ($(X1.west) + (-0.2,-0.55)$);
        \end{tikzpicture}
\end{align}

and

\begin{align}
    \kernel{T}^{\RV{Y}|\RV{D}} &= \prob{U}^{\RV{Y}'_A|\RV{D'}_A} \\
    &\overset{?}{=}
    \begin{tikzpicture}
            \path (0,0) node[dist,inner sep=-2pt] (H) {$\prob{U}^{\RV{H}'}$}
            + (0,-0.7) node (D) {$\RV{D}_1$}
            + (0,-1.2) node (D2) {$\RV{D}_2$}
            + (0,-1.6) node (D3) {$...$}
            + (0,-1.9) node (D4) {$\RV{D}_{|A|}$}
            ++ (0.5,0) coordinate (copy0)
            ++ (2,0.5) coordinate (placeholder)
            + (0,-0.95) node[kernel] (YD) {$\prob{U}^{\RV{Y'}_1|\RV{H'}\RV{D'}_1}$}
            + (-1.5,-0.8) node[copymap,label={$A$}] (copy2) {}
            + (0,-1.45) node[kernel] (YD2) {$\prob{U}^{\RV{Y'}_1|\RV{H'}\RV{D'}_1}$}
            + (0,-2.35) node[kernel] (YD3) {$\prob{U}^{\RV{Y'}_1|\RV{H'}\RV{D'}_1}$}
            + (1.9,-1.5) node (Y) {$\RV{Y'}$};
            \draw (H) to [out=0,in=180] (copy2) -- ($(YD.west) + (0,0.15)$);
            \draw (D) to [out=0,in=180] ($(YD.west)+(0,-0.15)$);
            \draw (D2) to [out=0,in=180] ($(YD2.west)+(0,-0.15)$);
            \draw (D4) to [out=0,in=180] ($(YD3.west)+(0,-0.15)$);
            \draw (copy2) to [out=-45,in=180] ($(YD2.west) + (0,0.15)$);
            \draw (copy2) to [out=-65,in=180] ($(YD3.west) + (0,0.15)$);
            \draw ($(copy2.west)+(-0.1,0.5)$) rectangle ($(Y) + (-0.5,-1.25)$);
            \draw ($(Y) + (-0.5,0)$) -- (Y);
            \draw (YD) -- ($(YD) + (1.4,0)$) (YD2) -- ($(YD2) + (1.4,0)$) (YD3) -- ($(YD3) + (1.4,0)$);
        \end{tikzpicture}\label{eq:fex_copyrep}
\end{align}

Where $\RV{H}'=\RV{Z}'\utimes\RV{W}'$.

The first follows straightforwardly from the mutual independence of the $\RV{X}'_i$'s. The following diagram uses ``$...$'' informally to indicate a missing section of diagram that continues ``as you'd expect''. Applying Lemma \ref{lem:representation_of_kernels}, we have

\begin{align}
    \begin{tikzpicture}
        \path (0,0) node[dist,inner sep=-2pt] (H) {$\prob{U}^{\RV{H}'}$}
        ++ (1,0) node[kernel] (Xzd) {$\prob{U}^{\RV{X}'_A|\RV{H}'}$}
        ++ (1.5,0.) node (X) {$\RV{X}'_A$};
        \draw (H) to [out=0,in=180] ($(Xzd.west) + (0,0)$);
        \draw (Xzd) -- (X);
    \end{tikzpicture} &=  \begin{tikzpicture}
        \path (0,0) node[dist,inner sep=-2pt] (H) {$\prob{U}^{\RV{H}'}$}
        ++ (0.9,0) node[copymap] (copy1) {}
        ++ (0.9,0) node[kernel] (Xzd) {$\prob{U}^{\RV{X}_1|\RV{Z}}$}
        + (0.7,0) node[copymap] (copy2) {}
        ++ (1.5,-0.5) node[kernel] (Xzd2) {$\prob{U}^{\RV{X}_2|\RV{Z}\RV{X}_1}$}
        +  (0.8,0) node[copymap] (copy3) {}
        ++ (1.5,0.5) node (X) {$\RV{X}_1$}
        + (0,-0.5) node (X2) {$\RV{X}_2$}
        + (-2.5,-1) node (X3) {$...$}
        + (0,-1.5) node (X4) {$\RV{X}_{|A|}$};
        \draw (H) -- (Xzd);
        \draw (copy1) to [out=-90,in=180] ($(Xzd2.west) + (0,0)$);
        \draw (copy2) to [out=-90,in=180] ($(Xzd2.west) + (0,0.15)$);
        \draw (Xzd) -- (X) (Xzd2) -- (X2);
        \draw (copy1) to [out=-90,in=115] ($(copy1.south) + (0.1,-0.8)$);
        \draw (copy2) to [out=-90,in=115] ($(copy2.south) + (0.1,-0.8)$);
        \draw (copy3) to [out=-90,in=115] ($(copy3.south) + (0.1,-0.4)$);
        \draw ($(X4.west) + (-0.2,0.2)$) to [out=-45,in=180] (X4.west);
    \end{tikzpicture}\\
    &= \begin{tikzpicture}
        \path (0,0) node[dist,inner sep=-2pt] (H) {$\prob{U}^{\RV{H}'}$}
        ++ (0.9,0) node[copymap] (copy1) {}
        ++ (0.9,0) node[kernel] (Xzd) {$\prob{U}^{\RV{X}_1|\RV{Z}}$}
        + (0.7,0) node[copymap] (copy2) {}
        ++ (1.5,-0.5) node[kernel] (Xzd2) {$\prob{U}^{\RV{X}_2|\RV{Z}}$}
        +  (0.8,0) node[copymap] (copy3) {}
        ++ (1.5,0.5) node (X) {$\RV{X}_1$}
        + (0,-0.5) node (X2) {$\RV{X}_2$}
        + (-2.5,-1) node (X3) {$...$}
        + (0,-1.5) node (X4) {$\RV{X}_{|A|}$};
        \draw (H) -- (Xzd);
        \draw (copy1) to [out=-90,in=180] ($(Xzd2.west) + (0,0)$);
        \draw[-{Rays[n=8]}] (copy2) to [out=-90,in=180] ($(Xzd2.west) + (0,0.15)$);
        \draw (copy1) to [out=-90,in=115] ($(copy1.south) + (0.1,-0.8)$);
        \draw[-{Rays[n=8]}] (copy2) to [out=-90,in=115] ($(copy2.south) + (0.1,-0.8)$);
        \draw[-{Rays[n=8]}] (copy3) to [out=-90,in=115] ($(copy3.south) + (0.1,-0.4)$);
        \draw (Xzd) -- (X) (Xzd2) -- (X2);
        \draw ($(X4.west) + (-0.2,0.2)$) to [out=-45,in=180] (X4.west);
    \end{tikzpicture}\\
      &= \begin{tikzpicture}
        \path (0,0) node[dist,inner sep=-2pt] (H) {$\prob{U}^{\RV{H}'}$}
        ++ (0.9,0) node[copymap] (copy1) {}
        ++ (0.9,0) node[kernel] (Xzd) {$\prob{U}^{\RV{X}_1|\RV{Z}}$}
        ++ (0,-0.5) node[kernel] (Xzd2) {$\prob{U}^{\RV{X}_1|\RV{Z}}$}
        ++ (1.5,0.5) node (X) {$\RV{X}_1$}
        + (0,-0.5) node (X2) {$\RV{X}_2$}
        + (-1.5,-1) node (X3) {$...$}
        + (0,-1.5) node (X4) {$\RV{X}_{|A|}$};
        \draw (H) -- (Xzd);
        \draw (copy1) to [out=-90,in=180] ($(Xzd2.west) + (0,0)$);
        \draw (copy1) to [out=-90,in=115] ($(copy1.south) + (0.1,-0.8)$);
        \draw (Xzd) -- (X) (Xzd2) -- (X2);
        \draw ($(X4.west) + (-0.2,0.2)$) to [out=-45,in=180] (X4.west);
    \end{tikzpicture}\\
    &=  \begin{tikzpicture}
            \path (0,0) node[dist,inner sep=-2pt] (H) {$\prob{U}^{\RV{H}}$}
            ++ (0.8,0) node[copymap,label={$A$}] (copy1) {}
            ++ (1.2,0.5) node[kernel] (XH) {$\prob{U}^{\RV{X}_1|\RV{Z}}$}
            + (0,-0.5) node[kernel] (XH2) {$\prob{U}^{\RV{X}_1|\RV{Z}}$}
            ++ (1.2,-0.25) node (X1) {$\RV{X}'_A$};
            \draw (H) -- (copy1) (copy1) to [out=35,in=180] (XH) (copy1) to [out=-35,in=180] (XH2);
            \draw ($(XH.east)$) to [out=-15,in=180] ($(X1.west) + (-0.2,0.25)$) ($(XH2.east)$) to [out=15,in=180] ($(X1.west)+(-0.2,-0.25)$);
            \draw ($(X1.west) + (-0.2,0)$) to (X1);
            \draw ($(copy1.west)+(-0.1,0.8)$) rectangle ($(X1.west) + (-0.2,-0.55)$);
        \end{tikzpicture}
\end{align}

As desired.

For $\prob{U}^{\RV{Y}'_A|\RV{D}'_A}$, define the ``interleaved'' random variable $\overset{~}{\RV{YD}'}_A=\utimes_{i\in A} \RV{Y}'_i\utimes\RV{D}'_i$. Note that this implies that $\prob{U}^{\RV{Y}'_A\RV{D}'_A}=\prob{U}^{\overset{~}{\RV{YD}'}_A}\rho$ where $\rho$ is a swap kernel that moves all the $\RV{D}'_i$s below the $\RV{Y}'_i$s. Then, applying what we just showed but substituting $\overset{~}{\RV{YD}'}_A$ for $\RV{X}'_A$,

\begin{align}
      \prob{U}^{\overset{~}{\RV{YD}'}_A} &= \begin{tikzpicture}
        \path (0,0) node[dist,inner sep=-2pt] (H) {$\prob{U}^{\RV{H}'}$}
        ++ (0.9,0) node[copymap] (copy1) {}
        ++ (0.9,0) node[kernel] (Xzd) {$\prob{U}^{\RV{Y}'_1\RV{D}'_1|\RV{Z}}$}
        ++ (0,-0.5) node[kernel] (Xzd2) {$\prob{U}^{\RV{Y}'_1\RV{D}'_1|\RV{Z}}$}
        ++ (2,0.5) node (X) {$\RV{Y'}_1\utimes\RV{D'}_1$}
        + (0,-0.5) node (X2) {$\RV{Y'}_2\utimes\RV{D'}_2$}
        + (-1,-1) node (X3) {$...$}
        + (0,-1.5) node (X4) {$\RV{Y'}_{|A|}\utimes\RV{D'}_{|A|}$};
        \draw (H) -- (Xzd);
        \draw (copy1) to [out=-90,in=180] ($(Xzd2.west) + (0,0)$);
        \draw (copy1) to [out=-90,in=115] ($(copy1.south) + (0.1,-0.8)$);
        \draw (Xzd) -- (X) (Xzd2) -- (X2);
        \draw ($(X4.west) + (-0.2,0.2)$) to [out=-45,in=180] (X4.west);
    \end{tikzpicture}\\
    &= \begin{tikzpicture}
        \path (0,0) node[dist,inner sep=-2pt] (H) {$\prob{U}^{\RV{H}'}$}
        + (0,-1.5) node[dist,inner sep=-3pt] (D) {$\prob{U}^{\RV{D}'_A}$}
        + (0.6,-1.35) node[copymap] (copy00) {}
        + (0.6,-1.5) node[copymap] (copy01) {}
        ++ (0.6,0) node[copymap] (copy1) {}
        ++ (1.2,0) node[kernel] (Xzd) {$\prob{U}^{\RV{Y}'_1|\RV{Z}}$}
        ++ (0,-1) node[kernel] (Xzd2) {$\prob{U}^{\RV{Y}'_1|\RV{Z}}$}
        ++ (1.5,1) node (X) {$\RV{Y'}_1$}
        + (0,-0.5) node (D1) {$\RV{D}'_1$}
        + (0,-1) node (X2) {$\RV{Y'}_2$}
        + (0,-1.5) node (D2) {$\RV{D'}_2$}
        + (-2,-2) node (X3) {$...$}
        + (0,-2.5) node (X4) {$\RV{Y'}_{|A|}$}
        + (0,-3) node (D4) {$\RV{D'}_{|A|}$};
        \draw (H) -- (Xzd);
        \draw (copy1) to [out=-65,in=180] ($(Xzd2.west) + (0,0)$);
        \draw (copy1) to [out=-90,in=115] ($(copy1.south) + (0.6,-1.8)$);
        \draw (Xzd) -- (X) (Xzd2) -- (X2);
        \draw ($(D.east) + (0,0.15)$) to [out=0,in=180] (copy00) to [out=65,in=180] ($(Xzd.west) + (0,-0.15)$);
        \draw ($(D.east) + (0,0.)$) to [out=0,in=180] (copy01) to [out=0,in=180] ($(Xzd2.west) + (0,-0.15)$);
        \draw (copy00) to [out=45,in=180] ($(copy00)+(0.7,0.85)$) -- (D1) (copy01) to [out=0,in=180] (D2);
        \draw (copy00) to [out=-45,in=115] ($(copy00.south) + (0.4,-0.5)$);
        \draw (copy01) to [out=-45,in=115] ($(copy01.south) + (0.2,-0.3)$);
        \draw ($(X4.west) + (-0.2,0.2)$) to [out=-45,in=180] (X4.west);
        \draw ($(D4.west) + (-0.2,0.2)$) to [out=-45,in=180] (D4.west);
    \end{tikzpicture}\\
    \implies \prob{U}^{\RV{Y}'_A\RV{D}'_A} &= \prob{U}^{\overset{~}{\RV{YD}'}_A}\rho\\
    &= \begin{tikzpicture}
        \path (0,0) node[dist,inner sep=-2pt] (H) {$\prob{U}^{\RV{H}'}$}
        + (0,-2.6) node[dist,inner sep=-3pt] (D) {$\prob{U}^{\RV{D}'_A}$}
        + (0.6,-2.45) node[copymap] (copy00) {}
        + (0.6,-2.7) node[copymap] (copy01) {}
        ++ (0.6,0) node[copymap] (copy1) {}
        ++ (1.2,0) node[kernel] (Xzd) {$\prob{U}^{\RV{Y}'_1|\RV{Z}}$}
        ++ (0,-0.5) node[kernel] (Xzd2) {$\prob{U}^{\RV{Y}'_1|\RV{Z}}$}
        ++ (2,0.5) node (X) {$\RV{Y'}_1$}
        + (0,-0.5) node (X2) {$\RV{Y'}_2$}
        + (-1.5,-1) node (X3) {$...$}
        + (0,-1.5) node (X4) {$\RV{Y'}_{|A|}$}
        + (0,-2.3) node (D1) {$\RV{D}'_1$}
        + (0,-2.8) node (D2) {$\RV{D'}_2$}
        + (-1,-3.3) node (D3) {$...$}
        + (0,-3.8) node (D4) {$\RV{D'}_{|A|}$};
        \draw (H) -- (Xzd);
        \draw (copy1) to [out=-65,in=180] ($(Xzd2.west) + (0,0)$);
        \draw (Xzd) -- (X) (Xzd2) to [out=0,in=180] (X2);
        \draw ($(D.east) + (0,0.15)$) to [out=0,in=180] (copy00) to [out=65,in=180] ($(Xzd.west) + (0,-0.15)$);
        \draw (copy1) to [out=-90,in=180] ($(copy1.south) + (0.6,-1.)$);
        \draw (copy00) to [out=35,in=180] ($(copy1.south) + (0.6,-1.1)$);
        \draw (copy01) to [out=35,in=180] ($(copy1.south) + (0.6,-1.2)$);
        \draw (copy00) to [out=-45,in=115] ($(copy00.south) + (0.3,-0.5)$);
        \draw (copy01) to [out=-45,in=105] ($(copy01.south) + (0.2,-0.35)$);
        \draw ($(D.east) + (0,0.)$) to [out=0,in=180] (copy01) to [out=0,in=180] ($(Xzd2.west) + (0,-0.15)$);
        \draw (copy00) to [out=0,in=180]  (D1) (copy01) to [out=-35,in=180] (D2);
        \draw ($(X4.west) + (-0.2,0.2)$) to [out=-45,in=180] (X4.west);
        \draw ($(D4.west) + (-0.2,0.2)$) to [out=-45,in=180] (D4.west);
        \draw[red] ($(H.west) + (-0.1,0.55)$) rectangle ($(X4.east) + (0.1,-0.4)$);
    \end{tikzpicture}
\end{align}

The red boxed kernel is the disintegration $\kernel{U}^{\RV{Y}'_A|\RV{D}'_A}$ and (notational difficulties aside) is equal to the kernel in Equation \ref{eq:fex_copyrep}.

This completes the proof that $(1)\implies (2)$.

$(2)\implies (1)$:

We will use integral notation. For all sets of events $\{J_i\in \sigalg{X}_1\}_B$, $\{K_i\in \sigalg{Y}_1\}_A$, $\mathbf{d}_A\in D$:

\begin{align}
  \kernel{T}^{\RV{X}\RV{Y}|\RV{D}}_{\mathbf{d}_A}((\times_{i\in B}J_i)\times (\times_{j\in A} K_j)) = \int_{H} \prod_{i\in B} \kernel{T}^{\RV{X}_1|\RV{H}}_h(J_i)\prod_{i\in A}\kernel{T}_{h,d_i}^{\RV{Y}_1|\RV{H}\RV{D}_1}(K_i)d\kernel{T}^{\RV{H}}(h)
\end{align}

For all $\{J_i\in \sigalg{X}_1\}_\mathbb{N}$, $\{K_i\in \sigalg{Y}_1\}_\mathbb{N}$, $\mathbf{d} \in D^{\mathbb{N}}$ define

\begin{align}
  \kernel{T}^{\prime \RV{X}'\RV{Y}'|\RV{D}'}_{\mathbf{d}}((\times_{i\in \mathbb{N}}J_i)\times (\times_{j\in \mathbb{N}} K_j)) = \int_{H} \prod_{i\in \mathbb{N}} \kernel{T}^{\RV{X}_1|\RV{H}}_h(J_i)\prod_{i\in \mathbb{N}}\kernel{T}_{h,d_i}^{\RV{Y}_1|\RV{H}\RV{D}_1}(K_i)d\kernel{T}^{\RV{H}}(h)
\end{align}

Marginalising over $\mathbb{N}\setminus A$ and $\mathbb{N}\setminus B$ is equivalent to choosing $K_i=Y_1$ for $i\not\in A$ and $J_i=X_1$ for $i\not\in B$. Then

\begin{align}
  \kernel{T}^{\prime \RV{X}'_B\RV{Y}'_A|\RV{D}'}_{\mathbf{d}}((\times_{i\in B}J_i\times X_1^{\mathbb{N}\setminus B})\times (\times_{j\in A} K_j\times Y_1^{\mathbb{N}\setminus A})) &= \int_{H} \prod_{i\in B} \kernel{T}^{\RV{X}_1|\RV{H}}_h(J_i) [\kernel{T}^{\RV{X}_1|\RV{H}}_h (X_1)]^{\mathbb{N}\setminus B} \prod_{i\in A}\kernel{T}_{h,d_i}^{\RV{Y}_1|\RV{H}\RV{D}_1}(K_i)[\kernel{T}^{\RV{Y}_1|\RV{H}}_h (Y_1)]^{\mathbb{N}\setminus A}d\kernel{T}^{\RV{H}}(h)\\
  &= \int_{H} \prod_{i\in B} \kernel{T}^{\RV{X}_1|\RV{H}}_h(J_i) \prod_{i\in A}\kernel{T}_{h,d_i}^{\RV{Y}_1|\RV{H}\RV{D}_1}(K_i)d\kernel{T}^{\RV{H}}(h)\\
  &= \kernel{T}^{\RV{X}\RV{Y}|\RV{D}}_{\mathbf{d}_A}((\times_{i\in B}J_i)\times (\times_{j\in A} K_j))
\end{align}

Thus $\kernel{T}'$ is an extension of $\kernel{T}$. Furthermore, for any swaps $\rho_r$, $\rho_s$ with associated kernels $R^{\RV{X}'}$, $S^{\RV{D}'}$,$S^{\RV{Y'}}$:

\begin{align}
  (S^{\RV{D'}\RV{Y'}}\kernel{T}'(R^{\RV{X'}}\otimes S^{\RV{Y}'}))^{\RV{X}'\RV{Y}'|\RV{D}'}_{\mathbf{d}}((\times_{i\in \mathbb{N}}J_i)\times (\times_{j\in \mathbb{N}} K_j)) &= \int_{H} \prod_{i\in \mathbb{N}} \kernel{T}^{\RV{X}_1|\RV{H}}_h(J_{\rho_r(i)})\prod_{i\in \mathbb{N}}\kernel{T}_{h,d_{\rho_s(i)}}^{\RV{Y}_1|\RV{H}\RV{D}_1}(K_{\rho_s(i)})d\kernel{T}^{\RV{H}}(h) \\
                            &= \int_{H} \prod_{i\in \rho_{r}(\mathbb{N})} \kernel{T}^{\RV{X}_1|\RV{H}}_h(J_{i})\prod_{i\in \rho_s(\mathbb{N})}\kernel{T}_{h,d_{i}}^{\RV{Y}_1|\RV{H}\RV{D}_1}(K_{i})d\kernel{T}^{\RV{H}}(h) \\
                            &= \kernel{T}^{\prime \RV{X}'\RV{Y}'|\RV{D}'}_{\mathbf{d}}((\times_{i\in \mathbb{N}}J_i)\times (\times_{j\in \mathbb{N}} K_j))
\end{align}

Therefore $\kernel{T}$ is infinitely doubly exchangeably extendable.
\end{proof}