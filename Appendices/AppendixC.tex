%!TEX root = ../main.tex

% Appendix C

\chapter{Proofs of key results in Chapter \ref{ch:other_causal_frameworks}} % Main appendix title

\label{AppendixC} % For referencing this appendix elsewhere, use \ref{AppendixA}

\section{Proofs related to causal Bayesian networks}\label{sec:cbn_proofs}

\begin{reptheorem}{th:causal_contractibility_cbn}
Given an uncertain unrolled causal Bayesian network $(\prob{P}_\cdot,C,\Omega,(\RV{V}_{ij})_{i\in A,j\in [n]},\RV{H},\graph{G})$, take $C'\subset C$ to be sequences of interventions that, for some $i\in A$, do not target a particular $\RV{V}_{ij}$ for any $j\in [n]$ and ensure every sequence $\RV{V}_{j[n]}$ has infinite support. Then $\RV{V}_{i[n]}\CI^e_{\prob{P}_{C'}}\text{Id}_C|(\RV{H},\mathrm{Pa}(\RV{V}_{i [n]}))$ and $\prob{P}_C^{\RV{V}_{i[n]}|\RV{H}\mathrm{Pa}(\RV{V}_{i [n]})}$ is IO contractible over $\RV{H}$. 
\end{reptheorem}

\begin{proof}
First we will prove $\RV{V}_{i[n]}\CI^e_{\prob{P}_{C'}}\text{Id}_C|(\RV{H},\mathrm{Pa}(\RV{V}_{i [n]}))$. This is equivalent to the claim that $\prob{P}_\alpha^{\RV{V}_{i[n]}|\RV{H}\mathrm{Pa}(\RV{V}_{i[n]})}$ is the same as $\prob{P}_{\alpha'}^{\RV{V}_{i[n]}|\RV{H}\mathrm{Pa}(\RV{V}_{i[n]})}$ for any $\alpha,\alpha'$. By assumption 5* of Definition \ref{def:unr_CBN}, for each $h\in H$
\begin{align}
    \RV{V}_{Aj}\CI^e_{\prob{P}_{\alpha,h}} \RV{V}_{A [n]\setminus\{j\}} |\text{Id}_C
\end{align}
which implies
\begin{align}
    &\RV{V}_{Aj}\CI^e_{\prob{P}_{\alpha}} \RV{V}_{A [n]\setminus\{j\}} |(\text{Id}_C,\RV{H})\\
    &\implies \RV{V}_{ij}\CI^e_{\prob{P}_{\alpha}} \RV{V}_{A [n]\setminus\{j\}} |(\RV{H},\mathrm{Pa}(\RV{V}_{i[n]}),\text{Id}_C)\label{eqApp:mutual_conditional_independence}
\end{align}
thus it is sufficient to show that, for any $\alpha,\alpha'\in C'$ and $j\in[n]$
\begin{align}
    \prob{P}_\alpha^{\RV{V}_{ij}|\RV{H}\mathrm{Pa}(\RV{V}_{ij})} &= \prob{P}_{\alpha'}^{\RV{V}_{ij}|\RV{H}\mathrm{Pa}(\RV{V}_{ij})}
\end{align}
By assumption, if $\pi_j(\alpha) =: (\mathrm{do}_{B_j},v_{B_j})$ and $\pi_j(\alpha') =: (\mathrm{do}_{B_j'},v_{B_j'}')$, $i\not\in B_j\cup B_j'$, and similarly replacing the $j$s with $k$s for any $k\in [n]$. Define $\alpha''$ such that, for some $k$, $\pi_k(\alpha'')=\pi_k(\alpha')$ and $\pi_j(\alpha'')=\pi_j(\alpha)$. Then by 4*, for all $h\in H$
\begin{align}
    \prob{P}_{\alpha}^{\RV{V}_{ij}|\RV{H}\mathrm{Pa}(\RV{V}_{ij})}(A|h,y) &= \prob{P}_{\alpha''}^{\RV{V}_{ij}|\RV{H}\mathrm{Pa}(\RV{V}_{ij})}(A|h,y)\\
    &= \prob{P}_{\alpha''}^{\RV{V}_{ik}|\RV{H}\mathrm{Pa}(\RV{V}_{ik})}(A|h,y)&\text{by 3*}\\
    &= \prob{P}_{\alpha'}^{\RV{V}_{ik}|\RV{H}\mathrm{Pa}(\RV{V}_{ik})}(A|h,y)&\text{by 4*}\\
    &= \prob{P}_{\alpha'}^{\RV{V}_{ij}|\RV{H}\mathrm{Pa}(\RV{V}_{ij})}(A|h,y)&\text{by 3*}\\
    \implies \prob{P}_\alpha^{\RV{V}_{ij}|\RV{H}\mathrm{Pa}(\RV{V}_{ij})} &= \prob{P}_{\alpha'}^{\RV{V}_{ij}|\RV{H}\mathrm{Pa}(\RV{V}_{ij})}
\end{align}

Next, IO contractibility of $\prob{P}_C^{\RV{V}_{i[n]}|\RV{H}\mathrm{Pa}(\RV{V}_{i [n]})}$ over $\RV{H}$. By Eq. \eqref{eqApp:mutual_conditional_independence}
\begin{align}
    \RV{V}_{ij}\CI^e_{\prob{P}_{\alpha}} (\RV{V}_{i [1,j)},\mathrm{Pa}(\RV{V}_{i [1,j)}) )|(\RV{H},\mathrm{Pa}(\RV{V}_{i[n]}),\text{Id}_C)
\end{align}
furthermore, by 3* and the assumption that no intervention $\alpha\in C'$ targets $\RV{V}_{ij}$ for any $j$, for any $\alpha\in C'$
\begin{align}
    \prob{P}_{\alpha}^{\RV{V}_{ij}|\RV{H}\mathrm{Pa}(\RV{V}_{ij})}(A|h,y) &= \prob{P}_{\alpha}^{\RV{V}_{ik}|\RV{H}\mathrm{Pa}(\RV{V}_{ik})}(A|h,y) 
\end{align}
thus $\prob{P}_C$ has independent and identical response functions conditional on $\RV{H}$, by assumption the inputs have infinite support and therefore by Theorem \ref{th:ciid_rep_kernel}, $\prob{P}_C^{\RV{V}_{i[n]}|\RV{H}\mathrm{Pa}(\RV{V}_{i [n]})}$ is IO contractible over $\RV{H}$.
\end{proof}

\section{Proofs related to precedent}\label{sec:proof_precedent}


\begin{reptheorem}{th:latent_to_observable}
Given a CIIR see-do model $(\prob{P}_\cdot,(\RV{E}_i,\RV{X}_i,\RV{Y}_i,\RV{Z}_i)_{i\in\mathbb{N}\cup\{c\}})$ with $E,X,Y$ and $Z$ all discrete, recall $\RV{H}$ is the directing random conditional of $(\prob{P}_\cdot,\RV{Z}_{\mathbb{N}},(\RV{E}_i,\RV{X}_i,\RV{Y}_i)_{i\in \mathbb{N}})$.

Let $I\subset \Delta(Y)^{XZ}$ be the event $\RV{H}^Y_{Xz}=\RV{H}^Y_{Xz'}$ for all $z,z'\in Z$; i.e. the event that $\RV{Y}_i$ is independent of $\RV{Z}_i$ conditional on $\RV{X}_i$ and $\RV{H}^Y_{XZ}$. Define $\prob{Q}_\alpha\in \Delta(\Omega)$ to be the probability measure such that, for all $A\in \sigalg{F}$
\begin{align}
\prob{Q}_\alpha(A) := \prob{P}_\alpha^{\mathrm{Id}_\Omega|\mathds{1}_I\circ \RV{H}}(A|1)
\end{align}
i.e. $\prob{Q}_\alpha$ is $\prob{P}_\alpha$ conditioned on $\RV{H}^Y_{XZ}\in I$, so $\RV{Y}_i\CI^e_{\prob{Q}_\cdot} \RV{Z}_i|(\RV{X}_i,\mathrm{Id}_C)$.

If the options $C$ have precedent with respect to $(\prob{Q}_\cdot,(\RV{E}_i,\RV{X}_i,\RV{Y}_i,\RV{Z}_i)_{i\in\mathbb{N}\cup\{c\}})$, and this model also satisfies conditional absolute continuity, then $(\prob{Q}_\cdot,\RV{X},\RV{Y})$ is also CIIR.
\end{reptheorem}

\begin{proof}
We apply the conditional absolute continuity condition to show that $\RV{Y}_i\CI^e_{\prob{Q}} \RV{E}_i|(\RV{Z}_i,\RV{X}_i,\RV{G},\text{Id}_C)$ for $i\in \mathbb{N}$. We then apply the precedent condition to extend this independence to $\RV{Y}_c\CI^e_{\prob{Q}} \RV{E}_c|(\RV{Z}_c,\RV{X}_c,\RV{G},\text{Id}_C)$ to complete the proof.

Note that by construction of $\prob{Q}_\alpha$ we have $\RV{Y}_i\CI^e_{\prob{Q}} \RV{Z}_i|(\RV{X}_i,\RV{G},\text{Id}_C)$. This in turn implies, for all $\alpha$ the following holds $\prob{Q}_\alpha$-almost surely:
\begin{align}
    \sum_{e\in E} \RV{G}^y_{exz}\frac{\RV{G}^x_{ez}\RV{G}^e_z}{\sum_{e'\in E}\RV{G}^x_{e'z}\RV{G}^{e'}_z}&\overset{\prob{Q}_\alpha}{\cong} \sum_{e\in E} \RV{G}^y_{exz'}\frac{\RV{G}^x_{ez'}\RV{G}^e_{z'}}{\sum_{e'\in E}\RV{G}^x_{e'z'}\RV{G}^{e'}_{z'}}\label{eqApp:polynomial_base}
\end{align}

Conditioning on $\RV{G}^Y_{EXZ}=g^Y_{EXZ}$

\begin{align}
    \sum_{e\in E} g^y_{exz}\frac{\RV{G}^x_{ez}\RV{G}^e_z}{\sum_{e'\in E}\RV{G}^x_{e'z}\RV{G}^{e'}_z}&\overset{\prob{P}_C}{\cong} \sum_{e\in E} \RV{g}^y_{exz'}\frac{\RV{G}^x_{ez'}\RV{G}^e_{z'}}{\sum_{e'\in E}\RV{G}^x_{e'z'}\RV{G}^{e'}_{z'}}\label{eqApp:polynomial}
\end{align}

Eq. \eqref{eqApp:polynomial} defines a polynomial constraint on $\RV{G}^{\RV{Ex}}_{\{z,z'\}}$ for each $x,z,z'$. If $g^y_{exz}=g^y_{e'xz}$ for all $e,e'$ then this constraint is trivial; if $g^y_{exz} = g^y_{exz'}$ also, then it is satisfied for every possible value of $\RV{G}^x_{E\{z,z'\}}$, otherwise it is unsatisfiable.

We will show that, unless $g^y_{exz}= g^y_{e'xz}$ for all $e,e'$ and $z$, that this constraint is nontrivial for some $z$. Consequently, the set of solutions for $\RV{G}^x_{EZ}$ subject to the restriction $g^y_{exz}\neq g^y_{e'xz}$ has Lebesgue measure 0. We will do this by showing that, assuming $g^y_{exz} > g^y_{e^<xz}$ for some $e,e^<$, we can find alternative realisations of $\RV{G}^e_{z}$ that lead to unequal values of the left hand side of Eq \eqref{eqApp:polynomial} without affecting the right hand side.

Let $g^x_{ez}$ and $g^e_z$ be a possible realisation of $\RV{G}^x_{ez}$ and $\RV{G}^e_z$. Assuming $g^y_{exz} > g^y_{e^<xz}$, either $g^x_{ez}=g^x_{e^<z}$, $g^x_{ez}< g^x_{e^<z}$ or $g^x_{ez}>g^x_{e^<z}$. Consider the first case, and take $g'$ such that $g^{\prime e}_{z}=0.5g^{e}_{z}$ and $g^{\prime e^<}_{z}=g^{e^<}_{z}+0.5g^{e}_{z}$ and equal to $g^{e''}_z$ for all other $e''\in E$. Note that $g^{\prime E}_z$ is also a possible realisation of $\RV{G}^e_z$, as it is everywhere positive and sums to 1, and $g'^{\prime e}_{z}<g^{e}_{z}$ almost surely as $g^{e}_{z}>0$ almost surely. Then
\begin{align}
    \frac{g^x_{ez}g^e_z}{\sum_{e'\in E}g^x_{e'z}g^{e'}_z} &> \frac{g^x_{ez}g^{\prime e}_z}{\sum_{e'\in E}g^x_{e'z}g^{\prime e'}_z}\\
    \frac{g^x_{e^<z}g^{e^<}_z}{\sum_{e'\in E}g^x_{e'z}g^{e'}_z} &< \frac{g^x_{e^<z}g^{\prime e^<}_z}{\sum_{e'\in E}g^x_{e'z}g^{\prime e'}_z}
\end{align}
because by assumption the denominator remains the same. But then
\begin{align}
    \sum_{e\in E} g^y_{exz}\frac{g^x_{ez}g^e_z}{\sum_{e'\in E}g^x_{e'z}g^{e'}_z}&> \sum_{e\in E} g^y_{exz'}\frac{g^x_{ez}g^{\prime e}_{z'}}{\sum_{e'\in E}g^x_{e'z'}g^{\prime e'}_{z'}}\label{eqApp:inequality}
\end{align}
because on the right side a smaller term in the sum receives more weight, a larger term receives less weight and all other terms are weighted equally.

Consider $g^x_{ez'}>g^x_{e^<z'}$. Then we still have
\begin{align}
    \frac{g^x_{ez}g^e_z}{\sum_{e'\in E}g^x_{e'z}g^{e'}_z} &> \frac{g^x_{ez}g^{\prime e}_z}{\sum_{e'\in E}g^x_{e'z}g^{\prime e'}_z}\\
    \frac{g^x_{e^<z}g^{e^<}_z}{\sum_{e'\in E}g^x_{e'z}g^{e'}_z} &< \frac{g^x_{e^<z}g^{\prime e^<}_z}{\sum_{e'\in E}g^x_{e'z}g^{\prime e'}_z}
\end{align}
For the second inequality, the right hand numerator grows and the denominator shrinks. For the first, note that
\begin{align}
    \frac{g^x_{ez}g^{\prime e}_z}{\sum_{e'\in E}g^x_{e'z}g^{\prime e'}_z} &= \frac{0.5 g^x_{ez}g^{e}_z}{\sum_{e'\in E}g^x_{e'z}g^{e'}_z - 0.5 g^{e}_z (g^x_{ez} - g^x_{e^< z})}
\end{align}
$g^e_z g^x_{ez}<1$ (an almost sure event) implies that the right hand denominator is greater than $0.5 \sum_{e'\in E}g^x_{e'z}g^{e'}_z$, and hence the right hand side is less than $\frac{g^x_{ez}g^e_z}{\sum_{e'\in E}g^x_{e'z}g^{e'}_z}$.

Thus the conclusion in Eq. \eqref{eqApp:inequality} follows for the same reasons as before. Considering $g^x_{ez'}< g^x_{e^<z'}$, analogous reasoning implies Eq. \eqref{eqApp:inequality} once again.

Thus, unless $g^y_{exz}=g^y_{e'xz}$ for all $e,e'$ and $z$, Eq. \eqref{eqApp:polynomial} implies a nontrivial constraint on $\RV{G}^x_{Ez}$ for some $z$. Thus for some $e,e'$,$z$, $x$ and $y$ the set of solutions $S:=\{g^X_{EZ}|\RV{G}^X_{EZ}=g^X_{EZ}\text{ satisfies Eq. \eqref{eqApp:polynomial} for all }x, z\land g^y_{exz}\neq g^y_{e'xz}\}$ has Lebesgue measure 0 \citep{okamoto_distinctness_1973}, and so by domination
\begin{align}
    \prob{Q}_{\alpha}^{\RV{G}^X_{EZ}|\RV{G}^{XY}_{EZ}}(S|g^{XY}_{EZ}) = 0
\end{align}
On the other hand, by assumption, the set $T:=\{g^E_{z}|\RV{G}^E_z=g^E_z\text{ satisfies Eq. \eqref{eqApp:polynomial}}\}$ has measure 1. Thus we conclude that with the exception of a $\prob{Q}_\alpha$ measure 0 set, $g^y_{exz}=g^y_{e'xz}$. That is, $\RV{Y}_i\CI^e_{\prob{Q}} \RV{E}_i|(\RV{Z}_i,\RV{X}_i,\RV{G},\text{Id}_C)$. By contraction with $\RV{Y}_i\CI^e_{\prob{Q}} \RV{Z}_i|(\RV{X}_i,\RV{G},\text{Id}_C)$, we have $\RV{Y}_i\CI^e_{\prob{Q}} (\RV{Z}_i,\RV{E}_i)|(\RV{X}_i,\RV{G},\text{Id}_C)$. 

By CIIR of the $(\RV{E}_i|(\RV{X}_i,\RV{Y}_i))$ pairs, we have for all $i$, $\prob{Q}_\alpha^{\RV{Y}_i\RV{X}_i|\RV{E}_i\RV{G}}\overset{\prob{Q}_\alpha^{\RV{E}_i|\RV{G}}}{\cong}\prob{Q}_\alpha^{\RV{Y}_c\RV{X}_c|\RV{E}_c\RV{G}}$. Because we have a representative version $\RV{G}^{XY}_E$ of $\prob{Q}_\alpha^{\RV{Y}_i\RV{X}_i|\RV{E}_i\RV{G}}$ for all $i\in \mathbb{N}$ (Theorem \ref{th:repr_cond}) and precedent implies that any set of measure 0 with respect to  $\prob{Q}_\alpha^{\RV{E}_i|\RV{G}}$ for all $i\in\mathbb{N}$ also has measure 0 with respect to $\prob{Q}_\alpha^{\RV{E}_c|\RV{G}}$, we have
\begin{align}
    \RV{G}^{XY}_{E} \overset{\prob{Q}_\alpha^{\RV{E}_c|\RV{G}}}{\cong} \prob{Q}_\alpha^{\RV{Y}_c\RV{X}_c|\RV{E}_c\RV{G}}
\end{align}
and thus
\begin{align}
    \RV{G}^Y_X \overset{\prob{Q}_\alpha^{\RV{X}_c|\RV{G}}}{\cong} \prob{Q}_\alpha^{\RV{Y}_c|\RV{X}_c\RV{G}}
\end{align}
completing the proof.
\end{proof}

% \subsection{Individual level conditionally independent and identical responses}\label{app:il_ciir}


% \begin{replemma}{lem:ind_to_cc}
% Given sequential input-output model $(\prob{P}_C,(\RV{D},\RV{I}),\RV{Y})$ with $\prob{P}_\alpha^{\RV{Y}|\RV{WDI}}$ IO contractible over $\RV{W}$, if $\RV{Y}\CI_{\prob{P}_C}^e (\RV{I},\text{Id}_C)|(\RV{W},\RV{D})$ and for any $j\in I$, $\sum_{\alpha\in C} \prob{P}_\alpha^{\RV{I}_i}(j)>0$, then there is some $\RV{W}'$ such that $\prob{P}_\alpha^{\RV{Y}|\RV{W}'\RV{D}}$ is also IO contractible over $\RV{W}$.
% \end{replemma}

% \begin{proof}
% Fix arbitrary $\nu\in \Delta(I^{\mathbb{N}})$ such that $\sum_{\alpha\in C} \prob{P}_\alpha^{\RV{I}_i} \gg \nu$. By assumption of IO contractibility of $\prob{P}_C^{\RV{Y}|\RV{WDI}}$ and Theorem \ref{th:ciid_rep_kernel}
% \begin{align}
%     \prob{P}_C^{\RV{Y}|\RV{WDI}} &\overset{\prob{P}_C}{\cong} \tikzfig{index_independence_1}\\
%     &\overset{\prob{P}_C}{\cong} \tikzfig{index_independence_2}
% \end{align}
% Where $\Pi_{D,i}:D^{\mathbb{N}}\kto D$ projects the $i$th coordinate, and similarly for $\Pi_{Y,i}$.

% In particular, for any $i\in \mathbb{N}$, $j\in I$, this holds for some $\nu$ such that $\nu(\Pi_{Y,i}^{-1} (j))=1$ and by extension for any finite $A\subset \mathbb{N}$ we can find $\nu$ such that $\nu(\Pi_{Y,i}^{-1} (j))=1$ for all $i\in A$, $j\in I$. Thus, for any $n\in \mathbb{N}$
% \begin{align}
%     \prob{P}_C^{\RV{Y}_{[n]}|\RV{W}\RV{D}_{[n]}\RV{I}_{[n]}} &\overset{\prob{P}_C}{\cong} \tikzfig{index_independence_3}\label{eq:follows_from_determinism}\\
%     &\overset{\prob{P}_C}{\cong} \tikzfig{index_independence_4}\label{eq:follows_from_equality}
% \end{align}

% where Equation \eqref{eq:follows_from_determinism} follows from Theorem \ref{th:fong_det_kerns} and Equation \eqref{eq:follows_from_equality} follows from the fact that Equation \eqref{eq:follows_from_determinism} holds for arbitrary $j\in I$.

% Thus by Lemma \ref{lem:infinitely_extended_kernels}
% \begin{align}
%     \prob{P}_C^{\RV{Y}|\RV{WD}} &= \tikzfig{index_independence_5}
% \end{align}
% Applying Theorem \ref{th:ciid_rep_kernel}, $\prob{P}_C^{\RV{Y}|\RV{WD}}$ is IO contractible over $\RV{W}$.
% \end{proof}


% \begin{reptheorem}{th:ind}
% Given a sequential input-output model $(\prob{P}_C,(\RV{D},\RV{I}),\RV{Y})$ on $(\Omega,\sigalg{F})$ with $Y$ standard measurable and $C$ countable, if there is some $\RV{J}$ such that for each $\alpha$
% \begin{align}
%     \prob{P}_\alpha^{\RV{Y}_i|\RV{J}\RV{I}_i\RV{D}_i} &= \prob{P}_\alpha^{\RV{Y}_i|\RV{J}\RV{I}_i\RV{D}_i} &\forall i,j\in \mathbb{N}\\
%     \RV{Y}_i&\CI^e_{\prob{P}_C} (\RV{I}_{\{i\}^{\complement}},\RV{D}_{\{i\}^{\complement}})|(\RV{J},\RV{I}_i,\RV{D}_i)
% \end{align}
% and
% \begin{align}
%     &\RV{Y}\CI^e_{\prob{P}_C} \RV{I} | \text{Id}_C\\
%     &\RV{YIJ}\CI^e_{\prob{P}_C} \RV{D}|\text{Id}_C\\
%     &\RV{YIJ}\CI^e_{\prob{P}_C} \text{Id}_C|\RV{D}\\
%     &\forall i,j\in \mathbb{N}: \sum_{\alpha\in C} \prob{P}_\alpha^{\RV{I}_i}(j)>0
% \end{align}
% then $\prob{P}_C^{\RV{Y}|\RV{JD}}$ is IO contractible over $\RV{J}$.
% \end{reptheorem}

% \begin{proof}
% For any $\alpha\in C$
% \begin{align}
%     \prob{P}_\alpha^{\RV{YJ}|\RV{I}} &= \tikzfig{kernel_fac_with_idents}\\
%     &= \tikzfig{kernel_fac_with_idents_indepped}
% \end{align}

% Define $\kernel{Q}$ by $\alpha\mapsto \prob{P}_\alpha$ and $\kernel{Q}^{\cdot|\cdot\text{Id}_C}$ by $\alpha\mapsto \prob{P}_\alpha^{*}$ and $\kernel{Q}^{\text{Id}_C}$ is an arbitrary distribution in $\Delta(C)$ with full support. Note that the support of $\kernel{Q}^{\RV{IDYJ}}$ is the union of the support of $\prob{P}^{\RV{IDYJ}}_\alpha$ for all $\alpha$. Then
% \begin{align}
%     \kernel{Q}^{\RV{YJ}|\RV{IC}} &\overset{\prob{Q}}{\cong} \tikzfig{kernel_fac_with_idents_kernelised}
% \end{align}

% By assumption $\RV{YI}\CI^e_{\prob{P}_C} \RV{D}|\text{Id}_C$, it is also the case that
% \begin{align}
%     \kernel{Q}^{\RV{Y}|\RV{ID}} &\overset{\prob{Q}}{\cong} \tikzfig{kernel_Q_fac_with_idents}\\
%     &\overset{\prob{Q}}{\cong} \tikzfig{kernel_Q_fac_with_idents_indepped}\\
%     &\overset{\prob{Q}}{\cong} \tikzfig{kernel_Q_fac_with_idents_subbed}
% \end{align}
% But
% \begin{align}
%     \kernel{Q}^{\RV{Y}|\RV{ID}}=\sum_{\alpha\in C} \prob{P}_\alpha^{\RV{Y}|\RV{ID}}\kernel{Q}^{\text{Id}_C}(\alpha)\\
%     &= \prob{P}_C^{\RV{Y}|\RV{ID}}\\
%     \implies \tikzfig{kernel_Q_fac_with_idents_subbed} &= \prob{P}_C^{\RV{Y}|\RV{ID}}
% \end{align}

% Furthermore, by assumption $\RV{Y}\CI^e_{\prob{P}_C} \RV{I} | \text{Id}_C$, so there is some $\kernel{K}:C\kto Y\times W$ such that
% \begin{align}
%     \kernel{Q}^{\RV{YJ}|\RV{IC}} &\overset{\prob{Q}}{\cong} \tikzfig{kernel_Q_indepped}\\
%     \implies \prob{P}_C^{\RV{YJ}|\RV{ID}} &= \tikzfig{kernel_Q_fac_with_idents_swapped}\\
%     &= \tikzfig{kernel_P_indep}
% \end{align}
% Then by Lemma \ref{lem:ind_to_cc}, $\prob{P}_C^{\RV{Y}|\RV{JD}}$ is IO contractible over $\RV{J}$.
% \end{proof}