
%!TEX root = ../main.tex

\chapter{Conclusion}\label{ch:discussion}

In this thesis, we considered how idealised decision problems could be used as a basis for causal modelling. Decision making models, as we defined them, are similar to classical statistical models with sample spaces, random variables and probability measures defined on the sample space, but they add to this an option set, with each option associated with a probability measure on the sample space. The point of this is that, in general, a decision maker knows something a ``classical statistician'' does not -- she knows what her set of options is. Decision making models also differ from the more common kinds of causal models -- potential outcomes and structural interventional models -- in that we suggest the former is interpreted alongside a \emph{decision procedure} (Section \ref{sec:actions}), while the interpretation of potential outcomes and structural interventions is somewhat more free-floating.

Our reason for pursuing this approach is that often (though not always), causal modelling is done to assist a decision maker to make a good choice from among their options. In this situation, we suggest that a decision maker already has a decision procedure on hand and in that case our approach demands less additional theory than the potential outcomes or structural interventional approach. A key question is whether this ``potential reduction in assumptions'' actually enables the construction of practically useful causal models using fewer assumptions.

A simple assumption a decision maker with access to data could make when assessing their options is that they can identify corresponding sequences of inputs and outputs that are related by a fixed stochastic function. These sequences span the given data as well as observations that arise as a result of the choice made by the decision maker. They can therefore determine how they should expect their choice to influence the world by assessing the relationship between inputs and outputs they observed in the data.  We proposed that this idea of a fixed but unknown functional relationship can be formalised using a sequential model that features conditionally independent and identical responses (CIIRs).

When a decision maker assumes that a model features CIIRs, the functional relationships themselves are unobserved. The decision maker doesn't necessarily have a clear means of interpreting arbitrary unobserved variables. We showed that the assumption of CIIRs is equivalent to a two symmetries of a decision model: firstly, it is equivalent to the \emph{IO contractible} of this model over some auxiliary variable, and secondly (under some side conditions) it is equivalent to the interchangeability of infinite conditioning sequences (Theorems \ref{th:data_ind_CC} and \ref{th:infinite_condition_swaps} respectively). We pointed out the second condition is often unreasonable.

The identified symmetries offer decision makers an alternative means of assessing unconfoundedness-like assumptions. In the potential outcomes framework, unconfoundedness is given as an assumption of independence between the potential outcomes and the corresponding inputs, and in the structural graphical models framework it is given by the assumption that all backdoor paths between the input and output variables are blocked by an observed variable. We showed that a decision maker might instead consider whether a set of problems are identical -- for example, predicting the consequences of their actions given the data and predicting held out observations given the same data. A decision maker might form an opinion about these questions without appealing to theories of counterfactuals or structural interventions.

On the other hand, the fact that this assumption of interchangeable data sequences seems to be mostly unreasonable means that it doesn't quite deliver on the ``useful'' front. People genuinely do make the assumption of unconfoundedness, but perhaps this could be interpreted as an assumption that the condition of interchangeable data sequences holds (in some sense) approximately. We identified the assumption of \emph{precedent} as a possible weakening of this condition. This assumption holds that, rather than \emph{every} input-output pair in the sequence of observations obey the same relation as the input-output pairs produced as a result of the decision maker's actions, only some unknown (but non-negligible) fraction obeys the same relation. This assumption is motivated to some extent by the observation that this is precisely what is implied in the structural interventional setting by the assumption of \emph{hidden confounders}.

The assumption of precedent has significant implications for inference when it is combined with an assumption of \emph{generic relations between conditionals}. Theorem \ref{th:latent_to_observable} showed that, given these assumptions, a conditional independence observed in the data implies an identical response function between inputs and outputs that arise as consequences of the decision maker's action. A weakness of this theorem is that the assumptions of precedent and generic relations between conditionals are themselves expressed in terms of an unobserved variable, and so we can't assume that a decision maker has any clear way of interpreting these assumptions. On top of this, the assumption of generic relations between conditionals is not particularly easy to understand by itself.

We offered the example of medical practitioners making prescriptions and observing patient outcomes to illustrate how this theorem works. In this case, one can informally understand the assumption of precedent as the assumption that, whatever the decision maker ends up doing, other medical practitioner have already sometimes acted in just the same way as the decision maker. There is an intuitive sense in which the choices made by practitioners are ``causally prior'' to the treatment and subsequent observation of patient outcomes (and also temporally prior).

Separately, a number of authors have suggested that generic relations between conditionals usually hold if the conditionals correspond to causal relations \citep{meek_strong_1995,lemeire_replacing_2013}. Our result suggests that such generic relations might, along with the assumption of precedent, be sufficient to support inference of the consequences of decisions from data. We have not established a perfect correspondence between the kinds of ``generic relations'' required by Theorem \ref{th:latent_to_observable} and the kinds of generic relations investigated by researchers studying the principle of independent causes and mechanisms. However, if such a correspondence holds, then we would have an understanding of why independent causes and mechanisms are important to the practice of data-driven decision making that does not depend on a structural theory of causation. Of particular interest is the question of whether the various methods developed to assess causal direction based on the principle of independent causes and mechanisms provides any reason to believe the required type of relation for Theorem \ref{th:latent_to_observable} holds -- see \citep{mooij_j.m._distinguishing_2016} for an overview of some of these methods.

Another extension of this line of work is to consider finite sample performance of inference based on the assumption of precedent. Intuitively, one might expect that inference based on the assumption of precedent is hard when the actions of interest are taken infrequently. Furthermore, we speculate that inference is also hard when the relation between conditionals is almost non-generic, which might happen when the data is produced by individuals controlling inputs in order to keep outputs in a desired range.

To express this theory, we made use of a string diagram notation for writing out some proofs and as a visual aid to understanding different kinds of decision models. String diagrams are simply a notation for reasoning using probability theory, and as such are a convenience, not a critical piece of the theory. Compared to the more common diagrammatic language of directed acyclic graphs (DAGs), the chief advantage of the string diagram notation is that it explicitly represents Markov kernels in the diagrams, and so it is possible (for example) to write that one diagram is equal to another different diagram without ambiguity. DAGs have an advantage over string diagrams in that correspondences between many structural properties of the diagram and properties of the model have been worked out -- for example, the correspondence between d-separation and conditional independence, as well as more sophisticated properties necessary and sufficient for identifiability \citep{tian2002general,shpitser_complete_2008}. A string diagram analogue of d-separation and its relation to different notions of conditional independence postulated by \citet{fritz_synthetic_2020} would further facilitate the use of string diagrams in causal reasoning.

The standard approaches to causal inference depend on a theory of causation -- this may be a theory of structural interventions, a vague notion of counterfactuals or something else. Such theories can be helpful to the extent that they make contact with intuitive ideas we have about causation and counterfactuals, but such intuitions are only really relevant to a small class of data-driven decision making problems. As we have shown, theories of causation are not needed to formally represent data-driven decision problems, nor are they needed to formulate substantive assumptions that license a decision maker to draw conclusions about the consequences of their choices from the data they have available. Theories of causation are troublesome to researchers who want to study problems of data-driven decision making in diverse contexts; it simply isn't clear how to generalise the relevant causal intuitions beyond problems in which they are apparently reliable. Our work calls into question whether relying on theories of causation is really necessary. We have shown that causal inference problems can be formalised, and in some cases solved,  without them, which allows one thus to sidestep the vexed question of exactly what they mean.