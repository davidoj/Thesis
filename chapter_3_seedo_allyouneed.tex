
%!TEX root = main.tex

\chapter{Chapter 3: See-do models}

\todo[inline]{These are ``todo'' notes. All such notes that involve theoretical development are also collected in an unordered list of outstanding theoretical questions}

\todo[inline]{The basic claim of this chapter is that see-do models are the basic type of thing that everyone who is studying ``causal inference'' is working with, even if they don't know it themselves}

Consider the following problem: you are presented with a collection $\RV{H}$ of hypotheses about how the world might function and a vector $\mathbf{x}$ of observational data which you know could have taken values in some space $X$. You want to determine which hypothesis $\RV{H}\in \RV{H}$ best describes the world. However you ultimately solve the problem, the next step you take will probably be to determine for each $\RV{H}\in \RV{H}$ a probability distribution $\kernel{P}_\RV{H}\in \Delta(\sigalg{X})$ that indicates how likely you would be to observe the various elements of $X$ were $\RV{H}$ in fact the case. This is a \emph{statistical model} -- an indexed set of probability distributions $\{\prob{P}_\RV{H}|\RV{H}\in \RV{H}\}$. Statistical models are ubiquitous in the field of statistics -- they are found in statistical decision theory where the elements of $\RV{H}$ are typically called ``states''\citep{wald_statistical_1950}, in Bayesian inference where the elements of $\RV{H}$ may be called ``parameters'' \citep{freedman_asymptotic_1963} and in frequentist inference where elements of $\RV{H}$ they may be called ``hypotheses'' \citep{fisher_statistical_1992}. 

These different approaches to statistics may have different notions of what the ``best hypothesis'' $\RV{H}$ is, may employ different estimation methods and may not even agree about what ``distributed according to $\kernel{P}_\RV{H}$'' means. Nonetheless, the interpretation of the statistical model in each case is roughly the same: supposing $\RV{H}\in\RV{H}$ is true, the data will be distributed according to $\kernel{P}_\RV{H}$. A statistical model takes a hypothesis and tells you what you are likely to \emph{see}.

Sometimes we are interested in modelling situations where we can also make some choices that also affect the eventual consequences. For example, I might hypothesise $\RV{H}_1$: the switch on the wall controls my light, $\RV{H}_2$: the switch on the wall does not control my light. Then, given $\RV{H}_1$ I can choose to toggle the switch, and I will see my light turn on, or I can choose not to toggle the switch and I will not see my light turn on. Given $\RV{H}_2$, neither choice will result in a light turned on. Choices are clearly different to hypotheses: the choice I make depends on what I want to happen, while whether or not a hypothesis is true has no regard for my ambitions.

A ``statistical model with choices'' is simply a map $\prob{T}:D\times \RV{H}\to \Delta(\sigalg{E})$ for some set of choices $D$, hypotheses $\RV{H}$ and outcome space $(E,\sigalg{E})$. We can also distinguish two types of outcomes: \emph{observations} which are given prior to a choice being made and \emph{consequences} which happen after a choice is made. Observations cannot be affected by the choices made, while consequences are not subject to this restriction. That is, observations are what we might \emph{see} before making a choice, which depends on the hypothesis alone, and if we are lucky we may be able to invert this dependence to learn something about the hypothesis from observations. On the other hand, the consequences of what we \emph{do} depends jointly on the hypothesis and the choice we make and we judge which choices are more desirable on the basis of which consequences we expect them to produce. 

What we are studying is a family of models that generalises of statistical models to include hypotheses, choices, observations and consequences. These models are referred to as \emph{see-do models}. Hypotheses, observations, consequences and choices are not individually new ideas. \emph{Statistical decision problems} \citep{wald_statistical_1950,savage_foundations_1972} extend statistical models with decisions and \emph{losses}. Like consequences, losses depend on which choices are made. However, unlike consequences, losses must be ordered and reflect the preferences of a decision maker. \emph{Influence diagrams} are directed graphs created to represent decision problems that feature ``choice nodes'', ``chance nodes'' and ``utility nodes''. An influence diagram may be associated with a particular probability distribution \cite{nilsson_evaluating_2013} or with a set of probability distributions \cite{dawid_influence_2002}.

See-do models have deep roots in decision theory. Decision theory asks, out of a set of available acts, which ones ought to be chosen. See-do models answer an intermediate question: out of a set of available acts, what are the consequences of each? This question is described by \citet{pearl_causality:_2009} as an ``interventional'' question.

See-do models depend cruicially on a set of choices $D$. While these models can obviously answer questions like ``what is likely to happen if I choose $d\in D$?'', this construction appears to rule out ``causal'' questions like ``Does rain cause wet roads?''. We define a restricted idea of causation called $D$-\emph{causation}. Roughly, if the roads get wet when it rains regardless of my choice of $d\in D$, then rain ``$D$-causes'' wet roads. $D$-causation is closely related to the idea \emph{limited invariance} put forward by \citet{heckerman_decision-theoretic_1995}.

The field of causal inference is additionally concerned with types of questions called ``counterfactual'' by Pearl. There is substantial theoretical interest in counterfactual questions, but counterfactual questions are much more rarely found in applications than interventional questions. Even though see-do models are motivated by the need to answer interventional questions, the theory develope here is surprisingly appliccable to counterfactuals as well. In particular, the theory of see-do models offers explanations for three key features of counterfacutla models:
\begin{itemize}
    \item \textbf{Apparent absence of choices}: \emph{Potential outcomes} models, which purportedly answer counterfactual questions, are standard statistical models \emph{without choices} \citep{rubin_causal_2005}
    \item \textbf{Deterministic dependence on unobserved variables}: Counterfactual models involve \emph{deterministic} dependence on unobserved variables \citep{pearl_causality:_2009,rubin_causal_2005,richardson2013single}
    \item \textbf{Residual dependence on observations}: Counterfactual questions depend on the given data \emph{even if the joint distribution of this data is known}. For example, \citet{pearl_causality:_2009} introduces a particular method for conditioning a known joint distribution on observations that he calls \emph{abduction}
\end{itemize}

Potential outcomes models lack a notion of ``choices'' because there is a generic method to ``add choices'' to a potential outcomes model, which is implicitly used whenever potential outcomes models are used. Furthermore, we show that a see-do model induces a potential outcome model if and only if it is a model of \emph{parallel choices}, and in this case the observed consequences depend deterministically on the unobserved potential outcomes in precisely the manner as given in \citet{rubin_causal_2005}. Parallel choices can be roughly understood as models of sequences of experiments where an action can be chosen for each experiment, and with the special properties that repeating the same action deterministically yields the same consequence, and the consequences of a sequence of actions doesn't depend on the order in which the actions are taken. That is, we show that the fundamental property of any ``counterfactual'' model is \emph{deterministic reproducibility} and \emph{action exchangeability}, and while these models may admit a ``counterfactual'' interpretation, they are fundamentally just a special class of see-do models.

\todo[inline]{But the proof is still in my notebook}

\todo[inline]{Interestingly, it seems to be possible to construct a see-do model where the ``hypothesis'' is a quantum state, and quantum mechanics + locality seems to rule out parallel choices in such models in a manner similar to Bell's theorem. ``Seems to'' because I haven't actually proven any of these things.}

The residual dependence on observations exhibited by counterfactual questions is a generic property of see-do models, and it is a particular property of \emph{decision problems} are notable in that it is often

\todo[inline]{Where to discuss the connections to statistical decision theory?}

See-do models are closely related to \emph{statistical decision theory} introduced by \citet{wald_statistical_1950} and elaborated by \citet{savage_foundations_1972} after Wald's death. See-do models equipped with a \emph{utility function} induce a slightly generalised form of statistical decision problems, and the complete class theorm is appliccable to these models.

A stylistic difference between see-do models and most other causal models is that see-do models explicitly represent both the observation model and the consequence model and their coupling, making them ``two picture'' causal models. Causal Bayesian Networks and Single World Interention Graphs \citep{richardson2013single} use ``one picture'' to represent the observation model and the consequence model. However, both of these approaches employ ``graph mutilation'', so one picture on the page actually corresponds to many pictures when combined with the mutilation rules. For more on how these different types of models relate, see Section \ref{sec:single_double_representation}. \citet{lattimore_replacing_2019}'s Bayesian causal inference employs two-picture causal models, as do ``twin networks'' \citep{pearl_causality:_2009}.


\section{Definition}

\todo[inline]{Terminology question: The variables $\RV{H}$ and $\RV{D}$ aren't necessarily \emph{random} in the commonsense understanding of the word. They're also defined on a \emph{kernel space} rather than a \emph{probability space}. They're currently called random variables simply by virtue of being measurable functions on the outcome space. I'm not a huge fan of ``quasirandom variables'', but it does capture the idea that these things are very similar to random variables but not exactly the same.}

\begin{definition}[See-Do model]\label{def:seedo}
A \emph{see-do model} $\langle\kernel{T},\RV{H},\RV{D},\RV{X},\RV{Y}\rangle$ is a kernel space (Definition \ref{def:kernel_space}) $(\kernel{T},H\times D,X\times Y)$ along with four random variables: the \emph{hypothesis} $\RV{H}:H\times D\times X\times Y\to H$, the \emph{choice} $\RV{D}:H\times D\times X\times Y\to D$, the \emph{observations} $\RV{X}:H\times D\times X\times Y\to X$ and the \emph{consequences} $\RV{Y}:H\times D\times X\times Y\to Y$, all given by the obvious projection maps. 

The spaces $H$, $D$, $X$ and $Y$ are the hypothesis, choice, observation and consequence spaces respectively.

A see-do model has the additional property that, holding the hypothesis fixed, the observations are independent of the choices - i.e. $\RV{X}\CI_{\kernel{T}} \RV{D}|\RV{H}$. We require that $H\times D$ is countable.
\end{definition}


\begin{theorem}[Observation and Consequence models]\label{th:obs_cmaps}
Any see-do model $(\kernel{T},\RV{H},\RV{D},\RV{X},\RV{Y})$ can be uniquely represented by the following pair of Markov kernels:
\begin{itemize}
    \item The \emph{observation map} $\kernel{T}^{\RV{X}|\RV{H}}$
    \item The \emph{consequence model} $\kernel{T}^{\RV{Y}|\RV{X}\RV{H}\RV{D}}$
\end{itemize}

Furthermore
\begin{align}
\kernel{T} = \begin{tikzpicture} \path (0,0) node (T) {$\RV{H}$}
        + (0,-1.15) node (D) {$\RV{D}$}
        ++ (0.5,0) node[copymap] (copy0) {}
        + (0.,-1.15) node[copymap] (copy2) {}
        ++ (0.7,0) node[kernel] (O) {$\kernel{T}^{\RV{X}|\RV{H}}$}
        ++ (0.7,0) node[copymap] (copy1) {}
        +  (0.9,-1) node[kernel] (C) {$\kernel{T}^{\RV{Y}|\RV{X}\RV{H}\RV{D}}$}
        ++ (1.9,0) node (X) {$\RV{X}$}
        +  (0,-1) node (Y) {$\RV{Y}$}
        + (0,0.5) node (H) {$\RV{H}$}
        + (0,-1.5) node (D2) {$\RV{D}$};
        \draw (T) -- (O) -- (X);
        \draw (copy0) to [out=-90,in=180] ($(C.west) + (0,0)$);
        \draw (D) to [out=0,in=180] ($(C.west) + (0,-0.15)$);
        \draw (copy1) to [out=-60,in=180] ($(C.west)+ (0,0.15)$);
        \draw (C) -- (Y);
        \draw (copy0) to [out = 65, in = 180] (H);
        \draw (copy2) to [out = -65, in = 180] (D2);
    \end{tikzpicture}
\end{align}
\end{theorem}

\todo[inline]{Maybe moves proofs out of main text}

\begin{proof}
By \ref{th:representaiton}, 

\begin{align}
\kernel{T} = \begin{tikzpicture} \path (0,0) node (T) {$\RV{H}$}
        + (0,-1.15) node (D) {$\RV{D}$}
        ++ (0.5,0) node[copymap] (copy0) {}
        + (0.,-1.15) node[copymap] (copy2) {}
        ++ (0.7,0) node[kernel] (O) {$\kernel{T}^{\RV{X}|\RV{H}\RV{D}}$}
        ++ (0.7,0) node[copymap] (copy1) {}
        +  (0.9,-1) node[kernel] (C) {$\kernel{T}^{\RV{Y}|\RV{X}\RV{H}\RV{D}}$}
        ++ (1.9,0) node (X) {$\RV{X}$}
        +  (0,-1) node (Y) {$\RV{Y}$}
        + (0,0.5) node (H) {$\RV{H}$}
        + (0,-1.5) node (D2) {$\RV{D}$};
        \draw (T) -- (O) -- (X);
        \draw[name path=P1] (copy0) to [out=-90,in=180] ($(C.west) + (0,0)$);
        \draw (D) to [out=0,in=180] ($(C.west) + (0,-0.15)$);
        \draw (copy1) to [out=-60,in=180] ($(C.west)+ (0,0.15)$);
        \draw (C) -- (Y);
        \draw (copy0) to [out = 65, in = 180] (H);
        \draw (copy2) to [out = -65, in = 180] (D2);
        \draw[name path=P2] (copy2) to [out = 65, in = 180] ($(O.west)+(0,-0.15)$);
    \end{tikzpicture}
\end{align}


By the assumption $\RV{X}\CI_{\kernel{T}} \RV{D}|\RV{H}$ and version 2 of conditional independence from Theorem \ref{th:ci_equivalence},

\begin{align}
\kernel{T} &= \begin{tikzpicture} \path (0,0) node (T) {$\RV{H}$}
        + (0,-1.15) node (D) {$\RV{D}$}
        ++ (0.5,0) node[copymap] (copy0) {}
        + (0.,-1.15) node[copymap] (copy2) {}
        ++ (0.7,0) node[kernel] (O) {$\kernel{T}^{\RV{X}|\RV{H}}$}
        ++ (0.7,0) node[copymap] (copy1) {}
        +  (0.9,-1) node[kernel] (C) {$\kernel{T}^{\RV{Y}|\RV{X}\RV{H}\RV{D}}$}
        ++ (1.9,0) node (X) {$\RV{X}$}
        +  (0,-1) node (Y) {$\RV{Y}$}
        + (0,0.5) node (H) {$\RV{H}$}
        + (0,-1.5) node (D2) {$\RV{D}$};
        \draw (T) -- (O) -- (X);
        \draw (copy0) to [out=-90,in=180] ($(C.west) + (0,0)$);
        \draw (D) to [out=0,in=180] ($(C.west) + (0,-0.15)$);
        \draw (copy1) to [out=-60,in=180] ($(C.west)+ (0,0.15)$);
        \draw (C) -- (Y);
        \draw (copy0) to [out = 65, in = 180] (H);
        \draw (copy2) to [out = -65, in = 180] (D2);
        \draw[-{Rays[n=8]}] (copy2) to [out = 65, in = 180] ($(O.west)+(-0.2,-0.5)$);
    \end{tikzpicture}\\
    &= \begin{tikzpicture} \path (0,0) node (T) {$\RV{H}$}
        + (0,-1.15) node (D) {$\RV{D}$}
        ++ (0.5,0) node[copymap] (copy0) {}
        + (0.,-1.15) node[copymap] (copy2) {}
        ++ (0.7,0) node[kernel] (O) {$\kernel{T}^{\RV{X}|\RV{H}}$}
        ++ (0.7,0) node[copymap] (copy1) {}
        +  (0.9,-1) node[kernel] (C) {$\kernel{T}^{\RV{Y}|\RV{X}\RV{H}\RV{D}}$}
        ++ (1.9,0) node (X) {$\RV{X}$}
        +  (0,-1) node (Y) {$\RV{Y}$}
        + (0,0.5) node (H) {$\RV{H}$}
        + (0,-1.5) node (D2) {$\RV{D}$};
        \draw (T) -- (O) -- (X);
        \draw (copy0) to [out=-90,in=180] ($(C.west) + (0,0)$);
        \draw (D) to [out=0,in=180] ($(C.west) + (0,-0.15)$);
        \draw (copy1) to [out=-60,in=180] ($(C.west)+ (0,0.15)$);
        \draw (C) -- (Y);
        \draw (copy0) to [out = 65, in = 180] (H);
        \draw (copy2) to [out = -65, in = 180] (D2);
    \end{tikzpicture}
\end{align}

\end{proof}

\begin{definition}[Consequence map]
Given a see-do model $(\kernel{T},\RV{H},\RV{D},\RV{X},\RV{Y})$, a \emph{consequence map} is a map $\kernel{C}:D\to \Delta(\sigalg{Y})$ where $D$ is a choice set and $Y$ is a consequence set.

The consequence model evaluated at any particular hypothesis $h\in H$, $\kernel{T}^{\RV{Y}|\RV{X}\RV{H}\RV{D}}_{\cdot,h,\cdot}$ is a consequence map.
\end{definition}

\todo[inline]{Not quite sure if this is the right place for the following definition}

The independence of observations and choices is preserved when we take the product of a see-do model and a \emph{prior} over hypotheses. Such a product produces a \emph{Bayesian see-do model}:

\begin{definition}[Bayesian See-Do Model]
A Bayesian See-Do Model $\langle\kernel{U},\RV{D},\RV{X},\RV{Y}\rangle$ is a Markov kernel space $(\kernel{U},D,X\times Y)$ with the property $\RV{X}\CI_{\kernel{U}}\RV{D}$, along with choices $\RV{D}$, observations $\RV{X}$ and consequences $\RV{Y}$, defined as before.
\end{definition}

\begin{theorem}[A see-do model with a prior is a Bayesian see-do model]
The product of a see-do model $\kernel{T}$ and a prior $\gamma\in \Delta(\sigalg{H})$
\begin{align}
    \kernel{U} &:= (\gamma\otimes \mathrm{Id}^D)\kernel{T}
\end{align}
Is a Bayesian see-do model.
\end{theorem}

\todo[inline]{Maybe moves proofs out of main text}

\begin{proof}

It nees to be shown that $\RV{X}\CI_{\kernel{U}}\RV{D}$.

By definition
\begin{align}
\kernel{U}^{\RV{X}|\RV{D}} &= \kernel{U}\kernel{F}^{\RV{X}}\\
                            &= (\gamma\otimes \mathrm{Id}^D)\kernel{T}\kernel{F}^{\RV{X}}\\
                            &= \begin{tikzpicture} \path (0,0) node[dist] (T) {$\gamma$}
                                    + (0,-1.15) node (D) {$\RV{D}$}
                                    ++ (0.5,0) node[copymap] (copy0) {}
                                    + (0.,-1.15) node[copymap] (copy2) {}
                                    ++ (0.7,0) node[kernel] (O) {$\kernel{T}^{\RV{X}|\RV{H}}$}
                                    ++ (0.7,0) node[copymap] (copy1) {}
                                    +  (0.9,-1) node[kernel] (C) {$\kernel{T}^{\RV{Y}|\RV{X}\RV{H}\RV{D}}$}
                                    ++ (1.9,0) node (X) {$\RV{X}$}
                                    +  (0,-1) node (Y) {};
                                    \draw (T) -- (O) -- (X);
                                    \draw (copy0) to [out=-90,in=180] ($(C.west) + (0,0)$);
                                    \draw (D) to [out=0,in=180] ($(C.west) + (0,-0.15)$);
                                    \draw (copy1) to [out=-60,in=180] ($(C.west)+ (0,0.15)$);
                                    \draw[-{Rays[n=8]}] (C) -- (Y);
                                \end{tikzpicture}\\
                            &= \begin{tikzpicture} \path (0,0) node[dist] (T) {$\gamma$}
                                    + (0,-1.15) node (D) {$\RV{D}$}
                                    ++ (0.5,0) coordinate (copy0)
                                    ++ (0.7,0) node[kernel] (O) {$\kernel{T}^{\RV{X}|\RV{H}}$}
                                    ++ (0.7,0) coordinate (copy1)
                                    ++ (1.9,0) node (X) {$\RV{X}$}
                                    +  (0,-1) node (Y) {};
                                    \draw (T) -- (O) -- (X);
                                    \draw[-{Rays[n=8]}] (D) -- (Y);
                                \end{tikzpicture}
\end{align}

Which implies $\RV{X}\CI_{\kernel{U}} \RV{D}$ by version (2) of conditional indpendence (Theorem \ref{th:ci_equivalence}).
\end{proof}
% Decisions are similar to the ``regime indicators'' found in \citet{dawid_decision-theoretic_2020}. They coincide precisely if we suppose that the observation and consequence spaces coincide ($X=Y$) and there exists an ``idle'' decision $d^*\in D$ such that $\kernel{C}_{(\cdot,d^*)} = \kernel{O}_{\cdot}$. However, in general we don't require that $\kernel{O}$ and $\kernel{C}$ are related in this manner. This assumption will be revisited in \todo[inline]{A section I haven't written yet}.

\subsubsection{Example}

Suppose we are betting on the outcome of the flip of a possibly biased coin with payout 1 for a correct guess and 0 for an incorrect guess, and we are given $N$ previous flips of the coin to inspect. This situation can be modeled by a hypothesis sufficient see-do model. Define $\kernel{B}:(0,1)\to \Delta(\{0,1\})$ by $\kernel{B}:\RV{H}\mapsto \mathrm{Bernoulli}(\RV{H})$. Then define ${\kernel{T}}$ by:

\begin{itemize}
    \item Choice set: $D=\{0,1\}$
    \item Observation set: $X=\{0,1\}^N$
    \item Consequence set: $Y=\{0,1\}$
    \item Hypothesis set: $H=(0,1)$
    \item Observation map: ${\kernel{T}}^{\RV{X}|\RV{H}}:\splitter{0.1}^N\kernel{B}$
    \item Consequence model: ${\kernel{T}}^{\RV{Y}|\RV{D}\RV{H}}:(h,d)\mapsto \mathrm{Bernoulli}(1-|d-h|)$
\end{itemize}

In this model, the chance $\RV{H}$ of the coin landing on heads is as much as we can hope to know about the success of our bet. $\RV{H}$ may be inferred from observation by some standard method, and 

\subsection{D-causation}

The choice set $D$ is a primitive element of a see-do model. However, while we claim that see-do models are the basic objects studied in causal inference, so far we have no notion of ``causation''. What we call $D$-\emph{causation} is one such notion. It is called $D$-causation because it is a notion of causation that depends on the set of choices available. A similar idea, called \emph{limited unresponsiveness}, is discussed extensively in the decision theoretic account of causation found in \citet{heckerman_decision-theoretic_1995}. The main difference is that see-do maps are fundamentally stochastic while Heckerman and Shachter work with ``states'' (approximately hypotheses in our terminology) that map decisions deterministically to consequences. In addition, while we define $D$-causation relative to a see-do map $\kernel{T}$, Heckerman and Shachter define limited unresponsiveness with respect to \emph{sets} of states.

Section \ref{sec:cbns_without_d} explores the difficulty of defining ``objective causation'' without reference to a set of choices. $D$ need not be interpreted as the set of choices available to an agent, but however we want to interpret it, all existing examples of causal models seem to require this set.

See Section \ref{ssec:random_variables} for the definition of random variables in Kernel spaces.

One way to motivate the notion of $D$-causation is to observe that for many decision problems, I may wish to include a very large set of choices $D$. Suppose I aim to have my light switched on, and there is a switch that controls the light. Often, the relevant choices for such a problem would appear to be $D_0=\{\text{flip the switch},\text{don't flip the switch}\}$. However, this doesn't come close to exhausting the set of things I might choose to do, and I might wish to consider a larger set of possibilities. For simplicity's sake, suppose I have instead the following set of options:

\begin{align*}
D_1:=&\{``\text{walk to the switch and press it with my thumb}'', \\
    &``\text{trip over the lego on the floor, hop to the light switch and stab my finger at it}'',\\
    &``\text{stay in bed}''\}
\end{align*}

If having the light turned on is all that matters, I could consider any acts in $D_1$ to be equivalent if, in the end, the light switch ends up in the same position. In this case, I could say that the light switch position $D_1$-causes the state of the light. Subject to the assumption that the light switch position $D_1$-causes the state of the light, I can reduce my problem to one of choosing from $D_0$ (noting that some choices correspond to mixtures of elements of $D_0$).

If I consider an even larger set of possible acts $D_2$, I might not accept that the switch position $D_2$-causes the state of the light. Let $D_2$ be the following acts:

\begin{align*}
D_2:=&\{``\text{walk to the switch and press it with my thumb}'', \\
    &``\text{trip over the lego on the floor, hop to the light switch and stab my finger at it}'',\\
    &``\text{stay in bed}'',\\
    &``\text{toggle the mains power, then flip the light switch}''\}
\end{align*}

In this case, it would be unreasonable to suppose that all acts that left the light switch in the ``on'' position would also result in the light being ``on''. Thus the switch does not $D_2$-cause the light to be on.

Formally, $D$-causation is defined in terms of conditional independence. Given a see-do model $\kernel{T}:H\times D\to \Delta(\sigalg{X}\otimes\sigalg{Y})$, define the \emph{consequence model} $\kernel{C}:H\times D\to \Delta(\sigalg{Y})$ as $\kernel{C}:=\kernel{T}^{\RV{Y}|\RV{H}\RV{D}}$.

\begin{definition}[$D$-causation]\label{def:d_cause}
Given a hypothesis $h\in H$ and a consequence model $\kernel{C}:H\times D\to \Delta(\mathcal{Y})$, random variables $\RV{Y}_1:Y\times D\to Y_1$, $\RV{Y}_2:Y\times D\to Y_2$ and $\RV{D}:Y\times D\to D$ (defined the usual way), $\RV{Y}_1$ $D$-causes $\RV{Y}_2$ iff $\RV{Y}_2\CI_{\kernel{C}} \RV{D}|\RV{Y}_1\RV{H}$.
\end{definition}

\subsection{D-causation vs Limited Unresponsiveness}

Heckerman and Shachter study deterministic ``consequence models''. Furthermore, what we call hypotheses $h\in H$, Heckerman and Schachter call states $s\in S$. Heckerman and Shachter's notion of causation is defined by \emph{limited unresponsiveness} rather than \emph{conditional independence}, which depends on a partition of states rather than a particular hypothesis.

\begin{definition}[Limited unresponsiveness]
    Given states $S$, deterministic consequence models $\kernel{C}_s:D\to \Delta(F)$ for each $s\in A$ and a random variables $\RV{Y}_1:F\to Y_1$, $\RV{Y}_2:F\to Y_2$, $\RV{Y}_1$ is unresponsive to $\RV{D}$ in states limited by $\RV{Y}_2$ if $\kernel{C}_{(s,d)}^{\RV{Y}_2|\RV{S}\RV{D}}=\kernel{C}_{(s,d')}^{\RV{Y}_2|\RV{S}\RV{D}}\implies \kernel{C}_{(s,d)}^{\RV{Y}_1|\RV{S}\RV{D}}=\kernel{C}_{(s,d')}^{\RV{Y}_1|\RV{S}\RV{D}}$ for all $d,d'\in D$, $s\in S$. Write $\RV{Y}_1\not\hookleftarrow_{\RV{Y}_2} \RV{D}$
\end{definition}

\begin{lemma}[Limited unresponsiveness implies $D$-causation]
For deterministic consequence models, $\RV{Y}_1\not\hookleftarrow_{\RV{Y}_2} \RV{D} $ implies $\RV{Y}_2$ $D$-causes $\RV{Y}_1$.
\end{lemma}

\begin{proof}
By the assumption of determinism, for each $s\in S$ and $d\in D$ there exists $y_1(s,d)$ and $y_2(s,d)$ such that $\kernel{C}^{\RV{Y}_1\RV{Y}_2|\RV{S}\RV{D}}_{s,d} = \delta_{y_1(s,d)}\otimes\delta_{y_2(s,d)}$.

By the assumption of limited unresponsiveness, for all $d,d'$ such that $y_2(s,d)=y_2(s,d')$, $y_1(s,d)=y_1(s,d')$ also. Define $f:Y_2\times S\to Y_1$ by $(s,y_1)\mapsto y(s,[y_1(s,\cdot)]^{-1}(y_1(s,d)))$ where $[y_1(s,\cdot)]^{-1}(a)$ is an arbitrary element of $\{d|y_1(s,d)=a\}$. For all $s,d$, $f(y_1(s,d),s)=y_2(s,d)$. Define $\kernel{M}:Y_2\times S\times D\to \Delta(\mathcal{Y}_1)$ by $(y_2,s,d)\mapsto \delta_{f(y_2,s)}$. $\kernel{M}$ is a version of $\kernel{C}^{\RV{Y}_1|\RV{Y}_2,\RV{S},\RV{D}}$ because, for all $A\in \mathcal{Y}_2$, $B\in \mathcal{Y}_1$, $s\in S$, $d\in D$:

\begin{align}
    \kernel{C}^{\RV{Y}_2|\RV{S}\RV{D}}_{(s,d)}\splitter{0.1}(\kernel{M}\otimes\mathrm{Id}) &= \int_A \kernel{M}(y_2',d,s;B) d\delta_{y_2(s,d)}(y_2') \\
                                                                                        &= \int_A \delta_{f(y_2',s)}(B) d\delta_{y_2(s,d)}(y_2') \\
                                                                                        &= \delta_{f(y_2(s,d),s)}(B)\delta_{y_2(s,d)}(A) \\
                                                                                        &= \delta_{y_1(s,d)}(B)\delta_{y_2(s,d)}(A)\\
                                                                                        &= \delta_{y_2(s,d)}\otimes\delta_{y_1(s,d)}(A\times B)
\end{align}

$\kernel{M}$ is clearly constant in $\RV{D}$. Therefore $\RV{Y}_1\CI_{\kernel{C}}\RV{D}|\RV{Y}_2\RV{S}$.
\end{proof}

However, despite limited unresponsiveness implying $D$-causation, it does not imply $D$-causation in mixtures of states\todo{define this}. Suppose $D=\{0,1\}$ where $1$ stands for ``toggle light switch'' and $0$ stands for ``do nothing''. Suppose $S=\{[0,0],[0,1],[1,0],[1,1]\}$ where $[0,0]$ represents ``switch initially off, mains off'' the other states generalise this in the obvious way. Finally, $\RV{F}\in\{0,1\}$ is the final position of the switch and $\RV{L}\in\{0,1\}$ is the final state of the light. We have

\begin{align}
    \kernel{C}^{\RV{L}\RV{F}|\RV{D}\RV{S}}_{d,[i,m]} = \delta_{(d\text{ XOR }i)\text{ AND }m}\otimes \delta_{(d\text{ XOR }i)\text{ AND }m}
\end{align}

Within states $[0,0]$ and $[1,0]$, the light is always off, so $\RV{F}=a\implies \RV{L}=0$ for any $a$. In states $[0,1]$ and $[1,1]$, $\RV{F}=1\implies \RV{L}=1$ and $\RV{F}=0\implies \RV{L}=0$. Thus $\RV{L}\not\hookleftarrow_{\RV{F}} \RV{D}$. However, suppose we take a mixture of consequence models:
\begin{align}
    \kernel{C}_\gamma &= \frac{1}{4}\kernel{C}_{\cdot,[0,0]} + \frac{1}{4}\kernel{C}_{\cdot,[0,1]} + \frac{1}{2}\kernel{C}_{\cdot,[1,1]}\\
    \kernel{C}^{\RV{F}\RV{L}|\RV{D}}_\gamma &= \frac{1}{4} \left[\begin{matrix}
                        1 & 0\\ 0 & 1
                      \end{matrix}\right]\otimes \left[\begin{matrix}
                        1 & 0\\ 1 & 0
                      \end{matrix}\right] + \frac{1}{4} \left[\begin{matrix}
                        1 & 0\\ 0 & 1
                      \end{matrix}\right]\otimes \left[\begin{matrix}
                        1 & 0\\ 0 & 1
                      \end{matrix}\right] + \frac{1}{2}\left[\begin{matrix}
                        0 & 1\\ 1 & 0
                      \end{matrix}\right]\otimes \left[\begin{matrix}
                        0 & 1\\ 1 & 0
                      \end{matrix}\right]
\end{align}

Then

\begin{align}
    [1,0]\kernel{C}^{\RV{F}\RV{L}|\RV{D}}_{\gamma} &= \frac{1}{4}[0,1]\otimes[1,0]+\frac{1}{4}[0,1]\otimes[0,1]+\frac{1}{2}[1,0]\otimes[1,0]\\
    [1,0]\splitter{0.1}(\kernel{C}^{\RV{F}|\RV{D}}_\gamma\otimes \kernel{C}^{\RV{L}|\RV{D}}_\gamma) &= (\frac{1}{2}[0,1]+\frac{1}{2}[1,0])\otimes(\frac{1}{4}[0,1]+\frac{3}{4}[1,0])\\
    \implies [1,0]\kernel{C}^{\RV{F}\RV{L}|\RV{D}}_{\gamma} &\neq [1,0] \splitter{0.1} (\kernel{C}^{\RV{F}|\RV{D}}_\gamma\otimes \kernel{C}^{\RV{L}|\RV{D}}_\gamma)
\end{align}

Thus under hypothesis mixture $\gamma$, $\RV{F}$ does not $D$-cause\todo{define this} $\RV{L}$ even though $\RV{F}$ $D$-causes $\RV{L}$ in all states $S$. The definition of $D$-causation was motivated by the idea that we could reduce a difficult decision problem with a large set $D$ to a simpler problem with a smaller ``effective'' set of decisions by exploiting conditional independence. Even if $\RV{X}$ $D$-causes $\RV{Y}$ in every $\RV{H}\in S$, $\RV{X}$ does not necessarily $D$-cause $\RV{Y}$ in mixtures of states in $S$. For this reason, we do not say that $\RV{X}$ $D$-causes $\RV{Y}$ in $S$ if $\RV{X}$ $D$-causes $\RV{Y}$ in every $\RV{H}\in S$, and in this way we differ substantially from \citet{heckerman_decision-theoretic_1995}.

Instead, we simply extend the definition of $D$-causation to mixtures of hypotheses: if $\gamma\in \Delta(\RV{H})$ is a mixture of hypotheses, define $\kernel{C}_\gamma:= (\gamma\otimes\textbf{Id})\kernel{C}$. Then $\RV{X}$ $D$-causes $\RV{Y}$ relative to $\gamma$ iff $\RV{Y}\CI_{\kernel{C}_\gamma} \RV{D}|\RV{X}$.

Theorem \ref{th:univ_d_causation} shows that under some conditions, $D$-causation can hold for arbitrary mixtures over subsets of the hypothesis class $\RV{H}$.

\begin{theorem}[Universal $D$-causation]\label{th:univ_d_causation}
If $\RV{X}\CI\RV{H}|\RV{D}$ for all $\RV{H},\RV{H}'\in S\subset \RV{H}$ and $\RV{X}$ $D$-causes $\RV{Y}$ in all $\RV{H}\in S$, then $\RV{X}$ $D$-causes $\RV{Y}$ with respect to all mixed consequence models $\kernel{C}_\gamma$ for all $\gamma\in \Delta(\RV{H})$ with $\gamma(S)=1$.
\end{theorem}

\begin{proof}

For $\gamma\in \Delta(\RV{H})$, define the mixture

\begin{align}
\kernel{C}_\gamma := \begin{tikzpicture}
    \path (0,0) node[dist] (g) {$\gamma$}
    + (0,-0.45) node (D) {$\RV{D}$}
    ++ (1,-0.3) node[kernel] (C) {$\kernel{C}$}
    ++ (1,0) node (F) {$\RV{F}$};
    \draw (g) to [out=0,in=180] ($(C.west) + (0,0.15)$) (D) -- ($(C.west) + (0,-0.15)$) (C) -- (F);
\end{tikzpicture}
\end{align}

Because $\kernel{C}_\RV{H}^{\RV{X}|\RV{D}} = \kernel{C}_{\RV{H}'}^{\RV{X}|\RV{D}}$ for all $\RV{H},\RV{H}'\in \RV{H}$, we have

\begin{align}
\begin{tikzpicture}
    \path (0,0) node[dist] (g) {$\gamma$}
    + (0.7,-0.15) node[copymap] (copy0) {}
    + (0,-0.45) node (D) {$\RV{D}$}
    ++ (1.5,-0.3) node[kernel] (C) {$\kernel{C}^{\RV{X}|\RV{D}\RV{H}}$}
    ++ (1,0) node (X) {$\RV{X}$}
    + (0,0.5) node (T) {$\RV{H}$};
    \draw (g) to [out=0,in=180] (copy0) -- ($(C.west) + (0,0.15)$) (D) -- ($(C.west) + (0,-0.15)$);
    \draw (C) -- (X);
    \draw (copy0) to [out=90,in=180] (T);
\end{tikzpicture} &= \begin{tikzpicture}
    \path (0,0) node[dist] (g) {$\gamma$}
    + (0,0.5) node[dist] (g2) {$\gamma$}
    + (0.7,-0.15) node[copymap] (copy0) {}
    + (0,-0.45) node (D) {$\RV{D}$}
    ++ (1.5,-0.3) node[kernel] (C) {$\kernel{C}^{\RV{X}|\RV{D}\RV{H}}$}
    ++ (1,0) node (X) {$\RV{X}$}
    + (0,0.3) node (T) {$\RV{H}$};
    \draw (g) to [out=0,in=180] (copy0) -- ($(C.west) + (0,0.15)$) (D) -- ($(C.west) + (0,-0.15)$);
    \draw (C) -- (X);
    \draw (g2) to [out=0,in=180] (T);
\end{tikzpicture} \label{eq:decompose_condi_x}
\end{align}

Also

\begin{align}
    \kernel{C}_\gamma^{\RV{XY}|\RV{D}} &= \begin{tikzpicture}
    \path (0,0) node[dist] (g) {$\gamma$}
    + (0,-0.45) node (D) {$\RV{D}$}
    ++ (1,-0.3) node[kernel] (C) {$\kernel{C}$}
    ++ (1,0) node[kernel] (F) {$\kernel{F}^{\RV{X}\utimes\RV{Y}}$}
    ++ (1,0.15) node (X) {$\RV{X}$}
    + (0,-0.3) node (Y) {$\RV{Y}$};
    \draw (g) to [out=0,in=180] ($(C.west) + (0,0.15)$) (D) -- ($(C.west) + (0,-0.15)$) (C) -- (F);
    \draw ($(F.east) + (0,0.15)$) -- (X) ($(F.east) + (0,-0.15)$) -- (Y);
\end{tikzpicture}\\
    &= \begin{tikzpicture}
    \path (0,0) node[dist] (g) {$\gamma$}
    + (0,-0.45) node (D) {$\RV{D}$}
    ++ (1,-0.3) node[kernel] (C) {$\kernel{C}^{\RV{XY}|\RV{D}\RV{H}}$}
    ++ (1,0.15) node (X) {$\RV{X}$}
    + (0,-0.3) node (Y) {$\RV{Y}$};
    \draw (g) to [out=0,in=180] ($(C.west) + (0,0.15)$) (D) -- ($(C.west) + (0,-0.15)$);
    \draw ($(C.east) + (0,0.15)$) -- (X) ($(C.east) + (0,-0.15)$) -- (Y);
\end{tikzpicture}\\
 &= \begin{tikzpicture}
    \path (0,0) node[dist] (g) {$\gamma$}
    + (0,-0.45) node (D) {$\RV{D}$}
    + (0.7,-0.45) node[copymap] (copy0) {}
    + (0.7,-0.15) node[copymap] (copy1) {}
    ++ (1.4,-0.3) node[kernel] (C) {$\kernel{C}^{\RV{X}|\RV{D}\RV{H}}$}
    + (0,0.6) coordinate (via0)
    + (0,-0.6) coordinate (via1)
    ++ (0.9,0) node[copymap] (copy2) {}
    ++ (0.7,0) node[kernel] (Yx) {$\kernel{C}^{\RV{Y}|\RV{X}\RV{D}\RV{H}}$}
    ++ (1.2,0.15) node (X) {$\RV{Y}$}
    + (0,-0.5) node (Y) {$\RV{X}$};
    \draw (g) to [out=0,in=180] (copy1) -- ($(C.west) + (0,0.15)$) (D) -- ($(C.west) + (0,-0.15)$) (C)--(Yx);
    \draw (copy0) to [out=-90,in=180] (via1) to [out=0,in=180] ($(Yx.west) + (0,-0.15)$) (copy1) to [out=90,in=180] (via0) to [out=0,in=180] ($(Yx.west) + (0,0.15)$);
    \draw ($(Yx.east) + (0,0.15)$) -- (X) (copy2) to [out=-90,in=180] (Y);
 \end{tikzpicture}\\
 &\overset{\RV{Y}\CI \RV{D}|\RV{X}\RV{H}}{=} \begin{tikzpicture}
    \path (0,0) node[dist] (g) {$\gamma$}
    + (0,-0.45) node (D) {$\RV{D}$}
    + (0.7,-0.15) node[copymap] (copy1) {}
    ++ (1.4,-0.3) node[kernel] (C) {$\kernel{C}^{\RV{X}|\RV{D}\RV{H}}$}
    ++ (0.9,0.1) node[copymap] (copy2) {}
    ++ (0.7,0.3) node[kernel] (Yx) {$\kernel{C}^{\RV{Y}|\RV{X}\RV{H}}$}
    ++ (1.2,0.15) node (X) {$\RV{Y}$}
    + (0,-0.5) node (Y) {$\RV{X}$};
    \draw (g) to [out=0,in=180] (copy1) -- ($(C.west) + (0,0.15)$) (D) -- ($(C.west) + (0,-0.15)$) (C) to [out=0,in=180] (copy2) to [out=0,in=180] (Yx);
    \draw (copy1) to [out=90,in=180] ($(Yx.west) + (0,0.15)$);
    \draw ($(Yx.east) + (0,0.15)$) -- (X) (copy2) to [out=-90,in=180] (Y);
 \end{tikzpicture} \\
 &\overset{\ref{eq:decompose_condi_x}}{=} \begin{tikzpicture}
    \path (0,0) node[dist] (g) {$\gamma$}
    + (0,-0.45) node (D) {$\RV{D}$}
    + (0.7,-0.15) node[copymap] (copy1) {}
    ++ (1.4,-0.3) node[kernel] (C) {$\kernel{C}^{\RV{X}|\RV{D}\RV{H}}$}
    + (1,0.6) node[dist] (g2) {$\gamma$}
    ++ (0.9,0.1) node[copymap] (copy2) {}
    ++ (1,0.3) node[kernel] (Yx) {$\kernel{C}^{\RV{Y}|\RV{X}\RV{H}}$}
    ++ (1.2,0.15) node (X) {$\RV{Y}$}
    + (0,-0.5) node (Y) {$\RV{X}$};
    \draw (g) to [out=0,in=180] (copy1) -- ($(C.west) + (0,0.15)$) (D) -- ($(C.west) + (0,-0.15)$) (C) to [out=0,in=180] (copy2) to [out=0,in=180] (Yx);
    \draw (g2) to [out=0,in=180] ($(Yx.west) + (0,0.15)$);
    \draw ($(Yx.east) + (0,0.15)$) -- (X) (copy2) to [out=-90,in=180] (Y);
 \end{tikzpicture}\\
 &= \overset{\ref{eq:decompose_condi_x}}{=} \begin{tikzpicture}
    \path (0,0) node (g) {}
    + (0,-0.45) node (D) {$\RV{D}$}
    + (0.7,-0.45) node[copymap] (copy1) {}
    ++ (1.4,-0.3) node[kernel] (C) {$\kernel{C}_\gamma^{\RV{X}|\RV{D}\RV{H}}$}
    + (1,0.6) node[dist] (g2) {$\gamma$}
    ++ (0.9,0.1) node[copymap] (copy2) {}
    ++ (1,0.3) node[kernel] (Yx) {$\kernel{C}^{\RV{Y}|\RV{X}\RV{H}}$}
    + (-0.5,0.6) coordinate (stop0)
    ++ (1.2,0.15) node (X) {$\RV{Y}$}
    + (0,-0.5) node (Y) {$\RV{X}$};
    \draw (D) -- ($(C.west) + (0,-0.15)$) (C) to [out=0,in=180] (copy2) to [out=0,in=180] (Yx);
    \draw (g2) to [out=0,in=180] ($(Yx.west) + (0,0.15)$);
    \draw ($(Yx.east) + (0,0.15)$) -- (X) (copy2) to [out=-90,in=180] (Y);
    \draw[-{Rays[n=8]}] (copy1) to [out=90,in=180] (stop0);
 \end{tikzpicture}\label{eq:is_conditional}
\end{align}
Equation \ref{eq:is_conditional} establishes that $(\gamma\otimes\textbf{Id}_X\otimes\stopper{0.3}_D)\kernel{C}^{\RV{Y}|\RV{X}\RV{H}}$ is a version of $\kernel{C}_\gamma^{\RV{Y}|\RV{X}\RV{D}}$, and thus $\RV{Y}\CI_{\kernel{C}_\gamma} \RV{D}|\RV{X}$.

This can also be derived from the semi-graphoid rules:

\begin{align}
    \RV{H}\CI \RV{D} \land \RV{H}\CI \RV{X} | \RV{D} &\implies \RV{H}\CI \RV{XD}\\
    &\implies \RV{H}\CI \RV{D}|\RV{X}\\
    \RV{D} \CI \RV{H}|\RV{X} \land \RV{D}\CI \RV{Y}|\RV{X}\RV{H} &\implies \RV{D}\CI \RV{Y}|\RV{X}\\
    &\implies \RV{Y}\CI\RV{D}|\RV{X}
\end{align}
\end{proof}

\subsection{Properties of D-causation}

If $\RV{X}$ D-causes $\RV{Y}$ relative to $\kernel{C}_\RV{H}$, then the following holds:

\begin{align}
    \kernel{C}_{\RV{H}}^{\RV{X}|\RV{D}} &= \begin{tikzpicture}
    \path (0,0) node (D) {$\RV{D}$}
    ++ (0.9,0) node[kernel] (Xd) {$\kernel{C}^{\RV{X}|\RV{D}}$}
    ++ (1.3,0) node[kernel] (Yd) {$\kernel{C}^{\RV{Y}|\RV{X}}$}
    ++ (0.9,0) node (Y) {$\RV{Y}$};
    \draw (D) -- (Xd) -- (Yd) -- (Y); 
    \end{tikzpicture}
\end{align}

This follows from version (2) of Definition \ref{def:conditional_independence}:

\begin{align}
    \kernel{C}_\RV{H}^{\RV{X}|\RV{D}} &= \begin{tikzpicture}
    \path (0,0) node (D) {$\RV{D}$}
    ++ (0.7,0) node[copymap] (copy0) {}
    ++ (0.7,0) node[kernel] (Xd) {$\kernel{C}^{\RV{X}|\RV{D}}$}
    + (0,0.5) coordinate (via1)
    ++ (1.3,0) node[kernel] (Yd) {$\kernel{C}^{\RV{Y}|\RV{X}\RV{D}}$}
    ++ (0.9,0) node (Y) {$\RV{Y}$};
    \draw (D) -- (Xd) -- (Yd) -- (Y);
    \draw (copy0) to [out=90,in=180] (via1) to [out=0,in=180] ($(Yd.west)+(0,0.15)$); 
    \end{tikzpicture}\\
     &= \begin{tikzpicture}
    \path (0,0) node (D) {$\RV{D}$}
    ++ (0.7,0) node[copymap] (copy0) {}
    ++ (0.7,0) node[kernel] (Xd) {$\kernel{C}^{\RV{X}|\RV{D}}$}
    + (1.3,0.5) coordinate (via1)
    ++ (1.3,0) node[kernel] (Yd) {$\kernel{C}^{\RV{Y}|\RV{X}}$}
    ++ (0.9,0) node (Y) {$\RV{Y}$};
    \draw (D) -- (Xd) -- (Yd) -- (Y);
    \draw[-{Rays[n=8]}] (copy0) to [out=90,in=180] (via1); 
    \end{tikzpicture}\\
    &= \begin{tikzpicture}
    \path (0,0) node (D) {$\RV{D}$}
    ++ (0.9,0) node[kernel] (Xd) {$\kernel{C}^{\RV{X}|\RV{D}}$}
    ++ (1.3,0) node[kernel] (Yd) {$\kernel{C}^{\RV{Y}|\RV{X}}$}
    ++ (0.9,0) node (Y) {$\RV{Y}$};
    \draw (D) -- (Xd) -- (Yd) -- (Y); 
    \end{tikzpicture}
\end{align}

D-causation is not transitive: if $\RV{X}$ D-causes $\RV{Y}$ and $\RV{Y}$ D-causes $\RV{Z}$ then $\RV{X}$ doesn't necessarily D-cause $\RV{Z}$.

\todo[inline]{Pearl's ``front door adjustment'' and general identification results make use of composing ``sub-consequence-kernels'' like this. Show, if possible, that Pearl's ``sub-consequence-kernels'' obey $D$-causation like relations}

\todo[inline]{Does this ``weak D-causation'' respect mixing under the same conditions as regular D-causation?}

\subsection{Decision sequences and parallel decisions}

Just as observations $\RV{X}$ can be a sequence of random variables $\RV{X}_1$, $\RV{X}_2$, ..., $\RV{D}$ can be a sequence of ``sub-choices'' $\RV{D}_1$, $\RV{D}_2$, ... . Note that by positing such a sequence there is not requirement that $\RV{D}_1$ comes ``before'' $\RV{D}_2$ or even that they have any ``before'' and ``after'' relations at all.

\todo[inline]{Define parallel decisions, show that they induce potential outcomes}

\subsection{Residual dependence on observations}

\begin{definition}[Hypothesis sufficiency]\label{def:hypothesis_sufficiency}
The hypothesis $\RV{H}$ is \emph{sufficient} for a see-do model if the consequence model has no dependence on observations $\RV{X}$ conditional on $\RV{H}$. That is, $\RV{Y}\CI_\kernel{T} \RV{X}|\RV{D}\RV{H}$. 

A hypothesis sufficient see-do model can be specified with:

\begin{itemize}
    \item Hypothesis space $\RV{H}$, choices $D$, observations $X$ and consequences $Y$
    \item Observation map $\kernel{T}^{\RV{X}|\RV{H}}$
    \item Reduced consequence model $\kernel{T}^{\RV{Y}|\RV{H}\RV{D}}$
\end{itemize}
\end{definition}

Given observations $\RV{X}$, assumed to be an IID sequence $\RV{X}_1,\RV{X}_2,...$ conditional on $\RV{H}$, a common ``causal inference problem'' is to estimate the ``true'' distribution of observations $\kernel{T}^{\RV{X}_i|\RV{H}}_{h^*}$ and from this to estimate the consequence model $\kernel{T}^{\RV{Y}|\RV{H}\RV{D}}_{h^*\cdot}$, if this is possible. This problem only makes sense if hypothesis sufficiency is assumed -- once $h^*$ is given, the consequence model of interest has no further dependence on $\RV{X}$. We show that all decision problems can be modeled by a hypothesis sufficient see-do model.


\subsubsection{Examples of hypothesis sufficient and insufficient see-do models}

Recall the previous example: suppose we are betting on the outcome of the flip of a possibly biased coin with payout 1 for a correct guess and 0 for an incorrect guess, and we are given $N$ previous flips of the coin to inspect. This situation can be modeled by a hypothesis sufficient see-do model. Define $\kernel{B}:(0,1)\to \Delta(\{0,1\})$ by $\kernel{B}:\RV{H}\mapsto \mathrm{Bernoulli}(\RV{H})$. Then define $\prescript{1}{}{\kernel{T}}$ by:

\begin{itemize}
    \item $D=\{0,1\}$
    \item $X=\{0,1\}^N$
    \item $Y=\{0,1\}$
    \item $H=(0,1)$
    \item $\prescript{1}{}{\kernel{T}}^{\RV{X}|\RV{H}}:\splitter{0.1}^N\kernel{B}$
    \item $\prescript{1}{}{\kernel{T}}^{\RV{Y}|\RV{D}\RV{H}}:(h,d)\mapsto \mathrm{Bernoulli}(1-|d-h|)$
\end{itemize}

In this model, the chance $\RV{H}$ of the coin landing on heads is as much as we can hope to know about how our bet will work out.

Suppose instead that in addition to the $N$ prior flips, we manage to look at the outcome of the flip on which we will bet. In this case, the situation can be modeled by the following hypothesis insufficient see-do model $\prescript{2}{}{\kernel{T}}$:

\begin{itemize}
    \item $D=\{0,1\}$
    \item $X=\{0,1\}^{N+1}$
    \item $Y=\{0,1\}$
    \item $H=(0,1)$
    \item $\prescript{2}{}{\kernel{T}}^{\RV{X}|\RV{H}}:\splitter{0.1}^{N+1}\kernel{B}$
    \item $\prescript{2}{}{\kernel{T}}^{\RV{Y}|\RV{X}\RV{H}\RV{D}}:(h,\mathbf{x},d)\mapsto \delta_{1-|d-x_{N+1}|}$
\end{itemize}

In this case, even if we are told the value of $\RV{H}$, we still benefit from using the observed data when making our decision.

It appears that it might be possible to model the second situation with a hypothesis sufficient model by including the result of the $N+1$th flip in the hypothesis. Define the new hypothesis space $H'=(0,1)\times\{0,1\}$ and define $\prescript{3}{}{\kernel{T}}$ by:

\begin{itemize}
    \item $D=\{0,1\}$
    \item $X=\{0,1\}^{N+1}$
    \item $Y=\{0,1\}$
    \item $H'=(0,1)\times\{0,1\}$
    \item $\prescript{3}{}{\kernel{T}}^{\RV{X}|\RV{H}'}:(\splitter{0.1}^N\kernel{B}\otimes \delta_{x_{N+1}}$
    \item $\prescript{3}{}{\kernel{T}}^{\RV{Y}|\RV{H}'\RV{D}}:(h,x_{N+1},d)\mapsto \delta_{1-|d-x_{N+1}|}$
\end{itemize}

However, $\RV{X}_{N+1}$ is related to the previous flips $\vecRV{X}_{<N}$ and $\prescript{3}{}{\kernel{T}}$ ignores this fact.  In particular, given any $\RV{H}'=(h,\_)$, $\RV{X}_{N+1}$ as well as $\RV{X}_{i}$, $i\leq N$ should all distributed according to Bernoulli($h$). Thus $\prescript{2}{}{\kernel{T}}$ is preferable to $\prescript{3}{}{\kernel{T}}$ because it represents more of the knowledge we have about the problem.

If a see-do model is employed in a \emph{decision problem} -- defined in the next section -- there is an alternative way to avoid hypothesis insufficiency that does not require throwing out some of the model structure.

\todo[inline]{The importance of this is that counterfactual questions are usually \emph{not} decision problems and so they do not have the possibility of avoiding insufficiency available; also \emph{in practice} counterfactual problems are usually hypothesis insufficient while decision problems are usually not.}

\subsection{Causal questions and decision functions}

\citet{pearl_book_2018} has proposed three types of causal question:
\begin{enumerate}
    \item Association: How are $\RV{W}$ and $\RV{Z}$ related? How would observing $\RV{W}$ change my beliefs about $\RV{Z}$?
    \item Intervention: What would happen if I do ... ? How can I make ... happen?
    \item Counterfactual: What if I had done ... instead of what I actually did?
\end{enumerate}

\emph{Causal decision problems} are, roughly speaking, ``interventional'' problems. In English, a causal decision problem roughly asks

\begin{quote}
    Given that I have data $\RV{X}$ and I know which values of $\RV{Y}$ I would like to see and some knowledge about how the world works, which of my available choices $D$ should I select?
\end{quote}

This type of question presupposes somewhat more than Pearl's prototypical interventional questions. First, it supposes that we have \emph{preferences} over the values that $\RV{Y}$ might take, which we need not have to answer the question ``What would happen if I do ...?''. Secondly, and crucially to our theory, causal decision problem suppose that we are given data and a set of choices. 

We will return to the question of preferences. For now, we will focus on the idea that a causal decision problem is about selecting a choice given data. That is, however the selection is made, the answer to a causal decision problem is always a \emph{decision function} $\kernel{D}:X\to \Delta(\sigalg{D})$.

\subsubsection{Avoiding insufficiency with decision functions}

\todo[inline]{Show that a decision problem with a hypothesis insufficient model induces an equivalent decision problem with a hypothesis sufficient model with an expanded set of choices, subject to some conditions.}

\subsubsection{Decision rules}

See-do models encode the relationship between observed data and consequences of decisions. In order to actually make decisions, we also require preferences over consequences. We suppose that a \emph{utility function} is given, and evaluate the desirability of consequences using \emph{expected utility}. A see-do model along with a utility allows us to evaluate the desirability of \emph{decisions rules} according to each hypothesis.

\begin{definition}[Utility function]
Given a See-Do Model $\kernel{T}:\RV{H}\times D\to \Delta(\sigalg{X}\otimes\sigalg{Y})$, a \emph{utility function} $u$ is a measurable function $Y\to \mathbb{R}$. 
\end{definition}

\begin{definition}[Expected utility]
Given a utility function $u:Y\to \mathbb{R}$ and probability measures $\mu,\nu\in \Delta(\sigalg{Y})$, the \emph{expected utility} of $\mu$ is $\mathbb{E}_{\mu}[u]$.

$\mu$ is \emph{preferred} to $\nu$ if $\mathbb{E}_{\mu}[u]\geq \mathbb{E}_{\nu}[u]$, and \emph{strictly preferred} if $\mathbb{E}_{\mu}[u]>\mathbb{E}_{\nu}[u]$.
\end{definition}

\begin{definition}[Decision rule]
Given a see-to map $\kernel{T}:\RV{H}\times D\to \Delta(\sigalg{X}\otimes\sigalg{Y})$, a \emph{decision rule} is a Markov kernel $X\to \Delta(\sigalg{D})$. A \emph{deterministic decision rule} is a decision rule that is deterministic.

\todo[inline]{Define deterministic Markov kernels}
\end{definition}

Expected utility together with a decision rule gives rise to the definition of \emph{risk}, which connects CSDT to classical statistical decision theory (SDT). For historical reasons, risks are minimised while utilities are maximised.

\begin{definition}[Risk]
Given a see-to map $\kernel{T}:\RV{H}\times D\to \Delta(\sigalg{X}\otimes\sigalg{Y})$, a utility $u:Y\to \mathbb{R}$ and the set of decision rules $\mathscr{U}$, the \emph{risk} is a function $l:\RV{H}\times \mathscr{U}\to \mathbb{R}$ given by

\begin{align}
    R(\RV{H},\kernel{U}) := - \int_X  \kernel{U}_x \kernel{T}^{\RV{Y}|\RV{D}\RV{X}\RV{H}}_{\cdot,x,\RV{H}} u d\kernel{T}^{\RV{X}|\RV{H}}_\RV{H}(x)
\end{align}

for $\RV{H}\in \RV{H}$, $\kernel{U}\in \mathscr{U}$. Here $\kernel{U}_x \kernel{T}^{\RV{Y}|\RV{D}\RV{X}\RV{H}}_{\cdot,x,\RV{H}} u$ is the product of the measure $\kernel{U}_x$, the kernel $\kernel{T}^{\RV{Y}|\RV{D}\RV{X}\RV{H}}_{\cdot,x,\RV{H}}:D\to \Delta(\sigalg{Y})$ and the function $u$.
\end{definition}

The loss induces a partial order on decision rules. If for all $\RV{H}$, $l(\RV{H},\kernel{U})\leq l(\RV{H},\kernel{U}')$ then $\kernel{U}$ is at least as good as $\kernel{U}'$. If, furthermore, there is some $\RV{H}_0$ such that $l(\RV{H}_0,\kernel{U})<l(\RV{H}_0,\kernel{U}')$ then $\kernel{U}$ is preferred to $\kernel{U}'$.

\begin{definition}[Induced statistical decision problem]
A see-do model $\kernel{T}:\RV{H}\times D\to \Delta(\sigalg{X}\otimes\sigalg{Y})$ along with a utility $u$ induces the \emph{statistical decision problem} $(\RV{H},\mathscr{U},R)$ with states $\RV{H}$, decisions $\mathscr{U}$ and risks $R$.

\todo[inline]{Statistical decision problems usually define the risk via the loss, but it is only possible to define a loss with a hypothesis sufficient model. We don't actually need a loss, though: the complete class theorem still holds via the induced risk and Bayes risk}

\end{definition}


\section{Existence of counterfactuals}

\todo[inline]{I'm struggling with how to explain this well.}

``Counterfactual'' or ``potential outcomes'' models in the causal inference literature are consequence models where choices can be considered in \emph{parallel}. 

Before defining parallel choices, we will consider a ``counterfactual model'' without parallel choices. Consider the following definitions, first from \citet{pearl_causality:_2009} pg. 203-204. I have preserved his notation, including not using any special fonts for things called ``variables'' because this term is used interchangeably with ``sets of variables'' and using special fonts for variables might give the impression that these should be treated as different things while using special fonts for sets of variables is inconsistent with my usual notation.

\todo[inline]{The real solution here is that Pearl's ``variable sets'' are actually ``coupled variables'', see Definition \ref{def:ctensor}, but I'd rather not change his definitions if I can avoid it}

\todo[inline]{put the following inside a quote environment somehow, the regular quote environment fails due to too much markup}
\vspace{1cm}

```
\paragraph{Definition 7.1.1 (Causal Model)}
A causal model is a triple
$M = \langle U, V, F\rangle$,
where:
\begin{enumerate}[label=(\roman*)]
    \item $U$ is a set of \emph{background} variables, (also called \emph{exogenous}), that are determined by factors outside the model;
    \item $V$ is a set $\{V_1 , V_2 ,..., V_n\}$ of variables, called \emph{endogenous}, that are determined by variables in the model -- that is, variables in $U\cup V$;
    \item $F$ is a set of functions $\{f_1 , f_2 ,..., f_n\}$ such that each $f_i$ is a mapping from (the respective domains of) $U_i \cup PA_i$ to $V_i$, where $U i \subseteq U$ and $PA_i \subseteq V \setminus V_i$ and the entire set $F$ forms a mapping from $U$ to $V$. In other words, each $f_i$ in $$v_i = f_i (pa_i , u_i ),\qquad  i\in 1, ... n,$$ assigns a value to $V_i$ that depends on (the values of) a select set of variables in $V \cup U$, and the entire set $F$ has a unique solution $V(u)$.
\end{enumerate}

\paragraph{Definition 7.1.2 (Submodel)}
Let $M$ be a causal model, $X$ a set of variables in $V$, and $x$ a particular realization of $X$. A submodel $M_x$ of $M$ is the causal model $$M_x =\{U, V, F_x\},$$ where $$F_x = \{ f_i : V_i \notin X\}\cup\{ X = x\}.$$

\paragraph{Definition 7.1.3 (Effect of Action)}
Let $M$ be a causal model, $X$ a set of variables in $V$, and $x$ a particular realization of $X$. The effect of action $do(X=x)$ on $M$ is given by the submodel $M_x$

\paragraph{Definition 7.1.4 (Potential Response)}
Let $X$ and $Y$ be two subsets of variables in $V$. The potential response of $Y$ to action $do(X = x)$, denoted $Y_x(u)$, is the solution for $Y$ of the set of equations $F_x$, that is, $Y_x(u) = Y_{M_x}(u)$.

\paragraph{Definition 7.1.6 (Probabilistic Causal Model)}
A probabilistic causal model is a pair $\langle M, P(u)\rangle$, where $M$ is a causal model and $P(u)$ is a probability function defined over the domain of U.
'''


\vspace{1cm}

Implicitly, Definition 7.1.3 proposes a set of ``actions'' that have ``effects'' given by $M_x$. It's not entirely clear what this set of actions should be -- the definition seems to suggest that there is an action for each ``realization'' of each variable in $V$, which would imply that the set of actions corresponds to the range of $V$. For the following discussion, we will call the set of actions $D$, whatever it actually contains (we have deliberately chosen to use the same letter as we use to represent choices or actions in see-do models).

Given $D$, Definition 7.1.3 appears to define a function $h:\mathscr{M}\times D\to \mathscr{M}$, where $\mathscr{M}$ is the space of causal models with background variables $U$ and endogenous variables $V$, such that for $M\in \mathscr{M}$, $do(X=x)\in D$, $h(M,do(X=x))=M_x$.

Definition 7.1.4 then appears to define a function $Y_\cdot(\cdot):D\times U\to Y$ (distinct from $Y$, which appears to be a function $U\to\text{something}$) and calls $Y_\cdot(\cdot)$ the ``potential response''. We could always consider the variable $\RV{V}:=\utimes_{i\in [n]} \RV{V}_i$ and define the ``total potential response'' $\mathbf{g}:=\RV{V}_\cdot(\cdot)$, which captures the potential responses of any subset of variables in $V$.

From this, we might surmise that in the Pearlean view, it is necessary that a ``counterfactual'' or ``potential response'' model has a probability measure $P$ on background variables $U$, a set of actions $D$ and a \emph{deterministic} potential response function $\mathbf{g}:D\times U\to V$.

Pearl's model also features a second deterministic function $\mathbf{f}:U\to Y$, and $G$ is derived from $F$ via the equation modifications permitted by $D$. It is straightforward to show that an arbitrary function $\mathbf{f}:U\to Y$ can be constructed from Pearl's set of functions $f_i$, and if $D$ may modify the set $F$ arbitrarily, then it appears that $\mathbf{g}$ can in principle be an arbitrary function $D\times U\to Y$ (though many possible choices would be quite unusual).

Pearl's counterfactual model seems to essentially be a deterministic map $\mathbf{g}:D\times U\to V$ along with a probability measure $P$ on $U$. Putting these together and marginalising over $U$ (as we might expect we want to do with ``background variables'') simply yields a consequence map $D\to \Delta(\sigalg{V})$, which doesn't seem to have any special counterfactual properties.

In order to pose counterfactual questions, Pearl introduces the idea of holding $U$ fixed:
\\
````
\paragraph{Definition 7.1.5 (Counterfactual)}
Let $X$ and $Y$ be two subsets of variables in $V$. The counterfactual sentence ``$Y$ would be $y$ (in situation $u$), had $X$ been $x$'' is interpreted as the equality $Y_x(u) = y$, with $Y_x(u)$
being the potential response of $Y$ to $X = x$.'
'''
\\

Holding $U$ fixed allows SCM counterfactual models to answer questions about what would have happened if we had taken different actions given the same background context. For example, we can compare $Y_x(u)$ with $Y_{x'}(u)$ and interpret the comparison as telling us what would have happened in the same situation $u$ if we did $x$ and, at the same time, what would happen if we did $x'$. It is the ability to consider different actions ``in exactly the same situation'' that makes these models ``counterfactual''.

One obvious question is: does $\mathbf{g}$ have to be deterministic? While SCMs are defined in terms of deterministic functions with noise arguments, it's not clear that this is a necessary feature of counterfactual models. If $\mathbf{g}$ were properly stochastic, what is the problem with considering $\mathbf{g}(x,u)$ and $\mathbf{g}(x',u)$ to represents what would happen in a fixed situation $u$ if I did $x$ and if I did $x'$ respectively? In fact, a nondeterministic $\mathbf{g}$  arguably fails to capture a key intuition of taking actions ``in exactly the same situation''. If I want to know the result of doing action $x$ and, in exactly the same situation, the result of doing action $x$, then one might intuitively think that the result should always be \emph{deterministically the same}. This property, which we call \emph{deterministic reproducibility}, does not hold if we consider a nondeterministic potential response map $\mathbf{g}$.

This idea of doing $x$ and, in the same situation, doing $x$ doesn't render very well in English. Furthermore, even though deterministic reproducibility seems to be an important property of counterfactual SCMs, they don't help very much to elucidate the idea. ``If I take action $x$ in situation $U$ I get $V_x(u)$ and if I take action $x$ in situation $U$ I get $V_x(u)$'' is just a redundant repetition. It seems that we want some way to express the idea of having two copies of $V_x(u)$ or, more generally, having multiple copies of a potential response function in such a way that we can make comparisons between their results.

The idea that we need \emph{can} be clearly expressed with a see-do model. 


\section{Functional Exchangeability, Imitability, Potential Outcomes}

A great deal of ``non-causal statistics'' deals with models of \emph{exchangeable random variables}. Random variables $\RV{X}_1$ and $\RV{X}_2$ both taking values in $X$ are exchangeable if $\prob{P}^{\RV{X}_1\RV{X}_2} = \prob{P}^{\RV{X}_2\RV{X}_1}$; more generally, a set of $n$ random variables is exchangeable if the joint distribution is unchanged by applying any permutation to the order of the variables. Exchangeability implies indifference to the ordering of \emph{observations}. For example:
\begin{itemize}
    \item Suppose the random variables $\RV{X}_1, \RV{X}_2, \RV{X}_3$ represent beliefs about the outcomes of flipping a coin twice. If we are not completely sure the coin is fair, then we might suppose that seeing heads for the first two flips will leave us slightly more confident in a heads than a tails outcome on the next flip, so $\prob{P}^{\RV{X}_3|\RV{X}_2\RV{X}_1}_{\text{HH}}$ is not quite the same as $\prob{P}^{\RV{X}_3}$. However, if $\RV{X}_1,\RV{X}_2$ and $\RV{X}_3$ are exchangeable then it must be the case that $\prob{P}^{\RV{X}_3|\RV{X}_2\RV{X}_1}_{HT}=\prob{P}^{\RV{X}_3|\RV{X}_1\RV{X}_2}_{HT}$; our views about the likelihood of a particular $\RV{X}_3$ after seeing a particular number of heads and a particular number of tails do not depend on the order of the heads and tails.
    \item Suppose $\RV{W}_1$ and $\RV{W}_2$ represent consecutive draws without replacement from an urn containing a finite number of red and white balls in the frequentist sense that given a large ensemble of draws from the urn, the sample fractions will converge to the respective probabilities of $\RV{W}_1$ and $\RV{W}_2$. While the changing number of balls in the urn makes $\RV{W}_2$ depend on $\RV{W}_1$, it can be verified that supposing an equal chance of drawing any particular ball, $\RV{W}_1$ and $\RV{W}_2$ are exchangeable, i.e. $\prob{P}^{\RV{W}_1\RV{W}_2}=\prob{P}^{\RV{W}_2\RV{W}_1}$. This means that the probability of drawing a certain number of reds and whites does not depend on the order in which the balls are drawn.
\end{itemize}

\emph{Functional exchangeability} is a causal version of exchangeability that applies to consequence maps. Where exchangeability implies indifference to the ordering of observations, functional exchangeability implies indifference to the ordering of \emph{decision-consequence pairs}. For example:

\begin{itemize}
    \item Suppose you are planning to conduct a sequence of experiments in which you will set a piece of equipment to a particular setting, represented by $\RV{D}_1,\RV{D}_2,\RV{D}_3$, and record the results. The random variables $\RV{Y}_1,\RV{Y}_2,\RV{Y}_3$ represent our beliefs about the results of the first, second and third repeats of the experiment prior to undertaking it. This experiment is functionally exchangeable if it is described by a consequence map $\kernel{C}:D_0^3\to \Delta(\sigalg{Y}^3)$ such that $\kernel{C}^{\RV{Y}_1\RV{Y}_2\RV{Y}_3|\RV{D}_1\RV{D}_2\RV{D}_3}=\kernel{C}^{\RV{Y}_2\RV{Y}_3\RV{Y}_1|\RV{D}_2\RV{D}_3\RV{D}_1}$ and likewise for other joint permutations of the decisions and consequences. As in the coin flipping example, this means that we ``learn the same thing'' from choosing settings $a$ and observing $p$ followed by choosing $b$ and observing $q$ as we would from choosing $b$ and observing $q$ followed by choosing $a$ and observing $p$
    \item Suppose $\RV{W}_1$ and $\RV{W}_2$ represent (in the frequentist sense) an ensemble of draws from one of two finite urns of red and white balls, with the urn choices given by $\RV{D}_1$, $\RV{D}_2$. Then $\RV{W}_1, \RV{W}_2$ and $\RV{D}_1, \RV{D}_2$ are conditionally exchangeable in the sense that $\prob{P}^{\RV{W}_1\RV{W}_2|\RV{D}_1\RV{D}_2}=\prob{P}^{\RV{W}_2\RV{W}_1|\RV{D}_2\RV{D}_1}$
\end{itemize}

\todo[inline]{Define permutation, set permutation}

\begin{definition}[Exchangeability]
A (possibly infinite) sequence of random variables $\RV{X}_A:=\utimes_{i\in A} \RV{X}_i$ taking values in $X$ on a probability space $(\prob{P},(\Omega,\sigalg{F}))$ is \emph{exchangeable} if for all $n\in A$, all permutations of indices $\sigma:[n]\to[n]$ and all $B\in X^n$ such that $B\times \in\sigalg{F}$, $\prob{P}^{\RV{X}_{[n]}\RV{X}_{A\setminus[n]}}(B\times X^{|A|-n}) =\prob{P}^{\RV{X}_{\sigma([n])}\RV{X}_{A\setminus[n]}}(B\times X^{|A|-n})$.

\todo[inline]{Define infinite coupled copymaps WRT kolmogorov extension theorem}
\end{definition}

De Finetti's representation theorem shows that any infinite exchangeable sequence of random variables can be represented as a mixture of independent and identically distributed sequences. This theorem connects exchangeable \emph{subjective Bayesian random variables} $\RV{X}^{(B)}_n$, which represent expectations of future $X^n$-valued observations, and independent and identically distributed \emph{frequentist random variables} $\RV{X}^{(F)}$, which represent almost sure $n\to\infty$ limits of proportions in $X^n$-valued observations. Conseuqences of the theorem are:

\begin{itemize}
    \item Any exchangeable sequence of random variables $\RV{X}_A$ that can be extended to an infinite sequence can be represented as a mixture $\mu\in\Delta(\theta)$ of coupled statistical models $\utimes_{i\in A} \kernel{O}$ where $\kernel{O}:\Theta\to \Delta(\sigalg{X})$
    \item Any statistical model $\kernel{O}:\Theta\to \Delta(\sigalg{X})$ along with a prior $\mu$ induces a family of exchangeable sequences of random variables $\RV{X}_A$, $A\subset\mathbb{N}$
\end{itemize}

If our task is to predict future $X$-valued observations and our predictions are the same regardless of the order in which the observations arrive, then the probability distribution representing predictions can be represented by a statistical model and a prior. A similar theorem exists connects ``functionally exchangeable'' consequence models and coupled 2-player statistical models. Coupled 2-player statistical models themselves are closely related to models typically used in causal inference.

\begin{theorem}[Representation of infinite exchangeable sequences\citep{hewitt_symmetric_1955}]
Let $(X,\sigalg{X})$ be a compact Hausdorff space with the Baire $\sigma$-algebra and $\prob{P}$ a measure on $X^\mathbb{N}$ with the product sigma algebra such that $\RV{X}_\mathbb{N}$ with each $\RV{X}_i$ given by the projection map $\pi_i$. Define $(\Delta(\sigalg{X}),\sigalg{E})$ to be the set of all probability measures on $X$ with the $\sigma$-algebra $\sigalg{E}$ being the coarsest algebra for which the maps $\mathrm{ev}_A:\Delta(\sigalg{X})\to \mathbb{R}$ given by $\mathrm{ev}_A:\rho\mapsto \rho(A)$ are measurable for all $A\in \sigalg{X}$. Then there exists a unique probability measure $\mu$ on $\Delta(\sigalg{X})$ with the given $\sigma$-algebra such that for all $n\in\mathbb{N}$, $C\in\sigalg{X^n}$:
\begin{align}
    \prob{P}(C\times X^{\mathbb{N}\setminus[n]}) = \int_{(\Delta(\sigalg{X}))} \prod_{i\in [n]} \rho(\RV{X}_i(C)) d\mu(\rho) 
\end{align}

Where $\RV{X}_i(A)$ is the image of $A$ under $\RV{X}_i$.
\end{theorem}

Examples of such spaces include bounded Borel subsets of $\mathbb{R}$ and bounded subsets of $\mathbb{N}$.

\begin{definition}[Functional Exchangeability]
A finite sequence of random variables $\RV{Y}_1,...,\RV{Y}_n:\Omega\to Y$ and a sequence of choice variables\todo{define choice variables; independent of Y conditional on D} $\RV{D}_1,...,\RV{D}_n:D\to D_0$ on kernel space $(\kernel{C},D,\Omega)$ along with is \emph{functionally exchangeable} if for all permutations $\sigma:[n]\to[n]$, $\kernel{C}^{\RV{Y}_1,...,\RV{Y}_n|\RV{D}_1,...,\RV{D}_n}=\kernel{C}^{\RV{Y}_{\sigma(1)},...,\RV{Y}_{\sigma(n)}|\RV{D}_{\sigma(1)},...,\RV{D}_{\sigma(n)}}$.

Graphically, functional exchangeability of $\RV{Y}_1, \RV{Y}_2$ and $\RV{D}_1, \RV{D}_2$ implies

\begin{align}
\begin{tikzpicture} \path (0,0) node (D1) {$\RV{D}_1$}
        + (0,-0.3) node (D2) {$\RV{D}_2$}
        ++ (2,-0.15) node[kernel] (C) {$\kernel{C}^{\RV{Y}_1\RV{Y}_2|\RV{D}_1\RV{D}_2}$}
        ++ (2,0.15) node (Y1) {$\RV{Y}_1$}
        +  (0,-0.3) node (Y2) {$\RV{Y}_2$};
        \draw (D1) -- ($(C.west)+(0,0.15)$) ($(C.east)+(0,0.15)$) -- (Y1);
        \draw (D2) -- ($(C.west)+(0,-0.15)$) ($(C.east)+(0,-0.15)$) -- (Y2);
    \end{tikzpicture} = \begin{tikzpicture} \path (0,0) node (D1) {$\RV{D}_2$}
        + (0,-0.3) node (D2) {$\RV{D}_1$}
        ++ (2.5,-0.15) node[kernel] (C) {$\kernel{C}^{\RV{Y}_1\RV{Y}_2|\RV{D}_1\RV{D}_2}$}
        ++ (2.5,0.15) node (Y1) {$\RV{Y}_2$}
        +  (0,-0.3) node (Y2) {$\RV{Y}_1$};
        \draw (D1) to [out=0, in=180] ($(C.west)+(0,-0.15)$) ($(C.east)+(0,-0.15)$) to [out=0, in=180] (Y1);
        \draw (D2) to [out=0, in=180] ($(C.west)+(0,0.15)$) ($(C.east)+(0,0.15)$) to [out=0, in=180] (Y2);
    \end{tikzpicture}
\end{align}
\todo[inline]{Include a lemma about swap maps and variable permutations}


\end{definition}

\begin{lemma}[Functionally exchangeable sequences with exchangeable choices induce exchangeable sequences]\label{lem:f-ex2ex}
Given functionally exchangeable sequences $\RV{Y}_{[n]}$ and $\RV{D}_{[n]}$ on $(\kernel{C},D_0^n,Y_0^n)$ with product $\sigma$-algebra, along with an exchangable measure $\prob{P}^{\RV{D}_{[n]}}$, define $\prob{P}'$ as follows:

\begin{align}
    \prob{P}' = 
    \begin{tikzpicture}
        \path (0,0) node[dist] (P) {$\prob{P}$}
        ++ (0.5,0) node[copymap] (copy0) {}
        ++ (0.5,0.5) node[kernel] (C) {$\kernel{C}$}
        ++ (1,0) node (Y) {$\RV{Y}_{[n]}$}
        + (0,-1) node (D) {$\RV{D}_{[n]}$};
        \draw (P) -- (copy0) (copy0) to [out=45,in=180] (C) (C) -- (Y);
        \draw (copy0) to [out=-45,in=180] (D);
    \end{tikzpicture}
\end{align}

Then sequence $\utimes_{i\in[n]}(\RV{Y}_i\otimes\RV{D}_i)$ (given by the obvious projection maps) on the probability space $(\prob{P}',Y_0^n\times D_0^n, \sigalg{Y}_0^n\otimes\sigalg{D}_0^n)$ is exchangeable.
\end{lemma}

\begin{proof}
For $i\in [n]$, $A_i\in \sigalg{Y}$, $B_i\in \sigalg{D}$ and arbitrary permutation $\sigma$ we have

\begin{align}
    \prob{P}^{\prime \RV{X}_{1}\RV{D}_1,...,\RV{X}_n\RV{D}_n}(\prod_{i\in[n]} A_i\times B_i) &= \int_{\prod_{i\in[n]}B_i} \prob{P}^{\prime \RV{X}_{[n]}|\RV{D}_{[n]}}_{d_{[n]}}(\prod_{i\in[n]} A_i) d\prob{P}^{\prime \RV{D}_{[n]}}(d_{[n]})\\
                                                                                             &= \int_{\prod_{i\in[n]}B_i} \kernel{C}^{\RV{X}_{[n]}|\RV{D}_{[n]}}_{d_{[n]}}(\prod_{i\in[n]} A_i) d\prob{P}^{\RV{D}_{[n]}}(d_{[n]})\label{eq:swapoutprime}\\
                                                                                             &= \int_{\prod_{i\in[n]}B_i} \kernel{C}^{\RV{X}_{\sigma([n])}|\RV{D}_{\sigma([n])}}_{d_{[n]}}(\prod_{i\in[n]} A_i) d\prob{P}^{\RV{D}_{\sigma([n])}}(d_{[n]})\label{eq:doubleswap}\\
                                                                                             &= \int_{\prod_{i\in[n]}B_i} \kernel{P}^{\prime \RV{X}_{\sigma([n])}|\RV{D}_{\sigma([n])}}_{d_{[n]}}(\prod_{i\in[n]} A_i) d\prob{P}^{\prime \RV{D}_{\sigma([n])}}\label{eq:swapforpprime}\\
                                                                                             &= \prob{P}^{\prime \RV{X}_{\sigma(1)}\RV{D}_{\sigma(1)},...,\RV{X}_{\sigma(n)}\RV{D}_{\sigma(n)}}(\prod_{i\in[n]} A_i\times B_i)
\end{align}

Where line \ref{eq:doubleswap} follows from exchangeability of $\prob{P}$ and functional exchangeability of $\kernel{C}$ and lines \ref{eq:swapoutprime} and \ref{eq:swapforpprime} follow from the fact that for any invertible function $f$ of $\RV{D}_{[n]}$ and random variable $\RV{Y}$, $\kernel{C}^{\RV{Y}|f(\RV{D}_{[n]})}$ is a version of $\prob{P}^{\prime \RV{Y}|f(\RV{D}_{[n]})}$ and $\prob{P}^{f(\RV{D}_{[n]})}=\prob{P}^{\prime \RV{Y}|f(\RV{D}_{[n]})}$.
\end{proof}

\begin{definition}[Non-interfering]
A pair of sequences $\RV{Y}_{[n]}$ and $\RV{D}_{[n]}$ on $(\kernel{C},D,\Omega)$ is \emph{noninterfering} if for all $U\subset [n]$, $\kernel{C}^{\RV{Y}_U|\RV{D}_U}$ exists and $\kernel{C}^{\RV{Y}_{U}|\RV{D}_{[n]}}=\kernel{C}^{\RV{Y}_U|\RV{D}_U}$. Non-interference implies that discard maps can ``fall through'' kernels:

\begin{align}
\begin{tikzpicture} \path (0,0) node (D1) {$\RV{D}_1$}
        + (0,-0.3) node (D2) {$\RV{D}_2$}
        ++ (2,-0.15) node[kernel] (C) {$\kernel{C}^{\RV{Y}_1\RV{Y}_2|\RV{D}_1\RV{D}_2}$}
        ++ (2,0.15) node (Y1) {$\RV{Y}_1$}
        +  (0,-0.3) node (Y2) {};
        \draw (D1) -- ($(C.west)+(0,0.15)$) ($(C.east)+(0,0.15)$) -- (Y1);
        \draw (D2) -- ($(C.west)+(0,-0.15)$);
        \draw[-{Rays[n=8]}] ($(C.east)+(0,-0.15)$) -- (Y2);
    \end{tikzpicture} &= \begin{tikzpicture} \path (0,0) node (D1) {$\RV{D}_1$}
        + (0,-0.3) node (D2) {$\RV{D}_2$}
        ++ (2,-0.15) node[kernel] (C) {$\kernel{C}^{\RV{Y}_1|\RV{D}_1}$}
        ++ (2,0.15) node (Y1) {$\RV{Y}_1$}
        +  (-3,-1) node (Y2) {$\RV{Y}_2$};
        \draw (D1) -- ($(C.west)+(0,0.15)$) ($(C.east)+(0,0.15)$) -- (Y1);
        \draw (D2) -- ($(C.west)+(0,-0.15)$);
        \draw[-{Rays[n=8]}] ($(C.east)+(0,-0.15)$) -- (Y2);
    \end{tikzpicture}
\end{align}

\end{definition}

\todo[inline]{I think it is the case that functionally exchangeable + non-interfering implies infinitely functionally exchangeably extendable, but not proved yet}

\begin{theorem}[Representation of functionally exchangeable sequences]
Let $(Y,\sigalg{Y})$ be a compact Hausdorff space with the Baire $\sigma$-algebra and $(D,\sigalg{D})$ a finite discrete space. Let $\kernel{C}$ be a Markov kernel $D^\mathbb{N}\to \Delta(\sigalg{Y}^\mathbb{N})$ and $\RV{Y}_\mathbb{N}$, $\RV{D}_\mathbb{N}$ a pair of functionally exchangeable sequences. Define $(F,\sigalg{F})$ to be the set of all Markov kernels $D\to \Delta(\sigalg{Y})$ with $\sigalg{F}$ the coarsest $\sigma$-algebra for which all evaluation maps $\mathrm{ev}_{d,A}:F\to \mathbb{R}$ given by $\mathrm{ev}_{d,A}:\kernel{H}\mapsto \kernel{H}_d(A)$ are measurable. Then there exists a unique probability measure $\nu$ on $\Delta(\sigalg{F})$ such that for all $n\in\mathbb{N}$, $d\in D^n$, $C\in \sigalg{Y}^n$:

\begin{align}
    \kernel{C}_d(C) = \int_{F} \prod_{i\in [n]} \kernel{H}_{\RV{D}_i(d)}(\RV{Y}_i(C)) d\nu(\kernel{H})
\end{align}
\end{theorem}

\begin{proof}
Let $\delta\in\Delta(\sigalg{D})$ be such that for all $n\in \mathbb{N}$, $\emptyset\neq B\in \sigalg{D}^n$ we have $\delta\splitter{0.1}_n(B)>0$. Such a measure exists by assumption on $D$. Define $delta_{\underline{n}}:=\delta\splitter{0.1}_n$. It is trivial to show that $\delta_{\underline{n}}$ is exchangeable. Define 
\begin{align}
\kernel{C}^n_{\delta} := \begin{tikzpicture}
        \path (0,0) node[dist] (P) {$\prob{P}$}
        ++ (0.5,0) node[copymap] (copy0) {}
        ++ (0.5,0.5) node[kernel] (C) {$\kernel{C}$}
        ++ (1,0) node (Y) {$\RV{Y}_{[n]}$}
        + (0,-1) node (D) {$\RV{D}_{[n]}$};
        \draw (P) -- (copy0) (copy0) to [out=45,in=180] (C) (C) -- (Y);
        \draw (copy0) to [out=-45,in=180] (D);
\end{tikzpicture}
\end{align}
 $:= \delta_{\underline{n}} \kernel{C}\utimes \mathrm{Id}_{D^n}$



.  By Lemma \ref{lem:f-ex2ex}, there is an exchangeable sequence $\utimes_{i\in[n]} \RV{Y}'_i\utimes \RV{D}'_i$ on the probability space $(\delta_{\underline{n}} \kernel{C}\utimes\mathrm{Id}_D,Y^n\times D^n, \sigalg{Y}^n\otimes\sigalg{D}^n)$.



\end{proof}

\begin{corollary}
Equivalently, for all $n\in \mathbb{N}$, let $\kernel{G}:F\to $

\begin{align}
\begin{tikzpicture}
    \path (0,0) node (D) {$\RV{D}_{[n]}$}
    ++ (2,0.) node[kernel] (C) {$\kernel{C}$}
    ++ (2,0.) node (Y) {$\RV{Y}_{[n]}$};
    \draw (D) -- (C) -- (Y);
\end{tikzpicture} &= \begin{tikzpicture} \path (0,0) node (D1) {$\RV{D}_1$}
        + (0,-0.5) node (dot) {...}
        + (0,-1) node (D2) {$\RV{D}_n$}
        ++ (2,0) node[kernel] (C) {$\kernel{H}^{\RV{Y}_1\RV{Y}_2|\RV{D}_1\RV{D}_2}$}
        ++ (2,0.15) node (Y1) {$\RV{Y}_1$}
        +  (0,-0.5) node (dot2) {...}
        +  (0,-1) node (Y2) {$\RV{Y}_n$};
        \draw (D1) -- ($(C.west)+(0,0.15)$) ($(C.east)+(0,0.15)$) -- (Y1);
        \draw (D2) -- ($(C.west)+(0,-0.15)$) ($(C.east)+(0,-0.15)$) -- (Y2);
    \end{tikzpicture}
\end{align}

\end{corollary}