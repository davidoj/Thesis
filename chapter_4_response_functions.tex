%!TEX root = main.tex

\chapter{Decision problems with independent and identical phenomena}\label{ch:evaluating_decisions}

Chapter \ref{ch:tech_prereq} introduced probability sets as generic tools for causal modelling, while Chapter \ref{ch:2p_statmodels} examined how probability set models can be used to construct mathematical models of decision problems. In general, few assumptions were made about the structure of the models in question. Section \ref{sec:cons_to_sdp} added some structure with \emph{see-do} models, which featured variables representing observations and consequences, and non-stochastic variables representing decisions and hypotheses. Extended conditional independence properties defined the roles of these different variables in the model. However, observations and consequences in see-do models could be just about anything -- they need not take values in the same set, nor be related to one another in any particular way, nor do observations need to form a sequence. Causal inference in practice often concerns the question of making choices in a context where observations and consequences are independent and identical, and the subject of this chapter is to examine probability sets that model decision problems with this kind of repeatability.

Repeatability in classical statistical models is often expressed by the assumption of \emph{exchangeability of observations}. This is the assumption that the measurement procedure produces a sequence of values that are ``all alike'' in the particular sense that any rearrangement of the sequence should be modeled with the same model (although note that exchangeability is only implied by this assumption if observations are modeled with a unique probability model, see \citet[pg. 463]{walley_statistical_1991}). An exchangeable probability distribution over a sequence of variables is a mixture of \emph{independent and identically distributed} distributions. Models with choices generally cannot be represented by a mixture of identically distributed sequences, because different choices will usually mean different actions will be taken and different distributions will result. The appropriate generalisation of independent and identical distributions seems to be independent and identical response functions -- that is, two elements of the sequence will have the same distribution over \emph{outputs} given identical \emph{inputs}. Chapter \ref{ch:other_causal_frameworks} reviews how this is a typical assumption of causal modelling frameworks, and this chapter investigates when models of sequences with choices will features independent and identical response functions.

The key result of this chapter is that a model of choices that results in a sequence of variables is representable as a mixture of independent and identical response functions is equivalent to the assumption of \emph{causal contractibility}. Causal contractibility is defined in Section \ref{sec:ccontracibility} and compared to prior work on similar questions. A representation theorem for causally contractible Markov kernels is proved in Section \ref{sec:rep_theorem}. It is applied to probability sets modelling ``one-shot'' choices with no dependence on prior data in Section \ref{sec:data_independent_actions}, and generalised to ``adaptive'' choices where actions may depend on prior data in Section \ref{sec:data_dependent}; the latter generalisation requires the notion of \emph{combs}, introduced in Section \ref{sec:def_combs}.

Causal contractibility is a more complicated assumption than exchangeability. An alternative to directly arguing for causal contractibility is given by Theorem \ref{th:cc_ind_treat}, which shows that if there are ``experimental identifiers'' that are exchangeable in the appropriate way, and if, given either the input variables or the choice nothing relevant can be gained by learning the other, then causal contractibility holds.  To my knowledge, this is the only example of a theorem of its type that proves the existence of ``causal effects'' in models featuring choices. Somewhat similar theorems exist that address the identification of potential outcomes, but they prove different things (the identifiability of a certain kind of latent variable, which is not identical to the consequences of a choice), and provide different conditions. A particular difference is that the conditions for Theorem \ref{th:cc_ind_treat} concern only \emph{observable} variables, and while randomization of actions is sometimes an admissible condition, it is not a necessary one.

\section{Informal explanation of the main theorems}

Two widely accepted facts about causal inference are:
\begin{enumerate}
    \item Randomized experiments are able to identify ``causal effects'' (more specifically: average treatment effects or interventional probabilities)
    \item The ``causal effects'' identified by randomized experiments represent in some sense the consequences of choices
\end{enumerate}
However the reasons given for this are incomplete. Results about the identifiability of potential outcomes such as those in \citet{rubin_causal_2005,imbens_causal_2015} rely on a tacit understanding of the relevance of potential outcomes to consequences of choices, and the same is true for the ``intervention'' operation in causal graphical models (see, for an example, \citet[Chap. 4]{pearl_book_2018}). Other arguments (considered in more detail in Section \ref{sec:prev_work}) consider the symmetry associated with physically swapping experimental subjects without altering the relevant model, or the idea that the process that gave rise to our observations is in some sense similar to the process that gives rise to the consequences of our choices, which begs the question: \emph{in what sense} are they similar?

A theory of causal inference seemingly ought to be able to explain these basic facts. Introducing them as \emph{postulates} seems to be a mistake -- it is possible, sometimes, to learn about the likely consequences of choices without randomized experimental data, and introducing a postulate depending on randomized experiment seems to demand a rigorous definition of ``randomized experiment''.

Causal contractibility is a notion of ``identifiability of causal effects'' that arises in decision problems that addresses both 1 and 2 together. It entails, roughly speaking, that the conditional probability arising from the long run limit of observations is the same as the function that should be used to evaluate actions under any prospective choice (see Theorem \ref{th:data_ind_CC}).

Theorem \ref{th:cc_ind_treat} provides sufficient (though not a necessary) conditions for causal contractibility. Roughly speaking again, these conditions are: a model $\prob{P}_C^{\RV{Y}|\RV{D}}$ is causally contractible if it is a sequence of experiments that all look alike \emph{and} the sequence of inputs is completely determined by the choice. These conditions can be satisfied with or without randomisation when the experimental design, execution and any subsequent decision making are all jointly controlled (a fact noted previously by \citet{kasy_why_2016}), but not when, for example, the decision making is being handled at a different time to the experimental design or execution. Randomisation can help in these situations because (again, roughly speaking) if you know something I don't then you can take good actions for reasons I can't understand, but no matter what you know a completely random action from you is as good as a completely random action from me, and so random experiments conducted by you are as good as random experiments conducted by me.

Causal contractibility is a very strong assumption, and it is likely that it is often not an acceptable one. Section \ref{sec:weaker_assumptions}


\section{Relevance to previous work}\label{sec:prev_work}

This chapter draws on three different lines of work. The first is the study of representations of symmetric of probability models. The equivalence between infinite exchangeable probability models and mixtures of independent and identically distributed models was shown by \cite{de_finetti_foresight_1992}. This result has been extended in many ways, including to finite sequences \citet{kerns_definettis_2006,diaconis_finite_1980} and for partially exchangeable arrays \citet{aldous_representations_1981}. A comprehensive overview of results is presented in \citet{kallenberg_probabilistic_2005}. Particularly similar to our result is the notion of ``partial exchangeability'' from \citet{diaconis_recent_1988}.

The second line of work is the study of exchangeability-like assumptions in causal models. \citet{lindley_role_1981} discuss models consisting of a sequence of exchangeabile observations along with ``one more observation'', a structure that is similar to the models with observations and consequences discussed in section \ref{sec:weaker_assumptions}. Lindley discusses the application of this model to questions of causation, but does not explore this deeply due to the perceived difficulty of finding a satisfactory definition of causation. \citet{rubin_causal_2005}'s discussion of causal inference with potential outcomes discusses the assumption of exchangeable potential outcomes, and \citet{imbens_causal_2015} uses this assumption to prove several identification results. However, as Lindley observed, this is an assumption of exchangeability of unobservable variables, and it is difficult to see how judgements about the measurement procedures being modeled do or do not imply exchangeability amongst such variables. \citet{saarela_role_2020}, using structural causal models, proposes \emph{conditional exchangeability}, which refers to the invariance of a joint distribution over outcomes under ``surgical switches'' of the values of causal variables of interest. This definition depends on having a structural model, a property not shared by the current work.

Exchangeability in the setting of causal models is often discussed in terms of the exchangeability of \emph{people} (or more generically, \emph{experimental setups}). This is in a sense the inverse of Rubin's position above -- a symmetry of a measurement procedure is given, but the implications regarding the corresponding mathematical model are unclear. \citet{hernan_beyond_2012,greenland_identifiability_1986,banerjee_chapter_2017,dawid_decision-theoretic_2020} all discuss assumptions along these lines.

The other component of causal contractibility is \emph{locality} (Definition \ref{def:caus_cont}). This is also suggested by existing work -- in particular, the stable unit treatment distribution assumption (SUTDA) in \citet{dawid_decision-theoretic_2020}, and the stable unit treatment value assumption (SUTVA) in \citep{rubin_causal_2005}:
\begin{blockquote}
(SUTVA) comprises two sub-assumptions. First, it assumes that \emph{there is no interference between units (Cox 1958)}; that is, neither $Y_i(1)$ nor $Y_i(0)$ is affected by what action any other unit received. Second, it assumes that \emph{there are no hidden versions of treatments}; no matter how unit $i$ received treatment $1$, the outcome that would be observed would be $Y_i(1)$ and similarly for treatment $0$.
\end{blockquote}

Finally, the idea of \emph{combs} in probabilistic models was first proposed by \citet{chiribella_quantum_2008} and an application to causal models was developed by \citet{jacobs_causal_2019}.

\section{Independent and Identical Response Functions}\label{sec:response_functions}

Independent and identical response functions are a precise version of what might be roughly thought of as a ``causal effect''. Generally, the ``causal effect of $\RV{D}$ on $\RV{Y}$'' is a ``fixed but unknown'' probabilistic map $D\kto Y$ that represents the probability of $\RV{Y}$ given some value of $\RV{D}$ \emph{no matter which choice is considered}. This is in line with a common interpretation of interventional probabilities, for example. Formally, this means that a model $\prob{P}_C$ of choices $C$ supports ``causal effects'' with respect to some sequence of pairs $(\RV{X}_i,\RV{Y}_i)_{i\in A}$ if it is a mixture of \emph{uniform independent and identical response functions} for these pairs. Because it is a mixture of response functions, the ``true'' response function is unknown, but if the true function were known then it would be identical for all pairs in the sequence, and the functions are uniform for all choices in the set $C$.

Start with a sequence of variable pairs $(\RV{X}_i,\RV{Y}_i)_{i\in \mathbb{N}}$ where $\RV{X}_i$ is the $i$th ``input'' and $\RV{Y}_i$ is the corresponding ``output'', each taking values in $X$ and $Y$ respectively. $(\RV{X}_i,\RV{Y}_i)_{i\in \mathbb{N}}$ are related by a sequence of ``independent and identical response functions'' (``IIRs'') if there is some variable $\RV{H}$ (generally a latent variable) such that $\prob{P}_C^{\RV{Y}_i|\RV{X}_i\RV{H}}=\prob_C^{\RV{Y}_j|\RV{X}_j\RV{H}}$ for all $i,j$ and for all $i$ $\RV{Y}_i\CI_{\prob{P}_C}^e (\RV{X}_{<i},\RV{Y}_{<i},\RV{C})|(\RV{H},\RV{X}_i)$.

The result of this chapter is that the pairs in a sequence $(\RV{X},\RV{Y}):=(\RV{X}_i,\RV{Y}_i)_{i\in \mathbb{N}}$ modeled by $\prob{P}_C$ are related by IIRs if and only if there is a causally contractible \emph{uniform comb} $\prob{P}_C^{\RV{Y}\combbreak\RV{X}}$. Combs are a generalisation of conditional probabilities and, and if we assume that actions are data-independent ($\RV{X}_i\CI_{\prob{P}_C}^e \RV{Y}_{<i} C|\RV{X}_{<i}$), this reduces to the assumption that the uniform conditional distribution $\prob{P}_C^{\RV{Y}|\RV{X}}$ is causally contractible. Many of the results will focus on the simpler case where we deal with the conditional probability instead of the comb -. Specifically, a representation theorem (Theorem \ref{th:ciid_rep_kernel}) is proven for general Markov kernels in Section \ref{sec:ccontracibility}, and this applied to models $\prob{P}_C$ with data-independent actions in Section \ref{sec:data_independent_actions}. Section \ref{sec:assessing}, discusses questions related to when the assumption of causal contractibility might be held to apply to a particular problem. Combs are introduced in Section \ref{sec:data_dependent}, and Theorem \ref{th:ciid_rep_kernel} is applied to models with data-dependent actions.

\subsection{Causally contractible Markov kernels}\label{sec:ccontracibility}

Causal contractibility is the conjunction of two properties of a Markov kernel $X^{\mathbb{N}}\kto Y^{\mathbb{N}}$: such a Markov kernel is causally contractible if it is \emph{local} and \emph{commutes with exchange}. 

Graphical notation can offer an intuitive picture of these two sub-assumptions. In the simplified case where we have $\kernel{K}:X^2\kto Y^2$, exchange commutativity for two inputs and outputs is given by the following equality:
\begin{align}
    \tikzfig{commutativity_of_exchange}
\end{align}
It expresses the idea that swapping the inputs is equivalent to swapping the outputs. Locality is given by the following pair of equalities:
\begin{align}
    \tikzfig{cons_locality_1}\\
    \tikzfig{cons_locality_2}
\end{align}
and expresses the notion that the outputs are independent of the non-corresponding input, conditional on the corresponding input.

\subsubsection{Definition of causal contractibility}

The general definitions follow.

\begin{definition}[Locality]\label{def:caus_cont}
A Markov kernel $\kernel{K}:X^{\mathbb{N}}\kto Y^{\mathbb{N}}$ is \emph{local} if for all $n\in \mathbb{N}$, $A_i\in \sigalg{Y}$, $(x_{[n]},x_{[n]^C})\in\mathbb{N}$ there exists $\kernel{L}:X^n\kto Y^n$ such that
\begin{align}
    \tikzfig{local_lhs} &= \tikzfig{local_rhs}\\
    &\iff\\
    \kernel{K}(\bigtimes_{i\in [n]} A_i\times Y^{\mathbb{N}}|x_{[n]},x_{[n]^C}) &= \kernel{L}(\bigtimes_{i\in [n]} A_i|x_{[n]})
\end{align}
\end{definition}

\begin{definition}[Exchange commutativity]\label{def:caus_exch}
A Markov kernel $\kernel{K}:X^{\mathbb{N}}\kto Y^{\mathbb{N}}$ \emph{commutes with exchange} if for all finite permutations $\rho:\mathbb{N}\to\mathbb{N}$, $A_i\in \sigalg{Y}$, $(x_{[n]},x_{[n]^C})\in\mathbb{N}$
\begin{align}
    \kernel{K}\mathrm{swap}_{\rho,Y} &=  \mathrm{swap}_{\rho,X} \kernel{K}\\
    &\iff\\
    \kernel{K}(\bigtimes_{i\in\mathbb{N}} A_{\rho(i)}|(x_i)_{i\in {\mathbb{N}}}) &= \kernel{K}(\bigtimes_{i\in\mathbb{N}} A_{i}|(x_{\rho(i)})_{i\in {\mathbb{N}}})
\end{align}
\end{definition}

Causal contractibility is the conjunction of both assumptions.
\begin{definition}[Causal contractibility]
A Markov kernel $\kernel{K}:X^{\mathbb{N}}\kto Y^{\mathbb{N}}$ is \emph{causally contractible} if it is local and commutes with exchange.
\end{definition}

\subsubsection{Properties of causally contractible Markov kernels}

A causally contractible Markov kernel $\kernel{K}:X^{\mathbb{N}}\kto Y^{\mathbb{N}}$ can be mapped to a kernel over a ``contracted'' domain $Y^n$, $\kernel{K}_n:X^{\mathbb{N}}\kto Y^n$, using the appropriate $\text{del}$ map. Theorem \ref{th:equal_of_condits} establishes that any contraction of $\kernel{K}$ to the same domain $Y^n$ yields the same result (see also Theorem \ref{th:equal_of_reduced_condits}). This feature is the motivation for the name \emph{causal contractibility}. 

Theorem \ref{th:no_implication} shows that exchange commutativity and locality are independent assumptions by constructing counterexamples.

Before these theorems are proved, the following definition and Lemma will prove helpful.

All swaps can be written as a product of transpositions, so proving that a property holds for all finite transpositions is enough to show it holds for all finite swaps. It's useful to define a notation for transpositions.
\begin{definition}[Finite transposition]
Given two equally sized sequences $A=(a_i)_{i\in [n]}$, $B=(b_i)_{i\in [n]}$, ${A\leftrightarrow B}:\mathbb{N}\to \mathbb{N}$ is the permutation that sends the $i$th element of $A$ to the $i$th element of $B$ and vise versa. Note that $A\leftrightarrow B$ is its own inverse.
\end{definition}

Lemma \ref{lem:infinitely_extended_kernels} is used to extend finite sequences to infinite ones, and is used in a number of upcoming theorems.

\begin{lemma}[Infinitely extended kernels]\label{lem:infinitely_extended_kernels}
Given a collection of Markov kernels $\kernel{K}_i:X^i\kto Y^i$ for all $i\in \mathbb{N}$, if we have for every $j>i$
\begin{align}
    \kernel{K}_j(\text{id}_{X_i}\otimes \text{del}_{X_{j-i}}) &= \kernel{K}_i\otimes \text{del}_{X_{j-i}}\label{eq:marginalise_comb}
\end{align} 
then there is a unique Markov kernel $\kernel{K}:X^{\mathbb{N}}\kto Y^{\mathbb{N}}$ such that for all $i,j\in \mathbb{N}$,$j>i$
\begin{align}
    \kernel{K}(\text{id}_{X_i}\otimes \text{del}_{X_{j-i}})&= \kernel{K}_i\otimes \text{del}_{X_{j-i}}
\end{align}
\end{lemma}

\begin{proof}
Take any $x\in X^{\mathbb{N}}$ and let $x_{|m}\in X^n$ be the first $n$ elements of $x$. By Equation \ref{eq:marginalise_comb}, for any $A_i\in \sigalg{Y}$, $i\in [m]$
\begin{align}
    \kernel{K}_n(\bigtimes_{i\in [m]}A_i\times Y^{n-m}|x_{|n}) &= \kernel{K}_m(\bigtimes_{i\in [m]}A_i|x_{|m})
\end{align}

Furthermore, by the definition of the $\mathrm{swap}$ map for any permutation $\rho:[n]\to[n]$
\begin{align}
    \kernel{K}_n\mathrm{swap}_{\rho}(\bigtimes_{i\in [m]}A_{\rho(i)}\times Y^{n-m}|x_{|n}) &= \kernel{K}_n(\bigtimes_{i\in [m]}A_{i}\times Y^{n-m}|x_{|n})
\end{align}
thus by the Kolmogorov Extension Theorem \citep{cinlar_probability_2011}, for each $x\in X^{\mathbb{N}}$ there is a unique probability measure $\prob{Q}_x\in \Delta(Y^{\mathbb{N}}$ satisfying
\begin{align}
    \prob{Q}_d(\bigtimes_{i\in [n]}A_i\times Y^{\mathbb{N}}) &= \kernel{K}_n(\bigtimes_{i\in [n]}A_{\rho(i)}|d_{|n})\label{eq:q_is_Markov}
\end{align}

Furthermore, for each $\{A_i\in\sigalg{Y}|i\in \mathbb{N}\}$, $n\in \mathbb{N}$ note that for $p>n$
\begin{align}
\prob{Q}_d(\bigtimes_{i\in[n]} A_i \times Y^{\mathbb{N}})&\geq \prob{Q}_d(\bigtimes_{i\in [p]} A_i\times Y^{\mathbb{N}})\\
&\geq \prob{Q}_d(\bigtimes_{i\in \mathbb{N}} A_i)
\end{align}
so by the Monotone convergence theorem, the sequence $\prob{Q}_d(\bigtimes_{i\in[n]} A_i)$ converges as $n\to \infty$ to $\prob{Q}_d(\bigtimes_{i\in\mathbb{N}} A_i)$. $d\mapsto \prob{Q}_d^{\RV{Z}_n}(\bigtimes_{i\in[n]} A_i)$ is measurable for all $n$, $\{A_i\in\sigalg{Y}|i\in \mathbb{N}\}$ by Equation \ref{eq:q_is_Markov}, and so $d\mapsto Q_d$ is also measurable.

Thus $d\mapsto Q_d$ is the desired $\prob{P}_C^{\RV{Y}_{\mathbb{N}}\combbreak \RV{D}_{\mathbb{N}}}:D^\mathbb{N}\kto Y^\mathbb{N}$.
\end{proof}

Theorem \ref{th:equal_of_condits} shows that, given a causally contractible kernel, the following operations yield equivalent results:
\begin{itemize}
    \item Marginalising all but the first $n$ outputs
    \item Marginalising all outputs except for the positions $A\subset\mathbb{N}$ where $|A|=n$, and swapping the first $n$ inputs with the elements of $A$
\end{itemize}

\begin{definition}[Marginalising kernel]
Given $(X,\sigalg{X})$ and $A\subset\mathbb{N}$, $\mathrm{marg}_A:X^\mathbb{N}\kto X^A$ is the Markov kernel given by
\begin{align}
    \bigotimes_{i\in \mathbb{N}} \text{switch}_{A,i}
\end{align}
where
\begin{align}
    \text{switch}_A &= \begin{cases}
                        \text{id}_X&i\in A\\
                        \text{del}_X&i\not\in A
                        \end{cases}
\end{align} 
\end{definition}

\begin{theorem}[Equality of equally sized contractions]\label{th:equal_of_condits}
A Markov kernel $\kernel{K}:X^{\mathbb{N}}\kto Y^{\mathbb{N}}$ is \emph{causally contractible} if and only if for every $n\in \mathbb{N}$ and every $A\subset\mathbb{N}$ there exists some $\kernel{L}:X^n\kto Y^n$ such that
\begin{align}
    \kernel{K} \text{marg}_A &= \text{swap}_{[n]\leftrightarrow A} \kernel{L}\otimes \text{del}_{X^{\mathbb{N}}}
\end{align}
\end{theorem}

\begin{proof}
Only if:
By exchange commutativity
\begin{align}
    \text{swap}_{[n]\leftrightarrow A} \kernel{K} &= \kernel{K} \text{swap}_{[n]\leftrightarrow A}
\end{align}
multiply both sides by $\text{swap}_{[n]\leftrightarrow A}$ on the right and, because $\text{swap}_{[n]\leftrightarrow A}$ is its own inverse,
\begin{align}
        \text{swap}_{[n]\leftrightarrow A} \kernel{K}\text{swap}_{[n]\leftrightarrow A} &= \kernel{K}
\end{align}
so
\begin{align}
    \kernel{K}\text{marg}_A &= \text{swap}_{[n]\leftrightarrow A} \kernel{K}\text{swap}_{[n]\leftrightarrow A}\text{marg}_A\\
    &= \text{swap}_{[n]\leftrightarrow A} \kernel{K}\text{marg}_{[n]}
\end{align}
By locality, there exists some $\kernel{L}:X^n\kto Y^n$ such that
\begin{align}
    \kernel{K} \text{marg}_{[n]} &= \kernel{K}(\text{id}_{[n]}\otimes \text{del}_{X^{\mathbb{N}}})\\
     &= \kernel{L}\otimes \mathrm{del}_{X^{\mathbb{N}}}
\end{align}
If:
Taking $A=[n]$ for all $n$ establishes locality.

For exchange commutativity, note that for all $x\in X^{\mathbb{N}}$, $n\in\mathbb{N}$, we have
\begin{align}
    \text{swap}_{A\leftrightarrow [n]} \kernel{K} \text{marg}_A &= \text{swap}_{A\leftrightarrow [n]} \kernel{K} \text{swap}_{A\leftrightarrow [n]} (\text{id}_{[n]}\otimes \text{del}_{X^{\mathbb{N}}})\\
     &= \kernel{K} \text{marg}_{[n]}\\
     &= \kernel{K}(\text{id}_{[n]}\otimes \text{del}_{X^{\mathbb{N}}})
\end{align}
Then by Lemma \ref{lem:infinitely_extended_kernels}
\begin{align}
    \text{swap}_{A\leftrightarrow [n]} \kernel{K} \text{swap}_{A\leftrightarrow [n]} &= \kernel{K}
\end{align}
Consider an arbitrary finite permutation $\rho:\mathbb{N}\to \mathbb{N}$. $\rho$ can be decomposed into a finite set of cyclic permutations on disjoint orbits. Each cyclic permutation is simply the composition of some set of transpositions, and so $\rho$ itself can be written as a composition of a sequence of transpositions. Thus for any finite $\rho:\mathbb{N}\to\mathbb{N}$
\begin{align}
    \text{swap}_{\rho} \kernel{K} \text{swap}_{\rho} &= \kernel{K}
\end{align}
\end{proof}

Theorem \ref{th:no_implication} shows that neither locality nor exchange commutativity is implied by the other.

\begin{theorem}\label{th:no_implication}
Exchange commutativity does not imply locality or vise versa.
\end{theorem}

\begin{proof}
First, a Markov kernel that exhibits exchange commutativity but not locality. Suppose $D=Y=\{0,1\}$ and $\kernel{K}:D^2\kto Y^2$ is given by
\begin{align}
    \kernel{K}(y_1,y_2|d_1,d_2) &= \llbracket (y_1,y_2)= (d_1+d_2,d_1+d_2) \rrbracket
\end{align}
then 
\begin{align}
    \kernel{K}(y_1,Y|d_1,d_2) &= \llbracket y_1 = d_1+d_2 \rrbracket
\end{align}
and there is no function depending on $y_1$ and $d_1$ only that is equal to this. Thus $\kernel{K}$ does not satisfy locality. 

However, taking $\rho$ to be the unique nontrivial swap $\{0,1\}\to \{0,1\}$
\begin{align}
    \text{swap}_{\rho,D}\kernel{K}(y_1,y_2|d_1,d_2) &= \kernel{K}(y_1,y_2|d_2,d_1)\\
    &= \llbracket (y_1,y_2)= (d_2+d_1,d_2+d_1) \rrbracket\\
    &= \llbracket (y_1,y_2)= (d_1+d_2,d_1+d_2) \rrbracket\\
    &= \llbracket (y_2,y_1)= (d_1+d_2,d_1+d_2) \rrbracket\\
    &= \kernel{K}\text{swap}_{\rho,Y}(y_1,y_2|d_1,d_2)
\end{align}
so $\kernel{K}$ commutes with exchange.

Next, a Markov kernel that satisfies locality but does not commute with exchange. Suppose again $D=Y=\{0,1\}$ and $\kernel{K}:D^2\kto Y^2$ is given by
\begin{align}
    \kernel{K}(y_1,y_2|d_1,d_2) &= \llbracket (y_1,y_2)= (0,1) \rrbracket
\end{align}

Then:
\begin{align}
    \kernel{K}(y_1|d_1,d_2) &= \llbracket y_1= 0 \rrbracket\\
    &= \kernel{K}(y_1|d_1)\\
    \kernel{K}(y_2|d_1,d_2)&= \llbracket y_2= 1 \rrbracket\\
    &= \kernel{K}(y_2|d_2)
\end{align}
so $\kernel{K}$ satisfies locality.

However, $\kernel{K}$ does not commute with exchange.
\begin{align}
    \text{swap}_{\rho(\RV{D})} \kernel{K}(y_1,y_2|d_1,d_2) &= \kernel{K}(y_1,y_2|d_2,d_1)\\
    &=\llbracket (y_1,y_2)= (0,1) \rrbracket\\
    &\neq \llbracket (y_2,y_1)= (0,1) \rrbracket\\
    &= \kernel{K}\text{swap}_{\rho(\RV{D})}(y_1,y_2|d_1,d_2)
\end{align}
\end{proof}

Theorem \ref{th:no_implication} presents abstract counterexamples to show that the assumptions of exchange commutativity and locality are indpedent. For some more practical examples, a model of the treatment of several patients who have already been examined might satisfy consequence locality but not exchange commutativity. Patient B's treatment could be assumed not to affect patient A, but the same results would not be expected from giving patient A's treatment to patient B as from giving patient A's treatment to patient A. 

A model of economic interventions might satisfy exchange commutativity but not locality. If a government prints money to make exactly $n$ payments of \$10 000 are made to a number of undistinguished recipients, the government cannot say much about the impact of who exactly receives the payment. However, the amount of inflation created by the payments depends on the number of payments made; making 100 such payments will have a negligible effect on inflation, while making payments to everyone in the country will have a substantial effect, and this will in turn affect the outcomes of the people who did or did not receive payment. \citet{dawid_causal_2000} offers the alternative example of herd immunity in vaccination campaigns as a situation where commutativity of exchange holds but locality does not.

Although locality seems to imply a lack of interference between inputs and outputs of different indices, it actually allows for some models with certain kinds of interference between actions and outcomes of different indices. For example: consider an experiment where I first flip a coin and record the results of this flip as the outcome of the ``step 1''. Subsequently, I can choose either to copy the outcome from step 1 to be the input for ``step 2'' (this is the choice $\RV{D}_1=0$), or flip a second coin use this as the input for step 2 (this is the choice $\RV{D}_1=1$). At the second step, I may further choose to copy the provisional results ($\RV{D}_2=0$) or invert them ($\RV{D}_2=1$). Then
\begin{align}
    \prob{P}_S^{\RV{Y}_1|\RV{D}}(y_1|d_1,d_2) &= 0.5\\
    \prob{P}_S^{\RV{Y}_2|\RV{D}}(y_2|d_1,d_2) &= 0.5
\end{align}
\begin{itemize}
    \item The marginal distribution of both experiments in isolation is $\text{Bernoulli}(0.5)$ no matter what choices I make, so a model of this experiment would satisfy Definition \ref{def:caus_cont}
    \item Nevertheless, the choice at step 1 affects the result of step 2
\end{itemize}

\subsubsection{Representation theorems for causally contractible Markov kernels}\label{sec:rep_theorem}

Theorem \ref{th:table_rep_kernel} shows that a causally contractible Markov kernel can be represented as the product of a column exchangeable probability distribution and a ``lookup function''. This representation is identical to the representation of potential outcomes models (see, for example, \citet{rubin_causal_2005}), but Theorem \ref{th:table_rep_kernel} applies to arbitrary kernels and the resulting representation will usually not be interpretable as a potential outcomes models. This theorem allows De Finetti's theorem to be applied to the column exchangeable probability distribution, which is a key step in proving the main result (Theorem \ref{th:ciid_rep_kernel}).

\begin{theorem}\label{th:table_rep_kernel}
A Markov kernel $\kernel{K}:X^{\mathbb{N}}\kto Y^{\mathbb{N}}$ is causally contractible if and only if there exists a column exchangeable probability distribution $\mu \Delta(Y^{|X|\times \mathbb{N}})$ such that
\begin{align}
    \kernel{K} &= \tikzfig{lookup_representation_kernel}\label{eq:lup_rep_kernel}\\
    &\iff\\
    \kernel{K}(A|(x_i)_{i\in \mathbb{N}}) &= \mu \Pi_{(x_i i)_{i\in\mathbb{N}}}(A)\forall A\in \sigalg{Y}^{\mathbb{N}}
\end{align}
Where $\Pi_{(d_i i)_{i\in\mathbb{N}}}:Y^{|X|\times \mathbb{N}}\to Y^{\mathbb{N}}$ is the function 
\begin{align}
    (y_{j i})_{j,i \in X\times  \mathbb{N}}\mapsto (y_{d_i i})_{i\in \mathbb{N}}
\end{align}
that projects the $(x_i,i)$ indices of $y$ for all $i\in \mathbb{N}$, and $\prob{F}_{\text{ev}}$ is the Markov kernel associated with the evaluation map
\begin{align}
    \text{ev}:X^\mathbb{N}\times Y^{X\times \mathbb{N}}&\to Y\\
    ((x_i)_\mathbb{N},(y_{ji})_{j,i\in X\times \mathbb{N}})&\mapsto (y_{x_i i})_{i\in \mathbb{N}}
\end{align}
\end{theorem}

\begin{proof}
Only if:
Choose $e:=(e_i)_{i\in\mathbb{N}}$ such that $e_{i+|X|j}$ is the $i$th element of $X$ for all $i,j\in \mathbb{N}$.

Define
\begin{align}
    \mu(\bigtimes_{(i,j)\in X\times \mathbb{N}} A_{ij}):=\kernel{K}(\bigtimes_{(i,j)\in X\times \mathbb{N}} A_{ij}|e)& \forall A_{ij}\in \sigalg{Y}
\end{align}

Now consider any $x:=(x_i)_{i\in \mathbb{N}}\in X^{\mathbb{N}}$. By definition of $e$, $e_{x_i i}=x_i$ for any $i,j\in \mathbb{N}$.

Define
\begin{align}
    \prob{Q}:X^{\mathbb{N}}\kto Y^{\mathbb{N}}\\
    \prob{Q}:= \tikzfig{lookup_representation_kernel}
\end{align}
and consider some $A\subset \mathbb{N}$, $|A|=n$ and $B:= (x_i,i))_{i\in A}$. Note that the subsequence of $e$ indexed by $B$, $e_B:=(e_{x_i i})_{i\in A}=x_A$. Thus given the swap map $\mathrm{swap}_{A\leftrightarrow B}:\mathbb{N}\to\mathbb{N}$ that sends the first element of $A$ to the first element of $B$ and so forth, $\mathrm{swap}_{A\leftrightarrow B} (e_B) = x_A$. For arbitrary $\{C_i\in \sigalg{Y}|i\in A\}$, define $C_A:=\mathrm{swap}_{[n]\leftrightarrow A} (\times_{i\in [n]} C_i\times Y^{\mathbb{N}})$. Then, for arbitrary $x\in X^{\mathbb{N}}$
\begin{align}
    \prob{Q}(C_A|x) &= \mu (\mathrm{ev}_x^{-1}(C_A))\label{eq:q_mu_rel}
\end{align}

The argument of $\mu$ is
\begin{align}
    \mathrm{ev}_x^{-1}(C_A)&=\{(y_{ji})_{j,i\in X\times\mathbb{N}}|(y_{x_i i})_{i\in\mathbb{N}}\in C_A\}\\
    &= \bigtimes_{i\in \mathbb{N}} \bigtimes_{j\in X} D_{ji}
\end{align}
where
\begin{align}
    D_{ji} = \begin{cases}
        C_i & (j,i)\in B\\
        Y & \text{otherwise}
    \end{cases}
\end{align}
and so
\begin{align}
    \text{swap}_{A\leftrightarrow B} (\mathrm{ev}_x^{-1}(C_A)) &= C_A\label{eq:swap_select_relation}
\end{align}

Substituting Equation \ref{eq:swap_select_relation} into \ref{eq:q_mu_rel}
\begin{align}
    \prob{Q}(C_A|x) &= \mu \text{swap}_{A\leftrightarrow B} (C_A)\\
    &= \kernel{K} \text{swap}_{A\leftrightarrow B} (C_A|e)\\
    &= \kernel{K}\text{swap}_{A\leftrightarrow B} (C_A|e_B,\text{swap}_{B\leftrightarrow A}(x)_B^C)&\text{by locality}\\
    &= \kernel{K}\text{swap}_{A\leftrightarrow B} (C_A|\text{swap}_{B\leftrightarrow A}(x))\\
    &= \text{swap}_{B\leftrightarrow A} \kernel{K}\text{swap}_{A\leftrightarrow B} (C_A|x)\\
    &= \kernel{K}(C_A|x)&\text{by commutativity of exchange}
\end{align}

Because this holds for all $x$, $A\subset\mathbb{N}$, by Lemma \ref{lem:infinitely_extended_kernels}

\begin{align}
    \prob{Q} &= \kernel{K}
\end{align}

Next we will show $\mu$ is column exchangeable. Consider any column swap $\text{swap}_{c}:X\times \mathbb{N}\to X\times \mathbb{N}$ that acts as the identity on the $X$ component and a finite permutation on the $\mathbb{N}$ component. From the definition of $e$, $\text{swap}_c(e)=e$. Thus by commutativity of exchange, for any $A\in \sigalg{Y}^{\mathbb{N}}$
\begin{align}
 \kernel{K}(A|e) &= \text{swap}_c\kernel{K}\text{swap}_c(A|e)\\
 &= \kernel{K}\text{swap}_c(A|\text{swap}_c(e))\\
 &= \kernel{K}\text{swap}_c(A|e)
\end{align}


If:
Suppose 
\begin{align}
    \kernel{K} &= \tikzfig{lookup_representation_kernel}
\end{align}
where $\mu$ is column exchangeable, and consider any two $x,x'\in X^{\mathbb{N}}$ such that some subsequences are equal $x_S=x'_T$ with $S,T\subset \mathbb{N}$ and $|S|=|T|=[n]$.

For any $\{A_i\in\sigalg{Y}|i\in S\}$, let $A_S = \text{swap}_{[n]\leftrightarrow S} \times_{i\in [n]} A_i\times Y^{\mathbb{N}}$, $A_T = \text{swap}_{S\leftrightarrow T} (A_S)$, $B=(x_i i)_{i\in S}$ and $C=(x_i i)_{i\in T}=(x_{\text{swap}_{S\leftrightarrow T}}(i) i)_{i\in S}$. By Equations \ref{eq:q_mu_rel} and \ref{eq:swap_select_relation}
\begin{align}
    \kernel{K}(A_S|x) &= \mu \text{swap}_{S\leftrightarrow B} (A_S)\\
    &= \mu \text{swap}_{T\leftrightarrow C} (A_T)&\text{ by column exchangeability of }\mu\\
    &= \kernel{K}(A_T|\text{swap}_{S\leftrightarrow T}(x))\\
    &=  \text{swap}_{S\leftrightarrow T}\kernel{K}(A_T| x)\\
    &= \text{swap}_{S\leftrightarrow T} \kernel{K} \text{swap}_{S\leftrightarrow T} (A_S| x)
\end{align}
so $\kernel{K}$ is causally contractible by Theorem \ref{th:equal_of_condits}.
\end{proof}

Theorem \ref{th:ciid_rep_kernel} is the main result of this section. It shows that a causally contractible Markov kernel $X^{\mathbb{N}}\kto Y^{\mathbb{N}}$ is representable as a ``prior'' $\mu\in \Delta(H)$ and a ``parallel product'' of Markov kernels $H\times X\kto Y$. These will be the response conditionals when Theorem \ref{th:ciid_rep_kernel} is applied to probability set models.

\begin{definition}[Measurable set of probability distributions]
Given a measurable set $(\Omega,\sigalg{F})$, the measurable set of distributions on $\Omega$, $\mathcal{M}_1(\Omega)$, is the set of all probability distributions on $\Omega$ equipped with the coarsest $\sigma$-algebra such that the evaulation maps $\eta_B:\nu\mapsto \nu(B)$ are measurable for all $B\in \sigalg{F}$.
\end{definition}

\begin{theorem}\label{th:ciid_rep_kernel}
Given a kernel $\kernel{K}:X^{\mathbb{N}}\kto Y^{\mathbb{N}}$, let $H:=\mathcal{M}_1(Y^X)$ be the measurable set of probability distributions on $(Y^X,\sigalg{Y}^X)$. $\kernel{K}$ is causally contractible if and only if there is some $\RV{H}:Y^{X\times\mathbb{N}}\to H$ and $\kernel{L}:H\times X\kto Y$ such that
\begin{align}
    \kernel{K} &= \tikzfig{do_model_representation_kernel}\\
    &\iff\\
    \kernel{K}(\bigtimes_{i\in\mathbb{N}}A_i|(x_i)_{i\in\mathbb{N}}) &= \int_H \prod_{i\in\mathbb{N}} \kernel{L}(A_i|h,x_i)\mu\kernel{F}_{\RV{H}}(\mathrm{d}h)
\end{align}
\end{theorem}

\begin{proof}
By Theorem \ref{th:table_rep_kernel}, we can represent the conditional probability $\kernel{K}$ as
\begin{align}
        \kernel{K} &= \tikzfig{lookup_representation_kernel}\label{eq:lookup_representation}
\end{align}
where $\mu$ is column exchangeable.

As a preliminary, we will show
\begin{align}
    \kernel{F}_{\mathrm{ev}} &= \tikzfig{lookup_rep_intermediate_kernel}\label{eq:ev_alternate_rep}
\end{align}
where  $\mathrm{evs}_{Y^D\times D}:Y^D\times D\to Y$ is the single-shot evaluation function
\begin{align}
    (x,(y_i)_{i\in X})\mapsto y_x
\end{align}

Recall that $\mathrm{ev}$ is the function
\begin{align}
    ((x_i)_\mathbb{N},(y_{ji})_{j,i\in X\times \mathbb{N}})&\mapsto (y_{x_i i})_{i\in \mathbb{N}}
\end{align}
By definition, for any $\{A_i\in\sigalg{Y}|i\in \mathbb{N}\}$
\begin{align}
    \kernel{F}_{\mathrm{ev}}(\bigtimes_{i\in \mathbb{N}}A_i|(x_i)_\mathbb{N},(y_{ji})_{i\in X\times \mathbb{N}}) &= \delta_{(y_{x_i i})_{i\in \mathbb{N}}}(\bigtimes_{i\in \mathbb{N}}A_i)\\
        &= \prod_{i\in \mathbb{N}} \delta_{y_{x_i i}} (A_i)\\
        &= \prod_{i\in \mathbb{N}} \kernel{F}_{\text{evs}} (A_i|x_i,(y_{ji})_{j\in X})\\
        &= \left(\bigotimes_{i\in\mathbb{N}} \kernel{F}_{\mathrm{evs}} \right)(\bigtimes_{i\in \mathbb{N}}A_i|(x_i)_\mathbb{N},(y_{ji})_{j\in X\times \mathbb{N}})
\end{align}
which is what we wanted to show.

Only if:
Define $\kernel{M}:H\kto Y^D$ by $\kernel{M}(A|h)=h(A)$ for all $A\in\sigalg{Y}^X$, $h\in H$. By the column exchangeability of $\mu$, from \citet[Prop. 1.4]{kallenberg_basic_2005} there is a directing random measure $\RV{H}:Y^{X\times\mathbb{N}}\to H$ such that
\begin{align}
    \mu &= \tikzfig{de_finetti_representation_kernel}\label{eq:df_rep_mu}\\
    &\iff\\
    \mu(\bigtimes_{i\in \mathbb{N}} A_i) &= \int_H \prod_{i\in \mathbb{N}} \kernel{M}(A_i|h) \mu\kernel{F}_{\RV{H}}(\mathrm{d}h)&\forall A_i\in\sigalg{Y}^X
\end{align}

By Equations \ref{eq:lookup_representation} and \ref{eq:ev_alternate_rep}
\begin{align}
    \kernel{K} &= \tikzfig{do_model_representation_kernel_pre}\\
    &:= \tikzfig{do_model_representation_kernel}\label{eq:lup_rep_combined}
\end{align}
Where we can connect the copied outputs of $\mu\kernel{F}_{\RV{H}}$ to the inputs of each $\kernel{M}$ ``inside the plate'' as the plates in Equations \ref{eq:ev_alternate_rep} and \ref{eq:df_rep_mu} are equal in number and each connected wire represents a single copy of $Y^D$.

If:
By assumption, for any $\{A_i\in \sigalg{Y}|i\in\mathbb{N}\}$, $x:=(x_i)_{i\in\mathbb{N}}\in X^{\mathbb{N}}$
\begin{align}
    \kernel{K}(\bigtimes_{i\in \mathbb{N}} A_i|x) &= \int_H \prod_{i\in \mathbb{N}}\kernel{L}(A_i|h,x_i)\mu(\mathrm{d}h)
\end{align}

Consider any $S,T\subset\mathbb{N}$ with $|S|=|T|$, and define $A_S:=\times_{i\in\mathbb{N}} B_i$ where $B_i=Y$ if $i\not\in S$, otherwise $A_i$ is an arbitrary element of $\sigalg{Y}$. Define $A_T:=\times_{i\in\mathbb{N}} B_{\mathrm{swap}_{S\leftrightarrow T}(i)}$.

\begin{align}
    \kernel{K}(A_S|x) &= \int_H \prod_{i\in S}\kernel{L}(A_i|h,x_i)\mu(\mathrm{d}h)\\
                      &= \int_H\prod_{i\in T}\kernel{L}(A_i|h,x_{\mathrm{swap}_{S\leftrightarrow T}(i)})\mu(\mathrm{d}h)\\
                      &= \mathrm{swap}_{S\leftrightarrow T}\kernel{K}(A_T|x)\\
                      &= \mathrm{swap}_{S\leftrightarrow T}\kernel{K}\mathrm{swap}_{S\leftrightarrow T}(A_S|x)
\end{align}
So by Theorem \ref{th:equal_of_condits}, $\kernel{K}$ is causally contractible.
\end{proof}

\subsection{Causal contractibility with data-independent actions}\label{sec:data_independent_actions}

Theorem \ref{th:ciid_rep_kernel} is proven for ``some causally contractible Markov kernel'', without a concrete role in a decision model. The simplest way to apply this to a decision model $\prob{P}_C$ is to assume that inputs are independent of previous observations -- that is, $\RV{D}_i\CI^e_{\prob{P}_C} (\RV{Y}_{<i},C)|\RV{D}_{<i}$. With this assumption, decision models with independent and identical response functions are those with \emph{causally contractible conditional probabilities} $\prob{P}_C^{\RV{Y}|\RV{D}}$. The assumption of data independence is a very limiting assumption -- see-do models, for example (Definition \ref{def:see_do_model}) do not satisfy it. The price to pay for the more general case is that we need to consider causally contractible \emph{combs} rather than causally contractible conditional probabilities, and this is addressed in Section \ref{sec:data_dependent}.

Given a sequence of variables $(\RV{D}_i,\RV{Y}_i)_{i\in \mathbb{N}}$ where the ``inputs'' are $\RV{D}:=(\RV{D}_i)_{i\in\mathbb{N}}$ and the ``outputs'' are $\RV{Y}=(\RV{Y}_i)_{i\in\mathbb{N}}$, say the inputs are independent of previous observations if $\RV{D}_i\CI^e_{\prob{P}_C} (\RV{Y}_{<i},C)|\RV{D}_{<i}$ for all $i\in\mathbb{N}$. This models an experiment where it may be possible to choose different inputs $\RV{D}$, but all the inputs are determined before the outputs $\RV{Y}$ are known. If a model satisfies this, then the dependence of $\RV{Y}$ on $\RV{D}$ is a sequence of independent and identical response functions if and only if the uniform conditional $\prob{P}_C^{\RV{Y}|\RV{D}}$ exists (Definition \ref{def:cprob_pset}) and is causally contractible.

We call a model $\prob{P}_C$ with sequential outputs $\RV{Y}$ and a corresponding sequence of data-independent inputs $\RV{D}$ a ``sequential just-do model''.

\begin{definition}[Sequential just-do model]
A \emph{sequential just-do model} is a triple $(\prob{P}_C,\RV{D},\RV{Y})$ where $\prob{P}_C$ is a probability set on $(\Omega,\sigalg{F})$, $\RV{D}$ is a sequence of ``inputs'' $\RV{D}:=(\RV{D}_i)_{i\in\mathbb{N}}$ and $\RV{Y}$ is a corresponding sequence of ``outputs'' $\RV{Y}=(\RV{Y}_i)_{i\in\mathbb{N}}$ where $\RV{D}_i:\Omega\to D$ and $\RV{Y}_i:\Omega\to Y$. Furthermore, it is required that $\RV{X}_i\CI^e_{\prob{P}_C} (\RV{Y}_{<i},C)|\RV{X}_{<i}$ for all $i\in \mathbb{N}$, and $\RV{Y}\CI^e_{\prob{P}_C} C|\RV{D}$.
\end{definition}

To apply Theorem \ref{th:ciid_rep_kernel} to a sequential just-do model, it is necessary to extend the model to a larger sample space including ``latent variables'' taking values in $Y^D$. A latent extension is a model over a larger collection of variables that reduces to the original model when we restrict our attention to the original collection of variables.

\begin{definition}[Latent extension]
Given a probability set $\prob{P}_C'$ on $(\Omega,\sigalg{F})$ and some measurable set $(G,\sigalg{G})$, a probability set $\prob{P}_C$ is a \emph{latent extension} of $\prob{P}_C'$ to $(\Omega\times G,\sigalg{F}\otimes \sigalg{G})$ if $\prob{P}_C \kernel{F}_{\Pi_{\Omega}} = \prob{P}_C'$.
\end{definition}

\begin{notation}[Variables on a latent extension]
Given a probability set $\prob{P}_C'$ on $(\Omega,\sigalg{F})$ and a latent extension $\prob{P}_C$ on $(\Omega\times G,\sigalg{F}\otimes \sigalg{G})$, every variable on the original sample space is given a primed name $\RV{X}'$, $\RV{Y}'$ etc., and corresponds to an unprimed variable on the larger space $\RV{X}:=\Pi_\Omega\circ \RV{X}'$.
\end{notation}

Theorem \ref{th:data_ind_CC} applies Theorem \ref{th:ciid_rep_kernel} to the case of a model with data-independent actions and derives the required conditional independences and equalities to show that a sequential just-do model $(\prob{P}_C,\RV{D},\RV{Y})$ with causally contractible $\prob{P}_C^{\RV{Y}|\RV{X}}$ satisfies the required conditional independences and equalities of conditional distributions for $\RV{X}$ and $\RV{Y}$ to be related by independent and identical response functions.

\begin{theorem}[Data-independent causal contractibility]\label{th:data_ind_CC}
Given a sequential just-do model $(\prob{P}_C',\RV{D}',\RV{Y}')$ on $(\Omega,\sigalg{F})$, then $\prob{P}_C^{\prime \RV{Y}'|\RV{D}'}$ is causally contractible if and only if there is a latent extension $\prob{P}_C$ of $\prob{P}_C'$ to $(\Omega\times Y^{D\times\mathbb{N}},\sigalg{F}\otimes\sigalg{Y}^{D\times\mathbb{N}})$ with some hypothesis $\RV{H}:\Omega\times Y^{D\times\mathbb{N}}\to H$ such that $\RV{Y}_i\CI^e_{\prob{P}_C'} (\RV{Y}_{<i},\RV{X}_{<i},C)|(\RV{X}_i,\RV{H})$ and $\prob{P}_C^{\RV{Y}_i|\RV{X}_i\RV{H}}=\prob{P}_C^{\RV{Y}_j|\RV{X}_j\RV{H}}$ for all $i,j\in \mathbb{N}$ and $\RV{H}\CI_{\prob{P}_C} (\RV{X},\RV{C})$.
\end{theorem}

\begin{proof}
If:
First, define the extension $\prob{P}_C$. From Theorem \ref{th:table_rep_kernel} and causal contractibility of $\prob{P}_C^{\prime \RV{Y}'|\RV{D}'}$ there is some $\mu\in \Delta(Y^{D\times\mathbb{N}})$ such that
\begin{align}
    \prob{P}_C^{\prime \RV{Y}'|\RV{D}'} &= \tikzfig{lookup_representation_variablised}\label{eq:lup_rep_varb}
\end{align}
Let $\prob{P}_C^{\RV{Y}^{D}|\RV{D}}=\mu\otimes \text{del}_{D^{\mathbb{N}}}$, $\prob{P}_C^{\RV{Y}|\RV{Y}^D\RV{D}}=\kernel{F}_{\mathrm{ev}}$ and $\RV{Y}^D:=\Pi_{Y^{D\times\mathbb{N}}}$, the projection $\Omega\times Y^{D\times\mathbb{N}}\to \Omega$. Let $\RV{W}=\Pi_{\Omega}$ and for each $\alpha\in C$, set 
\begin{align}
    \prob{P}_\alpha^{\RV{W}|\RV{Y}^D\RV{D}} &= \tikzfig{augmented_ccontracible}
\end{align}
Then 
\begin{align}
    \prob{P}_\alpha^{\RV{W}} &= \tikzfig{augmented_ccon2}\\
    &= \tikzfig{augmented_ccon3}\\
    &= \prob{P}_\alpha^{\prime \text{id}_\Omega}
    &= \prob{P}_\alpha
\end{align}

Before going further, it's necessary to check that there is some nonempty probability set $\prob{P}_C$ with these conditionals. By Theorem \ref{lem:valid_extendability}, because $\prob{P}_\alpha^{\RV{W}}=\prob{P}_{\alpha}'$ it is sufficient to show that $\prob{P}_\alpha^{\RV{Y}^D|\RV{W}}$ is valid. Because 
\begin{align}
    (\RV{W},\RV{Y}^D)(\Omega\times Y^{D\times\mathbb{N}})&=\Omega\times Y^{D\times\mathbb{N}}\\
    &=\RV{W}(\Omega\times Y^{D\times\mathbb{N}})\times \RV{Y}^D(\Omega\times Y^{D\times\mathbb{N}})
\end{align}
there are no impossible events, and so validity is guaranteed for any $\prob{P}_\alpha^{\RV{Y}^D|\RV{W}}$.

Thus $\prob{P}_C$ is a latent extension of $\prob{P}_C'$, and so $\prob{P}_C^{\RV{Y}|\RV{D}}$ is also causally contractible.

From Theorem \ref{th:ciid_rep_kernel} and by construction of $\prob{P}_C$, there exists a directing random measure $\RV{H}^*:Y^{D\times\mathbb{N}}\to H$ such that, defining $\RV{H}=\RV{H}^*\circ\Pi_{D\times\mathbb{N}}$
\begin{align}
    \prob{P}_C^{\RV{Y}|\RV{D}} &= \tikzfig{do_model_representation}
\end{align}
it remains to be shown that $\kernel{L}$ is a version of $\prob{P}^{\RV{Y}_i|\RV{D}_i\RV{H}}$ for all $i\in \mathbb{N}$ and $\RV{Y}_i\CI^e_{\prob{P}_C'} (\RV{Y}_{<i},\RV{D}_{<i},C)|(\RV{D}_i,\RV{H})$.

To show $\kernel{L}$ is a version of $\prob{P}^{\RV{Y}_i|\RV{H}\RV{D}_i}$ for all $i\in \mathbb{N}$:
\begin{align}
    \kernel{L}&=\tikzfig{kernel_l_broken_down}\\
              &=\tikzfig{kernel_l_broken_down2}\\
              &=\tikzfig{kernel_l_broken_down3}\label{eq:h_ci_x}\\
              &=\tikzfig{kernel_l_broken_down4}\label{eq:Y_ci_h_yd}\\
              &=\prob{P}_C^{\RV{Y}_i|\RV{H}\RV{D}_i}
\end{align}
Where \ref{eq:h_ci_x} follows from $\RV{H}\CI^e_{\prob{P}_C} (\RV{D}_i,\RV{C})$, which itself follows from $\RV{Y}^D_i\CI^e_{\prob{P}_C} (\RV{D}_i,\RV{C})$ which holds by construction. \ref{eq:Y_ci_h_yd} follows from $\RV{Y}_i\CI^e_{\prob{P}_C} (\RV{H},\RV{C})|(\RV{Y}^D_i,\RV{D}_i)$, which follows from $\RV{Y}_i$ being a deterministic function of $(\RV{Y}^D_i,\RV{D}_i)$.

For independence, note that
\begin{align}
     \prob{P}_C^{\RV{Y}_{<i}|\RV{H}\RV{X}_{<i}\RV{X}_i} &= \tikzfig{independence_inductive_base_0}\\
     &= \tikzfig{independence_inductive_base}
\end{align}
hence $\RV{Y}_{<i}\CI^e_{\prob{P}_C} (\RV{X}_i,\RV{C})|(\RV{H},\RV{X}_{<i})$

Then
\begin{align}
    \prob{P}_C^{\RV{Y}_i\RV{Y}_{<i}|\RV{H}\RV{D}_i\RV{D}_{<i}} &= \tikzfig{independence_inductive}\\
    \implies \prob{P}_C^{\RV{Y}_i|\RV{H}\RV{D}_i\RV{D}_{<i}} &\overset{\prob{P}_C}{\cong} \tikzfig{independence_inductive_last}
\end{align}
by Theorem \ref{th:higher_order_conditionals}. Hence $\RV{Y}_i\CI^e_{\prob{P}_C} (\RV{X}_{<i},\RV{Y}_{<i},\RV{C})|(\RV{H},\RV{X}_i)$.

Only if:
If $\prob{P}_C$ is a latent extension of $\prob{P}_C'$, then $\prob{P}_C^{\RV{Y}|\RV{D}}$ is causally contractible if and only if $\prob{P}_C^{\prime \RV{Y}'|\RV{D}'}$ is causally contractible. Thus it is sufficient to show $\prob{P}_C^{\RV{Y}|\RV{D}}$ is causally contractible.

By assumption, for all $i\in \mathbb{N}$
\begin{align}
    \prob{P}_C^{\RV{Y}_i|\RV{H}\RV{X}_{[i]}\RV{Y}_{<i}} &\overset{\prob{P}_C}{\cong} \text{del}_{X^{i-1}\times Y^{i-1}}\otimes \prob{P}_C^{\RV{Y}_1|\RV{H}\RV{X}_1}
\end{align}

Thus for all $n\in \mathbb{N}$ by repeated application of Theorem \ref{th:higher_order_conditionals}
\begin{align}
    \prob{P}_C^{\RV{Y}_{[n]}|\RV{H}\RV{X}_{[n]}} &\overset{\prob{P}_C}{\cong} \tikzfig{do_model_representation_finite}
\end{align}
thus by Lemma \ref{lem:infinitely_extended_kernels}
\begin{align}
    \prob{P}_C^{\RV{Y}_{\mathbb{N}}|\RV{H}\RV{X}_{\mathbb{N}}} &\overset{\prob{P}_C}{\cong} \tikzfig{do_model_representation_noprior}
\end{align}
and, because $\RV{H}\CI_{\prob{P}_C}^e(\RV{X},\RV{C})$
\begin{align}
    \prob{P}_C^{\RV{Y}_{\mathbb{N}}|\RV{X}_{\mathbb{N}}} &\overset{\prob{P}_C}{\cong} \tikzfig{do_model_representation}
\end{align}
causal contractibility follows from Theorem \ref{th:ciid_rep_kernel}.
\end{proof}

A consequence of Theorem \ref{th:equal_of_condits} applied to a just-do models $\prob{P}_C$ with causally contractible $\prob{P}_C^{\RV{Y}|\RV{D}}$ is that, for any $A,B\subset\mathbb{N}$ with $|A|=|B|$, $\prob{P}_C^{\RV{Y}_A|\RV{D}_A}=\prob{P}_C^{\RV{Y}_B|\RV{D}_B}$. A further consequence is the interchangeability of conditioning data -- for any $i\in \mathbb{N}$, $\prob{P}_C^{\RV{Y}_i|\RV{D}_i\RV{Y}_A\RV{D}_A}=\prob{P}_C^{\RV{Y}_i|\RV{D}_i\RV{Y}_B\RV{D}_B}$.

\begin{theorem}[Equality of subsequence conditionals]\label{th:equal_of_reduced_condits}
A sequential just-do model $(\prob{P}_C,\RV{D},\RV{Y})$ with $\prob{P}_C^{\RV{Y}|\RV{D}}$ causally contractible satisfies, for any $A,B\subset \mathbb{N}$ with $|A|=|B|$
\begin{align}
    \prob{P}_C^{\RV{Y}_A|\RV{D}_A} \overset{\prob{P}_C}{\cong} \prob{P}_C^{\RV{Y}_B|\RV{D}_B}
\end{align}
\end{theorem}

\begin{proof}
Only if:
For any $A,B\subset \mathbb{N}$, let $\text{swap}_{B\leftrightarrow A,D}:D^{\mathbb{N}}\kto D^{\mathbb{N}}$ be the transposiiton of $B$ with $A$ indices and $\text{swap}_{B\leftrightarrow A,Y}:Y^{\mathbb{N}}\kto Y^{\mathbb{N}}$ be the same defined on $Y$. By Theorem \ref{th:equal_of_condits}
\begin{align}
    \prob{P}_C^{\RV{Y}_A|\RV{D}_A}\otimes \text{del}_{D^{\mathbb{N}}} &=  \prob{P}_C^{\RV{Y}|\RV{D}} \text{marg}_A\\
     &= \text{swap}_{A\leftrightarrow[n],D} \prob{P}_C^{\RV{Y}_{[n]}|\RV{D}}\\
    &= \prob{P}_C^{\RV{Y}_{[n]}|\RV{D}_{[n]}}\otimes \text{del}_{D^{\mathbb{N}}}\\
    &= \text{swap}_{N\leftrightarrow[n],D} \prob{P}_C^{\RV{Y}_{[n]}|\RV{D}}\\
    &= \prob{P}_C^{\RV{Y}_B|\RV{D}_B}\otimes \text{del}_{D^{\mathbb{N}}}
\end{align}
\end{proof}

\subsubsection{Examples}\label{sec:examples}

Purely passive observations can be modeled with a probability set $\prob{P}_C$ where $|\prob{P}_C|=1$. In this case, a model that is exchangeable over the sequence of pairs $(\RV{D}_i,\RV{Y}_i)_{i\in \mathbb{N}}$ has $\prob{P}_C^{\RV{Y}|\RV{D}}$ causally contractible. This follows from the fact that

\begin{align}
    \prob{P}_C^{\RV{YD}} &= \tikzfig{do_model_rep_onechoice_combined}
\end{align}
and so
\begin{align}
    \tikzfig{do_model_rep_onechoice}
\end{align}
is a version of $\prob{P}_C^{\RV{Y}|\RV{D}}$.

Instead of passive observations only, a model might feature a subsequence of passive observations and a subsequence of active interventions. Say the passive observations are $(\RV{D},\RV{Y})_{i\in\mathbb{N}}$ and the active interventions are $(\RV{E},\RV{Z})_{i\in \mathbb{N}}$. By the previous argument, $\prob{P}_C^{\RV{Y}|\RV{D}}$ is causally contractible. We might further assume that $\prob{P}_C^{\RV{YZ}|\RV{DE}}$ is causally contractible -- that is, there is a independent and identical response function $\prob{P}_C^{\RV{Z}_i|\RV{E}_i\RV{H}}$ equal to $\prob{P}_C^{\RV{Y}_i|\RV{D}_i\RV{H}}$.

One consequence of this is ``observational imitation'': any choice $\alpha$ that makes $\prob{P}_\alpha^{\RV{D}\RV{E}}$ exchangeable also makes $\prob{P}_\alpha^{\RV{YZ}}$ exchangeable. That is, if for some permutation $\mathrm{swap}_\rho$
\begin{align}
    \prob{P}_\alpha^{\RV{DE}}\mathrm{swap}_\rho &= \prob{P}_\alpha^{\RV{DE}}
\end{align}
then by commutativity of exchange
\begin{align}
    \prob{P}_\alpha^{\RV{YZ}} &= \prob{P}_\alpha^{\RV{DE}} \prob{P}_C^{\RV{YZ}|\RV{DE}}\\
    &=  \prob{P}_\alpha^{\RV{DE}}\mathrm{swap}_\rho \prob{P}_C^{\RV{YZ}|\RV{DE}}\\
    &= \prob{P}_\alpha^{\RV{DE}} \prob{P}_C^{\RV{YZ}|\RV{DE}}\mathrm{swap}_\rho\\
    &= \prob{P}_C^{\RV{YZ}|\RV{DE}}\mathrm{swap}_\rho
\end{align}

However, the assumption that $\prob{P}_C^{\RV{YZ}|\RV{DE}}$ is causally contractible seems unreasonable in most situations. One implication of this assumption is (by Theorem \ref{th:equal_of_condits}):
\begin{align}
    \prob{P}^{\RV{Y}\RV{Z}_i|\RV{D}\RV{E}_i}_C &= \prob{P}^{\RV{Z}|\RV{E}}\\
    \implies \prob{P}^{\RV{Z}_i|\RV{E}_i\RV{D}\RV{Y}}_C &= \prob{P}^{\RV{Z}_{i}|\RV{E}_i\RV{E}_{\{i\}^C}\RV{Z}_{\{i\}^C}}
\end{align}
That is, the model must yield the same result when conditioned on either the observational results, or the results of other active interventions. It is rare to assume \emph{a priori} that observational and experimental data are equally informative. Such a conclusion could be drawn \emph{after} reviewing both sequences of data, see for example \citet{eckles_bias_2021}, or it might be rejected \citet{gordon_comparison_2018,gordon_close_2022}.

\begin{example}[Backdoor adjustment]\label{ex:backdoor}
If a sequential just-do model $(\prob{P}_C,(\RV{D},\RV{X}),\RV{Y})$ has $\prob{P}_C^{\RV{Y}|\RV{DX}}$ causally contractible as well as:
\begin{itemize}
    \item $\RV{X}_{i}\CI^e_{\prob{P}_C}\RV{D}_{i}C|\RV{H}$ ($\RV{X}_i$ is extended independent of $\RV{D}_i$ conditional on $\RV{H}$)
    \item $\prob{P}_C^{\RV{X}_{i}|\RV{H}}\cong \prob{P}_C^{\RV{X}_{1}|\RV{H}}$ (the distribution of $\RV{X}$ is exchangeable)
 \end{itemize}
Then the model exhibits a kind of ``backdoor adjustment'' \citet[Chap. 1]{pearl_causality:_2009}. Specifically
\begin{align}
    \prob{P}_\alpha^{\RV{Y}_{i}|\RV{D}_{i}\RV{H}}(A|d,h) &= \int_X \prob{P}_\alpha^{\RV{Y}_{i}|\RV{X}_{i}\RV{D}_{i}\RV{H}}(A|d,x,h)\prob{P}_\alpha^{\RV{X}_{i}|\RV{D}_{i}\RV{H}}(\mathrm{d}x|d,h)\\
    &= \int_X \prob{P}_C^{\RV{Y}_{1}|\RV{X}_{1}\RV{D}_{1}\RV{H}}(A|d,x,h)\prob{P}_C^{\RV{X}_{i}|\RV{H}}(\mathrm{d}x|h)\\
    &= \int_X \prob{P}_C^{\RV{Y}_{1}|\RV{X}_{1}\RV{D}_{1}\RV{H}}(A|d,x,h)\prob{P}_C^{\RV{X}_{1}|\RV{H}}(\mathrm{d}x|h)\label{eq:backdoor}
\end{align}
\end{example}


Equation \ref{eq:backdoor} is identical to the backdoor adjustment formula for an intervention on $\RV{D}_1$ targeting $\RV{Y}_1$ where $\RV{X}_1$ is a common cause of both.

\section{Causal contractibility in sequences of active choices}\label{sec:assessing}

\begin{example}[Confounding]\label{ex:confounding}
Given a sequential just-do model $(\prob{P}_{C},\RV{D},\RV{Y})$, we might be tempted to reason as follows: each $(\RV{D}_i,\RV{Y}_i)$ pair comes from one sub-experiment out of a long sequence that all look \emph{a priori} identical, therefore $\prob{P}_C^{\RV{Y}|\RV{D}}$ should be causally contractible. Not so fast!

Such a model could exhibit confounding. Suppose $(\prob{P}_{C},(\RV{D},\RV{U}),\RV{Y})$ really does feature a causally contractible $\prob{P}_C^{\RV{Y}|\RV{UD}}$ with single-valued $\RV{H}$ and all other variables binary. Take $\prob{P}_C^{\RV{Y}_i|\RV{U}_i\RV{D}_i}(1|\cdot,\cdot)$ as follows:
\begin{center}
\begin{tabular}{ c | c | c }
  & $\RV{D}_i=0$ & $\RV{D}_i=1$ \\\hline 
 $\RV{U}_i=0$ & 0 & 0 \\ \hline 
 $\RV{U}_i=1$ & 1 & 1    
\end{tabular}
\end{center}
and take $\prob{P}_\cdot^{\RV{D}_i|\RV{U}_i}(1|\cdot)$ to be
\begin{center}
\begin{tabular}{ c | c | c }
  & $\alpha=0$ & $\alpha=1$ \\\hline 
 $\RV{U}_i=0$ & 0 & 1 \\ \hline 
 $\RV{U}_i=1$ & 1 & 0    
\end{tabular}
\end{center}
then for any $i$, $\prob{P}_0^{\RV{Y}_i|\RV{D}_i}(1|0)=0$ but $\prob{P}_1^{\RV{Y}_i|\RV{D}_i}(1|0)=1$, so there cannot be a causally contractible $\prob{P}_C^{\RV{Y}|\RV{D}}$. Note that $\RV{U}$ need not be unobserved here.
\end{example}

So, a judgement of symmetry among sub-experiments is not enough for causal contractibility. What is enough?

In the following, it is helpful to assume that each sub-experiment has a ``unique identifier'' $\RV{I}_i$, with the sequence of all sub-experiment labels given by $\RV{I}$. With this, if $\prob{P}_C^{\RV{Y}|\RV{DI}}$ is assumed causally contractible, then it's possible to talk about the individual response functions $\prob{P}_C^{\RV{Y}_i|\RV{I}_i\RV{H}\RV{D}_i}$. These plays a role very similar to the $i$th vector of potential outcomes $\RV{Y}^D_i$. Because $\RV{I}_i$ is unique (i.e. never equal to $\RV{I}_j$ for $j\neq i$), only one observation of any individual is ever given, just like only one potential outcome is ever observed.

Theorem \ref{th:cc_ind_treat} presents a set of sufficient conditions for causal contractibility of $\prob{P}_C^{\RV{Y}|\RV{D}}$:
\begin{enumerate}
    \item There exist variables $\RV{I}$ representing ``unique experiment identifiers'' which satisfy the assumption that $\prob{P}_C^{\RV{Y}|\RV{DI}}$ is causally contractible (informally: it doesn't matter which order the experiments are conducted in, and treatments in each experiment do not affect any other experiments)
    \item The identifiers themselves are not informative regarding outcomes: $\RV{Y}\CI_{\prob{P}_C}^e \RV{I}|\RV{C})$
    \item The inputs $\RV{D}$ are substitutable for the choice $\RV{C}$ with respect to $\RV{Y}$ and $\RV{I}$: $\RV{YI}\CI^e_{\prob{P}_C} \RV{C}|\RV{D}$ and $\RV{YI}\CI^e_{\prob{P}_C} \RV{D}|\RV{C}$
\end{enumerate}
However, as we will show, even these conditions can be subtle to assess.

As an example of the application of Theorem \ref{th:cc_ind_treat}, consider an experiment where $n$ patients, each with an individual identifier $\RV{I}_i$, receive treatment $\RV{D}_i$ and experience outcome $\RV{Y}_i$. $\prob{P}_C^{\RV{Y}_{[n]}|\RV{D}_{[n]}\RV{I}_{[n]}}$ can be extended to an infinite sequence $\prob{P}_C^{\RV{Y}|\RV{DI}}$ that is causally contractible (see Assumption 1), and no matter which choice $\alpha\in C$ is given, all identifiers can be swapped without altering the distribution over consequences (see Assumption 2), and finally that the treatment vector $\RV{D}$ is a deterministic and invertible function of the choice $\alpha\in C$ then $\prob{P}_C^{\RV{Y}|\RV{D}}$ is causally contractible, and hence there are response functions $\prob{P}_C^{\RV{Y}_i|\RV{D}_i\RV{H}}$.

Theorem \ref{th:cc_ind_treat} can also be extended to the case where $\RV{D}$ is a function of the choice $\alpha$ and a ``random signal'' $\RV{R}$.

\begin{lemma}\label{lem:ind_to_cc}
Given sequential just-do model $(\prob{P}_C,(\RV{D},\RV{I}),\RV{Y})$ with $\prob{P}_C^{\RV{Y}|\RV{DI}}$ causally contractible, if $\RV{Y}\CI_{\prob{P}_C}^e (\RV{I},\RV{C})|\RV{D}$ and for any $j\in I$, $\sum_{\alpha\in C} \prob{P}_\alpha^{\RV{I}_i}(j)>0$, then $\prob{P}_C^{\RV{Y}|\RV{D}}$ is also causally contractible.
\end{lemma}

\begin{proof}
For arbitrary $\nu\in \Delta(I^{\mathbb{N}})$ such that $\sum_{\alpha\in C} \prob{P}_\alpha^{\RV{I}_i} \gg \nu$, by assumption of causal contractibility of $\prob{P}_C^{\RV{Y}|\RV{DI}}$ and Theorem \ref{th:ciid_rep_kernel}
\begin{align}
    \prob{P}_C^{\RV{Y}|\RV{DI}} &\overset{\prob{P}_C}{\cong} \tikzfig{index_independence_1}\\
    &\overset{\prob{P}_C}{\cong} \tikzfig{index_independence_2}
\end{align}
Where $\Pi_{D,i}:D^{\mathbb{N}}\kto D$ projects the $i$th coordinate, and similarly for $\Pi_{Y,i}$.

In particular, for any $i\in \mathbb{N}$, $j\in I$, this holds for some $\nu$ such that $\nu(\Pi_{Y,i}^{-1} (j)=1$ and by extension for any finite $A\subset \mathbb{N}$ we can find $\nu$ such that $\nu(\Pi_{Y,i}^{-1} (j)=1$ for all $i\in A$, $j\in I$. Thus for any $n\in \mathbb{N}$
\begin{align}
    \prob{P}_C^{\RV{Y}_{[n]}|\RV{D}_{[n]}\RV{I}_{[n]}} &\overset{\prob{P}_C}{\cong} \tikzfig{index_independence_3}\label{eq:follows_from_determinism}\\
    &\overset{\prob{P}_C}{\cong} \tikzfig{index_independence_4}\label{eq:follows_from_equality}
\end{align}

where Equation \ref{eq:follows_from_determinism} follows from Theorem \ref{th:fong_det_kerns} and Equation \ref{eq:follows_from_equality} follows from the fact that Equation \ref{eq:follows_from_determinism} holds for arbitrary $j\in I$.

Thus by Lemma \ref{lem:infinitely_extended_kernels}
\begin{align}
    \prob{P}_C^{\RV{Y}|\RV{D}} &= \tikzfig{index_independence_5}
\end{align}
Applying Theorem \ref{th:ciid_rep_kernel}, $\prob{P}_C^{\RV{Y}|\RV{D}}$ is causally contractible.
\end{proof}

\begin{theorem}\label{th:cc_ind_treat}
Given a sequential just-do model $(\prob{P}_C,(\RV{D},\RV{I}),\RV{Y})$ on $(\Omega,\sigalg{F})$ with $Y$ standard measurable and $C$ countable, $\prob{P}_C^{\RV{Y}|\RV{DI}}$ causally contractible $\RV{I}:\Omega\to I$ an identifier variable, if
\begin{align}
    &\RV{Y}\CI^e_{\prob{P}_C} \RV{I} | \RV{C}\\
    &\RV{YI}\CI^e_{\prob{P}_C} \RV{C}|\RV{D}\\
    &\RV{YI}\CI^e_{\prob{P}_C} \RV{D}|\RV{C}\\
    &\forall i,j\in \mathbb{N}: \sum_{\alpha\in C} \prob{P}_\alpha^{\RV{I}_i}(j)>0
\end{align}
then $\prob{P}_C^{\RV{Y}|\RV{D}}$ is causally contractible.
\end{theorem}

\begin{proof}
For any $\alpha\in C$
\begin{align}
    \prob{P}_\alpha^{\RV{Y}|\RV{I}} &= \tikzfig{kernel_fac_with_idents}\\
    &= \tikzfig{kernel_fac_with_idents_indepped}
\end{align}

Define $\kernel{Q}$ by $\alpha\mapsto \prob{P}_\alpha$ and $\kernel{Q}^{\cdot|\cdot\RV{C}}$ by $\alpha\mapsto \prob{P}_\alpha^{*}$ and $\kernel{Q}^{\RV{C}}$ is an arbitrary distribution in $\Delta(C)$ with full support. Note that the support of $\kernel{Q}^{\RV{IDY}}$ is the union of the support of $\prob{P}^{\RV{IDY}}_\alpha$ for all $\alpha$. Then
\begin{align}
    \kernel{Q}^{\RV{Y}|\RV{IC}} &\overset{\prob{Q}}{\cong} \tikzfig{kernel_fac_with_idents_kernelised}
\end{align}

By assumption $\RV{YI}\CI^e_{\prob{P}_C} \RV{D}|\RV{C}$, it is also the case that
\begin{align}
    \kernel{Q}^{\RV{Y}|\RV{ID}} &\overset{\prob{Q}}{\cong} \tikzfig{kernel_Q_fac_with_idents}\\
    &\overset{\prob{Q}}{\cong} \tikzfig{kernel_Q_fac_with_idents_indepped}\\
    &\overset{\prob{Q}}{\cong} \tikzfig{kernel_Q_fac_with_idents_subbed}
\end{align}
But
\begin{align}
    \kernel{Q}^{\RV{Y}|\RV{ID}}=\sum_{\alpha\in C} \prob{P}_\alpha^{\RV{Y}|\RV{ID}}\kernel{Q}^{\RV{C}}(\alpha)\\
    &= \prob{P}_C^{\RV{Y}|\RV{ID}}
    \implies \tikzfig{kernel_Q_fac_with_idents_subbed} &= \prob{P}_C^{\RV{Y}|\RV{ID}}
\end{align}

Furthermore, by assumption $\RV{Y}\CI^e_{\prob{P}_C} \RV{I} | \RV{C}$, so there is some $\kernel{K}:C\kto Y$ such that
\begin{align}
    \kernel{Q}^{\RV{Y}|\RV{IC}} &\overset{\prob{Q}}{\cong} \tikzfig{kernel_Q_indepped}\\
    \implies \prob{P}_C^{\RV{Y}|\RV{ID}} &= \tikzfig{kernel_Q_fac_with_idents_swapped}\\
    &= \tikzfig{kernel_P_indep}
\end{align}
Then by Lemma \ref{lem:ind_to_cc}, $\prob{P}_C^{\RV{Y}|\RV{D}}$ is causally contractible.
\end{proof}

Theorem \ref{th:cc_ind_treat} can be extended to the case where decisions $\RV{D}$ are a one-to-one deterministic function of the choice, or a random mixtures of one-to-one deterministic functions of the choice.

\begin{theorem}\label{cor:extend_to_randomised}
Consider a sequential just-do model $(\prob{P}_{C'},\RV{D},\RV{Y})$ where $\prob{P}_{C'}^{\RV{Y}|\RV{D}}$ is causally contractible, and construct a second model $(\prob{P}_{C},\RV{D},\RV{Y})$ where $\prob{P}_C$ is the union of $\prob{P}_{C'}$ and its convex hull. Then $\prob{P}_{C}^{\RV{Y}|\RV{D}}$ is also causally contractible.
\end{theorem}

\begin{proof}
For all $\alpha\in C$, there is some probability measure $\mu:C'\to [0,1]$ such that
\begin{align}
    \prob{P}_\alpha^{\RV{Y}|\RV{D}} &= \sum_{\beta\in C'} \mu(\beta) \prob{P}_\beta^{\RV{Y}|\RV{D}}\\
    &= \prob{P}_{C'}^{\RV{Y}|\RV{D}}
\end{align}
thus
\begin{align}
    \prob{P}_C^{\RV{Y}|\RV{D}} = \prob{P}_{C'}^{\RV{Y}|\RV{D}}
\end{align}
and in particular, $\prob{P}_C^{\RV{Y}|\RV{D}}$ is causally contractible.
\end{proof}

The assumption $\RV{Y}\CI^e_{\prob{P}_C} \RV{I} | \RV{C}$ can be understood in terms of \emph{permutability of identifiers}. An \emph{identifier variable} is a variable $\RV{I}$ that takes values in the set of finite permutations of $\mathbb{N}$. It is associated with a sequence $(\RV{I}_i)_{i\in \mathbb{N}}$ where $\RV{I}_i=\RV{I}(i)$. Each $\RV{I}_i$ takes values in $\mathbb{N}$ and $\RV{I}_i\neq \RV{I}_j$ for all $j\neq i$.

\begin{definition}[Identifier variable]
Given a probability set $\prob{P}_C$ on $(\Omega,\sigalg{F})$, let $I$ be the set of finite permutations $\mathbb{N}\to \mathbb{N}$. A variable $\RV{I}:\Omega\to I$ be a variable taking values in $I$ is an \emph{identifier variable}.
\end{definition}

If a uniform conditional probability is invariant to permutations of an index variable, then it is independent of that index variable.

\begin{lemma}\label{lem:ind}
Given a probability set $\prob{P}_C$ where $\RV{Y}\CI_{\prob{P}_C}^e \RV{C}|(\RV{D},\RV{I})$ and $\RV{I}:\Omega\to I$ is an identifier variable, if for each finite permutation $\rho:\mathbb{N}\to \mathbb{N}$
\begin{align}
    \prob{P}_\alpha^{\RV{Y}|\RV{I}} &= (\text{swap}_{\rho(I)}\otimes \text{Id}_X )\prob{P}_\alpha^{\RV{Y}|\RV{I}}
\end{align}
then $\RV{Y}\CI_{\prob{P}_C}^e \RV{I}|\RV{C}$.
\end{lemma}

\begin{proof}
By definition of the set $I$ of finite permutations, for every $\rho\in I$, $B\in\sigalg{Y}^{\mathbb{N}}$, $d\in D^{\mathbb{N}}$ there is a finite permutation $\rho^{-1}\in I$ such that $\rho\circ\rho^{-1}=\text{id}_{\mathbb{N}}$. Then
\begin{align}
    \prob{P}_\alpha^{\RV{Y}|\RV{I}}(B|\rho) &= (\kernel{F}_{\rho^{-1}}\otimes \text{Id}_X )\prob{P}_\alpha^{\RV{Y}|\RV{I}}(B|\rho)\\
    &= \prob{P}_\alpha^{\RV{Y}|\RV{I}}(B|\text{id}_{\mathbb{N}})
\end{align}
Therefore
\begin{align}
    \prob{P}_\alpha^{\RV{Y}|\RV{I}} &\overset{\prob{P}_C}{\cong} \text{erase}_{I}\otimes \prob{P}_\alpha^{\RV{Y}}
\end{align}
\end{proof}

Theorem \ref{cor:extend_to_randomised} can be used to argue that, given a sequence of experiments causally contractible under deterministic choices, adding random mixtures of these choices also yields a causally contractible sequence. \citet{kasy_why_2016} argues that as long as the experimenter controls the treatment assignment, causal effects are identified (i.e. the randomisation step is not strictly necessary). Example \ref{ex:randomised_experiment} shows that this argument might be supported, but Example \ref{ex:bad_randomised_experiment} shows that there are subtle ways that might lead to this argument failing. We consider a simpler case than \citet{kasy_why_2016}, where there are no covariates to worry about.

We assume an infinite sequence, which is clearly unreasonable. Extending the representation theorems to the case of finite sequences, using for example the result of \citet{diaconis_finite_1980} with establishes that finite exchangeable distributions are approximately mixtures of independent and identically distributed sequences, would allow some implausible assumptions in the following example to be removed.

\begin{example}\label{ex:randomised_experiment}
A sequential experiment is modeled by a probability set $\prob{P}_C$ with binary treatments $\RV{D}:=(\RV{D}_i)_{i\in \mathbb{N}}$ and binary outcomes $\RV{Y}:=(\RV{Y}_i)_{i\in\mathbb{N}}$. The set of choices $C$ is the set of all probability distributions$\Delta(D^N)$ for some $N\subset\mathbb{N}$ (this is to ensure $C$ is countable).

Each treatment $\RV{D}_i$ is given to a patient, and each patient provides a unique identifier $\RV{I}_i$ which for simplicity we assume is a number in $\mathbb{N}$ (instead of, say, a driver's license number and state of issue), and that (implausibly) there is a positive probability for $\RV{I}_i$ to take any value in $\mathbb{N}$ for any choice $\alpha$.

The treatments are decided as follows: the analyst consults the model $\prob{P}_C$, and, according to $\prob{P}_C$ and some previously agreed upon decision rule, comes up with a possibly stochastic sequence  of treatment distributions $\alpha:=(\mu_i)_{i\in N}$ with each $\mu_i$ in $\Delta(\{0,1\})$. If $\mu_i$ is deterministic -- that is, it puts probability 1 on some treatment $d_i$, the experiment administrator will assign patient $i$ the treatment $d_i$. Otherwise, if $\mu_i$ is nondeterministic, the administrator will consult some agreed-upon random number generator that yields treatment assignments according to $\mu_i$.

Let $C'\subset C$ be the deterministic elements of $C$, and assume that all elements of $C$ are a convex combination of elements of $C'$. The randomisation procedure is deemed sufficient to ensure that for any mixed $\alpha\in C$ where $\alpha = \sum_{\beta\in C'} \mu(\beta) \beta$, $\prob{P}_\alpha = \sum_{\beta\in C'} \prob{P}_\beta$.

Furthermore, assume $\prob{P}_\alpha^{\RV{Y}|\RV{DI}}$ is causally contractible for each $\alpha$. As discussed in Theorem \ref{th:equal_of_condits}, this means that for any $A,B\subset\mathbb{N}$, $|A|=|B|$, $\prob{P}_\alpha^{\RV{Y}_A|\RV{D}_A\RV{I}_A}=\prob{P}_C^{\RV{Y}_B|\RV{D}_B\RV{I}_B}$ and $\RV{Y}_A\CI^e_{\prob{P}_C} (\RV{D}_{A^C},\RV{I}_{A^C})|(\RV{D}_A,\RV{I}_A,\RV{C})$. Roughly: holding any subsequence of treatments and individuals fixed, the joint distribution of consequences is the same no matter where they appear in the sequence of experiments and no matter what treatments or individuals appear elsewhere.

The analyst constructing the model has no particular knowledge about any identifier, and so \emph{for any choice} reasons the associated model should be invariant to permutations of identifiers. This implies $\RV{Y}\CI^e_{\prob{P}_C} \RV{I}|\RV{C}$ (see Lemma \ref{lem:ind}). The assumption that this holds given any choice can be tricky -- not only must the identifiers appear symmetric to the analyst constructing the model, but nothing breaking this symmetry may be learned from the choice $\alpha$, see the Example \ref{ex:bad_randomised_experiment} next and the discussion afterwards. In this case, the reasoning is supported by the fact that the rule for selecting $\alpha$ is known in advance.

For the deterministic subset $C'\subset C$, $\RV{YI}\CI^e_{\prob{P}_{C'}} \RV{D}|\RV{C}$ as $\RV{D}$ is deterministic for all elements of $C'$, and $\RV{YI}\CI^e_{\prob{P}_{C'}} \RV{C}|\RV{D}$ is also because restricted to this subset, $\RV{D}$ is a one-to-one function of $\RV{C}$. By application of Theorem \ref{th:cc_ind_treat}, $\prob{P}_{C'}^{\RV{Y}|\RV{D}}$ is causally contractible, and by application of Corollary \ref{cor:extend_to_randomised}, so is $\prob{P}_{C}^{\RV{Y}|\RV{D}}$
\end{example}

Permutability of identifiers can fail when the rule for selecting $\alpha$ is not known in advance. The following example is extreme in order to illustrate the issue clearly. The distinction between the analyst and the administrator is also intended to make the example easier to parse. The key point is that, when the rule for selecting $\alpha$ is not known in advance, symmetries that are apparent at the time of model construction do not necessarily hold for every choice $\alpha$, and this remains true if e.g. the selection of choices leads to less extreme confounding or the analyst and the administrator are actually the same person.

\begin{example}\label{ex:bad_randomised_experiment}
The variables and treatment administration protocol is precisely as in Example \ref{ex:randomised_experiment} \emph{except} the administrator rather than the analyst selects the value of $\alpha$.

The administrator 

This change gives the administrator a degree of freedom: they can choose some $\alpha$ from the subset $F$, and moreover they can potentially do this with access to knowledge that the analyst lacks. Thus while they might all seem the same to the analyst, they will not all look the same to the administrator.

In particular, suppose the administrator examines every patient before making the final selection of $\alpha\in F$ and gets a covariate $\RV{U}_i$ for each one. Then the choice from the administrator's point of view is not elements of $F$ but maps $X^{\mathbb{N}}\to F$. The administrator may introduce confounding for the effect of $\RV{D}$ on $\RV{Y}$. Suppose (for the sake of a simple example) they are certain to introduce confounding of precisely the form in example \ref{ex:confounding} -- that is, they will either force $\RV{D}_i$ perfectly correlated with $\RV{U}_i$ or perfectly anticorrelated, and $\RV{Y}_i$ is always perfectly correlated with $\RV{U}_i$. The assumption that fails is \emph{label permutability}. \todo[inline]{Finish this} In particular, if $\prob{P}_\alpha^{\RV{YD}|\RV{I}}(1,1|i)=1$ and $\prob{P}_\alpha^{\RV{YD}|\RV{I}}(1,1|j)=0$, then . If the analyst anticipates this, then they may reject label permutability because, by hypothesis, the administrator may preferentially assign some 
\end{example}

\citet{kasy_why_2016} mentions that causal identifiability requires a ``controlled experiment'', not necessarily a ``randomised controlled experiment''. The preceding two examples establish that a ``controlled experiment'' means that the inputs to the experiment are under the control \emph{of the analyst}. Even that doesn't quite capture it; if the analyst and the administrator are the same person and they nevertheless perform some ad-hoc ``randomisation'' of decision functions, the points raised in Example \ref{ex:bad_randomised_experiment} still apply. More precisely, if there is a decision procedure involving $\prob{P}_C$ and a choice $\alpha$ output by this procedure, then the inputs $\RV{D}$ must be controlled by this $\alpha$. 

This requirement is not always satisfied for experiments in practice. Furthermore, this describes a situation in which the analyst is both the experiment designer and the consumer of the experiment data (in order to make some optimal decision down the line). If an experiment is conducted and the results are published in a paper for some reader to review, then the reader is in the position of the person who is here called the ``analyst'' -- while they don't get to design the experiment, it is on them to make use of the published experimental data in order to make decisions of their own. Importantly, they \emph{never} control the experimental inputs at all.

Randomisation can play an important role in situations like this where the analyst has little control over experimental inputs, and so under normal circumstances may not accept causal contractibility. An informal argument for this is: suppose I accept causal contractibility for an experiment I control with either deterministic or random inputs. I don't accept it for a similar experiment you control with deterministic inputs, because I think there's some chance you knew something I didn't when you chose the inputs. However, whatever you do or don't know, a random input controlled by you is as good as a random input controlled by me, so if I accept causal contractibility in the latter case then I also accept it in the former.

Dropping the assumption $\RV{YI}\CI^e_{\prob{P}_C} \RV{C}|\RV{D}$ means that, in general, one or both of $\prob{P}_C^{\RV{Y}|\RV{D}}$ or $\prob{P}_C^{\RV{Y}|\RV{ID}}$ may be ill-defined (note that the independence is merely a sufficient condition, not a necessary condition for these uniform conditional probabilities). The condition $\RV{YI}\CI^e_{\prob{P}_C} \RV{C}|\RV{D}$ alone also does \emph{not} imply the conclusion of Theorem \ref{th:cc_ind_treat}. 

Constructing the following example requires the hypotheses that any given identifier $i\in\mathbb{N}$ could be associated with one of two input-output maps $D\kto Y$. Thus the space of hypotheses is a sequence of binary values $H=\{0,1\}^{\mathbb{N}}$. Equipped with the product topology, $H$ is a countable product of separable, completely metrizable spaces and is therefore also separable and completely metrizable \citep[Thm. 16.4,Thm. 24.11]{willard_general_1970}. Thus $(H,\mathcal{B}(H))$ is a standard measurable space and, because it is uncountable, it is isomorphic to $([0,1],\mathcal{B}([0,1]))$.

\begin{example}
Take $Y=C=D=\{0,1\}$ and take $(H,\sigalg{H})$ to be $\{0,1\}^{\mathbb{N}}$ equipped with the product topology. For any $i\neq 1$, $\RV{Y}_i\RV{I}_i\RV{D}_i\CI^e_{\prob{P}_C} \RV{C}$, while $\prob{P}_\alpha^{\RV{D}_1}=\delta_\alpha$ and $\RV{I}_i\CI^e_{\prob{P}_C} \RV{C}$.

$\RV{YI}\CI^e_{\prob{P}_C} \RV{C}|\RV{D}$ follows from the fact that $\RV{C}$ can be (almost surely) written as a function of $\RV{D}$.

For all $i,\in \mathbb{N}$, $y,d\in \{0,1\}$, $h\in H$ set
\begin{align}
    \prob{P}_C^{\RV{Y}_i|\RV{H}\RV{I}_i\RV{D}_i}(y|h,j,d) &= \delta_1(p(j,h))\delta_d(y) + \delta_0(p(j,h))\delta_{1-d}(y)
\end{align}
where $p(j,h)$ projects the $j$-th component of $h$. That is, if $h$ maps $j$ to 1, $\RV{Y}$ goes with $\RV{D}$ while if $h$ maps $j$ to $0$, $\RV{Y}$ goes opposite $\RV{D}$. Suppose also 
\begin{align}
    \RV{Y}_i\CI_{\prob{P}_C}^e (\RV{X}_{<i},\RV{Y}_{<i},\RV{I}_{<i},\RV{C})|(\RV{X}_i,\RV{Y}_i,\RV{H})
\end{align}
Then $\prob{P}_C^{\RV{Y}|\RV{DI}}$ is causally contractible. Set $\prob{P}_{C}^{\RV{H}}$ to be the uniform measure on $(H,\sigalg{H})$ and for $i>1$
\begin{align}
    \prob{P}_C^{\RV{D}_i|\RV{I}_i\RV{H}}(d|j,h) &= \delta_{p(j,h)}(d)
\end{align}
that is, if $h$ maps $j$ to 1, $\RV{D}$ is 1 while if $h$ maps $j$ to $0$, $\RV{D}$ is 0. This also implies
\begin{align}
    \prob{P}_C^{\RV{I}_i|\RV{D}_i\RV{H}}(p(\cdot,h)^{-1}(d)|d,h) &= 1\label{eq:all_eq_d}
\end{align}

Then, for $i>1$
\begin{align}
    \prob{P}_\alpha^{\RV{Y}_i|\RV{H}\RV{D}_i}(y|h,d) &= \sum_{j\in \mathbb{N}} \delta_1(p(j,h))\delta_d(y)\prob{P}_C^{\RV{I}_i|\RV{D}_i\RV{H}}(j|d,h) + \delta_0(p(j,h))\delta_{1-d}(y)\prob{P}_C^{\RV{I}_i|\RV{D}_i\RV{H}}(j|d,h)\\
    &= \sum_{j\in \mathbb{N}} \delta_1(d)\delta_d(y)\prob{P}_C^{\RV{I}_i|\RV{D}_i\RV{H}}(j|d,h) + \delta_0(d)\delta_{1-d}(y)\prob{P}_C^{\RV{I}_i|\RV{D}_i\RV{H}}(j|d,h)&\text{by Eq \ref{eq:all_eq_d}}\\
    &= \delta_1(y)\\
    \implies \prob{P}_\alpha^{\RV{Y}_i|\RV{D}_i}(y|d) &= \delta_1(y)
\end{align}

For $q\in I$, set
\begin{align}
    \prob{P}_C^{\RV{I}|\RV{H}}(q|h)&= \begin{cases}
        0.5 & q=(1,2,3,4,...) \text{ or } (1,3,2,4,...)\\
        0&\text{otherwise}
    \end{cases}
\end{align}
and set
\begin{align}
    \prob{P}_C^{\RV{H}|\RV{D}}(h) &= \begin{cases}
        0.5 & h=(0,1,0,1,1,...)\text{ or }h=(0,0,1,1,1,...)\\
        0 &\text{otherwise}
    \end{cases}
\end{align}
Let $\overline{H}$ be the support of $\prob{P}_C^{\RV{H}|\RV{D}}(h)$.

Then for $i=1$
\begin{align}
    \prob{P}_\alpha^{\RV{Y}_1|\RV{D}_1}(y|h,d) &= \sum_{h\in H} \sum_{j\in \mathbb{N}} \prob{P}_\alpha^{\RV{I}_1|\RV{D}_1\RV{H}}(j|d,h)\prob{P}_C^{\RV{H}|\RV{D}_1}(h|d)\left(\delta_1(p(j,h))\delta_d(y) + \delta_0(p(j,h))\delta_{1-d}(y)\right)\\
    &= \sum_{h\in \overline{H}} 0.5( \delta_1(p(1,h))\delta_d(y) + \delta_0(p(1,h))\delta_{1-d}(y))\\
    &= \delta_{1-d}(y))\\
    &\neq  \prob{P}_\alpha^{\RV{Y}_i|\RV{D}_i}(y|h,d) & i\neq 1
\end{align}
Thus $\prob{P}_C^{\RV{Y}|\RV{D}}$ is not causally contractible by Theorem \ref{th:equal_of_condits}. 

However, given any finite permutation $\rho:\mathbb{N}\to\mathbb{N}$
\begin{align}
    \prob{P}_\alpha^{\RV{Y}|\RV{I}}(y|q) &= \sum_{h\in \overline{H}}\sum_{d\in\{0,1\}^{\mathbb{N}}} \prod_{i\in \mathbb{N}} \prob{P}_C^{\RV{Y}_i|\RV{I}_i\RV{D}_i\RV{H}}(y_i|q_i,d_i,h) \prob{P}_\alpha^{\RV{D}_i|\RV{I}_i\RV{H}}(d_i|q_i,h)\prob{P}_C^{\RV{H}}(h)\\
    &= \delta_{1-\alpha}(y_1)\delta_{(1)_{i\in\mathbb{N}}}(y_{>1})\\
    &= \prob{P}_\alpha^{\RV{Y}|\RV{I}}(y|\rho^{-1}(q))\\
    &= \kernel{F}_{\rho}\prob{P}_\alpha^{\RV{Y}|\RV{I}}(y|q)
\end{align}
\end{example}

\subsection{Prior work}

Assumption 1, that causal contractibility is satisfied jointly conditioning on decisions $\RV{D}$ and identifers $\RV{I}$, seems to be a background assumption in the literature, while assumptions similar to the second two are discussed explicitly. For example, \citet{greenland_identifiability_1986} explain
\begin{quote}
    Equivalence of response type may be thought of in terms of exchangeability of individuals: if the exposure states of the two individuals had been exchanged, the same data distribution would have resulted.
\end{quote}
Note that exchanging individuals involved in an experiment and exchanging the individuals' exposure states are two different things, and the former doesn't imply the latter -- for example, there might be some background trend such that individuals treated later experience different outcomes to individuals treated at the start. Assumptions 1 and 2 \emph{together} imply that permuting identifiers or permuting decisions both lead to the same distribution.

\citet{dawid_decision-theoretic_2020} suggests (with some qualifications) that ``post-treatment exchangeability'' for a decision problem regarding taking aspirin to treat a headache may be acceptable if the data are from
\begin{quote}
    A group of individuals whom I can regard, in an intuitive sense, as similar to myself, with headaches similar to my own.
\end{quote}
This seems on the face of it similar to assumption 2: that I can permute the identifies ``me'' and ``someone else'' without changing the model.

Finally, \citet{rubin_causal_2005} discusses two separate assumptions to justify causal identifiability:
\begin{quote}
    indexing of the units is, by definition, a random permutation of $1,..., N$, and thus any distribution on the science must be row-exchangeable [...] The second critical fact is that if the treatment assignment mechanism is ignorable (e.g., randomized), then when the expression for the assignment mechanism (2) is evaluated at the observed data, it is free of dependence on $Y_{mis}$
\end{quote}
Here ``the science'' means (roughly) \emph{the response function of each individual}, and exchangeability of these response functions is a similar assumption to permutability of individual identifiers (though we don't derive the exact correspondence here). Rubin's second condition is that treatment assignment is ignorable. Like Assumption 3, this assumption limits ``how much we can learn from the treatment assignment'', but again we don't derive the exact correspondence. Note that in Rubin's scheme the preliminary assumption of \emph{stable unit-treatment values} is made in order to establish the existence of individual response functions, which plays a similar role to Assumption 1. 

Rubin's result also differs substantially from this one in that it applies to \emph{identification of potential outcomes in randomised experiments}, while this result is about the existence of response conditionals \emph{when there are different choices that can be made}.

\subsubsection{Example: body mass index}

Given a sequential just-do model $(\prob{P}_C,(\RV{B},\RV{I}),\RV{Y})$ with $\RV{B}:=(\RV{B}_i)_{i\in M}$ representing body mass index of individual $\RV{I}_i$ and $\RV{Y}:=(\RV{Y}_i)_{i\in M}$ representing health outcomes of interest for the same individual, \citet{hernan_does_2008} noted that there are multiple different choices that can influence an individual's body mass index $\RV{B}_i$ in the same way. Thus $\RV{YI}\CI^e_{\prob{P}_C} \RV{C}|\RV{B}$ might generally be rejected, and so there may be no uniform conditional $\prob{P}_C^{\RV{Y}|\RV{IB}}$. In this case, $\prob{P}_C^{\RV{Y}|\RV{IB}}$ cannot be causally contractible because it doesn't exist.

Suppose instead a model $(\prob{P}_C,(\RV{D},\RV{I}),(\RV{B},\RV{Y}))$ is given, with $\RV{D}=(\RV{D}_i)_{i\in M}$ representing ``decisions'', appropriately fine-grained to satisfy
\begin{align}
    &\RV{YBI}\CI^e_{\prob{P}_C} \RV{C}|\RV{D}\\
    &\RV{YBI}\CI^e_{\prob{P}_C} \RV{D}|\RV{C}
\end{align}
and $\prob{P}_C^{\RV{YB}|\RV{ID}}$ causally contractible. Then by Theorem \ref{th:cc_ind_treat} $\prob{P}_C^{\RV{Y}|\RV{BD}}$ is also causally contractible. In general, there may be some $U\subset H$ such that for any $h\in U$ 
\begin{align}
    \prob{P}_C^{\RV{Y}_i|\RV{B}_i\RV{D}_i\RV{H}}(y|b,d,h) &= \prob{P}_C^{\RV{Y}_i|\RV{B}_i\RV{H}}(y|b,h)\label{eq:conditional_conditional_independence}
\end{align}
then, \emph{conditioning on }$\RV{H}\in U$, the resulting $\prob{P}_{C,\RV{H}\in U}^{\RV{Y}|\RV{B}}$ is causally contractible.
\todo[inline]{Defining conditioning}
So it may be possible to derive the fact that there is a independent and identical response conditional $\prob{P}_{C,\RV{H}\in U}^{\RV{Y}_i|\RV{H}\RV{B}_i}$ if $\RV{H}\in U$ is implied by available data, even if it is not assumed outright.

\section{Response conditionals with data-dependent actions}\label{sec:data_dependent}

The results of the previous section concern ``just-do'' models where actions have not dependence on previous data. Decision problems of interest actually have actions that depend on data -- what's really wanted are ``see-do'' models of some variety (see Definition \ref{def:see_do_model}). Here, Theorem \ref{th:data_ind_CC} is generalised to sequential see-do models with the use of \emph{probability combs}.

To begin with an example, consider a probability set $(\prob{P}_C,\RV{D},\RV{Y})$ with $\RV{D}:=(\RV{D}_i)_{i\in\mathbb{N}}$ and $\RV{Y}:=(\RV{Y}_i)_{i\in\mathbb{N}}$ as usual, and take a subsequence $(\RV{D}_i,\RV{Y}_i)_{i\in [2]}$ of length 2. Suppose $\prob{P}_C$ features independent and identical response conditionals in the sense that the following holds
\begin{align}
    \RV{Y}_i&\CI^e_{\prob{P}_C} (\RV{Y}_{<i},\RV{D}_{<i},\RV{C})|\RV{H}\RV{D}_i&\forall i\in \mathbb{N}\\
    \land \RV{H} &\CI^e_{\prob{P}_C} \RV{D} C\\
    \land \prob{P}_C^{\RV{Y}_i|\RV{H}\RV{D}_i} &= \prob{P}_C^{\RV{Y}_0|\RV{H}\RV{D}_0} & \forall i\in \mathbb{N}
\end{align}

Then, for arbitrary $\alpha\in C$
\begin{align}
    \prob{P}_\alpha^{\RV{Y}_{[2]}} &= \tikzfig{response_conditional_comb}
\end{align}
note that $\RV{D}_2$ depends on $\RV{Y}_1$ and $\RV{D}_1$. Instead of multiplying by a distribution over $(\RV{D}_1,\RV{D}_2)$, $\prob{P}_\alpha^{\RV{D}_2|\RV{Y}_1\RV{D}_1}$ has been ``inserted'' between the response conditionals $\prob{P}_C^{\RV{Y}_1|\RV{D}_1\RV{H}}$ and $\prob{P}_C^{\RV{Y}_2|\RV{D}_2\RV{H}}$. A comb is a Markov kernel that yields a probability distribution when another Markov kernel of appropriate type is inserted in this manner.

Given $\prob{P}_C^{\RV{Y}_1|\RV{D}_1\RV{H}}$ and $\prob{P}_C^{\RV{Y}_2|\RV{D}_2\RV{H}}$, define the comb
\begin{align}
    \prob{P}_C^{\RV{Y}_{[2]}\combbreak \RV{D}_{[2]}} := \tikzfig{causally_contractible_comb}
\end{align}
then $\prob{P}_C^{\RV{Y}_{[2]}\combbreak \RV{D}_{[2]}}$ is causally contractible. $\prob{P}_C^{\RV{Y}_{[2]}\combbreak \RV{D}_{[2]}}$ is \emph{not} a uniform conditional probability; in general 
\begin{align}
    \prob{P}_\alpha^{\RV{D}_1\RV{D}_2} \prob{P}_C^{\RV{Y}_{[2]}\combbreak \RV{D}_{[2]}}\neq \prob{P}_\alpha^{\RV{Y}_1\RV{Y}_2}
\end{align}

\subsection{Combs}\label{sec:def_combs}

Combs generalise conditional probabilities in this sense: given a conditional distribution and a marginal distribution of the right type, joining them together (with the semidirect product\ref{def:copyproduct}) I get a marginal distribution of a different type. Define ``1-combs'' as conditional probabilities and ``0-combs'' as conditional distributions. Then the previous observation can be restated as: given a 1-comb and a 0-comb of the right type,  joining them together yields a 0-comb of a different type. Higher order combs generalise this: given an $n$-comb and an $n-1$-comb of the right type, joining them yields an $n-1$ comb.

Joining combs uses an ``insert'' operation (Definition \ref{def:insert_discrete}).  A graphical depiction of this operation gives some intuition for why it is called ``insert'':
% \begin{align}
%     \prob{P}_\alpha^{\RV{XY}}&=\prob{P}_\alpha^{\RV{X}}\cprod\prob{P}_C^{\RV{Y}|\RV{X}}\\
%     &= \tikzfig{conditional_semidirect_product}
% \end{align}
% and the insert operation looks like
\begin{align}
    \prob{P}_\alpha^{\RV{Y}_{1}\RV{D}_2\RV{Y}_2|\RV{D}_1}&=\text{insert}(\prob{P}_\alpha^{\RV{D}_2|\RV{D}_1\RV{Y}_1},\prob{P}_C^{\RV{Y_{[2]}}\combbreak\RV{D}_{[2]}})\\
    &= \tikzfig{comb_insert_complicated}\label{eq:comb_insert_complicated}\\
    &= \tikzfig{comb_insert_gettingsimpler}\\
    &= \tikzfig{comb_insert_simple}\label{eq:comb_insert_simple}
\end{align}
While Equation \ref{eq:comb_insert_complicated} is a well-formed string diagram in the category of Markov kernels, Equation \ref{eq:comb_insert_simple} is not. In the case that all the underlying sets are discrete, Equation \ref{eq:comb_insert_simple} can be defined using an extended string diagram notation appropriate for the category of real-valued matrices \citep{jacobs_causal_2019}, though we do not introduce this extension here.

Formal definitions of finite and infinite combs follow, which will be used in Section \ref{sec:data_dependent_representation} to generalise Theorem \ref{th:data_ind_CC} to the data-dependent case.

\begin{definition}[Uniform $n$-Comb]
Given a probability set $\prob{P}_C$ with variables $\RV{Y}_i:\Omega\to Y$, $\RV{D}_i:\Omega\to D$ for $i\in [n]$ and uniform conditional probabilities $\{\RV{P}_C^{\RV{Y}_i|\RV{D}_{[i]}\RV{Y}_{[i-1]}}|i\in [n]\}$, the uniform $n$-comb $\prob{P}_C^{\RV{Y}_{[n]}\combbreak \RV{D}_{[n]}}:D^n\kto Y^n$ is the Markov kernel given by the recursive definition
\begin{align}
    \prob{P}_C^{\RV{Y}_{1}\combbreak \RV{D}_{1}} &= \prob{P}_C^{\RV{Y}_1|\RV{D}_1}\\
    \prob{P}_C^{\RV{Y}_{[m]}\combbreak \RV{D}_{[m]}} &= \tikzfig{comb_inductive}
\end{align}
\end{definition}

\begin{definition}[Uniform $\mathbb{N}$-comb]
Given a probability set $\prob{P}_C$ with variables $\RV{Y}_i:\Omega\to Y$ and $\RV{D}_i:\Omega\to D$ for $i\in \mathbb{N}$ and uniform conditional probabilities $\{\RV{P}_C^{\RV{Y}_i|\RV{D}_{[i]}\RV{Y}_{[i-1]}}|i\in \mathbb{N}\}$, the uniform $\mathbb{N}$-comb $\prob{P}_C^{\RV{Y}_{\mathbb{N}}\combbreak \RV{D}_{\mathbb{N}}}:D^\mathbb{N}\kto Y^\mathbb{N}$ is the Markov kernel such that for all $n\in \mathbb{N}$
\begin{align}
    \prob{P}_C^{\RV{Y}_{\mathbb{N}}\combbreak \RV{D}_{\mathbb{N}}}(\mathrm{id}_{Y^{n}}\otimes \mathrm{del}_{Y^{\mathbb{N}}}) &= \prob{P}_C^{\RV{Y}_{[n]}\combbreak \RV{D}_{[n]}}\otimes \mathrm{del}_{Y^{\mathbb{N}}}
\end{align}
\end{definition}

\begin{theorem}[Existence of $\mathbb{N}$-combs]
Given a probability set $\prob{P}_C$ with variables $\RV{Y}_i:\Omega\to Y$ and $\RV{D}_i:\Omega\to D$ for $i\in \mathbb{N}$ and uniform conditional probabilities $\{\RV{P}_C^{\RV{Y}_i|\RV{D}_{[i]}\RV{Y}_{[i-1]}}|i\in \mathbb{N}\}$, a uniform $\mathbb{N}$-comb $\prob{P}_C^{\RV{Y}_{\mathbb{N}}\combbreak \RV{D}_{\mathbb{N}}}:D^\mathbb{N}\kto Y^\mathbb{N}$ exists.
\end{theorem}

\begin{proof}
For each $n\in \mathbb{N}$ $m<n$, we have
\begin{align}
    \prob{P}_C^{\RV{Y}_{[n]}\combbreak \RV{D}_{[n]}}(\mathrm{id}_{Y^{n-m}})\otimes \mathrm{del}_{Y^m}) &= \prob{P}_C^{\RV{Y}_{[n-m]}\combbreak \RV{D}_{[n-m]}}\otimes \mathrm{del}_{Y^m}
\end{align}

Therefore the existence of $\prob{P}_C^{\RV{Y}_{\mathbb{N}}\combbreak \RV{D}_{\mathbb{N}}}$ is a consequence of Lemma \ref{lem:infinitely_extended_kernels}.
\end{proof}

For discrete sets, the insert operation has a compact definition

\begin{definition}[Comb insert - discrete]\label{def:insert_discrete}
Given an $n$-comb $\prob{P}_\alpha^{\RV{Y}_{[n]}\combbreak \RV{D}_{[n]}}$ and an $n-1$ comb $\prob{P}_\alpha^{\RV{D}_{[2,n]}|\RV{Y}_{[n-1]}}$, $(D,\sigalg{D})$ and $(Y,\sigalg{Y})$ discrete, for all $y_i\in Y$ and $d_i\in D$
\begin{align}
    \mathrm{insert}(\prob{P}_\alpha^{\RV{D}_{[2,n]}\combbreak \RV{Y}_{[n-1]}},\prob{P}_\alpha^{\RV{Y}_{[n]}\combbreak \RV{D}_{[n]}})(y_{[n]},d_{[2,n]}|d_1) &= \prob{P}_\alpha^{\RV{Y}_{[n]}\combbreak \RV{D}_{[n]}}(y_n|d_{[2,n]},d_1)\prob{P}_\alpha^{\RV{D}_{[1,n]}\combbreak\RV{Y}_{[n-1]}}(d_{[2,n]}|y_{[n-1]})
\end{align}
\end{definition}

\subsection{Response conditionals in models with data dependent actions}\label{sec:data_dependent_representation}

Theorem \ref{th:response_hdep} generalises Theorem \ref{th:data_ind_CC} to models $(\prob{P}_C,\RV{D},\RV{Y})$ with data-dependent actions, where instead the assumption that the uniform comb $\prob{P}_C^{\RV{Y}\combbreak \RV{D}}$ is causally contractible replaces the assumption that the conditional probability $\prob{P}_C^{\RV{Y}| \RV{D}}$ is causally contractible.

\begin{definition}[Sequential see-do model]
A \emph{sequential see-do model} is a triple $(\prob{P}_C,\RV{D},\RV{Y})$ where $\prob{P}_C$ is a probability set on $(\Omega,\sigalg{F})$, $\RV{D}$ is a sequence of ``inputs'' $\RV{D}:=(\RV{D}_i)_{i\in\mathbb{N}}$ and $\RV{Y}$ is a corresponding sequence of ``outputs'' $\RV{Y}=(\RV{Y}_i)_{i\in\mathbb{N}}$ where $\RV{D}_i:\Omega\to D$ and $\RV{Y}_i:\Omega\to Y$ and $\RV{Y}_i\CI^e_{\prob{P}_C} C|(\RV{D}_{[i]},\RV{Y}_{<i})$.
\end{definition}

\begin{theorem}[]\label{th:response_hdep}
Given a sequential see-do model $(\prob{P}_C',\RV{D}',\RV{Y}')$ on $(\Omega,\sigalg{F})$, then $\prob{P}_C^{\prime \RV{Y}'\combbreak \RV{D}'}$ is causally contractible if and only if there is a latent extension $\prob{P}_C$ of $\prob{P}_C'$ to $(\Omega\times H,\sigalg{F}\otimes\sigalg{Y}^{D\times\mathbb{N}})$ with hypothesis $\RV{H}:\Omega\times H\to H$ such that $\RV{Y}_i\CI^e_{\prob{P}_C'} (\RV{Y}_{<i},\RV{X}_{<i},C)|(\RV{X}_i,\RV{H})$ and $\prob{P}_C^{\RV{Y}_i|\RV{X}_i\RV{H}}=\prob{P}_C^{\RV{Y}_j|\RV{X}_j\RV{H}}$ for all $i,j\in \mathbb{N}$ and $\RV{H}\CI_{\prob{P}_C} (\RV{X},\RV{C})$.
\end{theorem}

\begin{proof}
If:
By assumption, there is some $\kernel{L}:H\times D\kto Y$ such that
\begin{align}
    \prob{P}_C^{\RV{Y}_i|\RV{H}\RV{D}_i} &= \kernel{L}
\end{align}
and $\RV{Y}_i\CI^e_{\prob{P}_C} (\RV{Y}_{<i},\RV{D}_{<i})|(\RV{D}_i,\RV{H})$. Thus
\begin{align}
    \prob{P}_C^{\RV{Y}_i|\RV{H}\RV{D}_i\RV{Y}_{<i}\RV{D}_{<i}} &= \kernel{L}\otimes \text{erase}_{Y^{i-1}\times D^{i-1}}
\end{align}
and so
\begin{align}
    \prob{P}_C^{\RV{Y}\combbreak \RV{D}} &= \tikzfig{do_model_representation}\label{eq:comb_representation_w_CI}
\end{align}
and so by Theorem \ref{th:ciid_rep_kernel}, $\prob{P}_C^{\RV{Y}\combbreak \RV{D}}$ is causally contractible.

Only if:
First, define the extension $\prob{P}_C$. From Theorem \ref{th:ciid_rep_kernel} and causal contractibility of $\prob{P}_C^{\prime \RV{Y}'\combbreak \RV{D}'}$ there is some $H$, $\mu\in \Delta(H)$ and $\kernel{L}:H\times D\kto Y$ such that
\begin{align}
    \prob{P}_C^{\prime \RV{Y}'\combbreak\RV{D}'} &= \tikzfig{do_model_representation_mu}
\end{align}
thus, by the definition of the comb insert operation
\begin{align}
    \prob{P}_\alpha^{\prime\RV{D}'_{[n]} \RV{Y}'_{[n]}} &= \prob{P}_\alpha^{\RV{D}_1}\odot \text{insert}(\prob{P}_\alpha^{\prime \RV{D}'_{[2,n]}\combbreak\RV{Y}'_{[n-1]}}, \prob{P}_C^{\prime \RV{Y}'_{[n]}\combbreak\RV{D}'_{[n]}}) 
\end{align}
Let
\begin{align}
    \prob{P}_C^{\RV{Y}_i|\RV{H}\RV{D}_i} &= \kernel{L}\label{eq:identical_response_assumption}
\end{align}
and let $\RV{Y}_i\CI^e_{\prob{P}_C} (\RV{Y}_{<i},\RV{D}_{<i})|(\RV{D}_i,\RV{H})$, and for all $\alpha$ set $\prob{P}_\alpha^{\RV{W}|\RV{DY}}=\prob{P}_\alpha^{\prime \RV{W}'|\RV{D'Y'}}$ for all $\RV{W}':\Omega\to W$ and $\prob{P}_\alpha^{\RV{D}_i|\RV{Y}_{<i}\RV{D}_{<i}}=\prob{P}_\alpha^{\prime \RV{D}_i'|\RV{Y}_{<i}'\RV{D}_{<i}''}$.

It remains to be shown that $\prob{P}_\alpha^{\RV{DY}}=\prob{P}_\alpha^{\prime \RV{DY}}$.

By Equation \ref{eq:identical_response_assumption} and $\RV{Y}_i\CI^e_{\prob{P}_C} (\RV{Y}_{<i},\RV{D}_{<i})|(\RV{D}_i,\RV{H})$, it follows (for identical reasons as Equation \ref{eq:comb_representation_w_CI}) that
\begin{align}
    \prob{P}_C^{\RV{Y}\combbreak \RV{D}} &= \tikzfig{do_model_representation}\\
    &= \tikzfig{do_model_representation_mu}\\
    &= \prob{P}_C^{\prime \RV{Y}'\combbreak\RV{D}'}
\end{align}

And so for all $n\in \mathbb{N}$
\begin{align}
    \prob{P}_\alpha^{\RV{D}_{[n]} \RV{Y}_{[n]}} &=  \prob{P}_\alpha^{\RV{D}_1}\odot \text{insert}(\prob{P}_\alpha^{ \RV{D}_{[2,n]}\combbreak\RV{Y}_{[n-1]}}, \prob{P}_C^{\prime \RV{Y}_{[n]}\combbreak\RV{D}_{[n]}}) \\
    &= \prob{P}_\alpha^{\RV{D}_1}\odot \text{insert}(\prob{P}_\alpha^{\prime \RV{D}'_{[2,n]}\combbreak\RV{Y}'_{[n-1]}}, \prob{P}_C^{\prime \RV{Y}'_{[n]}\combbreak\RV{D}'_{[n]}}) \\
    &= \prob{P}_\alpha^{\prime\RV{D}'_{[n]} \RV{Y}'_{[n]}}
\end{align}
\end{proof}

In contrast to the data-independent case where causal contractibility of $\prob{P}_C^{\RV{Y}|\RV{X}}$ implies the equivalence of all subsequence conditionals $\prob{P}_C^{\RV{Y}_A|\RV{X}_A}$ for all equally sized $A\subset\mathbb{N}$, a causally contractible comb $\prob{P}_C^{\RV{Y}\combbreak \RV{D}}$ does not generally imply that subsequence combs $\prob{P}_C^{\RV{Y}_A\combbreak \RV{D}_A}$ and $\prob{P}_C^{\RV{Y}_B\combbreak \RV{D}_B}$ are equivalent with $|A|=|B|$.


\subsection{Combs are the output of the ``fix'' operation}

There is a relationship between combs and the ``fix'' operation defined in \citet{richardson_nested_2017}. In particular, suppose we have a probability $\prob{P}_\alpha$ and a comb $\prob{P}_\alpha^{\RV{Y}_{[2]}|\RV{D}_{[2]}}$. Then (assuming discrete sets)
\begin{align}
    \prob{P}_\alpha^{\RV{Y}_{[2]}\combbreak \RV{D}_{[2]}}(y_1,y_2|d_1,d_2) &= \prob{P}_\alpha^{\RV{Y}_1|\RV{D}_1}(y_1|d_1)\prob{P}_\alpha^{\RV{Y}_2|\RV{D}_2}(y_2|d_2)\\
    &= \frac{\prob{P}_\alpha^{\RV{Y}_1|\RV{D}_1}(y_1|d_1)\prob{P}_\alpha^{\RV{D}_2|\RV{Y}_1\RV{D}_1}(d_2|y_1,d_1)\prob{P}_\alpha^{\RV{Y}_2|\RV{D}_2}(y_2|d_2)}{\prob{P}_\alpha^{\RV{D}_2|\RV{Y}_1\RV{D}_1}(d_2|y_1,d_1)}\\
    &= \frac{\prob{P}_\alpha^{\RV{Y}_{[2]}\RV{D}_2|\RV{D}_1}(y_1,y_2,d_2|d_1)}{\prob{P}_\alpha^{\RV{D}_2|\RV{Y}_1\RV{D}_1}(d_2|y_1,d_1)}
\end{align}
That is (at least in this case), the result of ``division by a conditional probability'' used in the fix operation is a comb. We speculate that the output of the fix operation is, in general, an $n$-comb, but we have not proven this.


\section{Weaker assumptions than causal contractibility}\label{sec:weaker_assumptions}

The results so far apply to purely observational models or to models where every ``input'' in the sequence is fixed at the point of choosing $\alpha$ (or a fixed random function is chosen at this point). Most of the interest in causal inference is how to use observational data -- which is plentiful -- to deduce consequences of choices. Suppose in the following that superscripter ``$o$'' refers to observational variables (obtained by some measurement procedure not responsive to choices) and ``$v$'' refers to interventional variables (obtained by some measurement procedure responsive to choices). That is $\RV{Y}^o:=(\RV{Y}_i^o)_{i\in \mathbb{N}}$ is a sequence of observantional variables, $\RV{Y}^v$ a sequence of interventional variables and $\RV{Y}^{o,v}:=(\RV{Y}^o_i,\RV{Y}^v_i)_{i\in\mathbb{N}}$ is a mixed sequence of both observational and interventional variables. $\RV{Y}_i^o$ and $\RV{Y}_i^v$ are assumed to take values in the same set $Y$.

One approach to bridging the gap between observations and interventions is to assume ``causal sufficiency'', which is tantamount (in the data-independent case) to assuming causal contractibility of $\prob{P}_C^{\RV{Y}^{o,v}|\RV{X}^{o,v}\RV{D}^{o,v}}$ with $\RV{D}^v$ responsive to choices and $\RV{X}^v$ unresponsive (see Example \ref{ex:backdoor}). As discussed, this is rarely a reasonable assumption -- it implies interchangeability between observational and interventional samples.

A weaker assumption that is often adopted is to consider models satisfying causal contractibility with respect to $\prob{P}_C^{\RV{Y}^{o,v}|\RV{U}^{o,v}\RV{D}^{o,v}}$, where $\RV{U}^{o,v}$ is unobserved. That is, while $\RV{U}^{o,v}$ appears in the model, it is not associted with any measurement procedure. This model still asserts that $(\RV{U}^o_i,\RV{X}^o_i,\RV{Y}^o_i)$ triples are interchangeable with $(\RV{U}^v_i,\RV{X}^v_i,\RV{Y}^v_i)$ triples, but neither of these are measurement outcomes. On the other hand, $(\RV{D}_i^o,\RV{Y}_i^o)$ pairs are not generally interchangeable with $(\RV{D}_i^v,\RV{Y}_i^v)$.

Consider models that satisfy causal contractibility with respect to $\prob{P}_C^{\RV{Y}^{o,v}|\RV{W}^{o,v}}$, where no comment is made about whether $\RV{W}^{o,v}$ is observed, unobseved or some function of observed and unobserved variables. This is a generalisation of the class of models discussed in the previous paragraph.  In isolation, this assumption is not especially interesting -- for example, the support of $\RV{W}^{o}_i$ and $\RV{W}^v_i$ might be disjoint. Suppose also, then, that $W$ is finite and $\RV{W}^o_i$ has full support. This assumption amounts to the assumption that, no matter what choice is made, ``nothing truly new can be done'' (which we call ``Ecclesiastes' assumption''\footnote{Ecclesiastes 1:9 reads ``Everything that happens has happened before; nothing is new, nothing under the sun.''\citep{noauthor_holy_1995}}). More precisely, for any choice $\alpha\in C$ and any consequence $\RV{Y}_i^v$, there is a random subsequence $\RV{Q}$ of indices $(1,2,3,....)$ such that the distribution $\prob{P}_\alpha^{\RV{Y}^{o,v}}$ is unchanged by permutations that only swap elements in the sequence $(RV{Y}^o_\RV{Q},\RV{Y}^v_i)$.

\begin{theorem}\label{th:condit_exchange}
Given just-do model $\prob{P}_C$ with $\prob{P}_C^{\RV{Y}^{o,v}|\RV{W}^{o,v}}$ causally contractible, $W$ finite and $\prob{P}_C^{\RV{W}^o|\RV{H}}(w|h)>0$ for all $w,h$, define $q:W^{\mathbb{N}}\times W\to (\{*\}\cup \mathscr{P}(\mathbb{N})$ by 
\begin{align}
    q:((w^o_j)_{\mathbb{N}},w^v_i)&\mapsto \{j|w^o_j=w^v_i\}
\end{align}
and take $\RV{Q}:=q\circ(\RV{W}^o,\RV{W}^v_i)$ for arbitrary $i\in \mathbb{N}$. For an index set $U\in\mathbb{N}$ Take $\text{swap}_{\cdot}:Y^{\mathbb{N}}\times Y^{\mathbb{N}}\to Y^{\mathbb{N}}\times Y^{\mathbb{N}}$ to be an arbitrary finite swap that acts as the identity on all indices $(j,x)\not\in \RV{Q}\times \{o\}\cup\{(i,v)\}$. Then $\prob{P}^{\RV{Y}^{o}\RV{Y}^v_i}\text{swap}_{\RV{Q}} = \prob{P}^{\RV{Y}^{o}\RV{Y}^v_i}$.
\end{theorem}

\begin{proof}
Note that for $B_j\in \sigalg{W}$, where $\rho_q:\mathbb{N}\times\{i,v\}\to \mathbb{N}\times\{i,v\}$ is the permutation function associated with $\text{swap}_{q}$
\begin{align}
    \prob{P}_\alpha^{\RV{W}^o\RV{W}^v_i}\text{swap}_{\RV{Q}} (\bigtimes_{j\in\mathbb{N}} B_j) &= \int_{W^{\mathbb{N}}}\int_{\mathscr{P}(\mathbb{N})} \prod_{k\not\in q\times\{o\}\cup\{(i,v)\}} \delta_{w_k}(B_k) \prod_{l\in q\times\{o\}\cup\{(i,v)\}} \delta_{\rho_q(w_l)} (B_l) \prob{P}_\alpha^{\RV{Q}|\RV{W}^o\RV{W}^v_i}(\mathrm{d}q|w)\prob{P}_\alpha^{}(\mathrm{d}w)\\
    &= \int_{W^{\mathbb{N}}}\int_{\mathscr{P}(\mathbb{N})} \prod_{k\not\in q\times\{o\}\cup\{(i,v)\}} \delta_{w_k}(B_k) \prod_{l\in q\times\{o\}\cup\{(i,v)\}} \delta_{w_l} (B_l) \prob{P}_\alpha^{\RV{Q}|\RV{W}^o\RV{W}^v_i}(\mathrm{d}q|w)\prob{P}_\alpha^{}(\mathrm{d}w)\label{eq:all_the_same}\\
    &= \prob{P}_\alpha^{\RV{W}^o\RV{W}^v_i}
\end{align}
where Eq. \ref{eq:all_the_same} follows from the fact that for every $k,l\in q\times\{o\}\cup\{(i,v)\}$, $w_k=w_l$.

Thus for $A\in \sigalg{Y}^{\mathbb{N}}$
\begin{align}
    \prob{P}_\alpha^{\RV{Y}^{o}\RV{Y}^v_i}\text{swap}_{\RV{Q}}(A) &= [\prob{P}_\alpha^{\RV{W}^o\RV{W}^v_i} \prob{P}_\alpha^{\RV{Y}^{o}\RV{Y}^v_i|\RV{Q}\RV{W}^o\RV{W}^v_i}\text{swap}_{\RV{Q}}](A)\\
    &= [\prob{P}_\alpha^{\RV{W}^o\RV{W}^v_i} \text{swap}_{\RV{Q}^{-1}} \prob{P}_\alpha^{\RV{Y}^{o}\RV{Y}^v_i|\RV{W}^o\RV{W}^v_i}\text{swap}_{\RV{Q}}](A)\\
    &= \prob{P}_\alpha^{\RV{Y}^{o}\RV{Y}^v_i}\label{eq:by_cc1}
\end{align}
Where Eq. \ref{eq:by_cc1} follows from causal contractibility of $\prob{P}_\alpha^{\RV{Y}^{o}\RV{Y}^v_i|\RV{W}^o\RV{W}^v_i}$.
\end{proof}

It also follows from Ecclesiastes' assumption and finite $W$ that if some $\RV{X}_i^o$, $\RV{Z}_i^o$ are \emph{deterministically} related given $\RV{W}$, then $\prob{P}_C^{\RV{Z}|\RV{X}}$ is causally contractible.

\begin{theorem}\label{th:det_obs_to_cons}
Given just-do model $\prob{P}_C$ with $\prob{P}_C^{\RV{X}^{o,v}\RV{Z}^{o,v}|\RV{W}^{o,v}}$ causally contractible, $W$ finite and $\prob{P}_C^{\RV{W}^o_i|\RV{H}}(w|h)>0$ for all $w,h$, if $\prob{P}_C^{\RV{Z}^o_0|\RV{X}^o_0\RV{H}}$ is deterministic then $\prob{P}_C^{\RV{Z}^{o,v}|\RV{X}^{o,v}}$ is causally contractible.
\end{theorem}

\begin{proof}
Because $\prob{P}_\alpha^{\RV{W}_0^o}\prob{P}_C^{\RV{Z}^o_0|\RV{X}^o_0\RV{W}^o_0\RV{H}}$ is deterministic, so is $\prob{P}_C^{\RV{Z}_0|\RV{X}_0\RV{W}_0\RV{H}}$.

Fix $h\in H$.  Suppose there is some $w,w'\in W$ such that
\begin{align}
    \prob{P}_C^{\RV{Z}_0|\RV{X}_0\RV{W}_0\RV{H}}(A|x,w,h) &\neq \prob{P}_C^{\RV{Z}_0|\RV{X}_0\RV{W}_0\RV{H}}(A|x,w',h)
\end{align}
then, by determinism, we can assume without loss of generality
\begin{align}
    \prob{P}_C^{\RV{Z}_0|\RV{X}_0\RV{W}_0\RV{H}}(A|x,w,h) = 1\\
    \prob{P}_C^{\RV{Z}_0|\RV{X}_0\RV{W}_0\RV{H}}(A|x,w',h) = 0
\end{align}
but $W$ is finite and $\prob{P}_C^{\RV{W}^o_i|\RV{H}}(w|h)>0$ for all $w$, so there is some $a>0$ such that $\prob{P}_C^{\RV{W}^o_i|\RV{H}}(w|h)\geq a$ for all $w$, and so
\begin{align}
    a \leq \sum_{w\in W} \prob{P}_C^{\RV{W}^o_0|\RV{X}^o_0,\RV{H}}(w|x,h)\prob{P}_C^{\RV{Z}_0|\RV{X}_0\RV{W}_0\RV{H}}(A|x,w,h)\leq 1-a
\end{align}
contradicting determinism of $\prob{P}_C^{\RV{Z}^o_0|\RV{X}^o_0\RV{H}}$.

Thus for all $w,w'$
\begin{align}
    \prob{P}_C^{\RV{Z}_0|\RV{X}_0\RV{W}_0\RV{H}}(A|x,w,h) &= \prob{P}_C^{\RV{Z}_0|\RV{X}_0\RV{W}_0\RV{H}}(A|x,w',h)
\end{align}
i.e. $\RV{Z}_0\CI^e_{\prob{P}_C} (\RV{W}_0,\RV{C})|(\RV{X}_0,\RV{H})$. But then there is some $\prob{P}_C^{\RV{Z}_0|\RV{X}_0\RV{H}}$ such that
\begin{align}
    \prob{P}_C^{\RV{Z}_0|\RV{X}_0\RV{W}_0\RV{H}} &= \prob{P}_C^{\RV{Z}_0|\RV{X}_0\RV{H}}\otimes \text{erase}_W\\
    \implies \prob{P}_\alpha^{\RV{Z}^v_i|\RV{X}^v_i\RV{H}} &= \prob{P}_C^{\RV{Z}_0|\RV{X}_0\RV{H}}
\end{align}
\end{proof}

Theorem \ref{th:det_obs_to_cons} doesn't hold in the case of approximate determinism, however. Intuitively, approximate determinism can hold if there is some value of $\RV{W}$ for which $\RV{Z}$ is not conditionally independent given $\RV{H}$ and $\RV{X}$, but it only ocurrs very rarely in observations. On the other hand, values of $\RV{W}$ rare in observations might, under some choices, become common. 

\begin{example}
Say $\prob{P}_C^{\RV{Z}^o_i|\RV{X}^o_i\RV{H}}$ is \emph{approximately deterministic} if $\prob{P}_C^{\RV{Z}^o_i|\RV{X}^o_i\RV{H}}(A|x,h)\in [0,\epsilon]\cup[1-\epsilon,1]$ for all $A\in\sigalg{Z}$, $x,h\in X\times H$.

Take $Z=X=W=H=\{0,1\}$. Set
\begin{align}
    \prob{P}_C^{\RV{Z}_0|\RV{X}_0\RV{W}_0\RV{H}}(1|1,1,1) = 1\\
    \prob{P}_C^{\RV{Z}_0|\RV{X}_0\RV{W}_0\RV{H}}(1|1,0,1) = 0
\end{align}
and
\begin{align}
    \prob{P}_C^{\RV{W}^o_0|\RV{H}}(1|1)=1-\epsilon
\end{align}
then
\begin{align}
    \prob{P}_C^{\RV{Z}_0|\RV{X}_0\RV{H}}(1|1,1) = 1-\epsilon
\end{align}
however, suppose there is some $\alpha$ such that
\begin{align}
    \prob{P}_\alpha^{\RV{W}^v_i|\RV{H}}(1|1)=0
\end{align}
then
\begin{align}
    \prob{P}_\alpha^{\RV{Z}_0|\RV{X}_0\RV{H}}(1|1,1) = 0\\
    &\neq \prob{P}_C^{\RV{Z}_0|\RV{X}_0\RV{H}}(1|1,1)
\end{align}
\end{example}