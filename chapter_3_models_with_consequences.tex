
%!TEX root = main.tex

\chapter{Models with consequences}\label{ch:2p_statmodels}

This thesis seeks to further understand causal inference problems using probability sets. This chapter discusses why we choose to use probability sets to understand causal inference problems and clarifies how mathematical models based on probability sets related to the real-world problems that we are ultimately interested in. Section \ref{sec:whats_the_point} explains why we focus on understanding \emph{decision problems} or problems involving the comparison of choices. Section \ref{sec:probability_sets} explains why we probability sets are particularly usef for reasoning about this kind of problem.

The reasons we provide for focussing on probability sets are not rigorous. We don't, for example, offer a set of axioms characterising a ``decision problem'' and show that any such problem can be represented by a probability set. In Section \ref{sec:how_represent_conseqeunces}, we consider some of the existing literature that addresses this question in a more rigorous manner. In particular, we show that the models of decision problems proposed by \citet{savage_foundations_1954} and by \citet{jeffrey_logic_1990} -- both of which are supported by axiomatic arguments -- induce probability sets. In addition, we show that probability sets of the appropriate type in conjunction with the principle of expected utility induces a \emph{statistical decision problem} as introduced by \citet{wald_statistical_1950}.

Section \ref{sec:variable} clarifies the role of \emph{variables} and \emph{measurement procedures} in probabilistic models of decision problems. In this work, as is standard in probability theory, variables are defined as measurable functions on a sample space. However, as is standard in statistical practice, variables are also associated with measurement procedures that interact with the real world somehow, and yield numbers that can be reasoned about mathematically.

The contents of this chapter are somewhat tangential to much of the work in the upcoming chapters. It is likely to be helpful to readers seeking clarification about the interpretation of some common model features such as choices and variables. Readers who want to explore alternative approaches to thinking about causal inference problems may also be interested in this chapter, as it explains the thinking behind the approach in this work.

\section{What is the point of causal inference?}\label{sec:whats_the_point}

Someone doing an introductory course in causal inference might encounter many problems about the identification of interventional probabilities or of treatment effects, all stated in the language of causal Bayesian networks, Potential Outcomes or some other mathematical framework for causal reasoning. However, the problems really came before the languages we can use to pose them: causal reasoning frameworks were invented because we wanted some way to talk rigorously about ``causal inference problems''.

There are many kinds of problems that can be posed using causal reasoning frameworks, but this work focuses exclusively on one of these. We consider problems in which we have a set of choices that we want to compare, and we are able to make comparisons only on the basis of a choice's consequences.

There are any different words we could use to describe this kind of problem: we want to assess the consequences of choices, or the effects of interventions, or the results of manipulations or the control of outcomes. Many texts on causal inference affirm that there are other ``causal'' questions we might want to ask, and problems of control are particularly important. \citet{rubin_causal_2005} argues that causal inference often to informs a decision maker by providing ``scientific knowledge'', but does not make recommendations by itself. \citep{imbens_causal_2015} introduces causal inference as the study of ``outcomes of manipulations'' and both \citep{spirtes_causation_1993,pearl_causality:_2009} highlight the universal relevance of understanding how to control certain outcomes, while also arguing that clarifying commonsense ideas of causation is an important aim of causal inference.

A short argument for our approach is: problems that involve comparing different choices on the basis of their consequences supported by data are an important class of problems, whether or not they shed any light on other issues like commonsense ideas of causality.

\subsection{Modelling decision problems}

People who need to make a decisions might (and often do) make them with no mathematical reasoning at all. However, this work is concerned with making decisions assisted by mathematical reasoning. In order to reason mathematically about a decision to be made, we assume that somehow, we have access to two sets:
\begin{enumerate}
    \item There is a set of choices $C$ that need to be compared
    \item There is a set of consequences $\Omega$ along with a utility function $u:\Omega\to \mathbb{R}$
\end{enumerate}

We are interested in understanding how $C$ may be related to $\Omega$ so that the order on $\Omega$ might induce some order on $C$. There are a great number of different ways that we could think of relating elements of $C$ to $\Omega$. For example, we could use a binary relation between the two sets, which given a total order on $\Omega$ would induce a preorder on $C$. However, in this work we assume that the kind of relation is either
\begin{itemize}
    \item A Markov kernel $C\kto \Omega$
    \item A set-valued function $C\to \mathscr{P}(\Delta(\Omega))$
\end{itemize}

That is, for each choice $c\in C$, we have either a probability distribution in $\Delta(\Omega)$ or a set of probability distributions. Usually, we will not know for certain the conseuqences that arise from any given choice, and yet we may have some views about which consequences are more likely than others. Probability has a long and successful history of representing uncertain knowledge of this type. There are many works that aim to show that any method for representing uncertain knowledge that adheres to certain principles must be a probability distribution \citet{de_finetti_foresight_1992,horvitz_framework_1986}, along with criticism of these principles \citet{halpern_counter_1999}. A notable alternative to representing uncertainty with a single probability distribution represents uncertainty with a set of probability distributions, which is a kind of \emph{vague probability} model \citep{walley_statistical_1991}. 

Perhaps more relevant to our aims are a number of works that establish conditions under which ``desirability'' relations over sets of choices or propositions must be represented by a probability distribution along with a utility function. These works are surveyed in Section \ref{sec:how_represent_conseqeunces}. Ultimately, however, the question of whether probability is the right choice to represent uncertain knowledge in decision models is not a key focus of this work. It is a conventional choice, and one that is accept here.

An important if rather obvious point is: if some choices must be evaluated, then a set of choices to evaluate must be available. The availability of a set of options is the feature that gives decision models their ``causal'' characteristics.

\subsection{Formal definitions}\label{sec:probability_sets}

We suppose that we are, at the outset, given a few basic ingredients: a set of choices $C$, a set of consequences $\Omega$ and a utility function $u:F\to \mathbb{R}$. We call these ingredients a ``decision problem''.

\begin{definition}[Decision problem]
A decision problem is a triple $(C,\Omega,u)$ consisting of a measurable set $(C,\sigalg{C})$ of choices, $(\Omega,\sigalg{F})$ consequences and a utility function $u:F\to \mathbb{R}$.
\end{definition}

Our task is to find a \emph{model} that relates $C$ to $\Omega$. We assume two forms of model -- a \emph{sharp model} associates each choice with a unique probability distribution, and a \emph{vague model} associates each choice with a set of probability distributions.

\begin{definition}[Sharp model]
Given a decision problem $(C,\Omega,u)$, a \emph{sharp model} is a function $C\to \Delta(\Omega)$.
\end{definition}

\begin{definition}[Vague model]
Given a decision problem $(C,\Omega,u)$, a \emph{vague model} is a function $C\to \mathscr{P}(\Omega)$.
\end{definition}

Both sharp models and vague models have probability sets induced by the range of the model.

\begin{definition}[Induced probability set (sharp model)]
Given a decision problem $(C,\Omega,u)$ and a sharp model $\prob{P}_\cdot:C\kto \Omega$, the induced probability set is $\prob{P}_C:=\{\prob{P}_\alpha|\alpha\in C\}$.
\end{definition}

\begin{definition}[Induced probability set (vague model)]
Given a decision problem $(C,\Omega,u)$ and a vague model $\prob{Q}_\cdot:C\kto \Omega$, the induced probability set is $\prob{P}_C:=\cup_{\alpha\in C} \prob{Q}_\alpha$.
\end{definition}

\section{Representation theorems for decision problems }\label{sec:how_represent_conseqeunces}

We assume decision models are probabilistic functions $C\to \Delta(\Omega)$ for some sample space $(\Omega,\sigalg{F})$ of ``consequences''. Probability distributions, and the principle of expected utility in particular, are common choices for evaluation under uncertainty. Representation theorems offer a more formal justification for this choice; they propose a collection of axioms regulating the sets of evaluations we want some decision evaluation model to admit, and then show that this model can be represented with (for example) a probability distribution along with a utility function. The desirability of some of the axioms in these theorems is not obvious. 

Here we review the representation theorems of \citet{savage_foundations_1954} and \citet{jeffrey_logic_1990}. We establish that both imply that choices are compared using a probabilistic function $C\to \Delta(\Omega)$ for a suitable selection of $C$ and $(\Omega,\sigalg{F})$, along with a ``desirability'' function which differs in type between the two theorems.

Lewis' \emph{causal decision theory} is also briefly reviewed. While the particular considerations that motivated this theory are not examined, this theory introduces \emph{dependency hypotheses}, which play a key role in the rest of this work.

The following discussion will often make reference to \emph{complete preference relations}. A complete preference relation is a relation $\succ,\prec,\sim$ on a set $A$ such that for any $a,b,c$ in $A$ we have:
\begin{itemize}
    \item Exactly one of $a\succ b$, $a\prec b$, $a\sim b$ holds
    \item $(a\succ b)\iff(b\prec a)$
    \item $a\succ b$ and $b\succ c$ implies $a\succ c$
\end{itemize}
In short, it is a total order without antisymmetry ($a$ and $b$ can be equally preferred even if they are not in fact equal).

This definition is meant to correspond to the common sense idea of having preferences over some set of things, where $\succ$ can be read as ``strictly better than'', $\prec$ read as ``strictly worse than'' and $\sim$ read as ``as good as''. Given any two things from the set, I can say which one I prefer, or if I prefer neither (and all of these are mutually exclusive). If I prefer $a$ to $a'$ then I think $a'$ is worse than $a$. Furthermore, if I prefer $a$ to $a'$ and $a'$ to $a''$ then I prefer $a$ to $a''$.

Define $a\preceq b$ to mean $a\prec b$ or $a \sim b$.

\subsection{von Neumann-Morgenstern utility}

\citet{von_neumann_theory_1944} proved that when the \emph{vNM axioms} hold (not defined here; see the original reference or \citet{steele_decision_2020}), an agent's preferences between ``lotteries'' (probability distributions in $\Delta(\Omega)$ for some $(\Omega,\sigalg{F})$) can be represented as the comparison of the expected value under each lottery of a utility function $u$ unique up to affine transformation. That is, for lotteries $\prob{P}_\alpha$ and $\prob{P}_{\alpha'}$, there exists some $u:\Omega\to \mathbb{R}$ unique up to affine transformation such that $\mathbb{E}_{\prob{P}_\alpha}[u]> \mathbb{E}_{\prob{P}_{\alpha'}}[u]$ if and only if $\prob{P}_{\alpha} \succ \prob{P}_{\alpha'}$.

In vNM theory, the set of lotteries is is the set of all probability measures on $(\Omega,\sigalg{F})$. Thus von Neumann-Morgenstern theorem gives conditions under which preferences \emph{over distributions of consequences} can be represented using expected utility. It is silent on the question of whether each choice should be mapped to a unique probability distribution over consequences.

\subsection{Savage decision theory}

Savage's decision theory establishes conditions under which, given \emph{acts} $C$, \emph{consequences} $\Omega$ and \emph{states} $(S,\sigalg{S})$ (which are ``possible mappings from acts to consequences''), the preference relation over acts can be represented with a probability distribution over states and a utility function $\Omega\to \mathbb{R}$. This is much closer to the subject of this work than the theorem of von Neumann and Morgernstern.

\begin{definition}[Elements of a Savage decision problem]
A \emph{Savage decision problem} features a measurable set of states $(S,\sigalg{S})$, a set of consequences $\Omega$ and a set of acts $C$ such that $|C|=\Omega^S$ and an evaluation function $T:S\times C\to F$ such that for any $f:S\to \Omega$ there exists $c\in C$ such that $T(\cdot,c)=f$.
\end{definition}

\begin{theorem}
Given any Savage decision problem $(S,\Omega,C,T)$ with a preference relation $(\prec,\sim)$ on $C$ that satisfies the \emph{Savage axioms} \ref{sec:savage_axioms}, there exists a unique probability distribution $\mu\in\Delta(\sigalg{S})$ and a utility $u:\Omega\to \mathbb{R}$ unique up to affine transformation such that
\begin{align}
    \alpha\preceq \alpha' &\iff \int_S u(T(s,\alpha))\mu(\mathrm{d}s) \leq \int_S u(T(s,\alpha'))\mu(\mathrm{d}s)&\forall \alpha,\alpha'\in C
\end{align}
\end{theorem}

\begin{proof}
\citet{savage_foundations_1954}
\end{proof}

If we equip consequences with a measures $(\Omega,\sigalg{F})$, Savage's setup implies the existence of a unique probabilistic function $C\to \Delta(\Omega)$ representing the ``probabilistic consequences'' of each choice.

\begin{theorem}
Given any Savage decision problem $(S,\Omega,C,T)$ with a preference relation $(\prec,\sim)$ on $C$ that satisfies the \emph{Savage axioms}, and a $\sigma$-algebra $\sigalg{F}$ on $\Omega$ such that $T$ is measurable, there is a probabilistic function $\prob{P}_{\cdot}:C\to \Delta(\Omega)$ and a utility $u:\Omega\to \mathbb{R}$ unique up to affine transformation such that
\begin{align}
    \alpha\preceq \alpha' &\iff \int_\Omega u(f)\prob{P}_\alpha(\mathrm{d}f) \leq \int_\Omega u(f)\prob{P}_{\alpha'}(\mathrm{d}f)&\forall \alpha,\alpha'\in C
\end{align}
\end{theorem}

\begin{proof}
Define $\prob{P}_\cdot:C\to \Delta(\Omega)$ by
\begin{align}
    \prob{P}_\alpha(A) &:= \mu (T_\alpha^{-1}(A))&\forall A\in \sigalg{F}
\end{align}
where $\RV{T}_\alpha:S\to F$ is the function $s\mapsto T(s,\alpha)$. $\prob{P}_\alpha$ is the pushforward of $T_\alpha$ under $\mu$.

Then 
\begin{align}
    \int_\Omega u(f)\prob{P}_\alpha(\mathrm{d}f) &= \int_S u \circ T_\alpha (s)\mu(\mathrm{d}s)\\
    &= \int_S u(T(s,\alpha))\mu(\mathrm{d}s)
\end{align}
\end{proof}

\subsubsection{Savage axioms}\label{sec:savage_axioms}

Careful analysis of Savage's theorem is outside the scope of this work, but given the relevant of Savage's representation theorem we will reproduce the axioms from \citet{savage_foundations_1954} with a small amount of commentary. Keep in mind that Savage's theorem establishes that the following are sufficient for representation with a probability set, not necessary, and furthermore the probability set representation of preferences satisfying these axioms is unique.

Given acts $C$, states $(S,\sigalg{S})$ and consequences $F$ and a map $T:S\times C\to F$, let all greek letters $\alpha,\beta$ etc. be elements of $C$. Savage's axioms are:
\begin{enumerate}[P1:]
    \item There is a complete preference relation $\preceq$ on $C$
    \begin{enumerate}[D1:]
        \item $\alpha\preceq \beta$ given $B\in \sigalg{S}$ if and only if $\alpha'\preceq \beta'$ for every $\alpha'$ and $\beta'$ such that $T(\alpha,s)=T(\alpha',s)$ for $s\in B$ and $T(\alpha',r)=T(\beta',r)$ for $r\not\in B$, and $\beta'\preceq \alpha'$ either for every such pair or for none.
    \end{enumerate}
    \item For every $\alpha,\beta$ and $B\in \sigalg{S}$, $\alpha\preceq \beta$ given $B$ or $\beta\preceq \alpha$ given $B$
    \begin{enumerate}[D2:]
        \item for $q,q'\in F$, $q\preceq q'$ if and only if $\alpha\preceq \alpha'$ where $T(\alpha,s)=q$ and $T(\alpha',s)=q'$ for all $s\in S$
        \item $B\in \sigalg{S}$ is null if and only if $\alpha\preceq \beta$ given B for every $\alpha,\beta\in C$
    \end{enumerate}
    \item If $T(\alpha,s)=q$ and $T(\alpha',s)=q'$ for every $s\in B$, $B\in \sigalg{S}$ non-null, then $\alpha\preceq \alpha'$ given $B$ if and only if $q\preceq q'$
    \begin{enumerate}[D4:]
        \item For $A,B\in \sigalg{S}$, $A\leqslant B$ if and only if $\alpha_A\preceq \alpha_B$ or $q\preceq q'$ for all $\alpha_A,\alpha_B\in C$, $q,q'\in F$ such that $T(\alpha_A,s) = q$ for $s\in A$, $T(\alpha_A,s')=q'$ for $s'\not\in A$, $T(\alpha_B,s)=q$ for $s\in B$, $T(\alpha_B,s')=q'$ for $s'\not\in B$. Read $\leqslant$ as ``is less probable than''
    \end{enumerate}
    \item For every $A,B\in\sigalg{S}$, $A\leqslant B$ or $B\leqslant A$
    \item For some $\alpha,\beta$, $\alpha\prec \beta$
    \item Suppose $\alpha\not\preceq \beta$. Then for every $\gamma$ there is a finite partition of $S$ such that if $\alpha'$ agrees with $\alpha$ and $\beta'$ agrees with $\beta$ except on some element $B$ of the partition, $\alpha'$ and $\beta'$ being equal to $\gamma$ on $B$, then $\alpha\not\preceq \beta'$ and $\alpha'\not\preceq \beta$
    \begin{enumerate}[D5:]
        \item $\alpha\preceq q$ for $q\in F$ given $B$ if and only if $\alpha\preceq \beta$ given $B$ where $T(\beta,s)=q$ for all $s\in S$
    \end{enumerate}
    \item If $\alpha\preceq T(\beta,s)$ given $B$ for every $s\in B$, then $\alpha\preceq \beta$ given $B$
    \begin{enumerate}[P7':]
        \item The proposition given by inverting every expression in D5 and P7
    \end{enumerate}
\end{enumerate}

Our initial view of decision problems was that the consequences $\Omega$ are a set of things we know how to rank and choices $C$ are the things we want to rank. This is not exactly Savage's setup -- he assumes a preference relation ranking ``acts'' $C$ to begin with. Furthermore, Savage also introduces a set of states $S$ and assumes that the set of acts corresponds to the set of all function $S\to \Omega$. Many decision problems might be able to be extended with states and the set of acts enriched so as to satisfy these requirements, but it is not obvious that this is always possible.

D1 formalises the idea of one act $\alpha$ being not preferred to another $\beta$ given the knowledge that the true state lies in the set $B$ (in short: ``given $B$'' or ``conditional on $B$''). P2 is sometimes called the ``sure thing principle'', as it implies the following: for any $\alpha, \beta$ if $\alpha$ is better than $\beta$ on some states and no worse on any other, then $\alpha\succ \beta$. In Savage's model, the ``likelihood'' that of any state cannot depend on the act chosen.

D4 + P4 defines the ``probability preorder'' $\leqslant$ on $(S,\sigalg{S})$ and assumes it is complete.

P5 is the requirement that the preference relation is non-trivial; not everything is equally desirable. This doesn't seem like it should be a practical requirement to me; we might hope that a model can distinguish between some of our options, but that doesn't mean we should assume it can. Savage claims that this requirement is ``innocuous'' because any exception must be trivial, but I'm not sure I agree.

P6 is a requirement of continuity; for any $\alpha\preceq \beta$, we can divide $S$ finely enough to squeeze a ``small slice'' of any third outcome $\gamma$ into the gap between the two.

P7 in combination with the other axioms forces preferences to be bounded.

\subsection{Jeffrey's decision theory}

Jeffrey's decision theory is an alternative to Savage's that starts from a different set of assumptions. One of the key differences is in what is assumed at the outset: where Savage assumes a set of states $S$, acts $C$ and consequences $\Omega$, Jeffrey's theory only considers a single space $\underline{\sigalg{F}}$, which is a complete atomless boolean algebra. Elements of $\underline{\sigalg{F}}$ are said to be propositions, although the structure of $\underline{\sigalg{F}}$ means we can't understand it as, for example, a set of propositions regarding the result of a particular measurement procedure (Section \ref{sec:variable}). The theory is set out in \citet{jeffrey_logic_1990}, and the key representation theorem proved in \citet{bolker_functions_1966}.

Recall that our fundamental problem is relating a set $C$ of things we can choose to a set $F$ of things we can compare. Jeffrey's theory uses a different strategy to accomplish this than Savages'; where identifies a set of acts $C$ with all functions $S\to F$ and proposes axioms that constrain a preference relation on $C$, Jeffrey assumes that choices are elements of the algebra $\underline{\sigalg{F}}$, along with propositions that do not correspond to choices. Jeffrey's axioms pertain to a preference relation on $\underline{\sigalg{F}}$. The ultimate result is, for our purposes, very similar.

\begin{theorem}\label{th:bolker_jeffrey}
Suppose there is a complete atomless Boolean algebra $\underline{\sigalg{F}}$ with a preference relation $\preceq$. If $\preceq$ satisfies the \emph{Bolker axioms} (Section \ref{sec:bolker_axioms}) then there exists a desirability function $\text{des}:\underline{\sigalg{F}}\to\mathbb{R}$ and a probability distribution $\mu\in \Delta(\underline{\sigalg{F}})$ such that for $A,B\in \underline{\sigalg{F}}$ and finite partition $D_1,...,D_n\in \underline{\sigalg{F}}$:
\begin{align}
    (A \preceq B) \iff \sum_{i}^n \text{des}(D_i) \mu(D_i|A) \leq \sum_{i}^n \text{des}(D_i) \mu(D_i|B) \label{eq:ev_dec_theory}
\end{align}
where $\mu(D_i|A):=\frac{\mu(A\cap D_i)}{\mu(A)}$ for $\mu(A)>0$, undefined otherwise.
\end{theorem}

\begin{proof}
\citet{bolker_functions_1966})
\end{proof}

As mentioned, in Jeffrey's theory the \emph{choices} $C$ are a subset of $\underline{\sigalg{F}}$. Thus we can deduce from a Jeffrey model a function $C\to \Delta(\underline{\sigalg{F}})$ that ``represents the consequences of choices'' in the sense of Theorem \ref{th:jeffrey_with_choices}.

\begin{theorem}\label{th:jeffrey_with_choices}
Suppose there is a complete atomless Boolean algebra $\underline{\sigalg{F}}$ with a preference relation $\preceq$ that satisfies the Bolker axioms, and a set of choices $C$ over which a preference relation is sought with $\mu(\alpha)>0$ for all $\alpha\in C$. Then there is a function $\prob{P}_\cdot:C\to \Delta(\underline{\sigalg{F}})$ such that for any $\alpha,\alpha'\in C$ and finite partition $D_1,...,D_n\in \underline{\sigalg{F}}$:
\begin{align}
    \alpha \preceq \alpha'\iff \sum_{i}^n \text{des}(D_i) \prob{P}_\alpha(D_i) \leq \sum_{i}^n \text{des}(D_i) \prob{P}_{\alpha'}(D_i)\label{eq:ev_with_choices}
\end{align}
Where $\mu$ and $\mathrm{des}$ are as in Theorem \ref{th:bolker_jeffrey}
\end{theorem}

\begin{proof}
Define $\prob{P}_\cdot$ by $\alpha\mapsto \mu(\cdot|\alpha)$. Then Equation \ref{eq:ev_with_choices} follows from Equation \ref{eq:ev_dec_theory}.
\end{proof}

\subsubsection{Bolker axioms}\label{sec:bolker_axioms}

$\underline{\sigalg{F}}$ a complete, atomless Boolean algebra with the impossible proposition. An example of such a set is constructed from the set of Lebesgue measurable sets on $[0,1]$ identifying any two sets that differ by a set of measure zero identified \citet{bolker_simultaneous_1967}. This is not a $\sigma$-algebra.
 
\begin{enumerate}[A1:]
    \item $\preceq$ is a complete preference relation
    \item $\underline{\sigalg{F}}$ is a complete, atomless Boolean algebra with the impossible proposition removed
    \item For $A,B\in \underline{\sigalg{F}}$, if $A\cap B=\emptyset$, then
    \begin{enumerate}[a)]
        \item If $A\succ B$ then $A\succ A\cup B \succ B$
        \item If $A\sim B$ then $A\sim A\cup B \sim B$
    \end{enumerate}
    \item Given $A\cap B=\emptyset$ and $A\sim B$, if $A\cup G\sim B\cup G$ for some $G$ where $A\cap G=B\cap G=\emptyset$ and $G\not\sim A$, then $A\cup G\sim B\cup G$ for every such $G$
    \begin{enumerate}[D1:]
        \item The supremum (infimum) of a subset $W\subset \underline{\sigalg{F}}$ is a set $G$ ($D$) such that for all $A\in W$, $G\subset A$ ($A\subset D$), and for any $E$ that also has this property, $G\subset E$ ($E\subset D$)
    \end{enumerate}
    \item Given $W:= \{W_i\}_{i\in M\subset \mathbb{N}}$ with $i<j\implies W_j\subset W_i$ and $W\subset \underline{\sigalg{F}}$ with supremum $G$ (infimum $D$), whenever $A\prec G \prec B$ ($A\prec D\prec B$) then there exists some $k\in M$ such that $i\geq k$ ($i\leq k$) implies $A\prec W_i \prec B$.
\end{enumerate}

Like Savage's theory, A1 requires the preference relation to be complete.

A3 is the assumption that the desirability of disjunctions of events lies between the desirability of each event; it is sometimes called ``averaging''. It notably rules out the following: if $A\succ B$ we cannot have $A\cup B\sim A$. In the Jeffrey-Bolker theory, propositions all have positive probabilities.

A4 allows a probability order to be defined on $\underline{\sigalg{F}}$. The conditions $A\cap B=\emptyset$, $A\sim B$, $A\cup G\sim B\cup G$ for some $G$ where $A\cap G=B\cap G=\emptyset$ and $G\not\sim A$ can be seen as a test for $A$ and $B$ being ``equally probable''. A4 requires that if $A$ and $B$ are rated as equally probable by one such test, then they are rated as equally probable by all such tests.

A5 is an axiom of continuity.

\subsection{Causal decision theory}

Causal decision theory was developed after both Jeffrey's and Savage's theory. A number of authors \citet{lewis_causal_1981,skyrms_causal_1982} felt that Jeffrey's theory erred by treating the consequences of a choice as an ``ordinary conditional probability''. \citet{lewis_causal_1981} suggested that causal decision theory can be used to evaluate choices when we are given a set $\Omega$ of consequences over which preferences are known, a set $C$ of choices and a set $H$ of dependency hypotheses (the letters have been changed to match usage in this work; in the original the consequences were called $S$, the choices $A$ and the dependency hypotheses $H$). Choices are then evaluated according to the causal decision rule. We have taken the liberty to state Lewis' rule in the language of the present work.

\begin{definition}[Causal decision rule]
Given a set $C$ of choices, sample space $(\Omega,\sigalg{F})$, variables $\RV{H}:\Omega\to H$ (the \emph{dependency hypothesis}) and $\RV{S}:\Omega\to S$ (the \emph{consequence}) and a utility $u:\Omega\to \mathbb{R}$, the \emph{causal utility} of a choice $\alpha\in C$ is given by
\begin{align}
    U(\alpha) := \int_S \int_H u(s) \prob{P}_\alpha^{\RV{S}|\RV{H}}(\mathrm{d}s|h) \prob{P}_C^{\RV{H}}(\mathrm{d}h)\label{eq:lewis_cdt}
\end{align}
For some probabilistic function $\prob{P}_\cdot:C\to \Delta(\Omega)$.
\end{definition}

The reasons why Lewis wanted to introduce dependency hypothesis and modify Jeffrey's rule to Equation \ref{eq:lewis_cdt} are controversial and do not come up in this work. However, causal decision theory is still relevant to this work in two ways: firstly, once again is a probabilistic function $\prob{P}_\cdot:C\to \Delta(\Omega)$. Secondly, causal decision theory introduces the notion of the dependency hypothesis $\RV{H}$. The dependency hypothesis is similar to the state in Savage's theory, however Lewis does not require a deterministic map from dependency hypotheses to consequences, nor does he require a choice to correspond to every possible function from dependency hypotheses to states.

Dependency hypotheses are quite an important idea in causal reasoning. Together Lewis' decision rule connect the theory of probability sets with \emph{statistical decision theory}, as Section \ref{sec:sdt} will show. Chapter \ref{ch:evaluating_decisions} goes into considerable detail concerning the question of when probability sets support certain types of dependency hypothesis. While they are typically not explicitly represented in common frameworks for causal inference, Chapter \ref{ch:other_causal_frameworks} discusses how dependency hypotheses are often implicit in these approaches, and shows how they can be made explicit.

\subsection{Statistical decision theory}\label{sec:sdt}

Statistical decision theory predates all of the theories discussed above; it was published in full by \citet{wald_statistical_1950}. While the preceding theories were all concerned with articulating a theory of rational decision under uncertainty, Wald presented statistical decision theory as a theory of statistical inference formulated as a two-player game. Statistical decision theory introduced many fundamental ideas that have since entered the ``water supply'' of machine learning theory such as \emph{decision rules} and \emph{risk} as a measure of the quality of a decision rule.

Statistical decision problems arise from a probability set combined with a utility in a manner that closely resembles the decision rule given by Equation \ref{eq:lewis_cdt}. 

A further property that may hold for some see-do models $\prob{P}^{\RV{X}|\RV{H}\square \RV{Y}|\RV{D}}$ is $\RV{Y}\CI^2_{\prob{P}} \RV{X}|(\RV{H},\RV{D})$. This expresses the view that the consequences are independent of the observations, once the hypothesis and the decision are fixed. Such a situation could hold in our scenario above, where the observations are trial data, the decisions are recommendations to care providers and the consequences are future patient outcomes. In such a situation, we might suppose that the trial data are informative about the consequences only via some parameter such as effect size; if the effect size can be deduced from $\RV{H}$ then our assumption corresponds to the conditional independence above.

Given a see-do model $\prob{P}^{\RV{X}|\RV{H}\square \RV{Y}|\RV{D}}$ along with the principle of expected utility to evaluate strategies, and the assumption $\RV{Y}\CI^2_{\prob{P}} \RV{X}|(\RV{H},\RV{D})$ we obtain a statistical decision problem in the form introduced by \citet{wald_statistical_1950}.

A \emph{statistical model} (or \emph{statistical experiment}) is a collection of probability distributions $\{\prob{P}_\theta\}$ indexed by some set $\Theta$. A statistical decision problem gives us an observation variable $\RV{X}:\Omega\kto X$ and a statistical experiment $\{\prob{P}^{\RV{X}}_\theta\}_\Theta$, a decision set $D$ and a loss $l:\Theta\times D\to \mathbb{R}$. A strategy $\model{S}^{\RV{D}|\RV{X}}_\alpha$ is evaluated according to the risk functional $R(\theta,\alpha):=\sum_{x\in X}\sum_{d\in D} \prob{P}^{\RV{X}}_\theta(x) S^{\RV{D}|\RV{X}}_\alpha (d|x) l(h,d)$. A strategy $\model{S}^{\RV{D}|\RV{X}}_\alpha$ is considered more desirable than $\model{S}^{\RV{D}|\RV{X}}_\beta$ if $R(\theta,\alpha)<R(\theta,\beta)$.

Suppose we have a see-do model $\prob{P}^{\RV{X}|\RV{H}\square \RV{Y}|\RV{D}}$ with $\RV{Y}\CI_{\model{P}} \RV{X}|(\RV{H,D})$, and suppose that the random variable $\RV{Y}$ is a ``negative utility'' function taking values in $\mathbb{R}$ for which \emph{low} values are considered desirable. Define a loss $l:H\times D\to \mathbb{R}$ by $l(h,d) = \sum_{y\in \mathbb{R}} y\model{P}^{\RV{Y}|\RV{H}\RV{D}}(y|h,d)$, we have 

\begin{align}
    \mathbb{E}_{\model{P}_{\alpha}}[\RV{Y}|h] &= \sum_{x\in X} \sum_{d\in D} \sum_{y\in Y} \model{P}^{\RV{X}|\RV{H}}(x|h) \model{Q}_\alpha^{\RV{D}|\RV{X}}(d|x) \model{P}^{\RV{Y}|\RV{HD}}(y|h,d)\\
    &= \sum_{x\in X} \sum_{d\in D} \model{P}^{\RV{X}|\RV{H}}(x|h) \model{Q}_\alpha^{\RV{D}|\RV{X}}(d|x) l(h,d)\\
    &= R(h,\alpha)
\end{align}

If we are given a see-do model where we interpret $\{\model{P}^{\RV{X}|\RV{H}}(\cdot|h)|h\in H\}$ as a statistical experiment and $\RV{Y}$ as a negative utility, the expectation of the utility under the strategy forecast given in equation \ref{eq:see_do_query} is the risk of that strategy under hypothesis $h$.

\section{Variables}\label{sec:variable}

In probability theory, it is standard to assume the existence of a probability space $(\mu,\Omega,\sigalg{F})$ and to define \emph{random variables} as measurable functions from $(\Omega,\sigalg{F})$ to $(\mathbb{R},\mathcal{B}(\mathbb{R}))$. However, variables aren't \emph{just} functions -- they're also typically understood to correspond to some measured aspect of the real world. For example, \citet{pearl_causality:_2009} offers the following two, purportedly equivalent, definitions of variables:
\begin{quote}
By a \emph{variable} we will mean an attribute, measurement or inquiry that may take on one of several possible outcomes, or values, from a specified domain. If we have beliefs (i.e., probabilities) attached to the possible values that a variable may attain, we will call that variable a random variable.
\end{quote}

\begin{quote}
This is a minor generalization of the textbook definition, according to which a random variable is a mapping from the sample space (e.g., the set of elementary events) to the real line. In our definition, the mapping is from the sample space to any set of objects called ``values,'' which may or may not be ordered.
\end{quote}

However, these quotes are describing different things -- in fact, they're not even describing the same \emph{kind} of thing. The first is talking about a \emph{measurement}, which is something we can do in the real world that produces as a result an element of a mathematical set. The second is talking about a \emph{function}, which is a purely mathematical object with a domain and a codomain and a mapping from the former into the latter.

The way we address this distinction is: a procedure that takes place in the real world and yields as results elements of a mathematical set is called a \emph{measurement procedure}. We suppose for a given problem that there is a ``complete measurement procedure'' $\proc{S}$, and the result of every specific measurement procedure of interest can be reconstructed from the result of the complete measurement procedure by applying a function to the latter result. The function $\RV{X}$ that yields the result of a specific measurement procedure $\proc{X}$ given the result of the complete measurement procedure $\proc{S}$ is the \emph{variable} associated with the measurement procedure $\proc{X}$.

In this way, the variable $\RV{X}$ -- which is by itself just a mathematical function -- is made relevant to the real-world by combining it with a total measurement procedure $\proc{S}$.

We can even use this scheme to address situations where it is possible to make different choices. We can simply posit a sub-procedure $\proc{C}$ that yields the choice $\alpha$ that we eventually make. However, modelling this can be tricky. If we want to use the consequence model to help make the decision, then it seems that the model of the decision procedure $\proc{C}$ will need to be self-referential. Furthermore, even if we have a model of $\proc{C}$ that says we will certainly decide on a particular element $\alpha^*$, we still need to map every element of $C$ to a consequence because this is what enables the comparison of elements of C. Thus, modelling $\proc{C}$ makes the model more complicated, and it's not obvious that this is answering the question that we are interested in. This complication is not obviously intractable, but we do not address it here. Instead, we simply assume that we have a collection $\proc{S}_\alpha$ of total measurement procedures that all yield elements of the same set, and each of which are executed if the choice made is $\alpha$.

\subsection{Variables and measurement procedures}

We illustrate this approach with the example of Newton's second law in the form $\RV{F}=\RV{MA}$. This model relates ``variables'' $\RV{F}$, $\RV{M}$ and $\RV{A}$. As \citet{feynman_feynman_1979} noted, in order to understand this law, we must bring some pre-existing understanding of force, mass and acceleration independent of the law itself. Furthermore, we contend, this knowledge cannot be expressed in any purely mathematical statement. In order to say what the net force on a given object is, even a highly knowledgeable physicist will have to go and do some measurements, which is a procedure that they carry out involving interacting with the real world somehow and obtaining as a result a vector representing the net forces on that object.

That is, the variables $\RV{F}$, $\RV{M}$ and $\RV{A}$ are referring to the \emph{results of measurement procedures}. We will introduce a separate notation to refer to these measurement procedures -- $\proc{F}$ is the procedure for measuring force, $\proc{M}$ and $\proc{A}$ for mass and acceleration respectively. A measurement procedure $\proc{F}$ is akin to \citet{menger_random_2003}'s notion of variables as ``consistent classes of quantities'' that consist of pairing between real-world objects and quantities of some type. Force $\proc{F}$ itself is not a well-defined mathematical thing, as measurement procedures are not mathematically well-defined. At the same time, the set of values it may yield \emph{are} well-defined mathematical things. No actual procedure can be guaranteed to return elements of a mathematical set known in advance -- anything can fail -- but we assume that we can study procedures reliable enough that we don't lose much by making this assumption.

Note that, because $\proc{F}$ is not a purely mathematical thing, we cannot perform mathematical reasoning with $\proc{F}$ directly. Rather, we introduce a variable $\RV{F}$ which, as we will see, is a well-defined mathematical object, assert that it corresponds to $\proc{F}$ and conduct our reasoning using $\RV{F}$.

\subsection{Measurement procedures}\label{sec:mprocs}

\begin{definition}[Measurement procedure]
A \emph{measurement procedure} $\proc{B}$ is a procedure that involves interacting with the real world somehow and delivering an element of a mathematical set $X$ as a result. A procedure $\proc{B}$ is said to takes values in a set $B$.
\end{definition}

We adopt the convention that the procedure name $\proc{B}$ and the set of values $B$ share the same letter.

\begin{definition}[Values yielded by procedures]
$\proc{B}\yields x$ is the proposition that the the procedure $\proc{B}$ will yield the value $x\in X$. $\proc{B}\yields A$ for $A\subset X$ is the proposition $\lor_{x\in A} \proc{B}\yields x$.
\end{definition}

\begin{definition}[Equivalence of procedures]\label{def:equality}
Two procedures $\proc{B}$ and $\proc{C}$ are equal if they both take values in $X$ and $\proc{B}\yields x\iff \proc{C}\yields x$ for all $x\in X$.
\end{definition}

If two involve different measurement actions in the real world but necessarily yield the same result, we say they are equivalent.

It is worth noting that this notion of equivalence identifies procedures with different real-world actions. For example, ``measure the force'' and ``measure everything, then discard everything but the force'' are often different -- in particular, it might be possible to measure the force only before one has measured everything else. Thus the result yielded by the first procedure could be available before the result of the second. However, if the first is carried out in the course of carrying out the second, they both yield the same result in the end and so we treat them as equivalent. 

Measurement procedures are like functions without well-defined domains. Just like we can compose functions with other functions to create new functions, we can compose measurement procedures with functions to produce new measurement procedures.

\begin{definition}[Composition of functions with procedures]
Given a procedure $\proc{B}$ that takes values in some set $B$, and a function $f:B\to C$, define the ``composition'' $f\circ \proc{B}$ to be any procedure $\proc{C}$ that yields $f(x)$ whenever $\proc{B}$ yields $x$. We can construct such a procedure by describing the steps: first, do $\proc{B}$ and secondly, apply $f$ to the value yielded by $\proc{B}$.
\end{definition}

For example, $\proc{MA}$ is the composition of $h:(x,y)\mapsto xy$ with the procedure $(\proc{M},\proc{A})$ that yields the mass and acceleration of the same object. Measurement procedure composition is associative:

\begin{align}
    (g\circ f)\circ\proc{B}\text{ yields } x &\iff B\text{ yields } (g\circ f)^{-1}(x) \\
    &\iff B\text{ yields } f^{-1}(g^{-1}(x))\\
    &\iff f\circ B \text{ yields } g^{-1}(x)\\
    &\iff g\circ(f\circ B)\text{ yields } x
\end{align}


One might wonder whether there is also some kind of ``tensor product'' operation that takes a standalone $\proc{M}$ and a standalone $\proc{A}$ and returns a procedure $(\proc{M},\proc{A})$. Unlike function composition, this would be an operation that acts on two procedures rather than a procedure and a function. Thus this ``append'' combines real-world operations somehow, which might introduce additional requirements (we can't just measure mass and acceleration; we need to measure the mass and acceleration of the same object at the same time), and may be under-specified. For example, measuring a subatomic particle's position and momentum can be done separately, but if we wish to combine the two procedures then we can get different results depending on the order in which we combine them.

Our approach here is to suppose that there is some complete measurement procedure $\proc{S}$ to be modeled, which takes values in the observable sample space $(\Psi,\sigalg{E})$ and for all measurement procedures of interest there is some $f$ such that the procedure is equivalent to $f\circ \proc{S}$ for some $f$. In this manner, we assume that any problems that arise from a need to combine real world actions have already been solved in the course of defining $\proc{S}$.

Given that measurement processes are in practice finite precision and with finite range, $\Psi$ will generally be a finite set. We can therefore equip $\Psi$ with the collection of measurable sets given by the power set $\sigalg{E}:=\mathscr{P}(\Psi)$, and $(\Psi,\sigalg{E})$ is a standard measurable space. $\sigalg{E}$ stands for a complete collection of logical propositions we can generate that depend on the results yielded by the measurement procedure $\proc{S}$.

One could also consider measurement procedures to produce results in $(\mathbb{R},\mathcal{B}(\mathbb{R}))$ (i.e. the reals with the Borel sigma-algebra) or a set isomorphic to it. This choice is often made in practice, and following standard practice we also often consider variables to take values in sets isomorphic to $(\mathbb{R},\mathcal{B}(\mathbb{R}))$. However, for measurement in particular this seems to be a choice of convenience rather than necessity -- for any measurement with finite precision and range, it is possible to specify a finite set of possible results.

\subsection{Observable variables}

Our \emph{complete} procedure $\proc{S}$ represents a large collection of subprocedures of interest, each of which can be obtained by composition of some function with $\proc{S}$. We call the pair consisting of a subprocedure of interest $\proc{X}$ along with the variable $\RV{X}$ used to obtain it from $\proc{S}$ an \emph{observable variable}.

\begin{definition}[Observable variable]
Given a measurement procedure $\proc{S}$ taking values in $(\Psi,\sigalg{E})$, an observable variable is a pair $(\RV{X}\circ \proc{S},\RV{X})$ where $\RV{X}:(\Psi,\sigalg{E})\to (X,\sigalg{X})$ is a measurable function and $\proc{X}:=\RV{X}\circ \proc{S}$ is the measurement procedure induced by $\RV{X}$ and $\proc{S}$.
\end{definition}

For the model $\RV{F}=\RV{MA}$, for example, suppose we have a complete measurement procedure $\proc{S}$ that yields a triple (force, mass, acceleration) taking values in the sets $X$, $Y$, $Z$ respectively. Then we can define the ``force'' variable $(\proc{F},\RV{F})$ where $\proc{F}:=\RV{F}\circ \proc{S}$ and $\RV{F}:X\times Y\times Z\to X$ is the projection function onto $X$.

A measurement procedure yields a particular value when it is completed. We will call a proposition of the form ``$\proc{X}$ yields $x$'' an \emph{observation}. Note that $\proc{X}$ need not be a complete procedure here. Given the complete procedure $\proc{S}$, a variable $\RV{X}:\Psi\to X$ and the corresponding procedure $\proc{X}=\RV{X}\circ\proc{S}$, the proposition ``$\proc{X}$ yields $x$'' is equivalent to the proposition ``$\proc{S}$ yields a value in $\RV{X}^{-1}(x)$''. Because of this, we define the \emph{event} $\RV{X}\yields x$ to be the set $\RV{X}^{-1}(x)$.

\begin{definition}[Event]
Given the complete procedure $\proc{S}$ taking values in $\Psi$ and an observable variable $(\RV{X}\circ \proc{S},\RV{X})$ for $\RV{X}:\Psi\to X$, the \emph{event} $\RV{X}\yields x$ is the set $\RV{X}^{-1}(x)$ for any $x\in X$.
\end{definition}

If we are given an observation ``$\proc{X}$ yields $x$'', then the corresponding event $\RV{X}\yields x$ is \emph{compatible with this observation}.

It is common to use the symbol $=$ instead of $\bowtie$ to stand for ``yields'', but we want to avoid this because $\RV{Y}=y$ already has a meaning, namely that $\RV{Y}$ is a constant function everywhere equal to $y$.

An \emph{impossible event} is the empty set. If $\RV{X}\yields x=\emptyset$ this means that we have identified no possible outcomes of the measurement process $\proc{S}$ compatible with the observation ``$\proc{X}$ yields $x$''. 

\subsection{Model variables}

Observable variables are special in the sense that they are tied to a particular measurement procedure $\proc{S}$. However, the measurement procedure $\proc{S}$ does not enter into our mathematical reasoning; it guides our construction of a mathematical model, but once this is done mathematical reasoning proceeds entirely with mathematical objects like sets and functions, with no further reference to the measurement procedure.

A \emph{model variable} is simply a measurable function with domain $(\Psi,\sigalg{E})$.

Model variables do not have to be derived from observable variables. We may instead choose a sample space for our model $(\Omega,\sigalg{F})$ that does not correspond to the possible values that $\proc{S}$ might yield. In that case, we require a surjective model variable $\RV{S}:\Omega\to \Psi$ called the complete observable variable, and every observable variable $(\RV{X}'\circ \proc{S},\RV{X}')$ is associated with the model variable $\RV{X}:=\RV{X}'\circ \RV{S}$.

An \emph{unobserved variable} is a variable whose set of possible values is not constrained by the results of the measurement procedure.

\begin{definition}[Unobserved variable]\label{def:unobserved_variable}
Given a sample space $(\Omega,\sigalg{F})$ and a complete observable variable $\RV{S}:\Omega\to\Psi$, a model variable $\RV{Y}:\Omega\to Y$ is \emph{unobserved} if $\RV{Y}(\RV{S}\yields s)=Y$ for all $s\in \Psi$.
\end{definition}

\subsection{Variable sequences and partial order}

Given $\RV{Y}:\Omega\to X$, we can define a sequence of variables: $(\RV{X},\RV{Y}):=\omega\mapsto (\RV{X}(\omega),\RV{Y}(\omega))$. $(\RV{X},\RV{Y})$ has the property that $(\RV{X},\RV{Y})\yields (x,y)= \RV{X}\yields x\cap \RV{Y}\yields y$, which supports the interpretation of $(\RV{X},\RV{Y})$ as the values yielded by $\RV{X}$ and $\RV{Y}$ together.

Define the partial order on variables $\varlessthan$ where $\RV{X}\varlessthan \RV{Y}$ can be read ``$\RV{X}$ is completely determined by $\RV{Y}$''.

\begin{definition}[Variables determined by another variable]\label{def:variable_po}
Given a sample space $(\Omega,\sigalg{F})$ and variables $\RV{X}:\Omega\to X$, $\RV{Y}:\Omega\to Y$, $\RV{X}\varlessthan \RV{Y}$ if there is some $f:Y\to X$ such that $\RV{X}=f\circ \RV{Y}$.
\end{definition}

Clearly, $\RV{X}\varlessthan(\RV{X},\RV{Y})$ for any $\RV{X}$ and $\RV{Y}$.

\subsection{Decision procedures}\label{sec:actions}

The kind of problem we want to solve requires us to compare the consequences of different choices from a set of possibilities $C$. We take the \emph{consequences of} $\alpha\in C$ to refer to the values obtained by some measurement procedure $\proc{S}_\alpha$ associated with the choice $\alpha$.

As we have said, what exactly a ``measurement procedure'' is is a bit vague -- it's ``what we actually do to get the numbers we associate with variables''. It seems we could describe the above in terms of a single measurement procedure $\proc{S}$, which involves:

\begin{enumerate}
    \item Choose $\alpha$
    \item Proceed according to $\proc{S}_\alpha$
\end{enumerate}

However, $\proc{S}$ is problematic to model. The model is often part of the process of choosing $\alpha$, and so a model of $\proc{S}$ that involves the step ``choose $\alpha$'' will be self-referential. Because of this, we don't try to model $\proc{S}$, and whether this changes anything is an open question.

\begin{definition}[Decision procedure]
A decision procedure is a collection $\{\proc{S}_\alpha\}_{\alpha\in C}$ of measurement procedures.
\end{definition}

Like measurement procedures, a decision procedure $\{\proc{S}_\alpha\}_{\alpha\in A}$ isn't a well-defined mathematical object; it's not really a ``set'', because the contents are real-world actions.