
%!TEX root = main.tex

\chapter{Discussion}\label{ch:discussion}

Decision making models differ from regular probabilistic models in that they have a domain of choices or options that must be compared to one another. 

In parallel, we've made use of a string diagram notation, both for writing out some proofs and for a visual aid to understanding different kinds of models. The string diagram aspect of this thesis can in principle be cleanly separated from the rest -- it is simply a notation for reasoning using probability theory. Compared to the more common diagrammatic language of directed acyclic graphs (DAGs), the chief advantage of the string diagram notation is that it explicitly represents Markov kernels in the diagrams, and so it is possible (for example) to write that one diagram is equal to another different diagram. This facilitates mathematical reasoning using diagrams alone. A key missing ingredient from the use of string diagrams in this thesis is an analogue of the \emph{d-separation} condition for traditional DAGs. Conditional independence is a very important feature in causal reasoning, and it would be useful to be able to map a collection of extended conditional independence statements to a diagrammatic representation. An analogue of d-separation could also be relevant to the abstract diagrammatic notion of conditional independence postulated by \citet{fritz_synthetic_2020}, which would allow for generalisation of conditional independence beyond the concrete category of Markov kernels.