
%!TEX root = main.tex

\chapter{Discussion}\label{ch:discussion}

Decision making models differ from regular probabilistic models in that they have a domain of choices or options that must be compared to one another. 

In parallel, we've made use of a string diagram notation, both for writing out some proofs and for a visual aid to understanding different kinds of models. The string diagram aspect of this thesis can in principle be cleanly separated from the rest -- it is simply a notation for reasoning using probability theory. Compared to the more common diagrammatic language of directed acyclic graphs (DAGs), the chief advantage of the string diagram notation is that it explicitly represents Markov kernels in the diagrams, and so it is possible (for example) to write that one diagram is equal to another different diagram. This facilitates mathematical reasoning using diagrams alone. A key missing ingredient from the use of string diagrams in this thesis is an analogue of the \emph{d-separation} condition for traditional DAGs. Conditional independence is a very important feature in causal reasoning, and it would be useful to be able to map a collection of extended conditional independence statements to a diagrammatic representation. An analogue of d-separation could also be relevant to the abstract diagrammatic notion of conditional independence postulated by \citet{fritz_synthetic_2020}, which would allow for generalisation of conditional independence beyond the concrete category of Markov kernels.

We focus on decision problems here, but the fact that we call the domain set $C$ a set of ``choices'' is only a convention. There is no obvious reason it could not, for example, be a set of counterfactual propositions. The string diagram notation, extended conditional independence and the theorems proved here would be just as true for models under this alternative interpretation, but they may not always be as relevant. For example, we might not expect counterfactual responses to be independent of responses in the real world.

Conditionally independent and identical response functions (CIIRs) are a fundamental notion to our work. The intuitive idea is: if I am trying to discover an unknown function, I need to feed it different inputs and examine its outputs. The assumption of CIIRs allows me to reason as if a series of trials is, precisely, a series of trials of different inputs given to the \emph{same} stochastic function.

Recall the symmetry identified in Chapter \ref{ch:evaluating_decisions}: the assumption of conditionally independent and identical response functions implies that infinite sequences of input-output pairs with sufficient support are interchangeable. This applies, for example, to sequences comprised of both experimental and passive observational data. To assume that the response of some variable to a treatment and covariates is identical for the experimental and the observational data is to hold that infinite sequences of each type of data are interchangeable. It is already widely accepted that this assumption is usually inappropriate for observational data, but it is often made nonetheless. We offer an alternative interpretation of this assumption when applied to observational data: it is often equivalent to assuming that the observational data is, in the limit, just as good as experimental data for the purposes of predicting consequences of actions.

There is clearly a need for weaker assumptions than conditionally independent and identical response functions. In Chapter \ref{ch:other_causal_frameworks}, we introduce the idea of \emph{preemption}. As an informal principle, the idea that whatever we can do has been done before is easy to understand, and seems like it might be applicable in many cases. We offer a tentative formalisation of this idea, but the question of how to precisely specify that something ``has been done before'' remains open. One interesting implication of our version of preemption is that, in conjunction with an assumption of a generic relationship between a ``choice-making'' function and the ``consequences of that choice'', we find that it's possible to reason from conditional independences in the data to the conclusion of identical response functions (Theorem \ref{th:det_obs_to_cons}). The notion of generic relationships between the functions that make up a causal graph has been used to motivate the assumption of \emph{causal faithfulness}, as well as to provide a basis for the inference of the direction of causal relationships. Th

One option identified in \citet{duvenaud_causal_2008} is to assume that inputs have no effect at all. In fact, in the guise of the \emph{null hypothesis} this is the ``default assumption'' employed in almost every experimental trial in existence (it was first given this name by \citet{fisher_design_1971}). By convention at least, this is a major alternative to the assumption of identical responses, particularly in cases where the data are not considered to be decisive one way or the other. 


