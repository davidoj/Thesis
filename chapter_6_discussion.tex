
%!TEX root = main.tex

\chapter{Discussion}\label{ch:discussion}

Decision making models differ from regular probabilistic models in that they have a domain of choices or options that must be compared to one another. 

In parallel, we've made use of a string diagram notation, both for writing out some proofs and for a visual aid to understanding different kinds of models. The string diagram aspect of this thesis can in principle be cleanly separated from the rest -- it is simply a notation for reasoning using probability theory. Compared to the more common diagrammatic language of directed acyclic graphs (DAGs), the chief advantage of the string diagram notation is that it explicitly represents Markov kernels in the diagrams, and so it is possible (for example) to write that one diagram is equal to another different diagram. This facilitates mathematical reasoning using diagrams alone. A key missing ingredient from the use of string diagrams in this thesis is an analogue of the \emph{d-separation} condition for traditional DAGs. Conditional independence is a very important feature in causal reasoning, and it would be useful to be able to map a collection of extended conditional independence statements to a diagrammatic representation. An analogue of d-separation could also be relevant to the abstract diagrammatic notion of conditional independence postulated by \citet{fritz_synthetic_2020}, which would allow for generalisation of conditional independence beyond the concrete category of Markov kernels.

We focus on decision problems here, but the fact that we call the domain set $C$ a set of ``choices'' is only a convention. There is no obvious reason it could not, for example, be a set of counterfactual propositions. The string diagram notation, extended conditional independence and the theorems proved here would be just as true for models under this alternative interpretation, but they may not always be as relevant. For example, we might not expect counterfactual responses to be independent of responses in the real world.

Recall the symmetry identified in Chapter \ref{ch:evaluating_decisions}: the assumption of conditionally independent and identical response functions implies that infinite sequences of input-output pairs with sufficient support are interchangeable. This applies, for example, to sequences comprised of both experimental and passive observational data. To assume that the response of some variable to a treatment and covariates is identical for the experimental and the observational data is to hold that infinite sequences of each type of data are interchangeable. It is already widely accepted that this assumption is usually inappropriate, but it is often made nonetheless. Theorem \todo[inline]{Write this theorem} offers an alternative interpretation of this assumption: that the authors claim their data is just as good as experimental data.

There is clearly a need for weaker assumptions than conditionally independent and identical response functions, and the obvious question is: what are the alternatives? 


 - Symmetry maybe not surprising, reinforces that usually can't just assume identification
 - What's the alternative? (unprincipled...also, precedent)
 - Precedent, faithfulness, invariance
 - Scale up validation?