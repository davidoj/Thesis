
%!TEX root = main.tex

\chapter{Imitablity and inferring causes from data}\label{ch:inferring_causes}


\section{Assumptions enabling learning}

\begin{itemize}
    \item Recall chapter 4: with no assumptions, no learning possible
    \item What kind of assumptions can be made? Chapter 5: setting, d-separation and ignorability
    \item Here we investigate:
    \begin{itemize}
        \item Conditional setting (chapter 4: decomposability)
        \item Imitability (chapter 3: double exchangeability)
    \end{itemize}
\end{itemize}

\section{Imitability}

\begin{itemize}
    \item Define ``perfect knowledge'' imitability
    \item Alternatively, could assume ``approximate imitability given actual knowledge'' (but we don't)
    \begin{itemize}
        \item Approximate imitability doesn't require double exchangeability
    \end{itemize}
    \item Define conditional setting
\end{itemize}

\section{Identification with imitability}

\begin{itemize}
    \item Define ``pre-intervention distribution RV''
    \item Define $D$-control
    \item Define identification
    \begin{itemize}
        \item Explain weak conditional setting, need for alternative notion of identification
    \end{itemize}
    \item Assumption: pre-intervention + ordinary RV $D$-controls consequence of interest; thm follows
    \item Reduction to SWIGs and hard intervention CBNs via actuator randomisation
    \begin{itemize}
        \item SWIG ``counterfactual'' RVs are different to regular counterfactuals and can be defined also in ``counterfactual-free'' CBNs
    \end{itemize}
\end{itemize}

\begin{itemize}
    \item $D$-control can be deduced from conditional independence under imitability + assumptions of coverage
    \item Similar to \cite{peters_causal_2016}
    \item What about faithfulness?? I've no idea
\end{itemize}