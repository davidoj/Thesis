\documentclass{book}

% If your paper is accepted, change the options for the package
% aistats2020 as follows:
%
% \usepackage[accepted]{aistats2020}
%
% This option will print headings for the title of your paper and
% headings for the authors names, plus a copyright note at the end of
% the first column of the first page.

% If you set papersize explicitly, activate the following three lines:
%\special{papersize = 8.5in, 11in}
%\setlength{\pdfpageheight}{11in}
%\setlength{\pdfpagewidth}{8.5in}

% If you use natbib package, activate the following three lines:
\usepackage[round]{natbib}
\renewcommand{\bibname}{References}
\renewcommand{\bibsection}{\subsubsection*{\bibname}}

% If you use BibTeX in apalike style, activate the following line:
%\bibliographystyle{apalike}

\usepackage[T1]{fontenc}    % use 8-bit T1 fonts
\usepackage{hyperref}       % hyperlinks
\usepackage{url}            % simple URL typesetting
\usepackage{booktabs}       % professional-quality tables
\usepackage{amsfonts}       % blackboard math symbols
\usepackage{nicefrac}       % compact symbols for 1/2, etc.
\usepackage{microtype}      % microtypography

% My packages

\usepackage[mathscr]{euscript}
\usepackage{graphicx}
\usepackage {tikz}
\usetikzlibrary {positioning}
\usetikzlibrary{shapes.misc}
\usetikzlibrary{shapes.geometric}
\usetikzlibrary{calc}
\usetikzlibrary{arrows.meta}
\usetikzlibrary{intersections}
\usepackage{amsthm}
\usepackage{amsmath}
\usepackage{amssymb}
\usepackage{dsfont}
\usepackage{stmaryrd }
\usepackage{csquotes}
\usepackage{wasysym}
\usepackage[]{todonotes}
\usepackage[shortlabels]{enumitem}
\usepackage{bm}
\usepackage{isomath}
\usepackage{mathtools}

\theoremstyle{plain}
\newtheorem{theorem}{Theorem}[section]
\newtheorem{corollary}[theorem]{Corollary}
\newtheorem{lemma}[theorem]{Lemma}
\newtheorem{proposition}[theorem]{Proposition}


\newtheorem{innercustomthm}{Theorem}
\newenvironment{customthm}[1]
  {\renewcommand\theinnercustomthm{#1}\innercustomthm}
  {\endinnercustomthm}

\theoremstyle{definition}
\newtheorem{definition}[theorem]{Definition}
\newtheorem{example}[theorem]{Example}

\DeclareMathAlphabet{\mathsfit}{T1}{\sfdefault}{\mddefault}{\sldefault}

\newcommand{\CI}{\mathrel{\text{\scalebox{1.07}{$\perp\mkern-10mu\perp$}}}}
\newcommand{\CII}{\mathrel{\text{\scalebox{1.07}{$\perp\mkern-10mu\perp\mkern-10mu\perp$}}}}
\newcommand{\RV}[1]{\ensuremath{\mathsf{#1}}}
\newcommand{\URV}[1]{\ensuremath{\underline{\RV{#1}}}}
\newcommand{\PA}[2]{\ensuremath{\text{Pa}_{#1}(#2)}}
\newcommand{\ND}[2]{\ensuremath{\text{ND}_{#1}(#2)}}
\newcommand{\CH}[2]{\ensuremath{\text{Ch}_{#1}(#2)}}
\newcommand{\DE}[2]{\ensuremath{\text{De}_{#1}(#2)}}
\newcommand{\ID}[1]{\ensuremath{\text{Id}_{#1}}}
\newcommand{\utimes}{\ensuremath{\underline{\otimes}}}
\newcommand{\prob}[1]{\ensuremath{\mathbb{#1}}}
\newcommand{\kernel}[1]{\ensuremath{\mathbb{#1}}}
\newcommand{\seedo}{\ensuremath{\mathbb{T}}}
\newcommand{\diagram}[1]{\ensuremath{\mathscr{#1}}}
\newcommand{\sigalg}[1]{\ensuremath{\mathcal{#1}}}
\newcommand{\vecRV}[1]{\ensuremath{\mathsfbfit{#1}}}
\newcommand{\vecVal}[1]{\ensuremath{\mathbf{#1}}}
\newcommand{\prodSet}[1]{\ensuremath{\mathbf{#1}}}
\newcommand{\indx}[1]{\ensuremath{\mathcal{#1}}}
\newcommand{\nod}[1]{\ensuremath{\mathsfit{#1}}}

\makeatletter
\newcommand*\bigcdot{\mathpalette\bigcdot@{.5}}
\newcommand*\bigcdot@[2]{\mathbin{\vcenter{\hbox{\scalebox{#2}{$\m@th#1\bullet$}}}}}
\makeatother

\tikzset{
	triangle/.style = {regular polygon, regular polygon sides=3 },
    node rotated/.style = {rotate=90},
    border rotated/.style = {shape border rotate=90},
    dist/.style = {triangle,draw,border rotated, inner sep=0pt},
    smalldist/.style = {triangle,draw,border rotated},
    kernel/.style={rectangle,draw,inner sep = 2pt},
    expectation/.style = {triangle,draw,inner sep=0pt,shape border rotate=270},
    copymap/.style = {circle,fill,inner sep=1pt}}

\newcommand\DCI{
	\begin{tikzpicture}[scale=0.35]
	\draw[->] (1,0) -- (0,0);
	\draw (0.6,0) -- (0.6,0.75);
	\draw (0.4,0) -- (0.4,0.75);
	\end{tikzpicture}
}

\newcommand\splitter[1]{%
\begin{tikzpicture}[scale=#1]
\draw (0,-1) -- (0,0);
\draw (0,0) to [bend right] (1,1);
\draw (0,0) to [bend left] (-1,1);
\end{tikzpicture}
}

\newcommand\stopper[1]{%
\begin{tikzpicture}[scale=#1]
\draw[-{Rays [n=8]}] (0,-1) -- (0,0);
\end{tikzpicture}
}

\newcommand\source[1]{%
\begin{tikzpicture}[scale=#1]
\path (0,0) node[prob,fill=gray] (P) {};
\draw (P) -- ($(P.east) + (1,0)$);
\end{tikzpicture}
}

\DeclareMathOperator*{\argmax}{arg\,max}
\DeclareMathOperator*{\argmin}{arg\,min}
\DeclareMathOperator*{\arginf}{arg\,inf}
\DeclareMathOperator*{\argsup}{arg\,sup}

\newcommand{\cheng}[1]{ {\color{purple}[{\bf Cheng:~{#1}}]} }

\title{Causal Statistical Decision Theory|What are interventions?}
\date{\today}

\author{ David Johnston }

\begin{document}

\maketitle


% \begin{abstract}
% We develop \emph{causal statistical decison theory} (CSDT) a novel theory of causal inference which we derive by introducing the idea that ``decisions have consequences'' to statistical decision theory. CSDT features \emph{causal theories} as the central object of study. We show that causal Bayesian networks have a natural representation as a causal theory and that potential outcomes models may arguably be represented as causal theories as well. In both cases the resulting theories feature unreasonably rich sets of decisions, which we suggest is because both approaches aim to produce reusable causal models. Using causal theories, we investigate reusability -- when can knowledge gained using one causal theory be applied to another -- and show that this is possible when the theories are related by a \emph{coarsening}.
% \end{abstract}
\tableofcontents



%!TEX root = main.tex


\chapter{Introduction}\label{ch:introduction}

\todo[inline]{I'm thinking of classification problems as types of prediction problems, though it's not really what ``prediction'' means. The key feature is that there is a ground truth that is not known at the time the prediction or class is offered, but will become fully known at some later point.}

Data driven prediction problems and data driven decision making problems have a lot in common. The outcomes some people are interested in predicting are often outcomes other people want to influence. A forecaster might want to predict the winner of the next election, while a party strategist is interested in maximising their party's chance of victory. A product manager may be simultaneously interested in accurately inferring the sentiment expressed in reviews of their product, and in making product changes that increase the frequency that this sentiment is positive. Furthermore, data relevant to prediction is often relevant to decision making and vise-versa. Political parties often reason that electorates in which their predicted chance of victory is very low are not worth investing campaign resources in, and if a forecaster learns of evidence that one party had adopted a particularly effective election strategy they might want to revisit their prediction of the eventual election winner. The overlap is not perfect: comprehensive electorate level polls are probably more useful to the forecaster while small-scale controlled experiments are probably more useful to the strategist.

A key difference between prediction and influence problems is the ``multiplicity of futures'' that each problem asks us to consider. A forecaster wants to identify -- loosely speaking -- the single most likely outcome, while a strategist must consider multiple options and identify the likely outcomes associated with each of these. As a consequence of this difference, the forecaster receives more complete feedback about the quality of their forecast than the strategist. Unlike the forecaster, all but one of the options that the strategist considers are never realised, and the world never offers feedback on these alternative options.

This difference suggests that it might be easier to assess the reliability of a predictive algorithm than the reliability of a decision-making algorithm, and this is be borne out in practice. Validating a predictive algorithm using data split into training and holdout sets is a ubiquitous in machine learning. For many data generating processes, appropriately conducted validation is widely considered to be a reliable indicator of an algorithm's performance for sufficiently similar data generating processes. In contrast, the most well-known condition that is widely accepted to yield reliable decision making algorithms is that the data used to draw inferences comes from a well-conducted controlled experiment. Data that satisfies this is much rarer than data that standard machine learning validation approaches can be applied to. There are approaches to causal inference that don't depend on experimental data, but they depend on other assumptions which are similarly applicable to a limited fraction of datasets. Alternatives to controlled experiments often come with the additional headache of being difficult to assess for a given dataset.

Some of the most far-reaching recent development in algorithmic decision making have involved only the elementary theory of randomised experiments. Operational advances that enable controlled experiments to be conducted at large scales have driven substantial changes in the operations of many online businesses \citep{kohavi_surprising_2017}, and Abhijit Banerjee and Esther Duflo were recently awarded a Nobel prize in part for their pioneering role in the use of large numbers of randomised controlled trials (RCTs) to assess the effectiveness of different kinds of development interventions \citep{zhang_abdul_2014}. Some fields of science have also been significantly affected by ``negative progress'' in the science of assessing experimental results. For example, in psychology, strong evidence has emerged that experimental findings from this field provide weaker evidence to a reader of the findings about the consequences of the reader's actions than many had believed \citet{open_science_collaboration_estimating_2015,stroebe_what_2019}. In a similar time frame, standards for what constitutes a ``well-conducted'' experiment have risen across many fields \citep{nosek_preregistration_2018,liberati_prisma_2009}.

An individual who wants to use data to make better decisions can consider running a controlled experiment of their own. This may not be possible, and even if it is, there may be large amounts of apparently relevant data available that seems wasteful to ignore on the basis of its non-experimental provenance. This individual might therefore be motivated to make some additional assumptions which allow them to draw conclusions about how to act from non-experimental data.

Some examples of assumptions this person could consider are (where * indicates that causal conclusions follow only when the data displays some key features):
\begin{itemize}
    \item There is an input variable independent of the \emph{potential outcomes} conditional on some covariates \citep[Chap. ~12]{imbens_causal_2015}, \citep[Chaps ~2, 3, 5]{angrist_mastering_2014}
    \item [*] There is a variable closely correlated with the \emph{potential outcomes} for each observation \citep[Chap. ~21]{imbens_causal_2015}
    \item \emph{Potential outcomes} are partially observed and vary in a simple way with some variable
    \item [*] The set of observed and unobserved variables have a known \emph{causal structure}
    \item The set of observed variables is \emph{causally sufficient} and the \emph{causal structure} is \emph{faithful} to the conditional independence structure of the observed variables

    \item DID (assumption: initial level accounts for confounders)
    \item Causal sufficiency + faithfulness
    \item Detailed causal structure w/identifiable substructure
\end{itemize}

A common feature of all of these: they're kinda confusing



 - Predictive machine learning has advanced tremendously w/combination of experimentation + theory
 - Causal inference has stuck close to established theory
 - Deriving assumptions is an exact science, making assumptions is not
 - Assumptions are so important in causal inference, and coming at it from a slightly different (decision theoretic) direction can a) expand the space of available assumptions and b) situating common assumptions in a different context


 Further reasons for alternative foundations
 - CGM theory has been undeniably productive -- mediation, confounding, ``m-structures'', causal discovery
 - Some outstanding questions:
 - Not always obvious how to map pre-formal knowledge to interventions
 - Not all variables are ``causally compatible''



\section{Theories of causal inference}

Beginning in the 1930s, a number of associations between cigarette smoking and lung cancer were established: on a population level, lung cancer rates rose rapidly alongside the prevalence of cigarette smoking. Lung cancer patients were far more likely to have a smoking history than demographically similar individuals without cancer and smokers were around 40 times as likely as demographically similar non-smokers to go on to develop lung cancer. In laborotory experiments, cells which were introduced to tobacco smoke developed \emph{ciliastasis}, and mice exposed to cigarette smoke tars developed tumors\citep{proctor_history_2012}. Nevertheless, until the late 1950s, substantial controversy persisted over the question of whether the available data was sufficient to establish that smoking cigarettes \emph{caused} lung cancer. Cigarette manufacturers famously argued against any possible connection \citep{oreskes_merchants_2011} and Roland Fisher in particular argued that the available data was not enough to establish that smoking actually caused lung cancer \citep{fisher_cancer_1958}. Today, it is widely accepted that cigarettes do cause lung cancer, along with other serious conditions such as vascular disease and chronic respiratory disease \citep{world_health_organisation_tobacco_nodate,wiblin_why_2016}.

The question of a causal link between smoking and cancer is a very important one to many different people. Individuals who enjoy smoking (or think they might) may wish to avoid smoking if cigarettes pose a severe health risk, so they are interested in knowing whether or not it is so. Additionally, some may desire reassurance that their habit is not too risky, whether or not this is true. Potential and actual investors in cigarette manufacturers may see health concerns as a barrier to adoption, and also may personally want to avoid supporting products that harm many people. Like smokers, such people might have some interest in knowing the truth of this question, and a separate interest in hearing that cigarettes are not too risky, whether or not this is true. Governments and organisations with a responsibility for public health may see themselves as having responsibility to discourage smoking as much as possible if smoking is severely detrimental to health. The costs and benefits of poor decisions about smoking are large: 8 million annual deaths are attributed to cigarette-caused cancer and vascular disease in 2018\citep{world_health_organisation_tobacco_nodate} while  global cigarette sales were estimated at US\$711 billion in 2020 \citep{noauthor_cigarettes_nodate} (a figure which might be substantially larger if cigarettes were not widely believed to be harmful).

The question of whether or not cigarette smoking causes cancer illustrates two key facts about causal questions: First, having the right answers to causal questions is of tremendous importance to huge numbers of people. Second, confusion over causal questions can persist even when a great deal of data and facts relevant to the question are agreed upon.

Causal conclusions are often justified on the basis of ad-hoc reasoning. For example \citet{krittanawong_association_2020} state:

\begin{quote}
[...] the potential benefit of increased chocolate consumption, reducing coronary artery disease (CAD) risk is not known. We aimed to explore the association between chocolate consumption and CAD.
\end{quote}

It is not clear whether Krittanawong et. al. mean that a negative association between chocolate consumption and CAD implies that increased chocolate consumption is likely to reduce coronary artery disease (which is suggested by the word ``benefit''), or that an association may be relevant to the question and the reader should draw their own conclusions. Whether the implication is being suggested by Krittanawong et. al. or merely imputed by na\"ive readers, it is being drawn on an ad-hoc basis -- no argument for the implication can be found in this paper. As \citet{pearl_causality:_2009} has forcefully argued, additional assumptions are always required to answer causal questions from associational facts, and stating these assumptions explicitly allows those assumptions to be productively scrutinised.

For causal questions that are controversial or difficult, it is tremendously advantageous to be able to address them transparently. Theories of causation enable this; given a theory of causation and a set of assumptions, if anyone claims that some conclusion follows it is publicly verifiable whether or not it actually does so. If the deduction is correct, then any remaining disagreement must be in the assumptions or in the theory. For people who are interested in understanding what is true, pinpointing disagreement can be enlightening. Someone could learn, for example, that there are assumptions that they find plausible that permit conclusions they did not initially believe. Alternatively, if a motivated conclusion follows only from implausible assumptions, hearing these assumptions explicitly might make the conclusion less attractive. 

Theories of causation help us to answer causal questions, which means that before we have any theory, we already have causal questions we want to answer. If potential outcomes notation and causal graphical models had never been invented there would still be just as many people who want to the answer to questions something like ``does smoking causes cancer?'', even if on-one could say what exactly they meant by ``causes'' and even if many people actually want answers to slightly different questions. Theories exist to serve our need for transparent answers to causal questions.

Potential outcomes and causal graphical models are prominent examples of ``practical theories'' of causation. I call them ``practical theories'' because most of the time we encounter them they are being used to answer ``practical'' questions like ``Does smoking cause cancer?'', or ``In general, when does data allow us to conclude that $X$ causes $Y$?'' It is less common to see the ``fundamental questions'' addressed, like ``Does the theory of causal graphical models offer an adequate account of what `cause' means?'', which is more often found in the field of philosophy. \citet{spirtes_causation_1993} explain their motivation to study what I call ``practical theories of causation'' as follows:

\begin{quote}
One approach to clarifying the notion of causation -- the philosophers’ approach ever since Plato -- is to try to define ``causation'' in other terms, to provide necessary and sufficient and noncircular conditions for one thing, or feature or event or circumstance, to cause another, the way one can define ``bachelor'' as ``unmarried adult male human.'' Another approach to the same problem -- the mathematician’s approach ever since Euclid -- is to provide axioms that use the notion of causation without defining it, and to investigate the necessary consequences of those assumptions. We have few fruitful examples of the first sort of clarification, but many of the second [...]
\end{quote}

I think what Spirtes, Glymour and Scheines (henceforth: SGS) mean here is that they \emph{define} a notion of causation -- because causal graphical models do define a notion of causation -- without interrogating whether it means the same thing as the word ``causation''. Incidentally, since publication of this paragraph, the notion of causation defined by causal graphical models has been subject to substantial interrogation by philosophers \citep{woodward_causation_2016}.

I am sympathetic to the argument that it does not matter a great deal whether ``causal-graphical-models-causation'' and ``causation'' mean the same thing in everyday language. It is common for words to have somewhat different meanings when used by specialists to when they are used by laypeople, and this isn't because the specialists are ignorant or confused about their subject. However, I think it matters a lot which causal questions can be transparently answered by ``causal-graphical-models-causation'', and so I believe that the notions of causation adopted by practical theories do warrant scrutiny.

I think one reason that SGS are keen to avoid dwelling on the definition of causation is that satisfactory definitions of causation are difficult. For example, causal graphical models depend on the notion of \emph{causal relationships} between variables. These may be defined as follows:

\begin{quote}
$\RV{X}_i$ is a \emph{cause} of $\RV{X}_j$ if there is an \emph{ideal intervention} on $\RV{X}_i$ that changes the value $\RV{X}_j$
\end{quote}

This definition is incomplete without a definition of ``ideal interventions''. Ideal interventions may be defined by their action in ``causally sufficient models'':
\begin{itemize}
    \item An $[\RV{X}_i,\RV{X}_j]$-ideal intervention is an operation whose result is determined by applying the \emph{do-calculus} to a \emph{causally sufficient} model $((\Omega,\mathcal{F},\prob{P}),\diagram{G},\vecRV{U})$
    \item A model $((\Omega,\mathcal{F},\prob{P}),\diagram{G},\vecRV{U})$ is $[\RV{X}_i,\RV{X}_j]$-causally sufficient if $\RV{U}$ contains $\RV{X}_i$, $\RV{X}_j$ and ``all intervenable variables that \emph{cause}'' both $\RV{X}_i$ and $\RV{X}_j$ \footnote{Weaker conditions for causal sufficiency are possible, but they don't avoid circularity \citep{shpitser_complete_2008}}
\end{itemize}

While I don't offer a definition of the \emph{do-calculus} in this introduction, it can be rigorously defined, see for example \citet{pearl_causality:_2009}. The problem is that the definition of a \emph{causally sufficient} model itself invokes the word \emph{cause}, which is what the original definition was trying to address. Circularity is a recognised problem with interventional definitions of causation \citep{woodward_causation_2016}. In Section \ref{sec:cbns_without_d}, I further show models with ideal interventions generally have counterintuitive properties. The purpose of a theory of causation like causal graphical models is to support transparent reasoning about causal questions, and a circular definition that leads to counterintuitive conclusions undermines this purpose.

As with Euclid's parallel postulate, I think it is reasonable to ask if the notion of ideal interventions and other causal definitions can be modified or avoided. Causal statistical decision theory (CSDT) is a theory of causation that is motivated by the problem of \emph{what is generally needed to answer causal questions} rather than \emph{what does ``causation'' mean?} Along similar lines to CSDT, \citet{dawid_decision-theoretic_2020} has observed that the problem of deciding how to act in light of data can be formalised without appeal to theories of causation. We develop this in substantial detail, showing how both \emph{interventional models} and \emph{counterfactual models} arise as special cases of CSDT.\todo{I want to revisit the claims about what I actually show, hopefully to add to it}

A key feature of CSDT is what I call the \emph{option set}. This is the set of decisions, acts or counterfactual propositions under consideration in a given problem. A causal graphical model and a potential outcomes model will both implicitly define an option set as a result of their basic definitions of causation, but CSDT demands that this is done explicitly. I argue that this is a key strength of CSDT, on the basis of the following claims which I defend in the following chapters:

\begin{itemize}
    \item Causal questions are not well-posed without an option set in the same way a function is not well-defined without its domain
    \item The option set need not correspond in any fixed manner to the set of observed variables
    \item The nature of the option set can affect the difficulty of causal inference questions
\end{itemize}


\todo[inline]{I commented out an additional section about potential outcomes and closest world counterfactuals, which is a second example of ``opaque causal definitions''. I'm interested if any readers think it would be good to have a second example}


% Potential outcomes basic assumptions

% \begin{itemize}
%     \item Potential outcomes defines ``the treatment effect of $\RV{X}_i$ on $\RV{X}_j$'' in terms of the value of $\RV{X}_j$ under the \emph{counterfacutal supposition} that $\RV{X}_i$ had taken a different value
% \end{itemize}

% In fact, the notion of ``ideal intervention'' often seems to underpin potential outcomes models as well. Work in the potential outcomes theory often uses the idea of the value of $\RV{X}_j$ under a counterfactual supposition concerning $\RV{X}_i$ interchangeably with the idea the response of $\RV{X}_j$ to an idealised intervention on $\RV{X}_i$ \citep{morgan_counterfactuals_2014,rubin_causal_2005,richardson2013single}. \cite{lewis_causation_1986} offered a definition of the value $\RV{X}_j$ under counterfactual suppositions in terms of the value it would take in the world that was ``closest'' to the real world but in which the value of $\RV{X}_i$ was altered. There are many ways that we could use to measure how close one world is to another, many of which need not invoke any notion of ``ideal intervention'', but I have never encountered practical work on causal inference that was based on considerations of such similarity measures.


\section{Causally compatible variables}\label{sec:cc_vars}


%!TEX root = main.tex


\chapter{Technical Prerequisites}\label{ch:tech_prereq}

Our approach to causal inference is (like most other approaches) based on probability theory. Many results and conventions will be familiar to readers, and these are collected in Section \ref{sec:standard_prob}.

Less likely to be familiar to readers is the string diagram notation we use to represent probabilistic functions. This is a notation created for reasoning about abstract Markov categories, and is somewhat different to existing graphical languages. The main difference is that in our notation wires represent variables and boxes (which are like nodes in directed acyclic graphs) represent probabilistic functions. Standard directed acyclic graphs annotate nodes with variable names and represent probabilistic functions implicitly. The advantage of explicitly representing probabilistic functions is that we can write equations involving graphics. It is introduced in Section \ref{ssec:mken_diagrams}.

We also extend the theory of probability to a theory of probability sets, which we introduce in Section \ref{sec:probability_sets}. This section goes over some ground already trodden by Section \ref{sec:standard_prob}; this structure was chosen so that people familiar with the Section \ref{sec:standard_prob} can skip to Section \ref{sec:probability_sets} for relevant generalisations to probability sets. Two key ideas introduced here are \emph{uniform conditional probability}, similar but not identical to conditional probability, and \emph{extended conditional independence} as introduced by \citet{constantinou_extended_2017}, similar but not identical to regular conditional independence.

We finally introduce the assumption of \emph{validity}, which ensures that probability sets constructed by ``assembling'' collections of uniform conditionals are non-empty.

This is a reference chapter -- a reader who is already quite familiar with probability theory may skip to Chapter \ref{ch:2p_statmodels}. Where necessary, references back to theorems and definitions in this chapter are given. In Chapter \ref{ch:evaluating_decisions}, we will introduce one additional probabilistic primitive: \emph{combs}, as we feel that additional context is helpful for understanding them.

\section{Conventions}

One of the unusual conventions in this thesis is the notation of uniform conditional probability. Given a set of probability distributions $\prob{P}_C:=\{\prob{P}_\alpha|\alpha\in C\}$ on a common sample space $(\Omega,\sigalg{F})$ with variables $\RV{X}:\Omega\to X$ and $\RV{Y}:\Omega\to Y$, $\prob{P}^{\RV{Y}|\RV{X}}_C$ represents a Markov kernel $X\kto Y$ that satisfies the definition of the distribution of $\RV{Y}$ given $\RV{X}$ (Definition \ref{def:disint}) for every $\alpha\in C$, while $\prob{P}_\alpha^{\RV{Y}|\RV{X}}$ is a conditional distribution only for $\alpha$. There are two unusual feature: firstly, it is more common to write a conditional distribution $\prob{P}(\RV{Y}|\RV{X})$ and secondly, the subscript indicating the ``domain of validity'' of the conditional probability is unusual.

Because this thesis uses sets of probability measures rather than single probability measures, in general a conditional distribution may be valid only for some subset of the probability measures, and always including a subscript indicating which subset or element for which a conditional distribution is valid avoids any ambiguity about this. Avoiding notation of the form $\prob{P}(\RV{Y}|\RV{X})$ is an aesthetic preference; writing a conditional distribution like this suggests $\prob{P}(\RV{Y}|\RV{X})$ is the result of function composition between $\prob{P}$ and some function denoted ``$\RV{Y}|\RV{X}$''. However, conditional probabilities are not given by composition of functions like this.

\begin{center}
\begin{tabular}{ |c|c|c| } 
 \hline
 Name & notation & meaning \\
 \hline
 Iverson bracket & $\llbracket \cdot \rrbracket$ & Function equal to 1 if $\cdot$ is true, false otherwise \\ 
 Identity function & $\mathrm{idf}_X$ & Identity function $X\to X$\\
 Identity kernel & $\mathrm{id}_{X}$ & Kernel associated with the identity function $X\to X$\\
 \hline
\end{tabular}
\end{center}

\section{Probability Theory}

\subsection{Standard Probability Theory}\label{sec:standard_prob}

\subsubsection{$\sigma$-algebras}

\begin{definition}[Sigma algebra]
Given a set $A$, a $\sigma$-algebra $\mathcal{A}$ is a collection of subsets of $A$ where
\begin{itemize}
	\item $A\in \mathcal{A}$ and $\emptyset\in \mathcal{A}$
	\item $B\in \mathcal{A}\implies B^C\in\mathcal{A}$
	\item $\mathcal{A}$ is closed under countable unions: For any countable collection $\{B_i|i\in Z\subset \mathbb{N}\}$ of elements of $\mathcal{A}$, $\cup_{i\in Z}B_i\in \mathcal{A}$ 
\end{itemize}
\end{definition}

\begin{definition}[Measurable space]
A measurable space $(A,\mathcal{A})$ is a set $A$ along with a $\sigma$-algebra $\mathcal{A}$.
\end{definition}

\begin{definition}[Sigma algebra generated by a set]
Given a set $A$ and an arbitrary collection of subsets $U\supset\mathscr{P}(A)$, the $\sigma$-algebra generated by $U$, $\sigma(U)$, is the smallest $\sigma$-algebra containing $U$.
\end{definition}

\paragraph{Common $\sigma$ algebras}

For any $A$, $\{\emptyset,A\}$ is a $\sigma$-algebra. In particular, it is the only sigma algebra for any one element set $\{*\}$.

For countable $A$, the power set $\mathscr{P}(A)$ is known as the discrete $\sigma$-algebra.

Given $A$ and a collection of subsets of $B\subset\mathscr{P}(A)$, $\sigma(B)$ is the smallest $\sigma$-algebra containing all the elements of $B$. 

If $A$ is a topological space with open sets $T$, $\mathcal{B}(\mathbb{R}):=\sigma(T)$ is the \emph{Borel $\sigma$-algebra} on $A$.

If $A$ is a separable, completely metrizable topological space, then $(A,\mathcal{B}(A))$ is a \emph{standard measurable set}. All standard measurable sets are isomorphic to either $(\mathbb{R},B(\mathbb{R}))$ or $(C,\mathscr{P}(C))$ for denumerable $C$ \citep[Chap. 1]{cinlar_probability_2011}.

\subsubsection{Probability measures and Markov kernels}

\begin{definition}[Probability measure]
Given a measurable space $(E,\sigalg{E})$, a map $\mu:\sigalg{E}\to [0,1]$ is a \emph{probability measure} if
\begin{itemize}
	\item $\mu(E)=1$, $\mu(\emptyset)=0$
	\item Given countable collection $\{A_i\}\subset\mathscr{E}$, $\mu(\cup_{i} A_i) = \sum_i \mu(A_i)$
\end{itemize}
\end{definition}

\begin{notation}[Set of all probability measures]
The set of all probability measures on $(E,\sigalg{E})$ is written $\Delta(E)$.
\end{notation}

\begin{definition}[Markov kernel]
Given measurable spaces $(E,\sigalg{E})$ and $(F,\sigalg{F})$, a \emph{Markov kernel} or \emph{stochastic function} is a map $\kernel{M}:E\times\sigalg{F}\to [0,1]$ such that
\begin{itemize}
	\item The map $\kernel{M}(A|\cdot):x\mapsto \kernel{M}(A|x)$ is $\sigalg{E}$-measurable for all $A\in \sigalg{F}$
	\item The map $\kernel{M}(\cdot|x):A\mapsto \kernel{M}(A|x)$ is a probability measure on $(F,\sigalg{F})$ for all $x\in E$
\end{itemize}
\end{definition}

\begin{notation}[Signature of a Markov kernel]
Given measurable spaces $(E,\sigalg{E})$ and $(F,\sigalg{F})$ and $\kernel{M}:E\times\sigalg{F}\to [0,1]$, we write the signature of $\kernel{M}:E\kto F$, read ``$\kernel{M}$ maps from $E$ to probability measures on $F$''.
\end{notation}

\begin{definition}[Deterministic Markov kernel]
A \emph{deterministic} Markov kernel $\kernel{A}:E\to \Delta(\mathcal{F})$ is a kernel such that $\kernel{A}_x(B)\in\{0,1\}$ for all $x\in E$, $B\in\mathcal{F}$.
\end{definition}

\paragraph{Common probability measures and Markov kernels}

\begin{definition}[Dirac measure]
The \emph{Dirac measure} $\delta_x\in \Delta(X)$ is a probability measure such that $\delta_x(A)=\llbracket x\in A \rrbracket$
\end{definition}

\begin{definition}[Markov kernel associated with a function]
Given measurable $f:(X,\sigalg{X})\to (Y,\sigalg{Y})$, $\kernel{F}_f:X\kto Y$ is the Markov kernel given by $x\mapsto \delta_{f(x)}$
\end{definition}

\begin{definition}[Markov kernel associated with a probability measure]
Given $(X,\sigalg{X})$, a one-element measurable space $(\{*\},\{\{*\},\emptyset\})$ and a probability measure $\mu\in \Delta(X)$, the associated Markov kernel $\kernel{Q}_\mu:\{*\}\kto X$ is the unique Markov kernel $*\mapsto \mu$
\end{definition}

\begin{lemma}[Products of functional kernels yield function composition]\label{lem:func_kern_product}
Given measurable $f:X\to Y$ and $g:Y\to Z$, $\kernel{F}_f\kernel{F}_g = \kernel{F}_{g\circ f}$.
\end{lemma}

\begin{proof}
\begin{align}
    (\kernel{F}_{f}\kernel{F}_g)_x(A) &= \int_X (\kernel{F}_g)_y(A) d(\kernel{F}_f)_x(y)\\
                                      &= \int_X \delta_{g(y)}(A) d\delta_{f(x)}(y)\\
                                      &= \delta_{g(f(x))} (A)\\
                                      &= (\kernel{F}_{g\circ f})_x(A) 
\end{align}
\end{proof}

\subsubsection{Variables, conditionals and marginals}

\begin{definition}[Variable]\label{def:variable}
Given a measurable space $(\Omega,\sigalg{F})$ and a measurable space of values $(X,\sigalg{X})$, an \emph{$X$-valued variable} is a measurable function $\RV{X}:(\Omega,\sigalg{F})\to (X,\sigalg{X})$.
\end{definition}

\begin{definition}[Sequence of variables]
Given a measurable space $(\Omega,\sigalg{F})$ and two variables $\RV{X}:(\Omega,\sigalg{F})\to (X,\sigalg{X})$, $\RV{Y}:(\Omega,\sigalg{F})\to (Y,\sigalg{Y})$, $(\RV{X},\RV{Y}):\Omega\to X\times Y$ is the variable $\omega\mapsto (\RV{X}(\omega),\RV{Y}(\omega))$.
\end{definition}

\begin{definition}[Marginal distribution]\label{def:pushforward}
Given a probability space $(\mu,\Omega,\sigalg{F})$ and a variable $\RV{X}:\Omega\to (X,\sigalg{X})$, the \emph{marginal distribution} of $\RV{X}$ with respect to $\mu$, $\mu^{\RV{X}}:\sigalg{X}\to [0,1]$ by $\mu^{\RV{X}}(A):=\mu(\RV{X}^{-1}(A))$ for any $A\in \sigalg{X}$.
\end{definition}

\begin{definition}[Conditional distribution]\label{def:disint}
Given a probability space $(\mu,\Omega,\sigalg{F})$ and variables $\RV{X}:\Omega\to X$, $\RV{Y}:\Omega\to Y$, the \emph{conditional distribution} of $\RV{Y}$ given $\RV{X}$ is any Markov kernel $\mu^{\RV{Y}|\RV{X}}:X\kto Y$ such that
\begin{align}
	\mu^{\RV{XY}}(A\times B)&=\int_{A} \mu^{\RV{Y}|\RV{X}}(B|x) \mathrm{d}\mu^{\RV{X}}(x) &\forall A\in \sigalg{X}, B\in \sigalg{Y}\\
	&\iff\\
	\mu^{\RV{XY}}&= \tikzfig{disint_def}\label{eq:conditional} 
\end{align}
\end{definition}

\subsubsection{Markov kernel product notation}\label{ssec:product_notation}

Three pairwise \emph{product} operations involving Markov kernels can be defined: measure-kernel products, kernel-kernel products and kernel-function products. These are analagous to row vector-matrix products, matrix-matrix products and matrix-column vector products respectively.

 , $\RV{T}:Y\to T$, $\kernel{M}:X\to \Delta(\sigalg{Y})$ and $\kernel{N}:Y\to \Delta(\sigalg{Z})$

\begin{definition}[Measure-kernel product]
Given $\mu\in \Delta(\mathcal{X})$ and $\kernel{M}:X\kto Y$, the \emph{measure-kernel product} $\mu\kernel{M}\in \Delta(Y)$ is given by
\begin{align}
\mu\kernel{M} (A) := \int_X \kernel{M}(A|x) \mu(\mathrm{d}x)
\end{align}
for all $A\in \sigalg{Y}$.
\end{definition}

\begin{definition}[Kernel-kernel product]
Given $\kernel{M}:X\kto Y$ and $\kernel{N}:Y\kto Z$, the \emph{kernel-kernel product} $\kernel{M}\kernel{N}:X\kto Z$ is given by
\begin{align}
\kernel{MN} (A|x) := \int_Y \kernel{N}(A|x) \kernel{M}(\mathrm{d}y|x)
\end{align}
for all $A\in \sigalg{Z}$, $x\in X$.
\end{definition}

\begin{definition}[Kernel-function product]
Given $\kernel{M}:X\kto Y$ and $f:Y\to Z$, the \emph{kernel-function product} $\kernel{M}f:X\to Z$ is given by
\begin{align}
\kernel{M}f (x) := \int_Y f(y)\kernel{N}(\mathrm{d}y|x)
\end{align}
for all $x\in X$.
\end{definition}

\begin{definition}[Tensor product]
Given $\kernel{M}:X\kto Y$ and $\kernel{L}:W\kto Z$, the tensor product $\kernel{M}\otimes\kernel{N}:X\times W\kto Y\times Z$ is given by
\begin{align}
	(\kernel{M}\otimes\kernel{L})(A\times B|x,w):=\kernel{M}(A|x)\kernel{L}(B|w)
\end{align}
For all $x\in X$, $w\in W$, $A\in \sigalg{Y}$ and $B\in \sigalg{Z}$.
\end{definition}

All products are associative \citep[Chapter 1]{cinlar_probability_2011}.

One application of the product notation is that marginal distributions can be alternatively defined in terms of a kernel product, as shown in Lemma \ref{lem:pushf_kprod}.

\begin{lemma}[Marginal distribution as a kernel product]\label{lem:pushf_kprod}
Given a probability space $(\mu,\Omega,\sigalg{F})$ and a variable $\RV{X}:\Omega\to (X,\sigalg{X})$, define $\kernel{F}_{\RV{X}}:\Omega\kto X$ by $\kernel{F}_{\RV{X}}(A|\omega)=\delta_{\RV{X}(\omega)}(A)$, then
\begin{align}
	\mu^{\RV{X}} = \mu\kernel{F}_{\RV{X}}
\end{align}
\end{lemma}

\begin{proof}
Consider any $A\in \sigalg{X}$.
\begin{align}
	\mu \kernel{F}_{\RV{X}}(A) &= \int_\Omega \delta_{\RV{X}(\omega)}(A) \mathrm{d}\mu(\omega)\\
	&= \int_{\RV{X}^{-1}(\omega)} \mathrm{d}\mu(\omega)\\
	&= \mu^{\RV{X}}(A)
\end{align}
\end{proof}

\subsubsection{Semidirect product}

Given a marginal $\mu^{\RV{X}}$ and a conditional $\mu^{\RV{Y}|\RV{X}}$, the product of the two yields the marginal distribution of $\RV{Y}$: $\mu^{\RV{Y}}=\mu^{\RV{X}}\mu^{\RV{Y}|\RV{X}}$. We define another product -- the \emph{semidirect} product $\odot$ -- as the product that yields the joint distribution of $(\RV{X},\RV{Y})$: $\mu^{\RV{XY}}=\mu^{\RV{X}}\odot\mu^{\RV{Y}|\RV{X}}$. The semidirect product is associative (Lemma \ref{lem:sdp_assoc})

\begin{definition}[Semidirect product]\label{def:copyproduct}
Given $\prob{K}:X\kto Y$ and $\prob{L}:Y\times X\kto Z$, the semidirect product $\prob{K}\cprod\prob{L}:X\to Y\times Z$ is given by
\begin{align}
    (\prob{K}\cprod\prob{L})(A\times B|x) &= \int_A \prob{L}(B|y,x)\prob{K}(dy|x)&\forall A\in \sigalg{Y},B\in\sigalg{Z}
\end{align}
\end{definition}


\begin{lemma}[Semidirect product is associative]\label{lem:sdp_assoc}
Given $\prob{K}:X\kto Y$, $\prob{L}:Y\times X\kto Z$ and $\prob{M}:Z\times Y\times X\kto W$
\begin{align}
    (\prob{K}\odot \prob{L})\odot \prob{Z} &= \prob{K}\odot(\prob{L}\odot\prob{Z})\\
\end{align}
\end{lemma}

\begin{proof}
\begin{align}
    (\prob{K}\odot \prob{L})\odot \prob{M} &= \tikzfig{odot_assoc_1}\\
                                            &=  \tikzfig{odot_assoc_2}\\
                                            &= \prob{K}\odot (\prob{L}\odot \prob{M})
\end{align}
\end{proof}

The semidirect product can be used to define a notion of almost sure equality: two kernels $\kernel{K}:X\kto Y$ and $\kernel{L}:X\kto Y$ are $\mu$-almost surely equal if $\mu\odot \kernel{K}=\mu\odot \kernel{L}$. This is identical to the notion of almost sure equality in \citet{cho_disintegration_2019}, who shows that under the assumption that $(Y,\sigalg{Y})$ is countably generated, $\kernel{K}\overset{\mu}{\cong}\kernel{L}$ if and only if $\kernel{K}=\kernel{L}$ $\mu$-almost everywhere.

\begin{definition}[Almost sure equality]\label{def:asequal_pspace}
Two Markov kernels $\kernel{K}:X\kto Y$ and $\kernel{L}:X\kto Y$ are almost surely equal $\overset{\prob{P}_C}{\cong}$ with respect to a probability space $(\mu,X,\sigalg{X})$, written $\kernel{K}\overset{\mu}{\cong}\kernel{L}$ if
\begin{align}
    \mu\odot \kernel{K}=\mu\odot \kernel{L}
\end{align}
\end{definition}

\begin{theorem}
Given $(\mu,X,\sigalg{X})$, $\kernel{K}:X\kto Y$ and $\kernel{L}:X\kto Y$, $\kernel{K}\overset{\mu}{\cong}\kernel{L}$ if and only if, defining $U:=\{x|\exists A\in\sigalg{Y}: \kernel{K}(A|x)\neq\kernel{L}(A|x)\}$, $\mu(U)=0$.
\end{theorem}

\begin{proof}
\citet{cho_disintegration_2019} proposition 5.4.
\end{proof}

We often want to talk about almost sure equality of two different versions $\kernel{K}$ and $\kernel{L}$ of a conditional distribution $\prob{P}^{\RV{Y}|\RV{X}}$ with respect to some ambient probability space $(\prob{P},\Omega,\sigalg{F})$. This simply means $\kernel{K}$ and $\kernel{L}$ satisfy Definition \ref{def:disint} with respect to $\prob{P}$, $\RV{X}$ and $\RV{Y}$, and they are almost surely equal with respect to the marginal $\prob{P}^{\RV{X}}$. Tthe relevant variables are usually obvious from the context and we leave them implicit and we will write $\kernel{K}\overset{\prob{P}}{\cong}\kernel{L}$. If the relevant marginal is ambiguous, we will instead write $\kernel{K}\overset{\prob{P}^{\RV{X}}}{\cong}\kernel{L}$

\begin{definition}[Almost sure equality with respect to a pair of variables]
Given $(\prob{P},\Omega,\sigalg{F})$ and $\RV{X}:\Omega\to X$, $\RV{Y}:\Omega\to Y$, two Markov kernels $\kernel{K}:X\kto Y$ and $\kernel{L}:X\kto Y$ are $\RV{X}$-almost surely equal with respect to $\prob{P}$, written $\kernel{K}\overset{\prob{P}}{\cong}\kernel{L}$, if they are almost surely equal with respect to the marginal $\prob{P}^{\RV{X}}$.
\end{definition}



\section{String Diagrams}\label{ssec:mken_diagrams}

We make use of string diagram notation for probabilistic reasoning. Graphical models are often employed in causal reasoning, and string diagrams are a kind of graphical notation for representing Markov kernels. The notation comes from the study of Markov categories, which are abstract categories that represent models of the flow of information. For our purposes, we don't use abstract Markov categories but instead focus on the concrete category of Markov kernels on standard measurable sets.

A coherence theorem exists for string diagrams and Markov categories. Applying planar deformation or any of the commutative comonoid axioms to a string diagram yields an equivalent string diagram. The coherence theorem establishes that any proof constructed using string diagrams in this manner corresponds to a proof in any Markov category \citep{selinger_survey_2011}. More comprehensive introductions to Markov categories can be found in \citet{fritz_synthetic_2020,cho_disintegration_2019}.

\subsection{Elements of string diagrams}\label{sec:string_diagram_elements}

In the string, Markov kernels are drawn as boxes with input and output wires, and probability measures (which are Markov kernels with the domain $\{*\}$) are represented by triangles:

\begin{align}
\kernel{K}&:=\begin{tikzpicture}[baseline={([yshift=-.5ex]current bounding box.center)}]
    \path (0,0) node (A) {}
    ++ (0.5,0) node[kernel] (K) {$\kernel{K}$}
    ++ (0.5,0) node (B) {};
    \draw (A) -- (K) -- (B);
\end{tikzpicture}\\
\mu&:= \begin{tikzpicture}[baseline={([yshift=-.5ex]current bounding box.center)}]
    \path (0,0) node[dist] (K) {$\kernel{P}$}
    ++ (0.5,0) node (B) {};
    \draw (K) -- (B);
\end{tikzpicture}
\end{align}

Given two Markov kernels $\kernel{L}:X\kto Y$ and $\kernel{M}:Y\kto Z$, the product $\kernel{L}\kernel{M}$ is represented by drawing them side by side and joining their wires:

\begin{align}
    \kernel{L}\kernel{M}:= \begin{tikzpicture}[baseline={([yshift=-.5ex]current bounding box.center)}]
    \path (0,0) node (A) {$X$}
    ++ (0.5,0) node[kernel] (K) {$\kernel{K}$}
    ++ (0.7,0) node[kernel] (M) {$\kernel{M}$}
    ++ (0.5,0) node (B) {$Z$};
    \draw (A) -- (K) -- (M) -- (B);
\end{tikzpicture}
\end{align}

Given kernels $\kernel{K}:W\kto Y$ and $\kernel{L}:X\kto Z$, the tensor product $\kernel{K}\otimes\kernel{L}:W\times X\kto Y\times Z$ is graphically represented by drawing kernels in parallel:

\begin{align}
    \kernel{K}\otimes \kernel{L}&:=\begin{tikzpicture}[baseline={([yshift=-.5ex]current bounding box.center)}]
    \path (0,0) node (A) {$W$}
    ++ (0.5,0) node[kernel] (K) {$\kernel{K}$}
    ++ (0.5,0) node (B) {$Y$};
    \path (0,-0.5) node (C) {$X$}
    ++ (0.5,0) node[kernel] (L) {$\kernel{L}$}
    ++ (0.5,0) node (D) {$Z$};
    \draw (A) -- (K) -- (B);
    \draw (C) -- (L) -- (D);
\end{tikzpicture}
\end{align}

Given $\prob{K}:X\kto Y$ and $\prob{L}:Y\times X\kto Z$, the semidirect product is graphically represented by connecting $\kernel{K}$ and $\kernel{L}$ and keeping an extra copy

\begin{align}
    \prob{K}\cprod\prob{L}:&= \text{copy}_X(\prob{K}\otimes \text{id}_X)(\text{copy}_Y\otimes\text{id}_X )(\text{id}_Y \otimes \prob{L})\\
                            &= \tikzfig{copy_product}
\end{align}

A space $X$ is identified with the identity kernel $\mathrm{id}^X:X\to \Delta(\sigalg{X})$. A bare wire represents the identity kernel:

\begin{align}
\mathrm{Id}^X:=\begin{tikzpicture}
\path (0,0) node (X) {$X$}
++(2,0) node (Y) {$X$};
\draw (X) -- (Y);
\end{tikzpicture}
\end{align}

Product spaces $X\times Y$ are identified with tensor product of identity kernels $\mathrm{id}^X\otimes \mathrm{id}^Y$. These can be represented either by two parallel wires or by a single wire representing the identity on the product space $X\times Y$:
\begin{align}
X\times Y \cong \mathrm{Id}^X\otimes \mathrm{Id}^Y &:= \begin{tikzpicture}
\path (0,0) node (E) {$X$}
++(1,0) node (F) {$X$}
(0,-0.5) node (F1) {$Y$}
+(1,0) node (G) {$Y$};
\draw (E) -- (F);
\draw (F1) -- (G);
\end{tikzpicture}\\
&= \begin{tikzpicture}
\path (0,0) node (X) {$X\times Y$}
++(2,0) node (Y) {$X\times Y$};
\draw (X) -- (Y);
\end{tikzpicture}
\end{align}

A kernel $\kernel{L}:X\to \Delta(\mathcal{Y}\otimes\mathcal{Z})$ can be written using either two parallel output wires or a single output wire, appropriately labeled:

\begin{align}
&\begin{tikzpicture}
\path (0,0) node (E) {$X$}
++ (1,0) node[kernel] (L) {$\kernel{L}$}
++ (1,0.15) node (F) {$Y$}
+(0,-0.3) node (G) {$Z$};
\draw (E) -- (L);
\draw ($(L.east) + (0,0.15)$) -- (F);
\draw ($(L.east)+ (0,-0.15)$) -- (G);
\end{tikzpicture}\\
&\equiv\\
&\begin{tikzpicture}
\path (0,0) node (E) {$X$}
++ (1,0) node[kernel] (L) {$\kernel{L}$}
++ (1.5,0) node (F) {$Y\times Z$};
\draw (E) -- (L) -- (F);
\end{tikzpicture}
\end{align}

We read diagrams from left to right (this is somewhat different to \citet{fritz_synthetic_2020,cho_disintegration_2019,fong_causal_2013} but in line with \citet{selinger_survey_2011}), and any diagram describes a set of nested products and tensor products of Markov kernels. There are a collection of special Markov kernels for which we can replace the generic ``box'' of a Markov kernel with a diagrammatic elements that are visually suggestive of what these kernels accomplish.

\subsection{Special maps}

\begin{definition}[Identity map]
The identity map $\text{id}_X:X\kto X$ defined by $(\text{id}_X)(A|x)= \delta_x(A)$ for all $x\in X$, $A\in\sigalg{X}$, is represented by a bare line.
\begin{align}
    \mathrm{id}_X&:=\begin{tikzpicture}[baseline={([yshift=-.5ex]current bounding box.center)}]
    \path (0,0) node (A) {$X$} ++ (0.5,0) node (B) {$X$};
    \draw (A) -- (B);
\end{tikzpicture}
\end{align}
\end{definition}

\begin{definition}[Erase map]
Given some 1-element set $\{*\}$, the erase map $\text{del}_X:X\kto \{*\}$ is defined by $(\text{del}_X)(*|x) = 1$ for all $x\in X$. It ``discards the input''. It looks like a lit fuse:
\begin{align}
    \text{del}_X&:=\begin{tikzpicture}[baseline={([yshift=-.5ex]current bounding box.center)}]
    \path (0,0) ++ (1,0) node (B) {$X$};
    \draw[-{Rays[n=8]}] (A) -- (B);
\end{tikzpicture}
\end{align}
\end{definition}

\begin{definition}[Swap map]
The swap map $\text{swap}_{X,Y}:X\times Y\kto Y\times X$ is defined by $(\text{swap}_{X,Y})(A\times B|x,y)=\delta_x(B)\delta_y(A)$ for $(x,y)\in X\times Y$, $A\in \sigalg{X}$ and $B\in \sigalg{Y}$. It swaps two inputs and is represented by crossing wires:
\begin{align}
    \text{swap}_{X,Y} &:=  \begin{tikzpicture}[baseline={([yshift=-.5ex]current bounding box.center)}]
        \path (0,0) node (A) {} 
        + (0,-0.5) node (B) {}
        ++ (1,0) node (C) {}
        + (0,-0.5) node (D) {};
        \draw (A) to [out=0,in=180] (D) (B) to [out=0, in=180] (C);
    \end{tikzpicture}
\end{align}
\end{definition}

\begin{definition}[Copy map]
The copy map $\text{copy}_X:X\kto X\times X$ is defined by $(\text{copy}_X)(A\times B|x)=\delta_x(A)\delta_x(B)$ for all $x\in X$, $A,B\in \sigalg{X}$. It makes two identical copies of the input, and is drawn as a fork:
\begin{align}
    \text{copy}_X&:=\begin{tikzpicture}[baseline={([yshift=-.5ex]current bounding box.center)}]
    \path (0,0) node (A) {$X$} 
    ++ (0.5,0) node[copymap] (copy0) {}
    ++ (0.5,0.15) node (B) {$X$}
    + (0,-0.3) node (C) {$X$};
    \draw (A) -- (copy0) to [out=45,in=180] (B) (copy0) to [out=-45, in=180] (C);
\end{tikzpicture}
\end{align}
\end{definition}

\begin{definition}[$n$-fold copy map]
The $n$-fold copy map $\text{copy}^n_X:X\kto X^n$ is given by the recursive definition
\begin{align}
    \text{copy}^1_X &= \text{copy}_X\\
    \text{copy}^n_X &= \tikzfig{n_fold_copy} &n>1
\end{align}
\end{definition}

\paragraph{Plates}

In a string diagram, a plate that is annotated $i\in A$ means the tensor product of the $|A|$ elements that appear inside the plate. A wire crossing from outside a plate boundary to the inside of a plate indicates an $|A|$-fold copy map, which we indicate by placing a dot on the plate boundary. For our purposes, we do not define anything that allows wires to cross from the inside of a plate to the outside; wires must terminate within the plate.

Thus, given $\kernel{K}_i:X\kto Y$ for $i\in A$,

\begin{align}
    \bigotimes_{i\in A} \kernel{K}_i &:= \tikzfig{plate_without_copymap}
    \text{copy}^{|A|}_X(\bigotimes_{i\in A} \kernel{K}_i) &:= \tikzfig{plate_with_copymap}
\end{align}

\subsection{Commutative comonoid axioms}

Diagrams in Markov categories satisfy the commutative comonoid axioms.

\begin{align}
    \tikzfig{ccom_lhs} = \tikzfig{ccom_rhs}\label{eq:ccom_1}
\end{align}
\begin{align}
    \tikzfig{ccom2_lhs} = \tikzfig{ccom2_mhs} = \tikzfig{ccom2_rhs}\label{eq:ccom2_del}
\end{align}
\begin{align}
    \tikzfig{ccom3_lhs} = \tikzfig{ccom3_rhs} \label{eq:ccom3_swap}
\end{align}
as well as compatibility with the monoidal structure
\begin{align}
    \tikzfig{mstruct1_lhs} &= \tikzfig{mstruct1_rhs}\\
    \tikzfig{mstruct2_lhs} &= \tikzfig{mstruct2_rhs}
\end{align}
and the naturality of \emph{del}, which means that
\begin{align}
    \tikzfig{naturality_lhs} &= \tikzfig{naturality_rhs}\label{eq:nat}
\end{align}


\subsection{Manipulating String Diagrams}\label{sssec:string_diagram_manipulation}

Planar deformations along with the applications of Equations \ref{eq:ccom_1} through to Equation \ref{eq:nat} are almost the only rules we have for transforming one string diagram into an equivalent one. One further rule is given by Theorem \ref{th:fong_det_kerns}.

\begin{theorem}[Copy map commutes for deterministic kernels \citep{fong_causal_2013}]\label{th:fong_det_kerns}
For $\kernel{K}:X\kto Y$
\begin{align}
	\tikzfig{deterministic_copymap_commute}
\end{align}
holds iff $\kernel{K}$ is deterministic.
\end{theorem}

\subsubsection{Examples}

String diagrams can always be converted into definitions involving integrals and tensor products. A number of shortcuts can help to make the translations efficiently.

For arbitrary $\kernel{K}:X\times Y\kto Z$, $\kernel{L}:W\kto Y$

\begin{align}
    \tikzfig{identity_tensor_L} &= (\text{id}_X\otimes \kernel{L})\kernel{K}\\
    [(\text{id}_X\otimes \kernel{L})\kernel{K}](A|x,w) &= \int_{Y}\int_X   \kernel{K}(A|x',y')\kernel{L}(\mathrm{d}y'|w)\delta_x(\mathrm{d}x')\\
                                           &= \int_Y  \kernel{K}(A|x,y') \kernel{L}(dy'|w)
\end{align}

That is, an identity map ``passes its input directly to the next kernel''. 

For arbitrary $\kernel{K}: X\times Y\times Y\kto Z$:

\begin{align}
 \tikzfig{identity_tensor_copy} &= (\text{id}_X\otimes \text{copy}_Y)\kernel{K}\\
 [(\text{id}_X\otimes \text{copy}_Y)\kernel{K}](A|x,y) &= \int_Y\int_Y \kernel{K}(A|x,y',y'') \delta_y(\mathrm{d}y')\delta_y(\mathrm{d}y'')\\
                                           &= \kernel{K}(A|x,y,y)
\end{align}

That is, the copy map ``passes along two copies of its input'' to the next kernel in the product. 

For arbitrary $\kernel{K}:X\times Y\kto Z$

\begin{align}
    \tikzfig{swap_example} &= \text{swap}_{YX} \kernel{K}\\
    (\text{swap}_{YX}\kernel{K})(A|y,x) &= \int_{X\times Y} \kernel{K}(A|x',y')\delta_y(\mathrm{d}y')\delta_x(\mathrm{d}x')\\
                                        &= \kernel{K}(A|x,y)
\end{align}

The swap map before a kernel switches the input arguments.

For arbitrary $\kernel{K}:X\kto Y\times Z$

\begin{align}
    \tikzfig{swap_example_2} &= \kernel{K}\text{swap}_{YZ}\\
    (\kernel{K}\text{swap}_{YZ})(A\times B|x) &= \int_{Y\times Z} \delta_{y}(B)\delta_{z}(A)\kernel{K}(\mathrm{d}y\times\mathrm{d}z|x)\\
    &= \int_{B\times A} \kernel{K}(\mathrm{d}y\times\mathrm{d}z|x)\\
    &= \kernel{K}(B\times A|x)
\end{align}

Given $\kernel{K}:X\kto Y$ and $\kernel{L}:Y\kto Z$:

\begin{align}
	(\kernel{K}\cprod \kernel{L})(\mathrm{id}_{Y}\otimes \mathrm{del}_Z) &= \tikzfig{semidirect_K_L}\\
	 &= \tikzfig{semidirect_K_L_2} &\text{by Eq. \ref{eq:nat}}\\
	 &= \tikzfig{semidirect_K_L_3} &\text{by Eq. \ref{eq:ccom2_del}}
\end{align}

Thus the action of the $\text{del}$ map is to marginalise over the deleted wire. With integrals, we can write

\begin{align}
	(\kernel{K}\cprod \kernel{L})(\mathrm{id}_{Y}\otimes \mathrm{del}_Z)(A\times\{*\}|x) &= \int_{Y}\int_{\{*\}}\delta_y(A)\delta_{*}(\{*\})\kernel{L}(\mathrm{d}z|y)\kernel{K}(\mathrm{d}y|x)\\
	&= \int_A \kernel{K}(\mathrm{d}y|x)\\
	&= \kernel{K}(A|x)
\end{align}

\section{Probability Sets}\label{sec:probability_sets}


A probability set is a set of probability measures. This section establishes a number of useful properties of conditional probability with respect to probability sets. Unlike conditional probability with respect to a probability space, conditional probabilities don't always exist for probability sets. Where they do, however, they are almost surely unique and we can marginalise and disintegrate them to obtain other conditional probabilities with respect to the same probability set.

\begin{definition}[Probability set]
A probability set $\prob{P}_C$ on $(\Omega,\sigalg{F})$ is a collection of probability measures on $(\Omega,\sigalg{F})$. In other words it is a subset of $\mathscr{P}(\Delta(\Omega))$, where $\mathscr{P}$ indicates the power set.
\end{definition}

Given a probability set $\prob{P}_C$, we define marginal and conditional probabilities as probability measures and Markov kernels that satisfy Definitions \ref{def:pushforward} and \ref{def:disint} respectively for \emph{all} base measures in $\prob{P}_C$. There are generally multiple Markov kernels that satisfy the properties of a conditional probability with respect to a probability set, and this definition ensures that marginal and conditional probabilities are ``almost surely'' unique (Definition \ref{def:asequal}) with respect to probability sets.

\begin{definition}[Marginal probability with respect to a probability set]
Given a sample space $(\Omega,\sigalg{F})$, a variable $\RV{X}:\Omega\to X$ and a probability set $\prob{P}_C$, the marginal distribution $\prob{P}_C^{\RV{X}}=\prob{P}_\alpha^{\RV{X}}$ for any $\prob{P}_\alpha\in\prob{P}_C$ if a distribution satisfying this condition exists. Otherwise, it is undefined.
\end{definition}

\begin{definition}[Uniform conditional distribution]\label{def:cprob_pset}
Given a sample space $(\Omega,\sigalg{F})$, variables $\RV{X}:\Omega\to X$ and $\RV{Y}:\Omega\to Y$ and a probability set $\prob{P}_C$, a uniform conditional distribution $\prob{P}_C^{\RV{Y}|\RV{X}}$ is any Markov kernel $X\kto Y$ such that $\prob{P}_C^{\RV{Y}|\RV{X}}$ is an $\RV{Y}|\RV{X}$ conditional probability of $\prob{P}_\alpha$ for all $\prob{P}_\alpha\in \prob{P}_C$. If no such Markov kernel exists, $\prob{P}_C^{\RV{Y}|\RV{X}}$ is undefined.
\end{definition}

Given a conditional distribution $\mu^{\RV{ZY}|\RV{X}}$ we can define a higher order conditional $\mu^{\RV{Z}|(\RV{Y}|\RV{X})}$, which is a version of $\mu^{\RV{Z}|\RV{XY}}$. This is useful because uniform conditionals don't always exist, but we can use higher order conditionals to show that if a probability set $\prob{P}_C$ has a uniform conditional $\prob{P}_C^{\RV{ZY}|\RV{X}}$ then it also has a uniform conditional $\prob{P}_C^{\RV{Z}|\RV{XY}}$ (Theorems \ref{th:ho_cond_psets} and \ref{th:higher_order_conditionals}). Given $\mu^{\RV{XY}|\RV{Z}}$ and $\RV{X}:\Omega\to X$, $\RV{Y}:\Omega\to Y$ standard measurable, it has recently been proven that a higher order conditional $\mu^{\RV{Z}|(\RV{Y}|\RV{X})}$ exists \citet{bogachev_kantorovich_2020}, Theorem 3.5.

\begin{definition}[Higher order conditionals]
Given a probability space $(\mu,\Omega,\sigalg{F})$ and variables $\RV{X}:\Omega\to X$, $\RV{Y}:\Omega\to Y$ and $\RV{Z}:\Omega\to Z$, a higher order conditional $\mu^{\RV{Z}|(\RV{Y}|\RV{X})}:X\times Y\to Z$ is any Markov kernel such that, for some $\mu^{\RV{Y}|\RV{X}}$, 
\begin{align}
    \mu^{\RV{ZY}|\RV{X}}(B\times C|x) &=\int_B \mu^{\RV{Z}|(\RV{Y}|\RV{X})}(C|x,y)\mu^{\RV{Y}|\RV{X}}(dy|x)\\ 
    &\iff\\
    \mu^{\RV{ZY}|\RV{X}} &= \tikzfig{disintegration_existence}\label{eq:disint_def}
\end{align}
\end{definition}

\begin{definition}[Uniform higher order conditional]\label{def:ho_cprob_pset}
Given a sample space $(\Omega,\sigalg{F})$, variables $\RV{X}:\Omega\to X$, $\RV{Y}:\Omega\to Y$ and $\RV{Z}:\Omega\to Z$ and a probability set $\prob{P}_C$, if $\prob{P}_C^{\RV{ZY}|\RV{X}}$ exists then a uniform higher order conditional $\prob{P}_C^{\RV{Z}|(\RV{Y}|\RV{X})}$ is any Markov kernel $X\times Y\kto Z$ that is a higher order conditional of some version of $\prob{P}_C^{\RV{ZY}|\RV{X}}$. If no $\prob{P}_C^{\RV{ZY}|\RV{X}}$ exists, $\prob{P}_C^{\RV{Z}|(\RV{Y}|\RV{X})}$ is undefined.
\end{definition}

\begin{definition}[Almost sure equality]
Two Markov kernels $\kernel{K}:X\kto Y$ and $\kernel{L}:X\kto Y$ are $\prob{P}_C,\RV{X},\RV{Y}$-almost surely equal if for all $A\in\sigalg{X}$, $B\in \sigalg{Y}$, $\alpha\in C$
\begin{align}
    \int_A \kernel{K}(B|x)\prob{P}_\alpha^{\RV{X}}(\mathrm{d}x) = \int_A\kernel{L}(B|x)\prob{P}_\alpha^{\RV{X}}(\mathrm{d}x)
\end{align}
we write this as $\kernel{K}\overset{\prob{P}_C}{\cong}\kernel{L}$, as the variables $\RV{X}$ and $\RV{Y}$ are clear from the context.
\end{definition}

Equivalently, $\kernel{K}$ and $\kernel{L}$ are almost surely equal if the set $C:\{x|\exists B\in\sigalg{Y}:\kernel{K}(B|x)\neq\kernel{L}(B|x)\}$ has measure 0 with respect to $\prob{P}_\alpha^{\RV{X}}$ for all $\alpha\in C$.

\subsection{Almost sure equality}

Two Markov kernels are almost surely equal with respect to a probability set $\prob{P}_C$ if the semidirect product $\odot$ of all marginal probabilities of $\prob{P}_\alpha^\RV{X}$ with each Markov kernel is identical.

\begin{definition}[Almost sure equality]\label{def:asequal}
Two Markov kernels $\kernel{K}:X\kto Y$ and $\kernel{L}:X\kto Y$ are almost surely equal $\overset{\prob{P}_C}{\cong}$ with respect to a probability set $\prob{P}_C$ and variable $\RV{X}:\Omega\to X$ if for all $\prob{P}_\alpha \in \prob{P}_C$,
\begin{align}
    \prob{P}^{\RV{X}}_\alpha\odot \kernel{K}=\prob{P}^{\RV{X}}_\alpha\odot \kernel{L}
\end{align}
\end{definition}

\begin{lemma}[Uniform conditional distributions are almost surely equal]
If $\kernel{K}:X\kto Y$ and $\kernel{L}:X\kto Y$ are both versions of $\prob{P}_C^{\RV{Y}|\RV{X}}$ then $\kernel{K}\overset{\prob{P}_C}{\cong}\kernel{L}$
\end{lemma}

\begin{proof}
For all $\prob{P}_\alpha \in \prob{P}_C$
\begin{align}
    \prob{P}^{\RV{X}}_\alpha\odot \kernel{K} &= \prob{P}^{\RV{XY}}_\alpha\\
    &= \prob{P}^{\RV{X}}_\alpha\odot \kernel{L}
\end{align}
\end{proof}

\begin{lemma}[Substitution of almost surely equal Markov kernels]\label{lem:sub_asequal}
Given $\prob{P}_C$, if $\kernel{K}:X\times Y \kto Z$ and $\kernel{L}:X\times Y \kto Z$ are almost surely equal $\kernel{K}\overset{\prob{P}_C}{\cong}\kernel{L}$, then for any $\prob{P}_\alpha\in \prob{P}_C$
\begin{align}
    \prob{P}_\alpha^{\RV{Y}|\RV{X}}\odot \kernel{K} &\overset{\prob{P}_C}{\cong} \prob{P}_\alpha^{\RV{Y}|\RV{X}}\odot \kernel{L}
\end{align}
\end{lemma}

\begin{proof}
For any $\prob{P}_\alpha\in\prob{P}_C$
\begin{align}
    \prob{P}_\alpha^{\RV{XY}}\odot \kernel{K} &\overset{\prob{P}_C}{\cong} (\prob{P}_\alpha^{\RV{X}}\odot \prob{P}_C^{\RV{Y}|\RV{X}})\odot \kernel{K}\\
                                              &\overset{\prob{P}_C}{\cong} \prob{P}_\alpha^{\RV{X}}\odot (\prob{P}_C^{\RV{Y}|\RV{X}}\odot \kernel{K})\\
                                              &\overset{\prob{P}_C}{\cong} \prob{P}_\alpha^{\RV{X}}\odot (\prob{P}_C^{\RV{Y}|\RV{X}}\odot \kernel{L})
\end{align}
\end{proof}

\begin{theorem}[Semidirect product of uniform conditional distributions is a joint uniform conditional distribution]\label{lem:joint_conditional}
Given a probability set $\prob{P}_C$ on $(\Omega,\sigalg{F})$, variables $\RV{X}:\Omega\to X$, $\RV{Y}:\Omega\to Y$ and uniform conditional distributions $\prob{P}_C^{\RV{Y}|\RV{X}}$ and $\prob{P}_C^{\RV{Z}|\RV{XY}}$, then $\prob{P}_C^{\RV{YZ}|\RV{X}}$ exists and is equal to
\begin{align}
    \prob{P}_C^{\RV{YZ}|\RV{X}} &\overset{\prob{P}_C}{\cong} \prob{P}_C^{\RV{Y}|\RV{X}}\odot \prob{P}_C^{\RV{Z}|\RV{XY}}
\end{align}
\end{theorem}

\begin{proof}
By definition, for any $\prob{P}_\alpha\in \prob{P}_C$
\begin{align}
    \prob{P}_\alpha^{\RV{XYZ}} &= \prob{P}_\alpha^{\RV{X}}\odot \prob{P}_\alpha^{\RV{YZ}|\RV{X}}\\
                               &= \prob{P}_\alpha^{\RV{X}}\odot(\prob{P}_\alpha^{\RV{Y}|\RV{X}}\odot \prob{P}_\alpha^{\RV{Z}|\RV{YX}})\\
                               &= \prob{P}_\alpha^{\RV{X}}\odot(\prob{P}_C^{\RV{Y}|\RV{X}}\odot \prob{P}_C^{\RV{Z}|\RV{YX}})
\end{align}
\end{proof}

\subsection{Extended conditional independence}\label{sec:eci}

Just like we defined uniform conditional probability as a version of ``conditional probability'' appropriate for probability sets, we need some version of ``conditional independence'' for probability sets. One such has already been given in some detail: it is the idea of \emph{extended conditional independence} defined in \citet{constantinou_extended_2017}.

We will first define regular conditional independence. We define it in terms of a having a conditional that ``ignores one of its inputs'', which, provided conditional probabilities exists, is equivalent to other common definitions (Theorem \ref{th:cho_ci_equiv}).

\begin{definition}[Conditional independence]\label{def:ci}
For a \emph{probability model} $\model{P}_{\alpha}$ and variables $\RV{A},\RV{B},\RV{Z}$, we say $\RV{B}$ is conditionally independent of $\RV{A}$ given $\RV{C}$, written $\RV{B}\CI_{\model{P}_{\alpha}}\RV{A}|\RV{C}$, if
\begin{align}
    \prob{P}^{\RV{Y}|\RV{WX}} &\overset{\prob{P}}{\cong} \tikzfig{cond_indep_erase}\\
    \iff
    \prob{P}^{\RV{Y}|\RV{WX}}(A|w,x) &\overset{\prob{P}}{\cong} \prob{K}(A|w)&\forall A\in \sigalg{Y}
\end{align}
\end{definition}

Conditional independence can equivalently be stated in terms of the existence of a conditional probability that ``ignores'' one of its inputs.

\begin{theorem}\label{th:cho_ci_equiv}
Given standard measurable $(\Omega,\sigalg{F})$, a probability model $\prob{P}$ and variables $\RV{W}:\Omega\to W$, $\RV{X}:\Omega\to X$, $\RV{Y}:\Omega\to Y$, $\RV{Y}\CI_{\prob{P}}\RV{X}|\RV{W}$ if and only if there exists some version of $\prob{P}^{\RV{Y}|\RV{WX}}$ and $\kernel{K}:W\kto Y$ such that
\begin{align}
    \prob{P}^{\RV{Y}|\RV{WX}} &\overset{\prob{P}}{\cong} \tikzfig{cond_indep_erase}\\
    \iff
    \prob{P}^{\RV{Y}|\RV{WX}}(A|w,x) &\overset{\prob{P}}{\cong} \prob{K}(A|w)&\forall A\in \sigalg{Y}
\end{align}
\end{theorem}

\begin{proof}
See \citet{cho_disintegration_2019}.
\end{proof}

Extended conditional independence as introduced by \citet{constantinou_extended_2017} is defined in terms of ``nonstochastic variables'' on the choice set C. A nonstochastic variable is essentially a variable defined on $C$ rather than on the sample space $\Omega$

\begin{definition}[Nonstochastic variable]
Given a sample space $(\Omega,\sigalg{F})$, a choice set $(C,\sigalg{C})$, a codomain $(X,\sigalg{X})$ and a probability set $\prob{P}_C$, a nonstochastic variable is a measurable function $\phi:C\to X$.
\end{definition}

In particular, we want to consider \emph{complementary} nonstochastic variable - that is, pairs of nonstochastic variables $\phi$ and $\xi$ such that the sequence $(\phi,\xi)$ is invertible. For example, if $\phi:=\mathrm{idf}_C$, then 

\begin{definition}[Complementary nonstochastic variables]
A pair of nonstochasti variables $\phi$ and $\xi$ are complementary if $(\phi,\xi)$ is invertible.
\end{definition}

\begin{notation}
The letters $\phi$ and $\xi$ are used to represent complementary nonstochastic variables.
\end{notation}


Unlike \citet{constantinou_extended_2017}, we limit ourselves to a definition of extended conditional independence where regular uniform conditional probabilities exist. Our definition is otherwise identical.

\begin{definition}[Extended conditional independence]\label{def:eci_orig}
Given a probability set $\prob{P}_C$, variables $\RV{X}$, $\RV{Y}$ and $\RV{Z}$ and complementary nonstochastic variables $\phi$ and $\xi$, the extended conditional independence $\RV{Y}\CI^e_{\prob{P}_C} \RV{X} \phi|\RV{Z} \xi$ holds if for each $a\in \xi(C)$, $\prob{P}_{\xi^{-1}(a)}^{\RV{Y}|\RV{XZ}}$ and $\prob{P}_{\xi^{-1}(a)}^{\RV{Y}|\RV{X}}$ exist and
\begin{align}
    \prob{P}_{\xi^{-1}(a)}^{\RV{Y}|\RV{XZ}} &\overset{\prob{P}_{\xi^{-1}(a)}}{\cong} \tikzfig{eci_def}\\
    &\iff\\
    \prob{P}_{\xi^{-1}(a)}^{\RV{Y}|\RV{XZ}}(A|x,z) &\overset{\prob{P}_{\xi^{-1}(a)}}{\cong} \prob{P}_{\xi^{-1}(a)}^{\RV{Y}|\RV{Z}}(A|z)&\forall A\in \sigalg{Y},(x,z)\in X\times Z\label{eq:eci}
\end{align}
\end{definition}

Very often, we consider a particular kind of extended conditional independence that does not explicitly make use of nonstochastic variables. We call this \emph{uniform conditional independence}.

\begin{definition}[Uniform conditional independence]\label{def:eci}
Given a probability set $\prob{P}_C$ and variables $\RV{X}$, $\RV{Y}$ and $\RV{Z}$, the uniform conditional independence $\RV{Y}\CI^e_{\prob{P}_C} \RV{X} C|\RV{Z}$ holds if $\prob{P}_C^{\RV{Y}|\RV{XZ}}$ and $\prob{P}_C^{\RV{Y}|\RV{X}}$ exist and
\begin{align}
    \prob{P}_C^{\RV{Y}|\RV{XZ}} &\overset{\prob{P}_C}{\cong} \tikzfig{eci_def}\\
    &\iff\\
    \prob{P}_C^{\RV{Y}|\RV{XZ}}(A|x,z) &\overset{\prob{P}_C}{\cong} \prob{P}_C^{\RV{Y}|\RV{Z}}(A|z)&\forall A\in \sigalg{Y},(x,z)\in X\times Z\label{eq:uci}
\end{align}
\end{definition}

For countable sets $C$ (which, recall, is an assumption we generally accept), as shown by \citet{constantinou_extended_2017} we can reason with collections of extended conditional independence statements as if they were regular conditional independence statements, with the provision that a complementary pair of nonstochastic variables must appear either side of the ``|'' symbol. 

\begin{enumerate}
    \item Symmetry: $\RV{X}\CI_{\prob{P}_C}^e \RV{Y} \phi|\RV{Z}\xi$ iff $\RV{Y}\CI_{\prob{P}_C}^e \RV{X} \phi|\RV{Z}\xi$
    \item $\RV{X}\CI_{\prob{P}_C}^e \RV{Y} C| \RV{Y} C$
    \item Decomposition: $\RV{X}\CI_{\prob{P}_C}^e \RV{Y} \phi|\RV{W}\xi$ and $\RV{Z}\varlessthan\RV{Y}$ implies $\RV{X}\CI_{\prob{P}_C}^e\RV{Z}\phi|\RV{W}\xi$
    \item Weak union:
    \begin{enumerate}
     	\item $\RV{X}\CI_{\prob{P}_C}^e \RV{Y} \phi|\RV{W}\xi$ and $\RV{Z}\varlessthan \RV{Y}$ implies $\RV{X}\CI_{\prob{P}_C}^e\RV{Y}\phi|(\RV{Z},\RV{W})\xi$
     	\item $\RV{X}\CI_{\prob{P}_C}^e \RV{Y} \phi|\RV{W}\xi$ and $\lambda\varlessthan \phi$ implies $\RV{X}\CI_{\prob{P}_C}^e\RV{Y}\phi|(\RV{Z},\RV{W})(\xi,\lambda)$
     \end{enumerate} 
    \item Contraction: $\RV{X}\CI_{\prob{P}_C}^e\RV{Z}\phi|\RV{W}\xi$ and $\RV{X}\CI_{\prob{P}_C}^e\RV{Y}\phi|(\RV{Z},\RV{W})\xi$ implies $\RV{X}\CI_{\prob{P}_C}^e(\RV{Y},\RV{Z})\phi|\RV{W}\xi$
\end{enumerate} 

The following forms of these properties are often used here:

\begin{enumerate}
    \item Symmetry: $\RV{X}\CI_{\prob{P}_C}^e \RV{Y} C|\RV{Z}$ iff $\RV{Y}\CI_{\prob{P}} \RV{X} C|\RV{Z}$
    \item Decomposition: $\RV{X}\CI_{\prob{P}_C}^e (\RV{Y},\RV{Z})C|\RV{W}$ implies $\RV{X}\CI_{\prob{P}}\RV{Y}C|\RV{W}$ and $\RV{X}\CI_{\prob{P}}\RV{Z}C|\RV{W}$
    \item Weak union: $\RV{X}\CI_{\prob{P}_C}^e(\RV{Y},\RV{Z})C|\RV{W}$ implies $\RV{X}\CI_{\prob{P}_C}^e\RV{Y}C|(\RV{Z},\RV{W})$
    \item Contraction: $\RV{X}\CI_{\prob{P}_C}^e\RV{Z}C|\RV{W}$ and $\RV{X}\CI_{\prob{P}_C}^e\RV{Y}C|(\RV{Z},\RV{W})$ implies $\RV{X}\CI_{\prob{P}_C}^e(\RV{Y},\RV{Z})C|\RV{W}$
\end{enumerate}

\subsection{Examples}

\begin{example}[Choice variable]\label{ex:choice_var}
Suppose we have a decision procedure $\proc{S}_C:=\{\proc{S}_\alpha|\alpha\in C\}$ that consists of a measurement procedure for each element of a denumerable set of choices $C$. Each measurement procedure $\proc{S}_\alpha$ is modeled by a probability distribution $\prob{P}_\alpha$ on a shared sample space $(\Omega,\sigalg{F})$ such that we have an observable ``choice'' variable $(\RV{D},\RV{D}\circ\proc{S}_\alpha)$ where $\RV{D}\circ\proc{S}_\alpha$ always yields $\alpha$.

Furthermore, Define $\RV{Y}:\Omega\to \Omega$ as the identity function. Then, by supposition, for each $\alpha\in A$, $\prob{P}_\alpha^{\RV{Y}\RV{C}}$ exists and for $A\in \sigalg{Y}$, $B\in \sigalg{C}$:

\begin{align}
    \prob{P}_\alpha^{\RV{YC}}(A\times B) &= \prob{P}_\alpha(A)\delta_\alpha(B)
\end{align}

This implies, for all $\alpha\in C$

\begin{align}
    \prob{P}_\alpha^{\RV{Y}|\RV{D}} &= \prob{P}_\alpha^{\RV{Y}}
\end{align}

Thus $\prob{P}_C^{\RV{Y}|\RV{D}}$ exists and

\begin{align}
    \prob{P}_C^{\RV{Y}|\RV{D}}(A|\alpha) &= \prob{P}_\alpha^{\RV{Y}} (A)&\forall A\in \sigalg{Y},\alpha\in C 
\end{align}

Because only deterministic marginals $\prob{P}_\alpha^{\RV{D}}$ are available, for every $\alpha\in C$ we have $\RV{Y}\CI_{\prob{P}_\alpha} \RV{D}$. This reflects the fact that \emph{after we have selected a choice $\alpha$} the value of $\RV{C}$ provides no further information about the distribution of $\RV{Y}$, because $\RV{D}$ is deterministic given any $\alpha$. It does not reflect the fact that ``choosing different values of $\RV{C}$ has no effect on $\RV{Y}$''.
\end{example}

\begin{theorem}[Uniform conditional independence representation]\label{th:uci_rep}
Given a probability set $\prob{P}_C$ with a uniform conditional probability $\prob{P}^{\RV{XY}|\RV{Z}}_C$,
\begin{align}
    \prob{P}^{\RV{XY}|\RV{Z}}_C &\overset{\prob{P}_C}{\cong} \tikzfig{eci_rep}\\
    &\iff\\
    \prob{P}^{\RV{XY}|\RV{Z}}_C(A\times B|z) &\overset{\prob{P}_C}{\cong} \prob{P}_C^{\RV{X}|\RV{Z}}(A|z)\prob{P}_C^{\RV{Y}|\RV{Z}}(B|z)&\forall A\in \sigalg{X},B\in \sigalg{Y},z\in Z
\end{align}
if and only if $\RV{Y}\CI_{\prob{P}_C}^e \RV{X}C|\RV{Z}$
\end{theorem}

\begin{proof}
If:
By Theorem \ref{th:higher_order_conditionals}
\begin{align}
    \prob{P}^{\RV{XY}|\RV{Z}}_C &= \tikzfig{eci_rep_1}\\
    &\overset{\prob{P}_C}{\cong} \tikzfig{eci_rep_2}\\
    &= \tikzfig{eci_rep}
\end{align}
Only if:
Suppose
\begin{align}
    \prob{P}^{\RV{XY}|\RV{Z}}_C &\overset{\prob{P}_C}{\cong} \tikzfig{eci_rep}
\end{align}
and suppose for some $\alpha\in C$, $A\times C\in \sigalg{X}\otimes\sigalg{Z}$, $B\in \sigalg{Y}$ $\prob{P}_\alpha^{\RV{XZ}}(A\times C)>0$ and
\begin{align}
    \prob{P}_C^{\RV{Y}|\RV{XZ}}(B|x,z) &> \prob{P}_C^{\RV{Y}|\RV{Z}}(B|z)& \forall (x,z)\in A\times C \label{eq:assume_ieq}
\end{align}
then
\begin{align}
    \prob{P}_\alpha^{\RV{XYZ}\RV{Z}}(A\times B\times C) &= \int_{A\times C} \prob{P}_C^{\RV{Y}|\RV{XZ}}(B|x,z)\prob{P}_C^{\RV{X}|\RV{Z}}(\mathrm{dx}|z)\prob{P}_\alpha^{\RV{Z}}(\mathrm{dz})\\
    &> \int_{A\times C} \prob{P}_C^{\RV{Y}|\RV{X}}(B|z)\prob{P}_C^{\RV{X}|\RV{Z}}(\mathrm{dx}|z)\prob{P}_\alpha^{\RV{Z}}(\mathrm{dz})\\
    &= \int_{C} \prob{P}_C^{\RV{XY}|\RV{X}}(A\times B|z)\prob{P}_\alpha^{\RV{Z}}(\mathrm{dz})\\
    &= \prob{P}_\alpha^{\RV{XYZ}\RV{Z}}(A\times B\times C)
\end{align}
a contradiction. An analogous argument follows if we replace ``$>$'' with ``$<$'' in Eq. \ref{eq:assume_ieq}.
\end{proof}


\subsection{Maximal probability sets and valid conditionals}

So far, we have been implicitly supposing that we first set up a probability set and from that set we may sometimes derive uniform conditional probabilities, extended conditional independences and so forth. However, sometimes we want to work backwards: start with a collection of uniform conditional probabilities, and work with the probability set implicitly defined by this collection. For example, when we have a Causal Bayesian Network, the collection of operations of the form ``$\mathrm{do}(\RV{X}=x)$'' specify a probability set by a collection of uniform conditional probabilities on variables other than $\RV{X}$, along with marginal probabilities of $\RV{X}$. Specifically:
\begin{align}
	\prob{P}_{\RV{X}=x}^{\RV{Y}|\mathrm{Pa}(\RV{Y})} &= \begin{cases}
	\prob{P}_{\mathrm{obs}}^{\RV{Y}|\mathrm{Pa}(\RV{Y})}&\RV{Y}\text{ is a causal variable and not equal to }\RV{X}\\
	\delta_x & \RV{Y}=\RV{X}
	\end{cases}
\end{align}

The qualification ``$\RV{Y}$ is a causal variable'' is usually not an explicit condition for causal Bayesian networks, but it is an important one. For example, $2\RV{X}$ is not equal to $\RV{X}$, but we cannot define a causal Bayesian network where both $\RV{X}$ and $2\RV{X}$ are causal variables, see Example \ref{ex:invalidity}.

When working backwards like this, we can run into a couple of problems: we may end up with a probability set where some probabilities are non-unique, or we might inadvertently define an empty probability set. \emph{Validity} is a condition that can ensure that we at least avoid the second problem.

Thus, if we start with a probability set, we know how to check if certain uniform conditional probabilities exist or not. However, there is a particular line of reasoning that comes up most often in the graphical models tradition of causal inference where we start with collections of conditional probabilities and assemble them into probability models as needed. A simple example of this is the causal Bayesian network given by the graph $\RV{X}\longrightarrowRHD \RV{Y}$ and some observational probability distribution $\prob{P}^{\RV{XY}}\in\Delta(X\times Y)$. Using the standard notion of ``hard interventions on $\RV{X}$'', this model induces a probability set which we could informally describe as the set $\prob{P}_\square:=\{\prob{P}_a^{\RV{XY}}|a\in X\cup\{*\}\}$ where $*$ is a special element corresponding to the observational setting. The graph $\RV{X}\longrightarrowRHD \RV{Y}$ implies the existence of the uniform conditional probability $\prob{P}_\square^{\RV{Y}|\RV{X}}$ under the nominated set of interventions, while the usual rules of hard interventions imply that $\prob{P}_a^{\RV{X}} = \delta_a$ for $a\in X$.

Reasoning ``backwards'' like this -- from uniform conditionals and marginals back to probability sets -- must be done with care. The probability set associated with a collection of conditionals and marginals may be empty or nonunique. Uniqueness may not always be required, but an empty probability set is clearly not a useful model.

Consider, for example, $\Omega=\{0,1\}$ with $\RV{X}=(\RV{Z},\RV{Z})$ for $\RV{Z}:=\text{id}_{\Omega}$ and any measure $\kappa\in \Delta(\{0,1\}^2)$ such that $\kappa(\{1\}\times \{0\})>0$. Note that $\RV{X}^{-1}(\{1\}\times \{0\})=\RV{Z}^{-1}(\{1\})\cap \RV{Z}^{-1}(\{0\})=\emptyset$. Thus for any probability measure $\mu\in \Delta(\{0,1\})$, $\mu^{\RV{X}}(\{1\}\times \{0\}) = \mu(\emptyset)=0 $ and so $\kappa$ cannot be the marginal distribution of $\RV{X}$ for any base measure at all.

We introduce the notion of \emph{valid distributions} and \emph{valid conditionals}. The key result here is: probability sets defined by collections of recursive valid conditionals and distributions are nonempty. While we suspect this condition is often satisfied by causal models in practice, we offer one example in the literature where it apparently is not. The problem of whether a probability set is valid is analogous to the problem of whether a probability distribution satisfying a collection of constraints exists discussed in \citet{vorobev_consistent_1962}. As that work shows, there are many questions of this nature that can be asked and that are not addressed by the criterion of validity.

There is also a connection between the notion of validity and the notion of \emph{unique solvability} in \citet{bongers_theoretical_2016}. We ask ``when can a set of conditional probabilities together with equations be jointly satisfied by a probability model?'' while Bongers et. al. ask when a set of equations can be jointly satisfied by a probability model.

\begin{definition}[Valid distribution]\label{def:valid_dist}
Given $(\Omega,\sigalg{F})$ and a variable $\RV{X}:\Omega\to X$, an $\RV{X}$-valid probability distribution is any probability measure $\prob{K}\in \Delta(X)$ such that $\RV{X}^{-1}(A)=\emptyset\implies \prob{K}(A) = 0$ for all $A\in\sigalg{X}$.
\end{definition}

\begin{definition}[Valid conditional]\label{def:valid_conditional_prob}
Given $(\Omega,\sigalg{F})$, $\RV{X}:\Omega\to X$, $\RV{Y}:\Omega\to Y$ a \emph{$\RV{Y}|\RV{X}$-valid conditional probability} is a Markov kernel $\prob{L}:X\kto Y$ that assigns probability 0 to impossible events, unless the argument itself corresponds to an impossible event:
\begin{align}
    \forall B\in \sigalg{Y}, x\in X: (\RV{X},\RV{Y})\yields \{x\}\times B = \emptyset \implies \left(\prob{L}(B|x) = 0\right) \lor \left(\RV{X}\yields \{x\} = \emptyset\right)
\end{align}
\end{definition}

When a probability distribution is interpreted as a Markov kernel, both of these definitions agree.

\begin{theorem}[Equivalence of validity definitions]\label{th:valid_agree}
Given $\RV{X}:\Omega\to X$, with $\Omega$ and $X$ standard measurable, a probability measure $\prob{P}^{\RV{X}}\in \Delta(X)$ is valid if and only if the conditional $\prob{P}^{\RV{X}|*}:=*\mapsto \prob{P}^{\RV{X}}$ is valid.
\end{theorem}

\begin{proof}
$*\yields *=\Omega$ necessarily. Thus validity of $\prob{P}^{\RV{X}|*}$ means 

\begin{align}
    \forall A\in \sigalg{X}: \RV{X}\yields A=\emptyset \implies \prob{P}^{\RV{X}|*}(A|*)&=0
\end{align}

But $\prob{P}^{\RV{X}|*}(A|*)=\prob{P}^{\RV{X}}(A)$ by definition, so this is equivalent to

\begin{align}
    \forall A\in \sigalg{X}: \RV{X}\yields A=\emptyset \implies \prob{P}^{\RV{X}}(A)&=0
\end{align}
\end{proof}

Conditionals can be used to define \emph{maximal probability sets}, which is the set of all probability distributions with those conditionals.

\begin{definition}[Maximal probability set]
Given $(\Omega,\sigalg{F})$, $\RV{X}:\Omega\to X$, $\RV{Y}:\Omega\to Y$ and a $\RV{Y}|\RV{X}$-valid conditional probability $\prob{L}:X\kto Y$ the maximal probability set $\prob{P}_C$ associated with $\prob{L}$ is the probability set such that for all $\prob{P}_\alpha\in \prob{P}_C$, $\prob{L}$ is a version of $\prob{P}_\alpha^{\RV{Y}|\RV{X}}$.
\end{definition}

Theorem \ref{lem:valid_extendability} shows that the semidirect product of any pair of valid conditional probabilities is itself a valid conditional. Suppose we have some collection of $\RV{X}_i|\RV{X}_{[i-1]}$-valid conditionals $\{\prob{P}_i^{\RV{X}_i|\RV{X}_{[i-1]}}|i\in [n]\}$; then recursively taking the semidirect product $\kernel{M}:=\prob{P}_1^{\RV{X}_1}\odot (\prob{P}_2^{\RV{X}_2|\RV{X}_{1}}\odot ...)$ yields a $\RV{X}_{[n]}$ valid distribution. Furthermore, the maximal probability set associated with $\kernel{M}$ is nonempty.

Collections of recursive conditional probabilities often arise in causal modelling -- in particular, they are the foundation of the structural equation modelling approach \citet{richardson2013single,pearl_causality:_2009}.

Note that validity is not a necessary condition for a conditional to define a non-empty probability set. Given some $\kernel{K}:X\kto Y$, $\kernel{K}$ might be an invalid conditional on if every value of $X$ is considered, but it might be valid on some subset of $X$. A marginal of $\RV{X}$ that assigns measure 0 to the subset of $X$ where $\kernel{K}$ is invalid can still define a valid distribution when combined with $\kernel{K}$. On the other hand, if $\kernel{K}$ is required to combine with arbitrary valid marginals of $\RV{X}$, then the validity of $\kernel{K}$ is necessary (Theorem \ref{th:valid_conditional_probability}).

\begin{theorem}[Semidirect product of valid conditional distributions is valid]\label{lem:valid_extendability}
Given $(\Omega,\sigalg{F})$, $\RV{X}:\Omega\to X$, $\RV{Y}:\Omega\to Y$, $\RV{Z}:\Omega\to Z$ (all spaces standard measurable) and any valid candidate conditional $\prob{P}^{\RV{Y}|\RV{X}}$ and $\prob{Q}^{\RV{Z}|\RV{Y}\RV{X}}$, $ \prob{P}^{\RV{Y}|\RV{X}}\odot \prob{Q}^{\RV{Z}|\RV{Y}\RV{X}}$ is also a valid candidate conditional.
\end{theorem}

\begin{proof}
Let $\prob{R}^{\RV{YZ}|\RV{X}}:=\prob{P}^{\RV{Y}|\RV{X}}\odot \prob{Q}^{\RV{Z}|\RV{Y}\RV{X}}$.

We only need to check validity for each $x\in \RV{X}(\Omega)$, as it is automatically satisfied for other values of $\RV{X}$.

For all $x\in \RV{X}(\Omega)$, $B\in \sigalg{Y}$ such that $\RV{X}\yields \{x\}\cap\RV{Y}\yields B=\emptyset$, $\prob{P}^{\RV{Y}|\RV{X}}(B|x)=0$ by validity. Thus for arbitrary $C\in \sigalg{Z}$
\begin{align}
    \prob{R}^{\RV{YZ}|\RV{X}}(B\times C|x) &= \int_B \prob{Q}^{\RV{Z}|\RV{YX}}(C|y,x)\prob{P}^{\RV{Y}|\RV{X}}(dy|x)\\
                                  &\leq \prob{P}^{\RV{Y}|\RV{X}}(B|x)\\
                                  &=0
\end{align}

For all $\{x\}\times B$such that $\RV{X}\yields \{x\}\cap\RV{Y}\yields B\neq \emptyset$ and $C\in \sigalg{Z}$ such that $(\RV{X},\RV{Y},\RV{Z})\yields \{x\}\times B\times C=\emptyset$, $\prob{Q}^{\RV{Z}|\RV{YX}}(C|y,x)=0$ for all $y\in B$ by validity. Thus:
\begin{align}
    \prob{R}^{\RV{YZ}|\RV{X}}(B\times C|x) &= \int_B \prob{Q}^{\RV{Z}|\RV{YX}}(C|y,x)\prob{P}^{\RV{Y}|\RV{X}}(dy|x)\\
                                            &=0
\end{align}
\end{proof}

\begin{corollary}[Valid conditionals are validly extendable to valid distributions]\label{corr:valid_extend_order1}
Given $\Omega$, $\RV{U}:\Omega\to U$, $\RV{W}:\Omega\to W$ and a valid conditional $\prob{T}^{\RV{W}|\RV{U}}$, then for any valid conditional $\prob{V}^{\RV{U}}$, $\prob{V}^{\RV{U}}\odot \prob{T}^{\RV{W}|\RV{U}}$ is a valid probability.
\end{corollary}

\begin{proof}
Applying Lemma \ref{lem:valid_extendability} choosing $\RV{X}=*$, $\RV{Y}=\RV{U}$, $\RV{Z}=\RV{W}$ and $\prob{P}^{\RV{Y}|\RV{X}}=\prob{V}^{\RV{U}|*}$ and $\prob{Q}^{\RV{Z}|\RV{YX}}=\prob{T}^{\RV{W}|\RV{U*}}$ we have $\prob{R}^{WU|*}:=\prob{V}^{\RV{U}|*}\odot \prob{T}^{\RV{W}|\RV{U}*}$ is a valid conditional probability. Then $\prob{R}^{\RV{WU}}\cong \prob{R}^{\RV{WU}|*}$ is valid by Theorem \ref{th:valid_agree}.
\end{proof}

\begin{theorem}[Validity of conditional probabilities]\label{th:valid_conditional_probability}
Suppose we have $\Omega$, $\RV{X}:\Omega\to X$, $\RV{Y}:\Omega\to Y$, with $\Omega$, $X$, $Y$ discrete. A conditional $\prob{T}^{\RV{Y}|\RV{X}}$ is valid if and only if for all valid distributions $\prob{V}^{\RV{X}}$, $\prob{V}^{\RV{X}}\odot \prob{T}^{\RV{Y}|\RV{X}}$ is also a valid distribution.
\end{theorem}

\begin{proof}
If: this follows directly from Corollary \ref{corr:valid_extend_order1}.

Only if: suppose $\prob{T}^{\RV{Y}|\RV{X}}$ is invalid. Then there is some $x\in X$, $y\in Y$ such that $\RV{X}\yields(x)\neq \emptyset$, $(\RV{X},\RV{Y})\yields(x,y)=\emptyset$ and $\prob{T}^{\RV{Y}|\RV{X}}(y|x)>0$. Choose $\prob{V}^{\RV{X}}$ such that $\prob{V}^{\RV{X}}(\{x\})=1$; this is possible due to standard measurability and valid due to $\RV{X}^{-1}(x)\neq \emptyset$. Then
\begin{align}
    (\prob{V}^{\RV{X}}\odot \prob{T}^{\RV{Y}|\RV{X}})(x,y) &= \prob{T}^{\RV{Y}|\RV{X}}(y|x) \prob{V}^{\RV{X}}(x)\\
                                                                     &= \prob{T}^{\RV{Y}|\RV{X}}(y|x)\\
                                                                     &>0
\end{align}
Hence $\prob{V}^{\RV{X}}\odot \prob{T}^{\RV{Y}|\RV{X}}$ is invalid.
\end{proof}


\begin{example}\label{ex:invalidity}
Body mass index is defined as a person's weight divided by the square of their height. Suppose we have a measurement process $\proc{S}=(\proc{W},\proc{H})$ and $\proc{B}=\frac{\proc{W}}{\proc{H}^2}$ - i.e. we figure out someone's body mass index first by measuring both their height and weight, and then passing the result through a function that divides the second by the square of the first. Thus, given the random variables $\RV{W},\RV{H}$ modelling $\proc{W},\proc{H}$, $\proc{B}$ is the function given by $\RV{B}=\frac{\RV{W}}{\RV{H}^2}$.

With this background, suppose we postulate a decision model in which body mass index can be directly controlled by a variable $\RV{C}$, while height and weight are not. Specifically, we have a probability set $\prob{P}_\square$ with
\begin{align}
    \prob{P}_\square^{\RV{B}|\RV{WHC}} &= \tikzfig{invalid_BMI_model} \label{eq:bmi_example}
\end{align}
Then pick some $w,h,x\in\mathbb{R}$ such that $\frac{w}{h^2}\neq x$ and $(\RV{W},\RV{H})\yields (w,h)\neq \emptyset$ (which is to say, our measurement procedure could potentially yield $(w,h)$ for a person's height and weight). We have $\prob{P}_\square^{\RV{B}|\RV{WHC}}(\{x\}|w,h,x)=1$, but 
\begin{align}
    (\RV{B},\RV{W},\RV{H})\yields \{(x,w,h)\} &= \{\omega|(\RV{W},\RV{H})(\omega) = (w,h),\RV{B}(\omega) = \frac{w}{h^2}\}\\
    &=\emptyset
\end{align}
so $\prob{P}_\square^{\RV{B}|\RV{WHC}}$ is invalid. Thus there is some valid $\mu^{\RV{WHC}}$ such that the probability set $\prob{P}_{\square}^{\RV{BWHC}} = \mu^{\RV{WHC}}\odot \prob{P}_\square^{\RV{Y}|\RV{X}}$ is empty.

Validity rules out conditional probabilities like \ref{eq:bmi_example}. We conjecture that in many cases this condition is implicitly taken into account -- it is obviously silly to posit a model in which body mass index can be controlled independently of height and weight. We note, however, that presuming the authors intended their model to be interpreted according to the usual semantics of causal Bayesian networks, the invalid conditional probability \ref{eq:bmi_example} would be used to evaluate the causal effect of body mass index in the causal diagram found in \citet{shahar_association_2009}.
\end{example}



\subsection{Existence of conditional probabilities}


\begin{lemma}[Conditional pushforward]\label{th:recurs_pushf}
Suppose we have a sample space $(\Omega,\sigalg{F})$, variables $\RV{X}:\Omega\to X$ and $\RV{Y}:\Omega\to Y$, $\RV{Z}:\Omega\to Z$ and a probability set $\prob{P}_C$ with conditional $\prob{P}_C^{\RV{X}|\RV{Y}}$ such that $\RV{Z}=f\circ \RV{Y}$ for some $f:Y\to Z$. Then there exists a conditional probability $\prob{P}_C^{\RV{Z}|\RV{X}}=\prob{P}_C^{\RV{Y}|\RV{X}}\kernel{F}_{f}$.
\end{lemma}

\begin{proof}
Note that $(\RV{X},\RV{Z})=(\text{id}_X\otimes f)\circ (\RV{X},\RV{Y})$. Thus, by Lemma \ref{lem:pushf_kprod}, for any $\prob{P}_\alpha\in \prob{P}_C$

\begin{align}
    \prob{P}_\alpha^{\RV{XZ}} = \prob{P}_\alpha^{\RV{XY}}\kernel{F}_{\text{id}_X\otimes f}
\end{align}

Note also that for all $A\in\sigalg{X}$, $B\in \sigalg{Z}$, $x\in X$, $y\in Y$:

\begin{align}
\prob{F}_{\text{id}_X\otimes f}(A\times B|x,y)&=\delta_x(A)\delta_{f(y)}(B)\\
&= \prob{F}_{\text{id}_X} (A|x)\otimes \prob{F}_f(B|y)\\
\implies \prob{F}_{\text{id}_X\otimes f} &= \prob{F}_{\text{id}_X} \otimes \prob{F}_f
\end{align}

Thus

\begin{align}
    \prob{P}_\alpha^{\RV{XZ}} &= (\prob{P}_\alpha^{\RV{X}}\odot \prob{P}_C^{\RV{Y}|\RV{X}})\kernel{F}_{\text{id}_X}\otimes \kernel{F}_f\\
    &= \tikzfig{conditional_pushforward}
\end{align}

Which implies $\prob{P}_C^{\RV{Y}|\RV{X}}\kernel{F}_{f}$ is a version of $\prob{P}_{\alpha}^{\RV{Z}|\RV{X}}$. Because this holds for all $\alpha$, it is therefore also a version of $\prob{P}_C^{\RV{Z}|\RV{X}}$.
\end{proof}

The following theorem is a standard result in many probability texts. In this work, the measurable spaces considered will all be standard measurable and so Theorem \ref{th:reg_cond} always applies. We will simply assume that conditional probabilities exist, and avoid referencing this theorem every time.

\begin{theorem}[Existence of regular conditionals]\label{th:reg_cond}
Suppose we have a sample space $(\Omega,\sigalg{F})$, variables $\RV{X}:\Omega\to X$ and $\RV{Y}:\Omega\to Y$ with $Y$ standard measurable and a probability model $\prob{P}_{\alpha}$ on $(\Omega,\sigalg{F})$. Then there exists a conditional $\prob{P}_\alpha^{\RV{Y}|\RV{X}}$.
\end{theorem}

\begin{proof}
\citet[Theorem 2.18]{cinlar_probability_2011}
\end{proof}

The following theorem was proved by \citet{bogachev_kantorovich_2020}.

\begin{theorem}\label{th:bogachev}
Given a Borel measurable map $m:X\kto Y\times Z$ let $\Pi_Y:Y\times Z\to Y$ be the projection onto $Y$. Then there exists a Borel measurable map $n:X\times Y\kto Y\times Z$ such that 
\begin{align}
    n(\Pi_y^{-1}(y)|x,y) &= 1\label{eq:proper}\\
    m(\RV{Y}^{-1}(A)\cap B|x) &= \int_A n(B|x,y) m\kernel{F}_{\Pi_Y}(dy|x)&\forall A\in \sigalg{Y},B\in\sigalg{Y\times Z}\label{eq:conditional1}
\end{align}
\end{theorem}

\begin{proof}
 \citet[Theorem 3.5]{bogachev_kantorovich_2020}
\end{proof}

The following corollary implies that, given a uniform conditional, higher order conditionals can generically be found for probability sets.

\begin{corollary}[Existence of higher order conditionals with respect to probability sets]\label{th:ho_cond_psets}
Take a sample space $(\Omega,\sigalg{F})$, variables $\RV{X}:\Omega\to X$ and $\RV{Y}:\Omega\to Y$, $\RV{Z}:\Omega\to Z$ and a probability set $\prob{P}_C$ with uniform conditional distribution $\prob{P}_C^{\RV{YZ}|\RV{X}}$, and $Y$ and $Z$ standard measurable. Then there exists a higher order uniform conditional $\prob{P}_C^{\RV{Z}|(\RV{Y}|\RV{X})}$.
\end{corollary}

\begin{proof}
Take $\prob{P}_C^{\RV{YZ}|\RV{X}}$ to be the Borel measurable map $m$ from Theorem \ref{th:bogachev}, and note that $\Pi_Y\circ (\RV{Y},\RV{Z})=\RV{Y}$. Then equation \ref{eq:conditional1} implies for all $A\in \sigalg{Y},B\in\sigalg{Y\times Z}$ there is some $n$ such that

\begin{align}
    \prob{P}_C^{\RV{YZ}|\RV{X}}(\RV{Y}^{-1}(A)\cap B|x) &= \int_A n(B|x,y) \prob{P}_C^{\RV{YZ}|\RV{X}}\kernel{F}_{\Pi_Y}(dy|x)\\
    &=\int_A n(B|x,y) \prob{P}_C^{\RV{Y}|\RV{X}}(dy|x)\label{eq:rec_push}
\end{align}
where Equation \ref{eq:rec_push} follows from Lemma \ref{th:recurs_pushf}.

Then, for any $\prob{P}_\alpha\in\prob{P}_C$
\begin{align}
    \prob{P}_C^{\RV{YZ}|\RV{X}}(\RV{Y}^{-1}(A)\cap B|x) &= \int_A n(B|x,y) \prob{P}_{\alpha}^{\RV{Y}|\RV{X}}(dy|x)
\end{align}
which implies $n$ is a version of $\prob{P}_C^{\RV{YZ}|(\RV{Y}|\RV{X})}$. By Lemma \ref{th:recurs_pushf}, $n\kernel{F}_{\Pi_Y}$ is a version of $\prob{P}_C^{\RV{Z}|(\RV{Y}|\RV{X})}$.
\end{proof}

We might be motivated to ask whether the higher order conditionals in Theorem \ref{th:ho_cond_psets} can be chosen to be valid. Despite Lemma \ref{lem:proper_implies_valid} showing that the existence of proper conditional probabilities implies the existence of valid ones, we cannot make use of this in the above theorem because Equation \ref{eq:proper} makes $n$ proper with respect to the ``wrong'' sample space $(Y\times Z, \sigalg{Y}\otimes\sigalg{Z})$ while what we would need is a proper conditional probability with respect to $(\Omega,\sigalg{F})$.

We can choose higher order conditionals to be valid in the case of discrete sets, and whether we can choose them to be valid in more general measurable spaces is an open question.

\begin{lemma}\label{lem:proper_implies_valid}
Given a probability space $(\mu,\Omega,\sigalg{F})$ and variables $\RV{X}:\Omega\to X$, $\RV{Y}:\Omega\to Y$, if there is a regular proper conditional probability $\mu^{|\RV{X}}:X\kto \Omega$ then there is a valid conditional distribution $\mu^{\RV{Y}|\RV{X}}$.
\end{lemma}

\begin{proof}
Take $\kernel{K}=\mu^{|\RV{X}}\kernel{F}_{\RV{Y}}$. We will show that $\kernel{K}$ is valid, and a version of $\mu^{\RV{Y}|\RV{X}}$.

Defining $\RV{O}:=\text{id}_{\Omega}$ (the identity function $\Omega\to \Omega$), $\mu^{|\RV{X}}$ is a version of $\mu^{\RV{O}|\RV{X}}$. Note also that $\RV{Y}=\RV{Y}\circ\RV{O}$. Thus by Lemma \ref{th:recurs_pushf}, $\kernel{K}$ is a version of $\mu^{\RV{Y}|\RV{X}}$.

It remains to be shown that $\kernel{K}$ is valid. Consider some $x\in X$, $A\in \sigalg{Y}$ such that $\RV{X}^{-1}(\{x\})\cap \RV{Y}^{-1}(A)=\emptyset$. Then by the assumption $\mu^{|\RV{X}}$ is proper
\begin{align}
    \kernel{K}(\RV{Y}\yields A|x) &= \delta_x(\RV{Y}^{-1}(A))\\
    &= 0
\end{align}

Thus $\kernel{K}$ is valid.
\end{proof}


\begin{theorem}[Higher order conditionals]\label{th:higher_order_conditionals}
Suppose we have a sample space $(\Omega,\sigalg{F})$, variables $\RV{X}:\Omega\to X$ and $\RV{Y}:\Omega\to Y$, $\RV{Z}:\Omega\to Z$ and a probability set $\prob{P}_C$ with conditional $\prob{P}_C^{\RV{YZ}|\RV{X}}$. Then $\prob{P}_C^{\RV{Z}|(\RV{Y}|\RV{X})}$ is a version of $\prob{P}_C^{\RV{Z}|\RV{Y}\RV{X}}$ 
\end{theorem}

\begin{proof}
For arbitrary $\prob{P}_{\alpha}\in \prob{P}_C$
\begin{align}
    \prob{P}_\alpha^{\RV{YZ}|\RV{X}} &= \tikzfig{higher_order_disint}\\
    \implies \prob{P}_\alpha^{\RV{XYZ}} &= \tikzfig{higher_order_disint_0}\\
    &= \tikzfig{higher_order_disint_1}\\
    &= \tikzfig{higher_order_disint_2}
\end{align}
Thus $\prob{P}_C^{\RV{Z}|(\RV{Y}|\RV{X})}$ is a version of $\prob{P}_{\alpha}^{\RV{Z}|\RV{Y}\RV{X}}$ for all $\alpha$ and hence also a version of $\prob{P}_C^{\RV{Z}|\RV{Y}\RV{X}}$.
\end{proof}


\begin{theorem}
Given probability gap model $\prob{P}_C$, $\RV{X}$, $\RV{Y}$, $\RV{Z}$ such that $\prob{P}_C^{\RV{Z}|\RV{YX}}$ exists, $\prob{P}_C^{\RV{Z}|\RV{Y}}$ exists iff $\RV{Z}\CI_{\prob{P}_C} \RV{X}|\RV{Y}$.
\end{theorem}

\begin{proof}
If:
If $\RV{Z}\CI_{\prob{P}_C} \RV{X}|\RV{Y}$ then by Theorem \ref{th:cho_ci_equiv}, for each $\prob{P}_\alpha\in \prob{P}_C$ there exists $\prob{P}_{\alpha}^{\RV{Z}|\RV{Y}}$ such that
\begin{align}
    \prob{P}_\alpha^{\RV{Y}|\RV{WX}} &= \tikzfig{cond_indep_erase}
\end{align}
\end{proof}


\begin{theorem}[Valid higher order conditionals]
Suppose we have a sample space $(\Omega,\sigalg{F})$, variables $\RV{X}:\Omega\to X$ and $\RV{Y}:\Omega\to Y$, $\RV{Z}:\Omega\to Z$ and a probability set $\prob{P}_C$ with regular conditional $\prob{P}_C^{\RV{YZ}|\RV{X}}$, $Y$ discrete and $Z$ standard measurable. Then there exists a valid regular $\prob{P}_C^{\RV{Z}|\RV{XY}}$.
\end{theorem}

\begin{proof}
By Theorem \ref{th:ho_cond_psets}, we have a higher order conditional $\prob{P}_C^{\RV{Z}|(\RV{Y}|\RV{X})}$ which, by Theorem \ref{th:higher_order_conditionals} is also a version of $\prob{P}_C^{\RV{Z}|\RV{XY}}$.

We will show that there is a Markov kernel $\kernel{Q}$ almost surely equal to $\prob{P}_C^{\RV{Z}|\RV{XY}}$ which is also valid. For all $x,y\in X\times Y$, $A\in\sigalg{Z}$ such that $(\RV{X},\RV{Y},\RV{Z})\yields\{(x,y)\}\times A=\emptyset$, let $\kernel{Q}(A|x,y)=\prob{P}_C^{\RV{Z}|\RV{XY}}(A|x,y)$.

By validity of $\prob{P}_C^{\RV{YZ}|\RV{X}}$, $x\in \RV{X}(\Omega)$ and $(\RV{X},\RV{Y},\RV{Z})\yields\{(x,y)\}\times A=\emptyset$ implies $\prob{P}_C^{\RV{YZ}|\RV{X}}(\{y\}\times A|x)=0$. Thus we need to show

\begin{align}
    \forall A\in \sigalg{Z}, x\in X, y\in Y: \prob{P}_C^{\RV{YZ}|\RV{X}}(\{y\}\times A|x)=0 \implies \left(\prob{Q}(A|x,y) = 0\right) \lor \left((\RV{X},\RV{Y})\yields \{(x,y)\} = \emptyset\right)
\end{align}

For all $x,y$ such that $\kernel{P}_{\{\}}^{\RV{Y}|\RV{X}}(\{y\}|x)$ is positive, we have $\model{P}^{\RV{YZ}|\RV{X}}(\{y\}\times A|x)=0\implies \prob{P}_\square^{\RV{Z}|\RV{XY}}(A|x,y)=0=:\kernel{Q}(A|x,y)$.

Furthermore, where $\kernel{P}_{\{\}}^{\RV{Y}|\RV{X}}(\{y\}|x)=0$, we either have $(\RV{X},\RV{Y},\RV{Z})\yields\{(x,y)\}\times A= \emptyset$ or can choose some $\omega\in (\RV{X},\RV{Y},\RV{Z})\yields\{(x,y)\}\times A$ and let $\kernel{Q}(\RV{Z}(\omega)|x,y) = 1$. This is an arbitrary choice, and may differ from the original $\prob{P}_C^{\RV{Z}|\RV{XY}}$. However, because $Y$ is discrete the union of all points $y$ where $\kernel{P}_{\{\}}^{\RV{Y}|\RV{X}}(\{y\}|x)=0$ is a measure zero set, and so $\kernel{Q}$ differs from $\kernel{P}_{\{\}}^{\RV{Y}|\RV{X}}$ on a measure zero set.
\end{proof}

%!TEX root = main.tex

\chapter{Chapter 3: See-do models}

Consider the following problem: you are presented with a collection $\Theta$ of hypotheses about how the world might function and a vector $\mathbf{x}$ of observational data which you know could have taken values in some space $X$. You want to determine which hypothesis $\theta\in \Theta$ best describes the world. However you ultimately solve the problem, the next step you take will probably be to determine for each $\theta\in \Theta$ a probability distribution $\kernel{P}_\theta\in \Delta(\sigalg{X})$ that indicates how likely you would be to observe the various elements of $X$ were $\theta$ in fact the case. This is a \emph{statistical model} -- an indexed set of probability distributions $\{\prob{P}_\theta|\theta\in \Theta\}$. Statistical models are ubiquitous in the field of statistics -- they are found in statistical decision theory where the elements of $\Theta$ are typically called ``states''\citep{wald_statistical_1950}, in Bayesian inference where the elements of $\Theta$ may be called ``parameters'' \citep{freedman_asymptotic_1963} and in frequentist inference where elements of $\Theta$ they may be called ``hypotheses'' \citep{fisher_statistical_1992}. 

These different approaches to statistics may have different notions of what the ``best hypothesis'' $\theta$ is, may employ different estimation methods and may not even agree about what ``distributed according to $\kernel{P}_\theta$'' means. Nonetheless, the interpretation of the statistical model in each case is roughly the same: supposing $\theta\in\Theta$ is true, the data will be distributed according to $\kernel{P}_\theta$. A statistical model takes a hypothesis and tells you what you are likely to \emph{see}.

Sometimes we are interested in modelling situations where we can also make some choices that also affect the eventual consequences. For example, I might hypothesise $\theta_1$: the switch on the wall controls my light, $\theta_2$: the switch on the wall does not control my light. Then, given $\theta_1$ I can choose to toggle the switch, and I will see my light turn on, or I can choose not to toggle the switch and I will not see my light turn on. Given $\theta_2$, neither choice will result in a light turned on. Choices are clearly different to hypotheses: the choice I make depends on what I want to happen, while whether or not a hypothesis is true has no regard for my ambitions.

A ``statistical model with choices'' is simply a map $\prob{T}:D\times \Theta\to \Delta(\sigalg{E})$ for some set of choices $D$, hypotheses $\Theta$ and outcome space $(E,\sigalg{E})$. We can also distinguish two types of outcomes: \emph{observations} which are given prior to a choice being made and \emph{consequences} which happen after a choice is made. Observations cannot be affected by the choices made, while consequences are not subject to this restriction. That is, observations are what we might \emph{see} before making a choice, which depends on the hypothesis alone, and if we are lucky we may be able to invert this dependence to learn something about the hypothesis from observations. On the other hand, the consequences of what we \emph{do} depends jointly on the hypothesis and the choice we make and we judge which choices are more desirable on the basis of which consequences we expect them to produce. 

What we are studying is a family of models that generalises of statistical models to include hypotheses, choices, observations and consequences. These models are referred to as \emph{see-do models}. Hypotheses, observations, consequences and choices are not individually new ideas. \emph{Statistical decision problems} \citep{wald_statistical_1950,savage_foundations_1972} extend statistical models with decisions and \emph{losses}. Like consequences, losses depend on which choices are made. However, unlike consequences, losses must be ordered and reflect the preferences of a decision maker. \emph{Influence diagrams} are directed graphs created to represent decision problems that feature ``choice nodes'', ``chance nodes'' and ``utility nodes''. An influence diagram may be associated with a particular probability distribution \cite{nilsson_evaluating_2013} or with a set of probability distributions \cite{dawid_influence_2002}.

See-do models have deep roots in decision theory. Decision theory asks, out of a set of available acts, which ones ought to be chosen. See-do models answer an intermediate question: out of a set of available acts, what are the consequences of each? This question is described by \citet{pearl_causality:_2009} as an ``interventional'' question. To model questions described by Pearl as ``counterfactual'', we can extend see-do models with a notion of \emph{parallel choices}.  This relies on interpreting the choice set $D$ as a set of \emph{counterfactual propositions} rather than a set of \emph{deicisions} or \emph{actions}. It may be possible to perform regular actions in sequence - I can do action ``1'', then action ``2'', and I can consider the results of performing both actions in sequence. In contrast, with counterfactual propositions I can consider the joint results of performing action ``1'' and performing action ``2'' \emph{instead of} action ``1''. can be thought of as imagining the results of performing one experiment multiple times, but making slightly different choices each time. I present a set of conditions formalising this general idea, and show that these conditions imply the existence of \emph{Potential Outcomes}. \todo[inline]{And that seem to have an interesting relationship to the assumptions in Bell-type theorems that rule out quantum behaviour of hidden variable theories, but at this stage I haven't fully worked this out and it isn't a high priority}

\section{Definition}

The primitives of see-do models are:

\begin{itemize}
    \item \emph{Hypotheses}: each hypothesis  
    \item \emph{Choices}; the set of options I have available
    \item \emph{Observations}; the observations I might be given
    \item \emph{Consequences}; what might arise as a result of my having chosen a particular option
    \item \emph{Uncertainty}; I am uncertain about which consequences are brought about by my available options, and I may be able to become less uncertain after considering the data I have been given
\end{itemize}

\emph{See-Do Models} formalise the above notions. See-Do Models use expected utility theory to formalise preferences and probability theory to model noisy observations and uncertainty over consequences. We make these choices because these are well understood and widely accepted tools for modelling preferences and uncertainty respectively. For a given CSDP, See-Do Models regard the \emph{option set}, the \emph{observation space} and the \emph{consequence space} to be fixed.

\begin{definition}[Option set]
An \emph{option set} $D$ is a finite or countable set of options. A decision maker may select any option from $D$ and in addition may select any mixture of options from $\Delta(\sigalg{D})$.
\end{definition}

\begin{definition}[Observation space]
The decision maker receives an \emph{observation}, which is an element of a standard measurable set $(X,\sigalg{X})$.
\end{definition}

\begin{definition}[Consequence space]
The decision maker's choice of option results in a \emph{consequence}, which is an element of a standard measurable set $(Y,\sigalg{Y})$.
\end{definition}

We allow for two types of uncertainty over which consequences will actually take place given the selection of a particular option. Consequences may be stochastic functions of the option selected and of the given data. Secondly, a decision maker may entertain a number of different \emph{hypotheses} about the relationship between data, decision and consequences. A decision maker may select a particular distribution of hypotheses called a \emph{prior} so that the only remaining uncertainty is stochastic, but we avoid assuming that a canonical prior is available at the outset.

\begin{definition}[See-Do Model]
A See-Do Model takes a hypothesis $\theta\in \Theta$ and an option $d\in D$ and returns a joint distribution over observations and consequences with the restriction that, given a particular hypothesis, observations are independent of options. Formally, a See-Do Model is a kernel space $(\kernel{T},X\times Y, \Theta\times D)$ where $\kernel{T}:\Theta\times D\to \Delta(\sigalg{X}\otimes\sigalg{Y})$ where $\Theta$ is the hypothesis space, $D$ is the option set, $X$ is the observation space and $Y$ is the consequence space. Defining $\RV{X},\RV{Y},\RV{D},\Theta$ to be projection maps to the associated spaces, a see-do model has the property $\RV{X}\CI_{\kernel{T}} \RV{D}|\Theta$. That is, for each hypothesis $\theta\in \Theta$, $\kernel{T}$ holds that the observations $\RV{X}$ are independent of the choice of option $\RV{D}$.

It is therefore possible to specify a see-do model $\kernel{T}$ with the following elements:
\begin{itemize}
    \item Hypothesis space $\Theta$, options $D$, observations $X$ and consequences $Y$
    \item Observation map $\kernel{T}^{\RV{X}|\Theta}$, which exists by virtue of the independence of observations from consequences
    \item Consequence map $\kernel{T}^{\RV{Y}|\Theta\RV{X}\RV{D}}$
\end{itemize}
\end{definition}

\begin{definition}[Hypothesis sufficiency]
According to the general definition of a see-do model, observations provide evidence about which hypothesis $\theta$ is correct \emph{and also} may directly affect the consequences. See-do models may be simplified if only the hypothesis and decision affects the consequence. The hypotheses are \emph{sufficient} for a See-Do Model $(\kernel{T},X\times Y, \Theta\times D)$ if $\RV{Y}\CI_{\kernel{T}} \RV{X}|\Theta\RV{D}$.

A hypothesis sufficient see-do model can be specified with:

\begin{itemize}
    \item Hypothesis space $\Theta$, options $D$, observations $X$ and consequences $Y$
    \item Observation map $\kernel{T}^{\RV{X}|\Theta}$, which exists by virtue of the independence of observations from consequences
    \item Consequence map $\kernel{T}^{\RV{Y}|\Theta\RV{D}}$
\end{itemize}
\end{definition}

\begin{definition}[Bayesian See-Do Model]
A Bayesian See-Do Model $\kernel{U}$ is a Markov kernel $D\to \Delta(\sigalg{X}\otimes\sigalg{Y})$ with the property $\RV{X}\CI_{\kernel{U}}\RV{D}$.

A Bayesian See-Do Model can be constructed from a see-do model $\kernel{T}$ by choosing an arbitrary prior $\gamma\in \Delta(\Theta)$ and taking the product:

\begin{align}
    \kernel{U} &= (\gamma\otimes \mathrm{Id}^D)\kernel{T}
\end{align}

For all $A\in \sigalg{X}$, $B\in\sigalg{Y}$, $d\in D$:
\begin{align}
    \kernel{U}_d(A\times B) &= \int_\Theta \int_A \kernel{T}_{\theta,x,d}^{\RV{Y}|\RV{X}\RV{D}\Theta} (B) d\kernel{T}_{\theta}^{\RV{X}|\Theta}(x) d\gamma(\theta)
\end{align}

In this case, $\kernel{U}^{\RV{Y}|\RV{X}\RV{D}\Theta}\overset{a.s.}{=}\kernel{T}^{\RV{Y}|\RV{X}\RV{D}\Theta}$ and $\kernel{U}^{\RV{X}|\Theta}\overset{a.s.}{=}\kernel{T}^{\RV{X}|\Theta}$


\end{definition}

\subsubsection{Examples of see-do models}

Suppose we are betting on the outcome of the flip of a possibly biased coin with payout 1 for a correct guess and 0 for an incorrect guess, and we are given $N$ previous flips of the coin to inspect. This situation can be modeled by a hypothesis sufficient see-do model. Define $\kernel{B}:(0,1)\to \Delta(\{0,1\})$ by $\kernel{B}:\theta\mapsto \mathrm{Bernoulli}(\theta)$. Then define $\kernel{T}^{(1)}$ by:

\begin{itemize}
    \item $D=\{0,1\}$
    \item $X=\{0,1\}^N$
    \item $Y=\{0,1\}$
    \item $\Theta=(0,1)$
    \item $\kernel{T}^{\RV{X}|\Theta(1)}:\splitter{0.1}^N\kernel{B}$
    \item $\kernel{T}^{\RV{Y}|\RV{D}\Theta(1)}:(\theta,d)\mapsto \mathrm{Bernoulli}(1-|d-\theta|)$
\end{itemize}

In this model, the chance $\theta$ of the coin landing on heads is as much as we can hope to know about how our bet will work out.

Suppose instead that in addition to the $N$ prior flips, we manage to sneak a look at the outcome of the flip on which we will bet. In this case, the situation can be modeled by the following hypothesis insufficient see-do model $\kernel{T}^{(2)}$:

\begin{itemize}
    \item $D=\{0,1\}$
    \item $X=\{0,1\}^{N+1}$
    \item $Y=\{0,1\}$
    \item $\Theta=(0,1)$
    \item $\kernel{T}^{\RV{X}|\Theta (2)}:\splitter{0.1}^{N+1}\kernel{B}$
    \item $\kernel{T}^{\RV{Y}|\RV{D}\RV{X}\Theta(2)}:(\theta,\mathbf{x},d)\mapsto \delta_{1-|d-x_{N+1}|}$
\end{itemize}

In this case, the observed data tells us more about how the bet will work out than the hypothesis alone.

It is also possible to model the second situation with a hypothesis sufficient model by including the result of the $N+1$th flip in the hypothesis. Define the new hypothesis space $\Theta'=(0,1)\times\{0,1\}$ and define $\kernel{T}^{(3)}$ by:

\begin{itemize}
    \item $D=\{0,1\}$
    \item $X=\{0,1\}^{N+1}$
    \item $Y=\{0,1\}$
    \item $\Theta'=(0,1)\times\{0,1\}$
    \item $\kernel{T}^{\RV{X}|\Theta'(3)}:(\splitter{0.1}^N\kernel{B}\otimes \delta_{x_{N+1}}$
    \item $\kernel{T}^{\RV{Y}|\RV{D}\Theta'(3)}:(\theta',x_{N+1},d)\mapsto \delta_{1-|d-x_{N+1}|}$
\end{itemize}

However, $\RV{X}_{N+1}$ is related to the previous flips $\vecRV{X}_{<N}$. In particular, given $\theta\in \Theta$, $\RV{X}_{N+1}$ should be distributed according to Bernoulli($\theta$). That is, defining $\Theta:\Theta'\times D\times X\times Y\to (0,1)$ to be the projection map that yields the parameter $\theta$, any Bayesian model $\kernel{U}$ should have the property $\kernel{U}^{\RV{X}_{N+1}|\Theta}=\kernel{B}$. Then for any $A\in \sigma(\{0,1\}^N), B\in \sigma(\{0,1\}), C\in \sigma(\{0,1\})$

\begin{align}
    \kernel{U}^{(3)}_d(A\times B\times C) &= \int_{\Theta} \int_A \int_{B}   \kernel{T}_{\theta',\mathbf{x},d}^{\RV{Y}|\RV{X}\RV{D}\Theta'(3)} (C) d\kernel{U}^{\RV{X}_{N+1}|\Theta}_\theta(x_{N+1}) d\kernel{T}_{\theta}^{\RV{X}|\Theta'(3)}(\mathbf{x}) d\kernel{U}^{\Theta}(\theta)\\
                                  &= \int_{\Theta} \int_A \int_{B}  \kernel{T}_{\theta',\mathbf{x},d}^{\RV{Y}|\RV{X}\RV{D}\Theta'(3)} (C) d\kernel{B}_\theta(x_{N+1}) d\kernel(\splitter{0.1}^{N}\kernel{B})(\mathbf{x}) d\kernel{U}^{\Theta}(\theta)\\
                                  &= \int_{\Theta} \int_{A\times B}  \kernel{T}_{\theta',\mathbf{x},d}^{\RV{Y}|\RV{X}\RV{D}\Theta'(2)} (C) d\kernel(\splitter{0.1}^{N+1}\kernel{B})(\mathbf{x}) d\kernel{U}^{\Theta}(\theta)\\
\end{align}

Which is equivalent to $\kernel{T}^{(2)}$ equipped with the prior $\kernel{U}^{\Theta}\in \Delta(\Theta)$. Thus the difference between $\kernel{T}^{(2)}$ and $\kernel{T}^{(3)}$ is whether a constraint is expressed in the model (as in $\kernel{T}^{(2)}$) or in over the class of allowable priors (as in $\kernel{T}^{(3)}$).

\todo[inline]{I don't have a theory of stochastic vs non-stochastic uncertainty, but it is the case in general that a hypothesis sufficient model with additional restrictions on the prior can be replaced by a hypothesis insufficient model with no restrictions on the prior. This is mainly relevant with regard to counterfactuals}






\begin{itemize}
    \item \emph{Option sets}; the set of options I have available
    \item \emph{Observations}; the observations I might be given
    \item \emph{Consequences}; what might arise as a result of my having chosen a particular option
    \item \emph{Preferences}; which consequences I want, and which I do not want
    \item \emph{Uncertainty}; I am uncertain about which consequences are brought about by my available options, and I may be able to become less uncertain after considering the data I have been given
\end{itemize}

\emph{See-Do Models} formalise the above notions. See-Do Models use expected utility theory to formalise preferences and probability theory to model noisy observations and uncertainty over consequences. We make these choices because these are well understood and widely accepted tools for modelling preferences and uncertainty respectively. For a given CSDP, See-Do Models regard the \emph{option set}, the \emph{observation space} and the \emph{consequence space} to be fixed.

\begin{definition}[Option set]
An \emph{option set} $D$ is a finite or countable set of options. A decision maker may select any option from $D$ and in addition may select any mixture of options from $\Delta(\sigalg{D})$.
\end{definition}

\begin{definition}[Observation space]
The decision maker receives an \emph{observation}, which is an element of a standard measurable set $(X,\sigalg{X})$.
\end{definition}

\begin{definition}[Consequence space]
The decision maker's choice of option results in a \emph{consequence}, which is an element of a standard measurable set $(Y,\sigalg{Y})$.
\end{definition}

We allow for two types of uncertainty over which consequences will actually take place given the selection of a particular option. Consequences may be stochastic functions of the option selected and of the given data. Secondly, a decision maker may entertain a number of different \emph{hypotheses} about the relationship between data, decision and consequences. A decision maker may select a particular distribution of hypotheses called a \emph{prior} so that the only remaining uncertainty is stochastic, but we avoid assuming that a canonical prior is available at the outset.

\begin{definition}[See-Do Model]
A See-Do Model takes a hypothesis $\theta\in \Theta$ and an option $d\in D$ and returns a joint distribution over observations and consequences with the restriction that, given a particular hypothesis, observations are independent of options. Formally, a See-Do Model is a Markov kernel $\kernel{T}:\Theta\times D\to \Delta(\sigalg{X}\otimes\sigalg{Y})$ where, given the obvious definitions of observations $\RV{X}$, consequences $\RV{Y}$, options $\RV{D}$ and hypothesis $\Theta$, $\RV{X}\CI_{\kernel{T}}\RV{D}|\Theta$. 

Letting $\kernel{O}$ be a version of $\kernel{T}^{\RV{X}|\Theta}$ and letting $\kernel{S}$ be a version of $\kernel{T}^{\RV{Y}|\RV{D}\RV{X}\Theta}$ we can write

\begin{align}
    \kernel{T} = 
    \begin{tikzpicture} \path (0,0) node (T) {$\Theta$}
        + (0,-1) node (D) {$\RV{D}$}
        ++ (0.5,0) coordinate (copy0)
        ++ (0.5,0) node[kernel] (O) {$\kernel{O}$}
        ++ (0.7,0) coordinate (copy1)
        +  (0.4,-1) node[kernel] (C) {$\kernel{S}$}
        ++ (1.1,0) node (X) {$\RV{X}$}
        +  (0,-1) node (Y) {$\RV{Y}$};
        \draw (T) -- (O) -- (X);
        \draw (copy0) to [out=-90,in=180] ($(C.west) + (0,0)$);
        \draw (D) to [out=0,in=180] ($(C.west) + (0,-0.15)$);
        \draw (copy1) to [out=-60,in=180] ($(C.west)+ (0,0.15)$);
        \draw (C) -- (Y);
    \end{tikzpicture}
\end{align}
\end{definition}

\begin{definition}[Hypothesis sufficiency]
According to the general definition, observations provide evidence about which hypothesis $\theta$ is correct \emph{and also} may directly affect the consequences. The hypotheses are \emph{sufficient} for a See-Do Model $\kernel{T}:\Theta\times D\to \Delta(\sigalg{X}\otimes\sigalg{Y})$ if $\RV{Y}\CI_{\kernel{T}} \RV{X}|\Theta\RV{D}$. In this case, letting $\kernel{O}$ be a version of $\kernel{T}^{\RV{X}|\Theta}$ and $\kernel{C}$ be a version of $\kernel{T}^{\RV{Y}|\RV{D}\Theta}$ we can write

\begin{align}
    \kernel{T} = 
    \begin{tikzpicture} \path (0,0) node (T) {$\Theta$}
        + (0,-1) node (D) {$\RV{D}$}
        ++ (0.5,0) coordinate (copy0)
        ++ (0.5,0) node[kernel] (O) {$\kernel{O}$}
        +  (0.,-1) node[kernel] (C) {$\kernel{C}$}
        ++ (1,0) node (X) {$\RV{X}$}
        +  (0,-1) node (Y) {$\RV{Y}$};
        \draw (T) -- (O) -- (X);
        \draw (copy0) to [out=-90,in=180] ($(C.west) + (0,0.1)$);
        \draw (D) to [out=0,in=180] ($(C.west) + (0,-0.1)$);
        \draw (C) -- (Y);    
    \end{tikzpicture}
\end{align}

To specify a see-do model with sufficient hypotheses we require an observation model $\kernel{O}:\Theta\to \Delta(\sigalg{X})$ and a consequence map $\kernel{C}:\Theta\times D\to \Delta(\sigalg{Y})$. To specify an insufficient model, we require an observation model (as before) and a \emph{state-dependent} consequence map $\kernel{S}:\Theta\times X\times D\to \Delta(\sigalg{Y})$.
\end{definition}

\begin{definition}[Bayesian See-Do Model]
A Bayesian See-Do Model $\kernel{T}$ is a Markov kernel $D\to \Delta(\sigalg{X}\otimes\sigalg{Y})$ with the property $\RV{X}\CI_{\kernel{T}}\RV{D}$. Letting $\kernel{V}\in\kernel{T}^{\RV{Y}|\RV{X}\RV{D}}$, it can be written in the form

\begin{align}
    \kernel{T} = 
    \begin{tikzpicture} \path (0,0) node[dist] (T) {$\gamma$}
        + (0,-1) node (D) {$\RV{D}$}
        ++ (0.5,0) coordinate (copy0)
        ++ (0.5,0) coordinate (O)
        +  (0.3,-1) node[kernel] (C) {$\kernel{V}$}
        ++ (1,0) node (X) {$\RV{X}$}
        +  (0,-1) node (Y) {$\RV{Y}$};
        \draw (T) -- (O) -- (X);
        \draw (copy0) to [out=-90,in=180] ($(C.west) + (0,0)$);
        \draw (D) to [out=0,in=180] ($(C.west) + (0,-0.15)$);
        \draw (C) -- (Y);
    \end{tikzpicture}
\end{align}

A Bayesian See-Do Model can be constructed from a See-Do Model $\kernel{T}'$ by choosing an arbitrary prior $\gamma\in \Delta(\Theta)$ and taking the product:

\begin{align}
    \kernel{T} &= (\gamma\otimes\mathrm{Id}^D)\kernel{T}'\\ 
               &= \begin{tikzpicture} \path (0,0) node[dist] (T) {$\gamma$}
        + (0,-1) node (D) {$\RV{D}$}
        ++ (0.5,0) coordinate (copy0)
        ++ (0.5,0) node[kernel] (O) {$\kernel{O}$}
        ++ (0.7,0) coordinate (copy1)
        +  (0.4,-1) node[kernel] (C) {$\kernel{S}$}
        ++ (1.1,0) node (X) {$\RV{X}$}
        +  (0,-1) node (Y) {$\RV{Y}$};
        \draw (T) -- (O) -- (X);
        \draw (copy0) to [out=-90,in=180] ($(C.west) + (0,0)$);
        \draw (D) to [out=0,in=180] ($(C.west) + (0,-0.15)$);
        \draw (copy1) to [out=-60,in=180] ($(C.west)+ (0,0.15)$);
        \draw (C) -- (Y);
    \end{tikzpicture}\label{eq:t_with_prior}\\
               &= \begin{tikzpicture} \path (0,0) node[dist] (T) {$\gamma \kernel{O}$}
        + (0,-1.3) node (D) {$\RV{D}$}
        ++ (0.8,0) coordinate (copy0)
        + (-0.2,-0.2) coordinate (r1)
        + (0,-0.3) coordinate (copy1)
        +  (0.9,-0.3) coordinate (c2)
        + (0.8,-1) node[kernel] (Tg) {$\kernel{T}^{\Theta|\RV{X}}$}
        +  (1.8,-1.3) node[kernel] (C) {$\kernel{S}$}
        ++ (2.4,0) node (X) {$\RV{X}$}
        +  (0,-1.3) node (Y) {$\RV{Y}$}
        + (0.3,-1.6) coordinate (r2);
        \draw (T) -- (X);
        \draw (copy0) to [out=-85,in=180] (Tg) (Tg) to [out=0,in=180] ($(C.west) + (0,0)$);
        \draw (D) to [out=0,in=180] ($(C.west) + (0,-0.15)$);
        \draw (copy1) to (c2) to [out=-0,in=180] ($(C.west)+ (0,0.15)$);
        \draw[red] (r1) rectangle (r2);
        \draw (C) -- (Y);
    \end{tikzpicture}
\end{align}

The existence of $\kernel{T}^{\Theta|\RV{X}}$ follows from the fact that $\Theta$ is independent of $\RV{D}$ in \ref{eq:t_with_prior}. Define $\kernel{V}$ as the kernel in the red box. Then

\begin{align}
    \kernel{T} &= \begin{tikzpicture} \path (0,0) node[dist] (T) {$\gamma\kernel{O}$}
        + (0,-1) node (D) {$\RV{D}$}
        ++ (0.5,0) coordinate (copy0)
        ++ (0.5,0) coordinate (O)
        +  (0.3,-1) node[kernel] (C) {$\kernel{V}$}
        ++ (1,0) node (X) {$\RV{X}$}
        +  (0,-1) node (Y) {$\RV{Y}$};
        \draw (T) -- (O) -- (X);
        \draw (copy0) to [out=-90,in=180] ($(C.west) + (0,0)$);
        \draw (D) to [out=0,in=180] ($(C.west) + (0,-0.15)$);
        \draw (C) -- (Y);
    \end{tikzpicture}
\end{align}

\end{definition}

\subsubsection{Examples of see-do models}

Suppose we are betting on the outcome of the flip of a possibly biased coin with payout 1 for a correct guess and 0 for an incorrect guess, and we are given $N$ previous flips of the coin to inspect. This situation can be modeled by a hypothesis sufficient see-do model. Define $\kernel{B}:(0,1)\to \Delta(\{0,1\})$ by $\kernel{B}:\theta\mapsto \mathrm{Bernoulli}(\theta)$. Then define $\kernel{T}^1$ by:

\begin{itemize}
    \item $D=\{0,1\}$
    \item $X=\{0,1\}^N$
    \item $Y=\{0,1\}$
    \item $\Theta=(0,1)$
    \item $\kernel{O}:\splitter{0.1}^N\kernel{B}$
    \item $\kernel{C}:(\theta,d)\mapsto \mathrm{Bernoulli}(1-|d-\theta|)$
\end{itemize}

Where $\otimes^N$ indicates the tensor product copied $N$ times. The chance $\theta$ of the coin landing on heads is as much as we can hope to know about how our bet will work out.

Suppose instead that in addition to the $N$ prior flips, we manage to sneak a look at the outcome of the flip on which we will bet. In this case, the situation can be modeled by the following hypothesis insufficient see-do model $\kernel{T}^2$:

\begin{itemize}
    \item $D=\{0,1\}$
    \item $X=\{0,1\}^{N+1}$
    \item $Y=\{0,1\}$
    \item $\Theta=(0,1)$
    \item $\kernel{O}:\splitter{0.1}^{N+1}\kernel{B}$
    \item $\kernel{S}:(\theta,\mathbf{x},d)\mapsto \delta_{1-|d-x_{N+1}|}$
\end{itemize}

It is also possible to model the second situation with a hypothesis sufficient model if we include the result of the $N+1$th flip in the hypothesis. Define $\kernel{T}^3$ by the elements:

\begin{itemize}
    \item $D=\{0,1\}$
    \item $X=\{0,1\}^{N+1}$
    \item $Y=\{0,1\}$
    \item $\Theta=(0,1)\times\{0,1\}$
    \item $\kernel{O}:(\splitter{0.1}^N\kernel{B}\otimes \delta_{x_{N+1}}$
    \item $\kernel{S}:(\theta,x_{N+1},d)\mapsto \delta_{1-|d-x_{N+1}|}$
\end{itemize}

However, $\RV{X}_{N+1}$ is related to the prior flips $\vecRV{X}_{<N}$. In particular, given $\theta\in \Theta$, $\RV{X}_{N+1}$ should be distributed according to Bernoulli($\theta$). If we require that any prior $\gamma$ over $\Theta\times \{0,1\}$ have this property, then definining $\kernel{B}^N:=\splitter{0.1}^N\kernel{B}$, the model will factorise as

\begin{align}
    (\gamma\otimes\mathrm{Id}^D)\kernel{T}^3 &= 
    \begin{tikzpicture} \path (0,0) node[dist,inner sep=-1pt] (T) {$\gamma^\Theta$}
        + (0,-1) node (D) {$\RV{D}$}
        ++ (0.5,0) coordinate (copy0)
        ++ (0.7,0) node[kernel] (O) {$\kernel{B}$}
        + (0,0.5) node[kernel] (B) {$\kernel{B}^N$}
        ++ (0.7,0) coordinate (copy1)
        +  (0.4,-1) node[kernel] (C) {$\kernel{S}$}
        ++ (1.1,0) node (X) {$\RV{X}_{N+1}$}
        +  (0,-1) node (Y) {$\RV{Y}$}
        + (0,0.5) node (Xn) {$\vecRV{X}_{\leq N}$};
        \draw (T) -- (O) -- (X);
        \draw (copy0) to [out=0,in=180] (B) -- (Xn);
        \draw (copy0) to [out=-90,in=180] ($(C.west) + (0,0)$);
        \draw (D) to [out=0,in=180] ($(C.west) + (0,-0.15)$);
        \draw (copy1) to [out=-60,in=180] ($(C.west)+ (0,0.15)$);
        \draw (C) -- (Y);
    \end{tikzpicture}\\
    &= (\gamma^\Theta\otimes\mathrm{Id}^D)\kernel{T}^2
\end{align}

The only real choice that can be made about the prior is $\gamma^\Theta$, and adding the requirement that $\RV{X}_{N+1}$ is distributed as Bernoulli($\theta$) to $\kernel{T}^3$ yields $\kernel{T}^2$.

\todo[inline]{I don't have a theory of stochastic vs non-stochastic uncertainty, but it is the case in general that a hypothesis sufficient model with additional restrictions on the prior can be replaced by a hypothesis insufficient model with no restrictions on the prior. This is mainly relevant with regard to counterfactuals}

\section{Causal Questions}

\citet{pearl_book_2018} has proposed three types of causal question:
\begin{enumerate}
    \item Association: How are $\RV{X}$ and $\RV{Y}$ related? How would observing $\RV{X}$ change my beliefs about $\RV{Y}$?
    \item Intervention: What would happen if I do ... ? How can I make $E$ happen?
    \item Counterfactual: What if I had done ... instead of what I actually did?
\end{enumerate}

I will initially focus on the second question type: ``How can I make $E$ happen?'', and later show how the approach for this type of question can be generalised to handle questions of the third type. Call this kind of question a \emph{causal decision problem}:

\begin{quote}\label{def:causal_decision_problem}
    Given available options, which ones are most likely to lead to a desirable result?
\end{quote}

Causal \emph{statistical} causal decision problems extend causal decision problems by introducing data:

\begin{quote}\label{def:causal_statistical_decision_problem}
    Given my available options and data, which options are likely to lead to a desirable result?
\end{quote}


\subsubsection{Decision rules}

See-do models encode the relationship between observed data and consequences of decisions. In order to actually make decisions, we also require preferences over consequences. We suppose that a \emph{utility function} is given, and evaluate the desirability of consequences using \emph{expected utility}. A see-do model along with a utility allows us to evaluate the desirability of \emph{decisions rules} according to each hypothesis.

\begin{definition}[Utility function]
Given a See-Do Model $\kernel{T}:\Theta\times D\to \Delta(\sigalg{X}\otimes\sigalg{Y})$, a \emph{utility function} $u$ is a measurable function $Y\to \mathbb{R}$. 
\end{definition}

\begin{definition}[Expected utility]
Given a utility function $u:Y\to \mathbb{R}$ and probability measures $\mu,\nu\in \Delta(\sigalg{Y})$, the \emph{expected utility} of $\mu$ is $\mathbb{E}_{\mu}[u]$.

$\mu$ is \emph{preferred} to $\nu$ if $\mathbb{E}_{\mu}[u]\geq \mathbb{E}_{\nu}[u]$, and \emph{strictly preferred} if $\mathbb{E}_{\mu}[u]>\mathbb{E}_{\nu}[u]$.
\end{definition}

\begin{definition}[Decision rule]
Given a see-to map $\kernel{T}:\Theta\times D\to \Delta(\sigalg{X}\otimes\sigalg{Y})$, a \emph{decision rule} is a Markov kernel $X\to \Delta(\sigalg{D})$. A \emph{deterministic decision rule} is a decision rule that is deterministic.

\todo[inline]{Define deterministic Markov kernels}
\end{definition}

Expected utility together with a decision rule gives rise to the definition of \emph{risk}, which connects CSDT to classical statistical decision theory (SDT). For historical reasons, risks are minimised while utilities are maximised.

\begin{definition}[Risk]
Given a see-to map $\kernel{T}:\Theta\times D\to \Delta(\sigalg{X}\otimes\sigalg{Y})$, a utility $u:Y\to \mathbb{R}$ and the set of decision rules $\mathscr{U}$, the \emph{risk} is a function $l:\Theta\times \mathscr{U}\to \mathbb{R}$ given by

\begin{align}
    R(\theta,\kernel{U}) := - \int_X  \kernel{U}_x \kernel{T}^{\RV{Y}|\RV{D}\RV{X}\Theta}_{\cdot,x,\theta} u d\kernel{T}^{\RV{X}|\Theta}_\theta(x)
\end{align}

for $\theta\in \Theta$, $\kernel{U}\in \mathscr{U}$. Here $\kernel{U}_x \kernel{T}^{\RV{Y}|\RV{D}\RV{X}\Theta}_{\cdot,x,\theta} u$ is the product of the measure $\kernel{U}_x$, the kernel $\kernel{T}^{\RV{Y}|\RV{D}\RV{X}\Theta}_{\cdot,x,\theta}:D\to \Delta(\sigalg{Y})$ and the function $u$.
\end{definition}

The loss induces a partial order on decision rules. If for all $\theta$, $l(\theta,\kernel{U})\leq l(\theta,\kernel{U}')$ then $\kernel{U}$ is at least as good as $\kernel{U}'$. If, furthermore, there is some $\theta_0$ such that $l(\theta_0,\kernel{U})<l(\theta_0,\kernel{U}')$ then $\kernel{U}$ is preferred to $\kernel{U}'$.

\begin{definition}[Induced statistical decision problem]
A see-do model $\kernel{T}:\Theta\times D\to \Delta(\sigalg{X}\otimes\sigalg{Y})$ along with a utility $u$ induces the \emph{statistical decision problem} $(\Theta,\mathscr{U},R)$ with states $\Theta$, decisions $\mathscr{U}$ and risks $R$.

\todo[inline]{Statistical decision problems usually define the risk via the loss, but it is only possible to define a loss with a hypothesis sufficient model. We don't actually need a loss, though: the complete class theorem still holds via the induced risk and Bayes risk}

\end{definition}


\todo[inline]{An alternative method of converting hypothesis insufficient to hypothesis sufficient models involves expanding the decision set; this is not appliccable to counterfactual models.}


A key difference between CSDT and other approaches to causal inference is that diagrams in CSDT feature two coupled maps $\kernel{O}$ and $\kernel{C}$, while most other approaches to causal inference represent both $\kernel{O}$ and $\kernel{C}$ in one diagram. \citet{lattimore_replacing_2019} is the only other example I am aware of that represents both $\kernel{O}$ and $\kernel{C}$. Nevertheless, ``one-picture'' causal models such as Causal Bayesian Networks, Single World Intervention Graphs \emph{do} represent observational distributions and interventional maps, and the two differ (see Section \ref{sec:single_double_representation})

A causal hypothesis class $\Theta$ induces a binary relation between observed probability distributions $\prob{O}_\theta$ and consequence maps $\prob{C}_\theta$. This approach is very agnostic about the actual relation induced -- we do not even insist that the range of the observed data $X$ is the same as the range of possible consequences $Y$ (though we will generally limit our attention to cases where the two coincide). 

In common with \citet{heckerman_decision-theoretic_1995}, decisions (or ``acts'') are primitive elements of See-Do Models. In contrast to our work, \citet{heckerman_decision-theoretic_1995} only discuss deterministic \emph{consequence maps}, while See-Do Models represent relations between consequence maps and observed probability.

Decisions are similar to the ``regime indicators'' found in \citet{dawid_decision-theoretic_2020}. They coincide precisely if we suppose that the observation and consequence spaces coincide ($X=Y$) and there exists an ``idle'' decision $d^*\in D$ such that $\kernel{C}_{(\cdot,d^*)} = \kernel{O}_{\cdot}$. However, in general we don't require that $\kernel{O}$ and $\kernel{C}$ are related in this manner. This assumption will be revisited in \todo[inline]{A section I haven't written yet}.

\subsection{D-causation}

While we take $D$ to be a primitive element of causal decision problems, and therefore a primitive of See-Do Models. Causes are not primitive, but we can offer a secondary notion of causation. We call this $D$-causation to stress the fact that it arises in a theory of causal inference in which the set $D$ of available decisions is primitive. A similar idea is discussed extensively in \citet{heckerman_decision-theoretic_1995}. The main differences are that what we call ``consequence maps'' map decisions to probability distributions over possible consequences while Heckerman and Shachter work with ``states'' that map decisions deterministically to consequences. In addition, while we define $D$-causation relative to a particular consequence map $\kernel{C}_\theta$, Heckerman and Shachter define it with respect to a \emph{set} of states.

Section \ref{sec:cbns_without_d} explores the difficulty of defining ``objective causation'' without reference to a set of basic decisions, acts or operations. $D$ need not be interpreted as the set of decisions an agent may make, but whatever interpretation it is assigned, all existing examples of causal models seem to require a ``domain set''.

See Section \ref{ssec:random_variables} for the definition of random variables.

\todo[inline]{Add definition of conditional independence, revise wire label definitions}

One way to motivate the notion of $D$-causation is to observe that for many decision problems, the full set $D$ may be extremely large. Suppose I aim to have my light switched on, and there is a switch that controls the light. Often, the relevant choice of acts for such a problem would appear to be $D_0=\{\text{flip the switch},\text{don't flip the switch}\}$. However, in principle I have a much larger range of options to choose from. For simplicity's sake, suppose I have instead the following set of options:

\begin{align*}
D_1:=&\{``\text{walk to the switch and press it with my thumb}'', \\
    &``\text{trip over the lego on the floor, hop to the light switch and stab my finger at it}'',\\
    &``\text{stay in bed}''\}
\end{align*}

If having the light turned on is all that matters, I could consider any acts in $D_1$ to be equivalent if they have the same ultimate impact on the position of the light switch. $D_0$ is a quotient over $D_1$ under this equivalence relation. 

If I hypothesize that, relative to $D_1$, the ultimate state of the light switch is all that matters to determine the ultimate state of the light, I can say that the light switch $D_1$-causes the state of the light. Given this $D_1$-causation, the $D_1$ decision problem can (subject to my hypothesis) be reduced to a $D_0$ decision between states of the light switch.

If I consider an even larger set of possible acts $D_2$, I might not accept the hypothesis of $D_2$-causation. Let $D_2$ be the following acts:

\begin{align*}
D_2:=&\{``\text{walk to the switch and press it with my thumb}'', \\
    &``\text{trip over the lego on the floor, hop to the light switch and stab my finger at it}'',\\
    &``\text{stay in bed}'',
    &``\text{toggle the mains power, then flip the light switch}''\}
\end{align*}

In this case, it would be unreasonable to hypothesize that all acts that left the light switch in the ``on'' position would also result in the light being ``on''. Thus the switch does not $D_2$-cause the light to be on.

Formally, $D$-causation is defined in terms of conditional independence:

\begin{definition}[$D$-causation]\label{def:d_cause}
Given a consequence map $\kernel{C}_\theta:D\to \Delta(\mathcal{Y})$, random variables $\RV{Y}_1:Y\times D\to Y_1$, $\RV{Y}_2:Y\times D\to Y_2$ and domain variable $\RV{D}:Y\times D\to D$ (Definition \ref{def:domain_variable}), $\RV{Y}_1$ $D$-causes $\RV{Y}_2$ iff $\RV{Y}_2\CI_{\kernel{C}_\theta} \RV{D}|\RV{Y}_1$.
\end{definition}

\subsection{D-causation vs Heckerman and Shachter}

Heckerman and Shachter study deterministic ``consequence maps''. Furthermore, what we call hypotheses $\theta\in\Theta$, Heckerman and Schachter call states $s\in S$. One could consider a state to be a hypothesis that is specific enough to yield a deterministic map from decisions to outcomes. Heckerman and Shachter's notion of causation is defined by \emph{limited unresponsiveness} rather than \emph{conditional independence}, which depends on a partition of states rather than a particular hypothesis.

\begin{definition}[Limited unresponsiveness]
    Given states $S$, deterministic consequence maps $\kernel{C}_s:D\to \Delta(F)$ for each $s\in A$ and a random variables $\RV{X}:F\to X$, $\RV{Y}:F\to Y$, $\RV{Y}$ is unresponsive to $\RV{D}$ in states limited by $\RV{X}$ if $\kernel{C}_{(s,d)}^{\RV{X}|\RV{D}}=\kernel{C}_{(s,d')}^{\RV{X}|\RV{D}\RV{S}}\implies \kernel{C}_{(s,d)}^{\RV{Y}|\RV{D}\RV{S}}=\kernel{C}_{(s,d')}^{\RV{Y}|\RV{D}\RV{S}}$ for all $d,d'\in D$, $s\in S$. Write $\RV{Y}\not\hookleftarrow_{\RV{X}} \RV{D}$
\end{definition}

\begin{lemma}[Limited unresponsiveness implies $D$-causation]
For deterministic consequence maps, $\RV{Y}\not\hookleftarrow_{\RV{X}} \RV{D} $ implies $\RV{X}$ $D$-causes $\RV{Y}$ in every state $s\in S$.
\end{lemma}

\begin{proof}
By the assumption of determinism, for each $s\in S$ and $d\in D$ there exists $x(s,d)$ and $y(s,d)$ such that $\kernel{C}^{\RV{X}\RV{Y}|\RV{D}\RV{S}}_{d,s} = \delta_{x(s,d)}\otimes\delta_{y(s,d)}$.

By the assumption of limited unresponsiveness, for all $d,d'$ such that $x(s,d)=x(s,d')$, $y(s,d)=y(s,d')$ also. Define $f:X\times S\to Y$ by $(s,x)\mapsto y(s,[x(s,\cdot)]^{-1}(x(s,d)))$ where $[x(s,\cdot)]^{-1}(a)$ is an arbitrary element of $\{d|x(s,d)=a\}$. For all $s,d$, $f(x(s,d),s)=y(s,d)$. Define $\kernel{M}:X\times D\times S\to \Delta(\mathcal{Y})$ by $(x,d,s)\mapsto \delta_{f(x,s)}$. $\kernel{M}$ is a version of $\kernel{C}^{\RV{Y}|\RV{X},\RV{D},\RV{S}}$ because, for all $A\in \mathcal{X}$, $B\in \mathcal{Y}$, $s\in S$, $d\in D$:

\begin{align}
    \kernel{C}^{\RV{X}|\RV{D}\RV{S}}_{(d,s)}\splitter{0.1}(\kernel{M}\otimes\mathrm{Id}) &= \int_A \kernel{M}(x',d,s;B) d\delta_{x(s,d)}(x') \\
                                                                                        &= \int_A \delta_{f(x',s)}(B) d\delta_{x(s,d)}(x') \\
                                                                                        &= \delta_{f(x(s,d),s)}(B)\delta_{x(s,d)}(A) \\
                                                                                        &= \delta_{y(s,d)}(B)\delta_{x(s,d)}(A)\\
                                                                                        &= \delta_{x(s,d)}\otimes\delta_{y(s,d)}(A\times B)
\end{align}

$\kernel{M}$ is also independent of $\RV{D}$, given the obvious labeling of inputs. Therefore $\RV{Y}\CI_{\kernel{C}_s}\RV{D}|\RV{X}$.
\end{proof}

However, despite limited unresponsiveness implying $D$-causation within every state, it does not imply $D$-causation in mixtures of states. Suppose $D=\{0,1\}$ where $1$ stands for ``toggle light switch'' and $0$ stands for ``do nothing''. Suppose $S=\{[0,0],[0,1],[1,0],[1,1]\}$ where $[0,0]$ represents ``switch initially off, mains off'' the other states generalise this in the obvious way. Finally, $\RV{F}\in\{0,1\}$ is the final position of the switch and $\RV{L}\in\{0,1\}$ is the final state of the light. We have

\begin{align}
    \kernel{C}^{\RV{L}\RV{F}|\RV{D}\RV{S}}_{d,[i,m]} = \delta_{(d\text{ XOR }i)\text{ AND }m}\otimes \delta_{(d\text{ XOR }i)\text{ AND }m}
\end{align}

Within states $[0,0]$ and $[1,0]$, the light is always off, so $\RV{F}=a\implies \RV{L}=0$ for any $a$. In states $[0,1]$ and $[1,1]$, $\RV{F}=1\implies \RV{L}=1$ and $\RV{F}=0\implies \RV{L}=0$. Thus $\RV{L}\not\hookleftarrow_{\RV{F}} \RV{D}$. However, suppose we take a mixture of consequence maps:
\begin{align}
    \kernel{C}_\gamma &= \frac{1}{4}\kernel{C}_{\cdot,[0,0]} + \frac{1}{4}\kernel{C}_{\cdot,[0,1]} + \frac{1}{2}\kernel{C}_{\cdot,[1,1]}\\
    \kernel{C}^{\RV{F}\RV{L}|\RV{D}}_\gamma &= \frac{1}{4} \left[\begin{matrix}
                        1 & 0\\ 0 & 1
                      \end{matrix}\right]\otimes \left[\begin{matrix}
                        1 & 0\\ 1 & 0
                      \end{matrix}\right] + \frac{1}{4} \left[\begin{matrix}
                        1 & 0\\ 0 & 1
                      \end{matrix}\right]\otimes \left[\begin{matrix}
                        1 & 0\\ 0 & 1
                      \end{matrix}\right] + \frac{1}{2}\left[\begin{matrix}
                        0 & 1\\ 1 & 0
                      \end{matrix}\right]\otimes \left[\begin{matrix}
                        0 & 1\\ 1 & 0
                      \end{matrix}\right]
\end{align}

Then

\begin{align}
    [1,0]\kernel{C}^{\RV{F}\RV{L}|\RV{D}}_{\gamma} &= \frac{1}{4}[0,1]\otimes[1,0]+\frac{1}{4}[0,1]\otimes[0,1]+\frac{1}{2}[1,0]\otimes[1,0]\\
    [1,0]\splitter{0.1}(\kernel{C}^{\RV{F}|\RV{D}}_\gamma\otimes \kernel{C}^{\RV{L}|\RV{D}}_\gamma) &= (\frac{1}{2}[0,1]+\frac{1}{2}[1,0])\otimes(\frac{1}{4}[0,1]+\frac{3}{4}[1,0])\\
    \implies [1,0]\kernel{C}^{\RV{F}\RV{L}|\RV{D}}_{\gamma} &\neq [1,0] \splitter{0.1} (\kernel{C}^{\RV{F}|\RV{D}}_\gamma\otimes \kernel{C}^{\RV{L}|\RV{D}}_\gamma)
\end{align}

Thus under hypothesis mixture $\gamma$, $\RV{F}$ does not $D$-cause $\RV{L}$ even though $\RV{F}$ $D$-causes $\RV{L}$ in all states $S$. The definition of $D$-causation was motivated by the idea that we could reduce a difficult decision problem with a large set $D$ to a simpler problem with a smaller ``effective'' set of decisions by exploiting conditional independence. Even if $\RV{X}$ $D$-causes $\RV{Y}$ in every $\theta\in S$, $\RV{X}$ does not necessarily $D$-cause $\RV{Y}$ in mixtures of states in $S$. For this reason, we do not say that $\RV{X}$ $D$-causes $\RV{Y}$ in $S$ if $\RV{X}$ $D$-causes $\RV{Y}$ in every $\theta\in S$, and in this way we differ substantially from \citet{heckerman_decision-theoretic_1995}.

Instead, we simply extend the definition of $D$-causation to mixtures of hypotheses: if $\gamma\in \Delta(\Theta)$ is a mixture of hypotheses, define $\kernel{C}_\gamma:= (\gamma\otimes\textbf{Id})\kernel{C}$. Then $\RV{X}$ $D$-causes $\RV{Y}$ relative to $\gamma$ iff $\RV{Y}\CI_{\kernel{C}_\gamma} \RV{D}|\RV{X}$.

Theorem \ref{th:univ_d_causation} shows that under some conditions, $D$-causation can hold for arbitrary mixtures over subsets of the hypothesis class $\Theta$.

\begin{theorem}[Universal $D$-causation]\label{th:univ_d_causation}
If $\kernel{C}^{\RV{X}|\RV{D}}_{\theta} = \kernel{C}^{\RV{X}|\RV{D}}_{\theta'}$ for all $\theta,\theta'\in S\subset \Theta$ and $\RV{X}$ $D$-causes $\RV{Y}$ in all $\theta\in S$, then $\RV{X}$ $D$-causes $\RV{Y}$ with respect to all mixed consequence maps $\kernel{C}_\gamma$ for all $\gamma\in \Delta(\Theta)$ with $\gamma(S)=1$.
\end{theorem}

\begin{proof}

For $\gamma\in \Delta(\Theta)$, define the mixture

\begin{align}
\kernel{C}_\gamma := \begin{tikzpicture}
    \path (0,0) node[dist] (g) {$\gamma$}
    + (0,-0.45) node (D) {$\RV{D}$}
    ++ (1,-0.3) node[kernel] (C) {$\kernel{C}$}
    ++ (1,0) node (F) {$\RV{F}$};
    \draw (g) to [out=0,in=180] ($(C.west) + (0,0.15)$) (D) -- ($(C.west) + (0,-0.15)$) (C) -- (F);
\end{tikzpicture}
\end{align}

Because $\kernel{C}_\theta^{\RV{X}|\RV{D}} = \kernel{C}_{\theta'}^{\RV{X}|\RV{D}}$ for all $\theta,\theta'\in \Theta$, we have

\begin{align}
\begin{tikzpicture}
    \path (0,0) node[dist] (g) {$\gamma$}
    + (0.7,-0.15) coordinate (copy0)
    + (0,-0.45) node (D) {$\RV{D}$}
    ++ (1.5,-0.3) node[kernel] (C) {$\kernel{C}^{\RV{X}|\RV{D}\Theta}$}
    ++ (1,0) node (X) {$\RV{X}$}
    + (0,0.5) node (T) {$\Theta$};
    \draw (g) to [out=0,in=180] (copy0) -- ($(C.west) + (0,0.15)$) (D) -- ($(C.west) + (0,-0.15)$);
    \draw (C) -- (X);
    \draw (copy0) to [out=90,in=180] (T);
\end{tikzpicture} &= \begin{tikzpicture}
    \path (0,0) node[dist] (g) {$\gamma$}
    + (0,0.5) node[dist] (g2) {$\gamma$}
    + (0.7,-0.15) coordinate (copy0)
    + (0,-0.45) node (D) {$\RV{D}$}
    ++ (1.5,-0.3) node[kernel] (C) {$\kernel{C}^{\RV{X}|\RV{D}\Theta}$}
    ++ (1,0) node (X) {$\RV{X}$}
    + (0,0.3) node (T) {$\Theta$};
    \draw (g) to [out=0,in=180] (copy0) -- ($(C.west) + (0,0.15)$) (D) -- ($(C.west) + (0,-0.15)$);
    \draw (C) -- (X);
    \draw (g2) to [out=0,in=180] (T);
\end{tikzpicture} \label{eq:decompose_condi_x}
\end{align}

Also

\begin{align}
    \kernel{C}_\gamma^{\RV{XY}|\RV{D}} &= \begin{tikzpicture}
    \path (0,0) node[dist] (g) {$\gamma$}
    + (0,-0.45) node (D) {$\RV{D}$}
    ++ (1,-0.3) node[kernel] (C) {$\kernel{C}$}
    ++ (1,0) node[kernel] (F) {$\kernel{F}^{\RV{X}\utimes\RV{Y}}$}
    ++ (1,0.15) node (X) {$\RV{X}$}
    + (0,-0.3) node (Y) {$\RV{Y}$};
    \draw (g) to [out=0,in=180] ($(C.west) + (0,0.15)$) (D) -- ($(C.west) + (0,-0.15)$) (C) -- (F);
    \draw ($(F.east) + (0,0.15)$) -- (X) ($(F.east) + (0,-0.15)$) -- (Y);
\end{tikzpicture}\\
    &= \begin{tikzpicture}
    \path (0,0) node[dist] (g) {$\gamma$}
    + (0,-0.45) node (D) {$\RV{D}$}
    ++ (1,-0.3) node[kernel] (C) {$\kernel{C}^{\RV{XY}|\RV{D}\Theta}$}
    ++ (1,0.15) node (X) {$\RV{X}$}
    + (0,-0.3) node (Y) {$\RV{Y}$};
    \draw (g) to [out=0,in=180] ($(C.west) + (0,0.15)$) (D) -- ($(C.west) + (0,-0.15)$);
    \draw ($(C.east) + (0,0.15)$) -- (X) ($(C.east) + (0,-0.15)$) -- (Y);
\end{tikzpicture}\\
 &= \begin{tikzpicture}
    \path (0,0) node[dist] (g) {$\gamma$}
    + (0,-0.45) node (D) {$\RV{D}$}
    + (0.7,-0.45) coordinate (copy0)
    + (0.7,-0.15) coordinate (copy1)
    ++ (1.4,-0.3) node[kernel] (C) {$\kernel{C}^{\RV{X}|\RV{D}\Theta}$}
    + (0,0.6) coordinate (via0)
    + (0,-0.6) coordinate (via1)
    ++ (0.9,0) coordinate (copy2)
    ++ (0.7,0) node[kernel] (Yx) {$\kernel{C}^{\RV{Y}|\RV{X}\RV{D}\Theta}$}
    ++ (1.2,0.15) node (X) {$\RV{Y}$}
    + (0,-0.5) node (Y) {$\RV{X}$};
    \draw (g) to [out=0,in=180] (copy1) -- ($(C.west) + (0,0.15)$) (D) -- ($(C.west) + (0,-0.15)$) (C)--(Yx);
    \draw (copy0) to [out=-90,in=180] (via1) to [out=0,in=180] ($(Yx.west) + (0,-0.15)$) (copy1) to [out=90,in=180] (via0) to [out=0,in=180] ($(Yx.west) + (0,0.15)$);
    \draw ($(Yx.east) + (0,0.15)$) -- (X) (copy2) to [out=-90,in=180] (Y);
 \end{tikzpicture}\\
 &\overset{\RV{Y}\CI \RV{D}|\RV{X}\Theta}{=} \begin{tikzpicture}
    \path (0,0) node[dist] (g) {$\gamma$}
    + (0,-0.45) node (D) {$\RV{D}$}
    + (0.7,-0.15) coordinate (copy1)
    ++ (1.4,-0.3) node[kernel] (C) {$\kernel{C}^{\RV{X}|\RV{D}\Theta}$}
    ++ (0.9,0.1) coordinate (copy2)
    ++ (0.7,0.3) node[kernel] (Yx) {$\kernel{C}^{\RV{Y}|\RV{X}\Theta}$}
    ++ (1.2,0.15) node (X) {$\RV{Y}$}
    + (0,-0.5) node (Y) {$\RV{X}$};
    \draw (g) to [out=0,in=180] (copy1) -- ($(C.west) + (0,0.15)$) (D) -- ($(C.west) + (0,-0.15)$) (C) to [out=0,in=180] (copy2) to [out=0,in=180] (Yx);
    \draw (copy1) to [out=90,in=180] ($(Yx.west) + (0,0.15)$);
    \draw ($(Yx.east) + (0,0.15)$) -- (X) (copy2) to [out=-90,in=180] (Y);
 \end{tikzpicture} \\
 &\overset{\ref{eq:decompose_condi_x}}{=} \begin{tikzpicture}
    \path (0,0) node[dist] (g) {$\gamma$}
    + (0,-0.45) node (D) {$\RV{D}$}
    + (0.7,-0.15) coordinate (copy1)
    ++ (1.4,-0.3) node[kernel] (C) {$\kernel{C}^{\RV{X}|\RV{D}\Theta}$}
    + (1,0.6) node[dist] (g2) {$\gamma$}
    ++ (0.9,0.1) coordinate (copy2)
    ++ (1,0.3) node[kernel] (Yx) {$\kernel{C}^{\RV{Y}|\RV{X}\Theta}$}
    ++ (1.2,0.15) node (X) {$\RV{Y}$}
    + (0,-0.5) node (Y) {$\RV{X}$};
    \draw (g) to [out=0,in=180] (copy1) -- ($(C.west) + (0,0.15)$) (D) -- ($(C.west) + (0,-0.15)$) (C) to [out=0,in=180] (copy2) to [out=0,in=180] (Yx);
    \draw (g2) to [out=0,in=180] ($(Yx.west) + (0,0.15)$);
    \draw ($(Yx.east) + (0,0.15)$) -- (X) (copy2) to [out=-90,in=180] (Y);
 \end{tikzpicture}\\
 &= \overset{\ref{eq:decompose_condi_x}}{=} \begin{tikzpicture}
    \path (0,0) node (g) {}
    + (0,-0.45) node (D) {$\RV{D}$}
    + (0.7,-0.45) coordinate (copy1)
    ++ (1.4,-0.3) node[kernel] (C) {$\kernel{C}_\gamma^{\RV{X}|\RV{D}\Theta}$}
    + (1,0.6) node[dist] (g2) {$\gamma$}
    ++ (0.9,0.1) coordinate (copy2)
    ++ (1,0.3) node[kernel] (Yx) {$\kernel{C}^{\RV{Y}|\RV{X}\Theta}$}
    + (-0.5,0.6) coordinate (stop0)
    ++ (1.2,0.15) node (X) {$\RV{Y}$}
    + (0,-0.5) node (Y) {$\RV{X}$};
    \draw (D) -- ($(C.west) + (0,-0.15)$) (C) to [out=0,in=180] (copy2) to [out=0,in=180] (Yx);
    \draw (g2) to [out=0,in=180] ($(Yx.west) + (0,0.15)$);
    \draw ($(Yx.east) + (0,0.15)$) -- (X) (copy2) to [out=-90,in=180] (Y);
    \draw[-{Rays[n=8]}] (copy1) to [out=90,in=180] (stop0);
 \end{tikzpicture}\label{eq:is_conditional}
\end{align}
Equation \ref{eq:is_conditional} establishes that $(\gamma\otimes\textbf{Id}_X\otimes\stopper{0.3}_D)\kernel{C}^{\RV{Y}|\RV{X}\Theta}$ is a version of $\kernel{C}_\gamma^{\RV{Y}|\RV{X}\RV{D}}$, and thus $\RV{Y}\CI_{\kernel{C}_\gamma} \RV{D}|\RV{X}$.

This can also be derived from the semi-graphoid rules:

\begin{align}
    \Theta\CI \RV{D} \land \Theta\CI \RV{X} | \RV{D} &\implies \Theta\CI \RV{XD}\\
    &\implies \Theta\CI \RV{D}|\RV{X}\\
    \RV{D} \CI \Theta|\RV{X} \land \RV{D}\CI \RV{Y}|\RV{X}\Theta &\implies \RV{D}\CI \RV{Y}|\RV{X}\\
    &\implies \RV{Y}\CI\RV{D}|\RV{X}
\end{align}
\end{proof}

\subsection{Properties of D-causation}

If $\RV{X}$ D-causes $\RV{Y}$ relative to $\kernel{C}_\theta$, then the following holds:

\begin{align}
    \kernel{C}_{\theta}^{\RV{X}|\RV{D}} &= \begin{tikzpicture}
    \path (0,0) node (D) {$\RV{D}$}
    ++ (0.9,0) node[kernel] (Xd) {$\kernel{C}^{\RV{X}|\RV{D}}$}
    ++ (1.3,0) node[kernel] (Yd) {$\kernel{C}^{\RV{Y}|\RV{X}}$}
    ++ (0.9,0) node (Y) {$\RV{Y}$};
    \draw (D) -- (Xd) -- (Yd) -- (Y); 
    \end{tikzpicture}
\end{align}

This follows from version (2) of Definition \ref{def:conditional_independence}:

\begin{align}
    \kernel{C}_\theta^{\RV{X}|\RV{D}} &= \begin{tikzpicture}
    \path (0,0) node (D) {$\RV{D}$}
    ++ (0.7,0) coordinate (copy0)
    ++ (0.7,0) node[kernel] (Xd) {$\kernel{C}^{\RV{X}|\RV{D}}$}
    + (0,0.5) coordinate (via1)
    ++ (1.3,0) node[kernel] (Yd) {$\kernel{C}^{\RV{Y}|\RV{X}\RV{D}}$}
    ++ (0.9,0) node (Y) {$\RV{Y}$};
    \draw (D) -- (Xd) -- (Yd) -- (Y);
    \draw (copy0) to [out=90,in=180] (via1) to [out=0,in=180] ($(Yd.west)+(0,0.15)$); 
    \end{tikzpicture}\\
     &= \begin{tikzpicture}
    \path (0,0) node (D) {$\RV{D}$}
    ++ (0.7,0) coordinate (copy0)
    ++ (0.7,0) node[kernel] (Xd) {$\kernel{C}^{\RV{X}|\RV{D}}$}
    + (1.3,0.5) coordinate (via1)
    ++ (1.3,0) node[kernel] (Yd) {$\kernel{C}^{\RV{Y}|\RV{X}}$}
    ++ (0.9,0) node (Y) {$\RV{Y}$};
    \draw (D) -- (Xd) -- (Yd) -- (Y);
    \draw[-{Rays[n=8]}] (copy0) to [out=90,in=180] (via1); 
    \end{tikzpicture}\\
    &= \begin{tikzpicture}
    \path (0,0) node (D) {$\RV{D}$}
    ++ (0.9,0) node[kernel] (Xd) {$\kernel{C}^{\RV{X}|\RV{D}}$}
    ++ (1.3,0) node[kernel] (Yd) {$\kernel{C}^{\RV{Y}|\RV{X}}$}
    ++ (0.9,0) node (Y) {$\RV{Y}$};
    \draw (D) -- (Xd) -- (Yd) -- (Y); 
    \end{tikzpicture}
\end{align}

D-causation is not transitive: if $\RV{X}$ D-causes $\RV{Y}$ and $\RV{Y}$ D-causes $\RV{Z}$ then $\RV{X}$ doesn't necessarily D-cause $\RV{Z}$.

%!TEX root = main.tex

\chapter{Chapter 4: See-do models compared to causal graphical models and potential outcomes}\label{ch:4}
\input{chapter_5_inferenceprinciples}

\bibliographystyle{plainnat}
\bibliography{references}

\appendix
\newpage
\section*{Appendix:}

% \input{appendix_AIstats}

\end{document}
