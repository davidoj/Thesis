\documentclass{book}

% If your paper is accepted, change the options for the package
% aistats2020 as follows:
%
% \usepackage[accepted]{aistats2020}
%
% This option will print headings for the title of your paper and
% headings for the authors names, plus a copyright note at the end of
% the first column of the first page.

% If you set papersize explicitly, activate the following three lines:
%\special{papersize = 8.5in, 11in}
%\setlength{\pdfpageheight}{11in}
%\setlength{\pdfpagewidth}{8.5in}

% If you use natbib package, activate the following three lines:
\usepackage[round]{natbib}
\renewcommand{\bibname}{References}
\renewcommand{\bibsection}{\subsubsection*{\bibname}}

% If you use BibTeX in apalike style, activate the following line:
%\bibliographystyle{apalike}

\usepackage[T1]{fontenc}    % use 8-bit T1 fonts
\usepackage{hyperref}       % hyperlinks
\usepackage{url}            % simple URL typesetting
\usepackage{booktabs}       % professional-quality tables
\usepackage{amsfonts}       % blackboard math symbols
\usepackage{nicefrac}       % compact symbols for 1/2, etc.
\usepackage{microtype}      % microtypography

% My packages
\usepackage{tikzit}
\input{diagrams.tikzstyles}
\usepackage[mathscr]{euscript}
\usepackage{graphicx}
\usepackage {tikz}
\usetikzlibrary {positioning}
\usetikzlibrary{shapes.misc}
\usetikzlibrary{shapes.geometric}
\usetikzlibrary{calc}
\usetikzlibrary{arrows.meta}
\usetikzlibrary{intersections}
\usepackage{amsthm}
\usepackage{amsmath}
\usepackage{amssymb}
\usepackage{dsfont}
\usepackage{stmaryrd }
\usepackage{csquotes}
\usepackage{wasysym}
\usepackage[]{todonotes}
\usepackage[shortlabels]{enumitem}
\usepackage{bm}
\usepackage{isomath}
\usepackage{mathtools}
\usepackage{algpseudocode}
\usepackage{algorithm}

\makeatletter
\newcommand{\newreptheorem}[2]
  {\newtheorem*{rep@#1}{\rep@title}\newenvironment{rep#1}[1]
  {\def\rep@title{#2 \ref*{##1}}\begin{rep@#1}}{\end{rep@#1}}}
\makeatother

\theoremstyle{plain}
\newtheorem{theorem}{Theorem}[section]
\newtheorem{corollary}[theorem]{Corollary}
\newtheorem{lemma}[theorem]{Lemma}
\newtheorem{proposition}[theorem]{Proposition}
\newreptheorem{theorem}{Theorem}
\newreptheorem{lemma}{Lemma}

\newtheorem{innercustomthm}{Theorem}
\newenvironment{customthm}[1]
  {\renewcommand\theinnercustomthm{#1}\innercustomthm}
  {\endinnercustomthm}

\theoremstyle{definition}
\newtheorem{definition}[theorem]{Definition}
\newtheorem{example}[theorem]{Example}
\newtheorem{notation}[theorem]{Nxample}


\DeclareMathAlphabet{\mathsfit}{T1}{\sfdefault}{\mddefault}{\sldefault}

\newcommand{\CI}{\mathrel{\text{\scalebox{1.07}{$\perp\mkern-10mu\perp$}}}}
\newcommand{\CII}{\mathrel{\text{\scalebox{1.07}{$\perp\mkern-10mu\perp\mkern-10mu\perp$}}}}
\newcommand{\RV}[1]{\ensuremath{\mathsf{#1}}}
\newcommand{\URV}[1]{\ensuremath{\underline{\RV{#1}}}}
\newcommand{\PA}[2]{\ensuremath{\text{Pa}_{#1}(#2)}}
\newcommand{\ND}[2]{\ensuremath{\text{ND}_{#1}(#2)}}
\newcommand{\CH}[2]{\ensuremath{\text{Ch}_{#1}(#2)}}
\newcommand{\DE}[2]{\ensuremath{\text{De}_{#1}(#2)}}
\newcommand{\ID}[1]{\ensuremath{\text{Id}_{#1}}}
\newcommand{\utimes}{\ensuremath{\underline{\otimes}}}
\newcommand{\prob}[1]{\ensuremath{\mathbb{#1}}}
\newcommand{\disint}[1]{\ensuremath{\overline{\prob{#1}}}}
\newcommand{\kernel}[1]{\ensuremath{\mathbb{#1}}}
\newcommand{\model}[1]{\ensuremath{\mathbb{#1}}}
\newcommand{\diagram}[1]{\ensuremath{\mathscr{#1}}}
\newcommand{\sigalg}[1]{\ensuremath{\mathcal{#1}}}
\newcommand{\vecRV}[1]{\ensuremath{\mathsfbfit{#1}}}
\newcommand{\vecVal}[1]{\ensuremath{\mathbf{#1}}}
\newcommand{\prodSet}[1]{\ensuremath{\mathbf{#1}}}
\newcommand{\indx}[1]{\ensuremath{\mathcal{#1}}}
\newcommand{\nod}[1]{\ensuremath{\mathsfit{#1}}}
\newcommand{\kto}{\ensuremath{\rightarrowtriangle}}
\newcommand{\proc}[1]{\ensuremath{\mathscr{#1}}}
\newcommand{\yields}{\ensuremath{\bowtie}}
\newcommand{\submodel}{\ensuremath{\sqsubset}}
\newcommand{\seedo}[5]{\ensuremath{\model{#1}^{\RV{#3}|\RV{#2}\square\RV{#5}|\RV{#4}}}}
\newcommand{\rseedo}[6]{\ensuremath{\model{#1}^{\RV{#3}|\RV{#2}\framebox{#6}\RV{#5}|\RV{#4}}}}
\newcommand{\set}{\ensuremath{\bowtie}}
\newcommand{\cprod}{\ensuremath{\odot}}
\newcommand{\bigcprod}{\ensuremath{\bigodot}}
\newcommand{\combprod}{\ensuremath{\underline{\cprod}}}
\newcommand{\combbreak}{\ensuremath{\wr}}
\newcommand{\bigcombprod}{\ensuremath{\underline{\bigcprod}}}
\newcommand{\varlessthan}{\ensuremath{\preccurlyeq}}
\algnewcommand\algorithmicassert{\texttt{assert}}
\algnewcommand\Assert[1]{\State \algorithmicassert(#1)}%


\providecommand\longrightarrowRHD{\relbar\joinrel\relbar\joinrel\mathrel\RHD}
\providecommand\longleftarrowRHD{\mathrel\LHD\joinrel\relbar\joinrel\relbar}

\makeatletter
\newcommand*\bigcdot{\mathpalette\bigcdot@{.5}}
\newcommand*\bigcdot@[2]{\mathbin{\vcenter{\hbox{\scalebox{#2}{$\m@th#1\bullet$}}}}}
\makeatother

\tikzset{
    triangle/.style = {regular polygon, regular polygon sides=3 },
    node rotated/.style = {rotate=90},
    border rotated/.style = {shape border rotate=90},
    dist/.style = {triangle,draw,border rotated, inner sep=0pt},
    smalldist/.style = {triangle,draw,border rotated},
    kernel/.style={rectangle,draw,inner sep = 2pt},
    expectation/.style = {triangle,draw,inner sep=0pt,shape border rotate=270},
    copymap/.style = {circle,fill,inner sep=1pt}}

\newcommand\DCI{
    \begin{tikzpicture}[scale=0.35]
    \draw[->] (1,0) -- (0,0);
    \draw (0.6,0) -- (0.6,0.75);
    \draw (0.4,0) -- (0.4,0.75);
    \end{tikzpicture}
}

\newcommand\splitter[1]{%
\begin{tikzpicture}[scale=#1]
\draw (0,-1) -- (0,0);
\draw (0,0) to [bend right] (1,1);
\draw (0,0) to [bend left] (-1,1);
\end{tikzpicture}
}

\newcommand\stopper[1]{%
\begin{tikzpicture}[scale=#1]
\draw[-{Rays [n=8]}] (0,-1) -- (0,0);
\end{tikzpicture}
}

\newcommand\swap[1]{%
\begin{tikzpicture}[scale=#1]
\draw (0,0) to [out=90, in=270] (0.5,1);
\draw (0.5,0) to [out=90,in=270] (0,1);
\end{tikzpicture}
}

\newcommand\source[1]{%
\begin{tikzpicture}[scale=#1]
\path (0,0) node[prob,fill=gray] (P) {};
\draw (P) -- ($(P.east) + (1,0)$);
\end{tikzpicture}
}

\DeclareMathOperator*{\argmax}{arg\,max}
\DeclareMathOperator*{\argmin}{arg\,min}
\DeclareMathOperator*{\arginf}{arg\,inf}
\DeclareMathOperator*{\argsup}{arg\,sup}

\newcommand{\cheng}[1]{ {\color{purple}[{\bf Cheng:~{#1}}]} }

\title{Causal Statistical Decision Theory|What are interventions?}
\date{\today}

\author{ David Johnston }

\begin{document}

\maketitle


% \begin{abstract}
% We develop \emph{causal statistical decison theory} (CSDT) a novel theory of causal inference which we derive by introducing the idea that ``decisions have consequences'' to statistical decision theory. CSDT features \emph{causal theories} as the central object of study. We show that causal Bayesian networks have a natural representation as a causal theory and that potential outcomes models may arguably be represented as causal theories as well. In both cases the resulting theories feature unreasonably rich sets of decisions, which we suggest is because both approaches aim to produce reusable causal models. Using causal theories, we investigate reusability -- when can knowledge gained using one causal theory be applied to another -- and show that this is possible when the theories are related by a \emph{coarsening}.
% \end{abstract}
\tableofcontents



%!TEX root = main.tex


\chapter{Introduction}\label{ch:introduction}

\todo[inline]{I'm thinking of classification problems as types of prediction problems, though it's not really what ``prediction'' means. The key feature is that there is a ground truth that is not known at the time the prediction or class is offered, but will become fully known at some later point.}

Data driven prediction problems and data driven decision making problems have a lot in common. The outcomes some people are interested in predicting are often outcomes other people want to influence. A forecaster might want to predict the winner of the next election, while a party strategist is interested in maximising their party's chance of victory. A product manager may be simultaneously interested in accurately inferring the sentiment expressed in reviews of their product, and in making product changes that increase the frequency that this sentiment is positive. Furthermore, data relevant to prediction is often relevant to decision making and vise-versa. Political parties often reason that electorates in which their predicted chance of victory is very low are not worth investing campaign resources in, and if a forecaster learns of evidence that one party had adopted a particularly effective election strategy they might want to revisit their prediction of the eventual election winner. The overlap is not perfect: comprehensive electorate level polls are probably more useful to the forecaster while small-scale controlled experiments are probably more useful to the strategist.

A key difference between prediction and influence problems is the ``multiplicity of futures'' that each problem asks us to consider. A forecaster wants to identify -- loosely speaking -- the single most likely outcome, while a strategist must consider multiple options and identify the likely outcomes associated with each of these. As a consequence of this difference, the forecaster receives more complete feedback about the quality of their forecast than the strategist. Unlike the forecaster, all but one of the options that the strategist considers are never realised, and the world never offers feedback on these alternative options.

This difference suggests that it might be easier to assess the reliability of a predictive algorithm than the reliability of a decision-making algorithm, and this is be borne out in practice. Validating a predictive algorithm using data split into training and holdout sets is a ubiquitous in machine learning. For many data generating processes, appropriately conducted validation is widely considered to be a reliable indicator of an algorithm's performance for sufficiently similar data generating processes. In contrast, the most well-known condition that is widely accepted to yield reliable decision making algorithms is that the data used to draw inferences comes from a well-conducted controlled experiment. Data that satisfies this is much rarer than data that standard machine learning validation approaches can be applied to. There are approaches to causal inference that don't depend on experimental data, but they depend on other assumptions which are similarly applicable to a limited fraction of datasets. Alternatives to controlled experiments often come with the additional headache of being difficult to assess for a given dataset.

Some of the most far-reaching recent development in algorithmic decision making have involved only the elementary theory of randomised experiments. Operational advances that enable controlled experiments to be conducted at large scales have driven substantial changes in the operations of many online businesses \citep{kohavi_surprising_2017}, and Abhijit Banerjee and Esther Duflo were recently awarded a Nobel prize in part for their pioneering role in the use of large numbers of randomised controlled trials (RCTs) to assess the effectiveness of different kinds of development interventions \citep{zhang_abdul_2014}. Some fields of science have also been significantly affected by ``negative progress'' in the science of assessing experimental results. For example, in psychology, strong evidence has emerged that experimental findings from this field provide weaker evidence to a reader of the findings about the consequences of the reader's actions than many had believed \citet{open_science_collaboration_estimating_2015,stroebe_what_2019}. In a similar time frame, standards for what constitutes a ``well-conducted'' experiment have risen across many fields \citep{nosek_preregistration_2018,liberati_prisma_2009}.

An individual who wants to use data to make better decisions can consider running a controlled experiment of their own. This may not be possible, and even if it is, there may be large amounts of apparently relevant data available that seems wasteful to ignore on the basis of its non-experimental provenance. This individual might therefore be motivated to make some additional assumptions which allow them to draw conclusions about how to act from non-experimental data.

Some examples of assumptions this person could consider are (where * indicates that causal conclusions follow only when the data displays some key features):
\begin{itemize}
    \item There is an input variable independent of the \emph{potential outcomes} conditional on some covariates \citep[Chap. ~12]{imbens_causal_2015}, \citep[Chaps ~2, 3, 5]{angrist_mastering_2014}
    \item [*] There is a variable closely correlated with the \emph{potential outcomes} for each observation \citep[Chap. ~21]{imbens_causal_2015}
    \item \emph{Potential outcomes} are partially observed and vary in a simple way with some variable
    \item [*] The set of observed and unobserved variables have a known \emph{causal structure}
    \item The set of observed variables is \emph{causally sufficient} and the \emph{causal structure} is \emph{faithful} to the conditional independence structure of the observed variables

    \item DID (assumption: initial level accounts for confounders)
    \item Causal sufficiency + faithfulness
    \item Detailed causal structure w/identifiable substructure
\end{itemize}

A common feature of all of these: they're kinda confusing



 - Predictive machine learning has advanced tremendously w/combination of experimentation + theory
 - Causal inference has stuck close to established theory
 - Deriving assumptions is an exact science, making assumptions is not
 - Assumptions are so important in causal inference, and coming at it from a slightly different (decision theoretic) direction can a) expand the space of available assumptions and b) situating common assumptions in a different context


 Further reasons for alternative foundations
 - CGM theory has been undeniably productive -- mediation, confounding, ``m-structures'', causal discovery
 - Some outstanding questions:
 - Not always obvious how to map pre-formal knowledge to interventions
 - Not all variables are ``causally compatible''



\section{Theories of causal inference}

Beginning in the 1930s, a number of associations between cigarette smoking and lung cancer were established: on a population level, lung cancer rates rose rapidly alongside the prevalence of cigarette smoking. Lung cancer patients were far more likely to have a smoking history than demographically similar individuals without cancer and smokers were around 40 times as likely as demographically similar non-smokers to go on to develop lung cancer. In laborotory experiments, cells which were introduced to tobacco smoke developed \emph{ciliastasis}, and mice exposed to cigarette smoke tars developed tumors\citep{proctor_history_2012}. Nevertheless, until the late 1950s, substantial controversy persisted over the question of whether the available data was sufficient to establish that smoking cigarettes \emph{caused} lung cancer. Cigarette manufacturers famously argued against any possible connection \citep{oreskes_merchants_2011} and Roland Fisher in particular argued that the available data was not enough to establish that smoking actually caused lung cancer \citep{fisher_cancer_1958}. Today, it is widely accepted that cigarettes do cause lung cancer, along with other serious conditions such as vascular disease and chronic respiratory disease \citep{world_health_organisation_tobacco_nodate,wiblin_why_2016}.

The question of a causal link between smoking and cancer is a very important one to many different people. Individuals who enjoy smoking (or think they might) may wish to avoid smoking if cigarettes pose a severe health risk, so they are interested in knowing whether or not it is so. Additionally, some may desire reassurance that their habit is not too risky, whether or not this is true. Potential and actual investors in cigarette manufacturers may see health concerns as a barrier to adoption, and also may personally want to avoid supporting products that harm many people. Like smokers, such people might have some interest in knowing the truth of this question, and a separate interest in hearing that cigarettes are not too risky, whether or not this is true. Governments and organisations with a responsibility for public health may see themselves as having responsibility to discourage smoking as much as possible if smoking is severely detrimental to health. The costs and benefits of poor decisions about smoking are large: 8 million annual deaths are attributed to cigarette-caused cancer and vascular disease in 2018\citep{world_health_organisation_tobacco_nodate} while  global cigarette sales were estimated at US\$711 billion in 2020 \citep{noauthor_cigarettes_nodate} (a figure which might be substantially larger if cigarettes were not widely believed to be harmful).

The question of whether or not cigarette smoking causes cancer illustrates two key facts about causal questions: First, having the right answers to causal questions is of tremendous importance to huge numbers of people. Second, confusion over causal questions can persist even when a great deal of data and facts relevant to the question are agreed upon.

Causal conclusions are often justified on the basis of ad-hoc reasoning. For example \citet{krittanawong_association_2020} state:

\begin{quote}
[...] the potential benefit of increased chocolate consumption, reducing coronary artery disease (CAD) risk is not known. We aimed to explore the association between chocolate consumption and CAD.
\end{quote}

It is not clear whether Krittanawong et. al. mean that a negative association between chocolate consumption and CAD implies that increased chocolate consumption is likely to reduce coronary artery disease (which is suggested by the word ``benefit''), or that an association may be relevant to the question and the reader should draw their own conclusions. Whether the implication is being suggested by Krittanawong et. al. or merely imputed by na\"ive readers, it is being drawn on an ad-hoc basis -- no argument for the implication can be found in this paper. As \citet{pearl_causality:_2009} has forcefully argued, additional assumptions are always required to answer causal questions from associational facts, and stating these assumptions explicitly allows those assumptions to be productively scrutinised.

For causal questions that are controversial or difficult, it is tremendously advantageous to be able to address them transparently. Theories of causation enable this; given a theory of causation and a set of assumptions, if anyone claims that some conclusion follows it is publicly verifiable whether or not it actually does so. If the deduction is correct, then any remaining disagreement must be in the assumptions or in the theory. For people who are interested in understanding what is true, pinpointing disagreement can be enlightening. Someone could learn, for example, that there are assumptions that they find plausible that permit conclusions they did not initially believe. Alternatively, if a motivated conclusion follows only from implausible assumptions, hearing these assumptions explicitly might make the conclusion less attractive. 

Theories of causation help us to answer causal questions, which means that before we have any theory, we already have causal questions we want to answer. If potential outcomes notation and causal graphical models had never been invented there would still be just as many people who want to the answer to questions something like ``does smoking causes cancer?'', even if on-one could say what exactly they meant by ``causes'' and even if many people actually want answers to slightly different questions. Theories exist to serve our need for transparent answers to causal questions.

Potential outcomes and causal graphical models are prominent examples of ``practical theories'' of causation. I call them ``practical theories'' because most of the time we encounter them they are being used to answer ``practical'' questions like ``Does smoking cause cancer?'', or ``In general, when does data allow us to conclude that $X$ causes $Y$?'' It is less common to see the ``fundamental questions'' addressed, like ``Does the theory of causal graphical models offer an adequate account of what `cause' means?'', which is more often found in the field of philosophy. \citet{spirtes_causation_1993} explain their motivation to study what I call ``practical theories of causation'' as follows:

\begin{quote}
One approach to clarifying the notion of causation -- the philosophers’ approach ever since Plato -- is to try to define ``causation'' in other terms, to provide necessary and sufficient and noncircular conditions for one thing, or feature or event or circumstance, to cause another, the way one can define ``bachelor'' as ``unmarried adult male human.'' Another approach to the same problem -- the mathematician’s approach ever since Euclid -- is to provide axioms that use the notion of causation without defining it, and to investigate the necessary consequences of those assumptions. We have few fruitful examples of the first sort of clarification, but many of the second [...]
\end{quote}

I think what Spirtes, Glymour and Scheines (henceforth: SGS) mean here is that they \emph{define} a notion of causation -- because causal graphical models do define a notion of causation -- without interrogating whether it means the same thing as the word ``causation''. Incidentally, since publication of this paragraph, the notion of causation defined by causal graphical models has been subject to substantial interrogation by philosophers \citep{woodward_causation_2016}.

I am sympathetic to the argument that it does not matter a great deal whether ``causal-graphical-models-causation'' and ``causation'' mean the same thing in everyday language. It is common for words to have somewhat different meanings when used by specialists to when they are used by laypeople, and this isn't because the specialists are ignorant or confused about their subject. However, I think it matters a lot which causal questions can be transparently answered by ``causal-graphical-models-causation'', and so I believe that the notions of causation adopted by practical theories do warrant scrutiny.

I think one reason that SGS are keen to avoid dwelling on the definition of causation is that satisfactory definitions of causation are difficult. For example, causal graphical models depend on the notion of \emph{causal relationships} between variables. These may be defined as follows:

\begin{quote}
$\RV{X}_i$ is a \emph{cause} of $\RV{X}_j$ if there is an \emph{ideal intervention} on $\RV{X}_i$ that changes the value $\RV{X}_j$
\end{quote}

This definition is incomplete without a definition of ``ideal interventions''. Ideal interventions may be defined by their action in ``causally sufficient models'':
\begin{itemize}
    \item An $[\RV{X}_i,\RV{X}_j]$-ideal intervention is an operation whose result is determined by applying the \emph{do-calculus} to a \emph{causally sufficient} model $((\Omega,\mathcal{F},\prob{P}),\diagram{G},\vecRV{U})$
    \item A model $((\Omega,\mathcal{F},\prob{P}),\diagram{G},\vecRV{U})$ is $[\RV{X}_i,\RV{X}_j]$-causally sufficient if $\RV{U}$ contains $\RV{X}_i$, $\RV{X}_j$ and ``all intervenable variables that \emph{cause}'' both $\RV{X}_i$ and $\RV{X}_j$ \footnote{Weaker conditions for causal sufficiency are possible, but they don't avoid circularity \citep{shpitser_complete_2008}}
\end{itemize}

While I don't offer a definition of the \emph{do-calculus} in this introduction, it can be rigorously defined, see for example \citet{pearl_causality:_2009}. The problem is that the definition of a \emph{causally sufficient} model itself invokes the word \emph{cause}, which is what the original definition was trying to address. Circularity is a recognised problem with interventional definitions of causation \citep{woodward_causation_2016}. In Section \ref{sec:cbns_without_d}, I further show models with ideal interventions generally have counterintuitive properties. The purpose of a theory of causation like causal graphical models is to support transparent reasoning about causal questions, and a circular definition that leads to counterintuitive conclusions undermines this purpose.

As with Euclid's parallel postulate, I think it is reasonable to ask if the notion of ideal interventions and other causal definitions can be modified or avoided. Causal statistical decision theory (CSDT) is a theory of causation that is motivated by the problem of \emph{what is generally needed to answer causal questions} rather than \emph{what does ``causation'' mean?} Along similar lines to CSDT, \citet{dawid_decision-theoretic_2020} has observed that the problem of deciding how to act in light of data can be formalised without appeal to theories of causation. We develop this in substantial detail, showing how both \emph{interventional models} and \emph{counterfactual models} arise as special cases of CSDT.\todo{I want to revisit the claims about what I actually show, hopefully to add to it}

A key feature of CSDT is what I call the \emph{option set}. This is the set of decisions, acts or counterfactual propositions under consideration in a given problem. A causal graphical model and a potential outcomes model will both implicitly define an option set as a result of their basic definitions of causation, but CSDT demands that this is done explicitly. I argue that this is a key strength of CSDT, on the basis of the following claims which I defend in the following chapters:

\begin{itemize}
    \item Causal questions are not well-posed without an option set in the same way a function is not well-defined without its domain
    \item The option set need not correspond in any fixed manner to the set of observed variables
    \item The nature of the option set can affect the difficulty of causal inference questions
\end{itemize}


\todo[inline]{I commented out an additional section about potential outcomes and closest world counterfactuals, which is a second example of ``opaque causal definitions''. I'm interested if any readers think it would be good to have a second example}


% Potential outcomes basic assumptions

% \begin{itemize}
%     \item Potential outcomes defines ``the treatment effect of $\RV{X}_i$ on $\RV{X}_j$'' in terms of the value of $\RV{X}_j$ under the \emph{counterfacutal supposition} that $\RV{X}_i$ had taken a different value
% \end{itemize}

% In fact, the notion of ``ideal intervention'' often seems to underpin potential outcomes models as well. Work in the potential outcomes theory often uses the idea of the value of $\RV{X}_j$ under a counterfactual supposition concerning $\RV{X}_i$ interchangeably with the idea the response of $\RV{X}_j$ to an idealised intervention on $\RV{X}_i$ \citep{morgan_counterfactuals_2014,rubin_causal_2005,richardson2013single}. \cite{lewis_causation_1986} offered a definition of the value $\RV{X}_j$ under counterfactual suppositions in terms of the value it would take in the world that was ``closest'' to the real world but in which the value of $\RV{X}_i$ was altered. There are many ways that we could use to measure how close one world is to another, many of which need not invoke any notion of ``ideal intervention'', but I have never encountered practical work on causal inference that was based on considerations of such similarity measures.


\section{Causally compatible variables}\label{sec:cc_vars}


%!TEX root = main.tex

\todo[inline][Todo: I need the following theorem in this chapter]
\begin{theorem}[Representation]\label{th:representaiton}
\todo[inline]{Representation theorem: can uniquely define kernel $P^{\RV{X}|\RV{Y}}$ with $P^{\RV{Z}|\RV{Y}}$ and $P^{\RV{X}|\RV{Z}\RV{Y}}$ }
\end{theorem}


\subsection{Probability Theory}

Given a set $A$, a $\sigma$-algebra $\mathcal{A}$ is a collection of subsets of $A$ where
\begin{itemize}
	\item $A\in \mathcal{A}$ and $\emptyset\in \mathcal{A}$
	\item $B\in \mathcal{A}\implies B^C\in\mathcal{A}$
	\item $\mathcal{A}$ is closed under countable unions: For any countable collection $\{B_i|i\in Z\subset \mathbb{N}\}$ of elements of $\mathcal{A}$, $\cup_{i\in Z}B_i\in \mathcal{A}$ 
\end{itemize}

A measurable space $(A,\mathcal{A})$ is a set $A$ along with a $\sigma$-algebra $\mathcal{A}$. Sometimes the sigma algebra will be left implicit, in which case $A$ will just be introduced as a measurable space.

\paragraph{Common $\sigma$ algebras}

For any $A$, $\{\emptyset,A\}$ is a $\sigma$-algebra. In particular, it is the only sigma algebra for any one element set $\{*\}$.

For countable $A$, the power set $\mathscr{P}(A)$ is known as the discrete $\sigma$-algebra.

Given $A$ and a collection of subsets of $B\subset\mathscr{P}(A)$, $\sigma(B)$ is the smallest $\sigma$-algebra containing all the elements of $B$. 

Let $T$ be all the open subsets of $\mathbb{R}$. Then $\mathcal{B}(\mathbb{R}):=\sigma(T)$ is the \emph{Borel $\sigma$-algebra} on the reals. This definition extends to an arbitrary topological space $A$ with topology $T$.

A \emph{standard measurable set} is a measurable set $A$ that is isomorphic either to a discrete measurable space $A$ or $(\mathbb{R}, \mathcal{B}(\mathbb{R}))$. For any $A$ that is a complete separable metric space, $(A,\mathcal{B}(A))$ is standard measurable. 

Given a measurable space $(E,\mathcal{E})$, a map $\mu:\mathcal{E}\to [0,1]$ is a \emph{probability measure} if
\begin{itemize}
	\item $\mu(E)=1$, $\mu(\emptyset)=0$
	\item Given countable collection $\{A_i\}\subset\mathscr{E}$, $\mu(\cup_{i} A_i) = \sum_i \mu(A_i)$
\end{itemize}

Write by $\Delta(\mathcal{E})$ the set of all probability measures on $\mathcal{E}$.

A particular probability measure we will often discuss is the \emph{Dirac measure}. For any $x\in X$, the Dirac measure $\delta_x\in \Delta(\sigalg{X})$ is the probability measure where $\delta_x(A)=0$ if $x\not\in A$ and $\delta_x(A)=1$ if $x\in A$.

Given another measurable space $(F,\mathcal{F})$, a \emph{stochastic map} or \emph{Markov kernel} is a map $\kernel{M}:E\times\mathcal{F}\to [0,1]$ such that
\begin{itemize}
	\item The map $\kernel{M}(\cdot;A):x\mapsto \kernel{M}(x;A)$ is $\mathcal{E}$-measurable for all $A\in \mathcal{F}$
	\item The map $\kernel{M}_x:A\mapsto \kernel{M}(x;A)$ is a probability measure on $F$ for all $x\in E$
\end{itemize}

Extending the subscript notation, for $\kernel{C}:X\times Y\to \Delta(\mathcal{Z})$  and $x\in X$ we will write $\kernel{C}_{x,\cdot}$ for the ``curried'' map $y\mapsto \kernel{C}_{x,y}$. If $\kernel{C}$ is a Markov kernel with respect to $(X\times Y, \sigalg{X}\otimes\sigalg{Y}),(Z,\sigalg{Z})$ then it is straightforward to show that $\kernel{C}_{x,\cdot}$ is a Markov kernel with respect to $(Y,\sigalg{Y}),(Z,\sigalg{Z})$.

This yields the notational conventions for arbitrary kernel $\kernel{C}$:

\begin{itemize}
	\item $\kernel{C}$ with no subscripts is a Markov kernel
	\item $\kernel{C}_{\cdot,a,b}$ with at least one $\cdot$ subscript is a Markov kernel
	\item $\kernel{C}_y$ with no $\cdot$ subscripts is a probability measure
\end{itemize}

The map $x\mapsto \kernel{M}_x$ is of type $E\to \Delta(\mathcal{F})$. We will abuse notation somewhat to write $\kernel{M}:E\to \Delta(\mathcal{F})$. In this sense, we view Markov kernels as maps from elements of $E$ to probability measures on $\mathcal{F}$. This is simply a convention that helps us to think about constructions involving Markov kernels, and it is equally valid to view Markov kernels as maps from elements of $\mathcal{F}$ to measurable functions $E\to[0,1]$, a view found in \citet{clerc_pointless_2017}, or simply in terms of their definition above.

Given an indiscrete measurable space $(\{*\},\{\{*\},\emptyset\})$, we identify Markov kernels $\kernel{N}:\{*\}\to \Delta(\mathcal{E})$ with the probability measure $\kernel{N}_*$. In addition, there is a unique Markov kernel $\stopper{0.2}:E\to \Delta(\{\{*\},\emptyset\})$ given by $x\mapsto \delta_*$ for all $x\in E$ which we will call the ``discard'' map.

Two Markov kernels $\kernel{M}X\to \Delta(\sigalg{Y})$ and $\kernel{N}:X\to \Delta(\sigalg{Y})$ are equal iff for all $x\in X$, $A\in \sigalg{Y}$
\begin{align}
	\kernel{M}_x(A) = \kernel{N}_x(A)
\end{align}

We will typically be more concerned with ``almost sure'' equality than exact equality, which will be defined later.

\subsection{Product Notation}\label{ssec:product_notation}

Probability measures, Markov kernels and measurable functions can be combined to yield new probability measures, Markov kernels or measurable functions. Given $\mu\in \Delta(\mathcal{X})$, $\RV{T}:Y\to T$, $\kernel{M}:X\to \Delta(\sigalg{Y})$ and $\kernel{N}:Y\to \Delta(\sigalg{Z})$ define:

The \textbf{measure-kernel} product $\mu \kernel{M}:\sigalg{Y}\to [0,1]$ where for all $A\in\sigalg{Y}$,

\begin{align}
\mu\kernel{M} (A) := \int_X \kernel{M}_x (A) d\mu(x)
\end{align}

The \textbf{kernel-function} product $\kernel{M} \RV{T}:X\to T$ where for all $x\in X$:

\begin{align}
\kernel{M}\RV{T}(x) := \int_Y T(y) d\kernel{M}_x(y)
\end{align}


The \textbf{kernel-kernel} product $\kernel{M}\kernel{N}:X\to \Delta(\sigalg{Z})$ where for all $x\in X$, $A\in \sigalg{Z}$:

\begin{align}
(\kernel{M}\kernel{N})_x(A) &:= \int_Y \kernel{N}_y(A) d\kernel{M}_x(y)
 \end{align} 

All kernel products are associative \citep{cinlar_probability_2011}. An intuition for this notation can be gained from thinking of probability measures $\mu\in \Delta(\mathcal{X})$ as row vectors, Markov kernels $\kernel{M},\kernel{N}$ as matrices and measurable functions $\RV{T}:Y\to T$ as column vectors and kernel products are vector-matrix and matrix-matrix products. If the $X,Y,Z$ and $T$ are discrete spaces then this analogy is precise.

Finally, the \textbf{tensor product} $\kernel{M}\otimes \kernel{N}:X\times Y\to \Delta(\sigalg{Y}\otimes\sigalg{Z})$ is yields the kernel that applies $\kernel{M}$ and $\kernel{N}$ ``in parallel''. For all $x\in X$, $y\in Y$, $G\in \sigalg{Y}$ and $H\in \sigalg{Z}$:

\begin{align}
(\kernel{M}\otimes \kernel{N})_{x,y}(G\times H) := \kernel{M}_x(G)\kernel{N}_y(H)
\end{align}

\subsection{String Diagrams}\label{ssec:mken_diagrams}

Some constructions are unwieldly in product notation; for example, given $\mu\in \Delta(\mathcal{E})$ and $\kernel{M}:E\to (\mathcal{F})$, it is not straightforward to write an expression using kernel products and tensor products that represents the ``joint distribution'' given by $A\times B\mapsto \int_A \kernel{M}(x;B)d\mu$.

An alternative notation known as \emph{string diagrams} provides greater expressive capability than product notation while being more visually clear than integral notation. \citet{cho_disintegration_2019} provides an extensive introduction to string diagram notation for probability theory.

Key features of string diagrams include:
\begin{itemize}
	\item String diagrams as they are used in this work can always be interpreted as a mixture of kernel-kernel products and tensor products of Markov kernels
	\item String diagrams are the subject of a coherence theorem: two string diagrams that differ only by planar deformation are always equal \citep{selinger_survey_2010}. This also holds for a number of additional transformations detailed below
	\begin{itemize}
		\item Informally, diagrams that look like they should be the same are in fact the same
	\end{itemize}
\end{itemize}

\subsubsection{Elements of string diagrams}

The basic elements of a string diagram are Markov kernels. Diagrams representing Markov kernels can be assembled into larger diagrams by taking regular products or tensor products.

Indiscrete spaces play a key role in string diagrams. An indiscrete space is any one element measurable space $(\{*\},\{\emptyset,\{*\}\})$ which admits the unique probability measure $\mu:\{\emptyset,\{*\}\}\to(0,1)$ given by $\mu(\emptyset)=0$, $\mu(\{*\})=1$. Any probability measure $\mu\in \Delta(\sigalg{X})$ can be interpreted as a Markov kernel $\mu':\{*\}\to \Delta(\mathcal{X})$ where $\mu'_*=\mu$ (note that $*$ is the \emph{only} argument $\mu'$ can be given).


A Markov kernel $\kernel{M}:X\to \Delta(\mathcal{Y})$ can always be represented as a rectangular box with input and output wires labeled with the relevant spaces:

\begin{align}
\begin{tikzpicture}
\path (0,0) node (A) {$X$}
++(0.75,0) node[kernel] (B) {$\kernel{M}$}
++(0.75,0) node (C) {$Y$};
\draw (A) -- (B) -- (C);
\end{tikzpicture}
\end{align}

Note that we will later substitute labelling wires with spaces for labelling them with random variable names.

Probability measures are kernels with an indiscrete domain $\mu \in \Delta(\mathcal{X})$ can be written as triangles:
\begin{align}
\begin{tikzpicture}
\path (0,0) node[dist] (B) {$\mu$}
++(0.75,0) node (C) {$X$};
\draw (B) -- (C);
\end{tikzpicture}\label{eq:prob_meas_sd}
\end{align}

Note that Eq \ref{eq:prob_meas_sd} technically represents a Markov kernel $\mu':\{*\}\to\Delta(\mathcal{X})$, but for our purposes this distinction isn't practically important.

We do \emph{not} define kernel-function products for string diagrams. While kernel-kernel products always yield Markov kernels as a result, and measure-kernel products can be reinterpreted as kernel-kernel products, kernel-function products do not admit such a reinterpretation. \citet{cho_disintegration_2019} defines the operation of \emph{conditioning} using kernel-function products, but this will take extra work to incorporate into our notation which hasn't yet been done.

\paragraph{Elementary operations}

Kernel-kernel products have a visually similar representations in string diagram notation to the previously introduced product notation. Given $\kernel{M}:X\to\Delta(\mathcal{Y})$ and $\kernel{N}:Y\to \Delta(\mathcal{Z})$, we have 

\begin{align}
\kernel{M}\kernel{N} := \begin{tikzpicture}
 \path (0,0) node (E) {$X$}
 ++ (1,0) node[kernel] (M) {$\kernel{M}$}
 ++ (1,0) node[kernel] (N) {$\kernel{N}$}
 ++(1,0) node (G) {$Z$};
 \draw (E) -- (M) -- (N) -- (G);
\end{tikzpicture}\label{eq:sd_composition}
\end{align}

For $\mu\in \Delta(\mathcal{E})$,

\begin{align}
\mu\kernel{M} &:= \begin{tikzpicture}
 \path (0,0) node[dist] (M) {$\mu$}
 ++ (1,0) node[kernel] (N) {$\kernel{M}$}
 ++(1,0) node (G) {$Z$};
 \draw (M) -- (N) -- (G);
\end{tikzpicture}
\end{align}

Tensor products in string diagram notation are represented by vertical juxtaposition. For $\kernel{O}:Z\to \Delta(\mathcal{W})$:

\begin{align}
\kernel{M}\otimes\kernel{O}&:= \begin{tikzpicture}
\path (0,0) node (E) {$X$}
++(1,0) node[kernel] (M) {$\kernel{M}$}
++(1,0) node (F) {$Y$}
(0,-0.5) node (F1) {$Z$}
++(1,0) node[kernel] (N) {$\kernel{O}$}
+(1,0) node (G) {$W$};
\draw (E) -- (M) -- (F);
\draw (F1) -- (N) -- (G);
\end{tikzpicture}
\end{align}

A space $X$ is identified with the identity kernel $\mathrm{Id}^X:X\to \Delta(\sigalg{X})$, $x\mapsto \delta_x$. A bare wire represents an identity kernel or, equivalently, the space given by its labels:

\begin{align}
\mathrm{Id}^X:=\begin{tikzpicture}
\path (0,0) node (X) {$X$}
++(2,0) node (Y) {$X$};
\draw (X) -- (Y);
\end{tikzpicture}
\end{align}

Product spaces $X\times Y$ are identified with tensor products of identity kernels $X\times Y \cong \kernel{I}^X\otimes \kernel{I}^Y$. These can be represented either by two parallel wires or by a single wire equipped with appropriate labels:
\begin{align}
X\times Y \cong \mathrm{Id}^X\otimes \mathrm{Id}^Y &:= \begin{tikzpicture}
\path (0,0) node (E) {$X$}
++(1,0) node (F) {$X$}
(0,-0.5) node (F1) {$Y$}
+(1,0) node (G) {$Y$};
\draw (E) -- (F);
\draw (F1) -- (G);
\end{tikzpicture}\\
&= \begin{tikzpicture}
\path (0,0) node (X) {$X\times Y$}
++(2,0) node (Y) {$X\times Y$};
\draw (X) -- (Y);
\end{tikzpicture}
\end{align}

A kernel $\kernel{L}:X\to \Delta(\mathcal{Y}\otimes\mathcal{Z})$ can be written using either two parallel output wires or a single output wire, appropriately labeled:

\begin{align}
&\begin{tikzpicture}
\path (0,0) node (E) {$X$}
++ (1,0) node[kernel] (L) {$\kernel{L}$}
++ (1,0.15) node (F) {$Y$}
+(0,-0.3) node (G) {$Z$};
\draw (E) -- (L);
\draw ($(L.east) + (0,0.15)$) -- (F);
\draw ($(L.east)+ (0,-0.15)$) -- (G);
\end{tikzpicture}\\
&\equiv\\
&\begin{tikzpicture}
\path (0,0) node (E) {$X$}
++ (1,0) node[kernel] (L) {$\kernel{L}$}
++ (1.5,0) node (F) {$Y\times Z$};
\draw (E) -- (L) -- (F);
\end{tikzpicture}
\end{align}

\paragraph{Markov kernels with special notation}

A number of Markov kernels are given special notation distinct from the generic ``box'' above. This notation facilitates intuitive visual representation.

As has already been noted, the identity kernel $\textbf{Id}:X\to \Delta(X)$ maps a point $x$ to the measure $\delta_x$ that places all mass on the same point:

\begin{align}
\textbf{Id} : x\mapsto \delta_x \equiv \begin{tikzpicture}\path (0,0) node (X) {$X$} + (1,0) node (X1) {$X$}; \draw (X)--(X1); \end{tikzpicture}\label{eq:identity}
\end{align}

The identity kernel is an identity under left and right products:

\begin{align}
	(\kernel{K}\textbf{Id})_w(A) &= \int_X \textbf{Id}_x(A) d\kernel{K}_w (x) \\
							 	 &= \int_X \delta_x(A) d\kernel{K}_w(x)\\
							 	 &= \int_A d\kernel{K}_w(x)\\
							 	 &= \kernel{K}_w(A)\\
	(\textbf{Id}\kernel{K})_w(A) &= \int_X \kernel{K}_x (A) d\textbf{Id}_w(x)\\
								 &= \int_X  \kernel{K}_x(A) d\delta_w(x)\\
								 &= \kernel{K}_w(A)								  
\end{align}

The copy map $\splitter{0.1}:X\to \Delta(\mathcal{X}\times \mathcal{X})$ maps a point $x$ to two identical copies of x:
\begin{align}
 \splitter{0.1}: x\mapsto \delta_{(x,x)} \equiv \begin{tikzpicture}
 \path (0,0) node (X) {$X$} ++ (0.5,0) coordinate (copy0) ++ (0.5,0.25) node (X1) {$X$} ++(0,-0.5) node (X2) {$X$};\draw (X)--(copy0) to [bend left] (X1) (copy0) to [bend right] (X2);
 \end{tikzpicture}\label{eq:copy}
 \end{align} 

The copy map ``copies'' its arguments to kernels or under the right product:

\begin{align}
	\int_(X\times X) \kernel{K}_{x',x''}(A) d\splitter{0.1}_x (x',x'') &= \int_(X\times X) \kernel{K}_{x',x''}(A) d\delta_{(x,x)}(x',x'')\\
															&= \kernel{K}_{x,x}(A)
\end{align}

The swap map $\sigma:X\times Y\to \Delta(\mathcal{Y}\otimes\mathcal{X})$ swaps its inputs:

\begin{align}
\sigma := (x,y)\to \delta_{(y,x)} \equiv \begin{tikzpicture}
\path (0,0) node (X) {$X$}
+(1,0.3) node (X1) {$X$}
(0,0.3) node (Y) {$Y$}
+(1,-0.3) node (Y1) {$Y$};
\draw (X)--(X1) (Y) -- (Y1);
\end{tikzpicture}\label{eq:swap}
\end{align}

Under products are taken with the swap map, arguments are interchanged. For $\kernel{K}:X\times Y\to \Delta(\sigalg{Z})$ and $\kernel{L}:Z\to \Delta(\sigalg{X}\otimes\sigalg{Y})$, $A\in \sigalg{X}$, $B\in\sigalg{Y}$:

\begin{align}
	(\sigma\kernel{K})_{y,x}(A) &= \int_(X\times Y) \kernel{K}_{x',y'}(A) d\sigma_{(y,x)}(x',y') &= \int_(X\times Y) \kernel{K}_{x',y'}(A) d\delta_{(x,y)}(x',y')\\
													   &= \kernel{K}_{x,y}(A)\\
	(\kernel{L}\sigma)_{z}(B\times A) &= \int_{X\times Y} \sigma_{x',y'}(B\times A) d\kernel{L}_z(x',y')\\
	&= \int_{X\times Y} \delta_{(y',x')} (B\times A) d\kernel{L}_z(x',y')\\
	&= \kernel{L}_z(A\times B)
\end{align}

The discard map $\stopper{0.2}:X\to \Delta(\{*\})$ maps every input to $\delta_{*}$, the unique probability measure on the indiscrete set $\{\emptyset,\{*\}\}$.
\begin{align}
\stopper{0.2}: x\mapsto \delta_{*} \equiv \begin{tikzpicture}
 \draw[-{Rays [n=8]}] (0,0) node (X) {$X$} (X) -- (1,0);
\end{tikzpicture}\label{eq:discard}
\end{align}

Any measurable function $g:W\to X$ has an associated Markov kernel $\kernel{F}^g:W\to \Delta(\mathcal{X})$ given by $\kernel{F}^g:w\mapsto \delta_{g(w)}$. Given a probability measue $\mu\in \Delta(\sigalg{W})$, $\mu g$ is a measure-function product while $\mu \kernel{F}^g$ is commonly called the pushforward measure $g_\# \mu$. We will generalise this slightly to the notion of \emph{pushforward kernels}.

\begin{definition}[Kernel associated with a function]\label{def:functional_kernel}
Given a measurable function $g:W\to X$, define the function induced kernel $\kernel{F}^{g}:W\to \Delta(\mathcal{X})$ to be the the Markov kernel $w\mapsto \delta_{g(w)}$ for all $w\in W$.
\end{definition}

\begin{definition}[Pushforward kernel]
Given a kernel $\kernel{M}:V\to \Delta(\mathcal{W})$ and a measurable function $g:W\to X$, the \emph{pushforward kernel} $g_\# \kernel{M}:V\to \Delta(\mathcal{X})$ is the kernel $g_\# \kernel{M}$ such that $(g_\# \kernel{M})_a(B) = \kernel{M}_a(g^{-1}(B))$ for all $a\in V$, $B\in \sigalg{X}$.
\end{definition}

\begin{lemma}[Pushforward kernels are functional kernel products]\label{lem:pushf_funk}
Given a kernel $\kernel{M}:V\to \Delta(\mathcal{W})$ and a measurable function $g:W\to X$, $g_\# \kernel{M} = \kernel{M} \kernel{F}^{g}$.
\end{lemma}

\begin{proof}
for any $a\in V$, $B\in \sigalg{X}$:
\begin{align}
	(\kernel{M}\kernel{F}^g)_a(B) &= \int_W \delta_{g(y)}(B) d\kernel{M}_a(y)\\
								&= \int_W \delta_{y}(g^{-1}(B)) d\kernel{M}_a(y)\\
								&= \int_{g^{-1}(B)} d\kernel{M}_a(y)\\
								&= (g_{\#} \kernel{M})_a (B)
\end{align}
\end{proof}

\subsubsection{Working With String Diagrams}\label{sssec:string_diagram_manipulation}

todo:
\begin{itemize}
\item Infinite copy map
\item De Finetti's representation theorem
\end{itemize}

There are a relatively small number of manipulation rules that are useful for string diagrams. In addition, we will define graphically analogues of the standard notions of \emph{conditional probability}, \emph{conditioning}, and infinite sequences of exchangeable random variables.

\paragraph{Axioms of Symmetric Monoidal Categories}

For the following, we either omit labels or label diagrams with their domain and codomain spaces, as we are discussing identities of kernels rather than identities of components of a condtional probability space. Recalling the unique Markov kernels defined above, the following equivalences, known as the \emph{commutative comonoid axioms}, hold among string diagrams:

\begin{align}
	\begin{tikzpicture}[scale=0.8]
	\path (0,0) node (X) {} 
	++ (0.5,0) coordinate (copy0)
	+ (1.5,0.5) node (X1) {}
	++ (0.5,-0.5) coordinate (copy1)
	+(1,0.5) node (X2) {}
	+(1,-0.5) node (X3) {};
	\draw (X) -- (copy0) to [bend left] (X1) (copy0) to [bend right] (copy1) to [bend left] (X2) (copy1) to [bend right] (X3);
	\end{tikzpicture}
	=
	\begin{tikzpicture}[scale=0.8]
	\path (0,0) node (X) {} 
	++ (0.5,0) coordinate (copy0)
	+ (1.5,-0.5) node (X1) {}
	++ (0.5,0.5) coordinate (copy1)
	+(1,0.5) node (X2) {}
	+(1,-0.5) node (X3) {};
	\draw (X) -- (copy0) to [bend right] (X1) (copy0) to [bend left] (copy1) to [bend left] (X2) (copy1) to [bend right] (X3);
	\end{tikzpicture}
	:=
	\begin{tikzpicture}[scale=0.8]
	\path (0,0) node (X) {} 
	++ (0.5,0) coordinate (copy0)
	+ (1,0.5) node (X1) {}
	+(1,0) node (X2) {}
	+(1,-0.5) node (X3) {};
	\draw (X) -- (copy0) to [bend left] (X1) (copy0) to (X2) (copy0) to [bend right] (X3);
	\end{tikzpicture}\label{eq:ccom1}
\end{align}

\begin{align}
	\begin{tikzpicture}[scale=0.8]
	\path (0,0) node (X) {}
	++(0.5,0) coordinate (copy0)
	+ (1,0.5) node (S) {}
	+(1,-0.5) node (X1) {};
	\draw (X) -- (copy0) to [bend right] (X1);
	\draw[-{Rays [n=8]}] (copy0) to [bend left] (S);
	\end{tikzpicture}
	= 
	\begin{tikzpicture}[scale=0.8]
	\path (0,0) node (X) {}
	++(0.5,0) coordinate (copy0)
	+ (1,-0.5) node (S) {}
	+(1,0.5) node (X1) {};
	\draw (X) -- (copy0) to [bend left] (X1);
	\draw[-{Rays [n=8]}] (copy0) to [bend right] (S);
	\end{tikzpicture}
	=
	\begin{tikzpicture}[scale=0.8]
	\path (0,0) node (X) {}
	++ (1,0) node (X1) {};
	\draw (X) -- (X1);
	\end{tikzpicture}\label{eq:ccom2}
\end{align}

\begin{align}
	\begin{tikzpicture}[scale=0.8]
	\path (0,0) node (X) {$\RV{X}$}
	++(0.5,0) coordinate (copy0)
	+ (1,0.5) node (X2) {$\RV{X}$}
	+(1,-0.5) node (X1) {$\RV{X}$};
	\draw (X) -- (copy0) to [bend right] (X1);
	\draw (copy0) to [bend left] (X2);
	\end{tikzpicture}
=
	\begin{tikzpicture}[scale=0.8]
	\path (0,0) node (X) {}
	++(0.5,0) coordinate (copy0)
	+ (1.2,0.5) node (X2) {}
	+(1.2,-0.5) node (X1) {};
	\draw (X) -- (copy0) .. controls (0.75,0.4) .. (X1.west);
	\draw (copy0) .. controls (0.75,-0.4) .. (X2.west);
	\end{tikzpicture}
\label{eq:ccom3}
\end{align}

The discard map $\stopper{0.2}$ can ``fall through'' any Markov kernel:

\begin{align}
\begin{tikzpicture}
\path (0,0) node (X) {}
++(0.7,0) node[kernel] (A) {$\kernel{A}$}
++(0.7,0) node (S) {};
\draw (X) -- (A);
\draw[-{Rays [n=8]}] (A) -- (S);
\end{tikzpicture}
= 
\begin{tikzpicture}
\path (0,0) node (X) {}
++(0.7,0) node (S) {};
\draw[-{Rays [n=8]}] (X) -- (S);
\end{tikzpicture}\label{eq:termobj1}
\end{align}

Combining \ref{eq:ccom2} and \ref{eq:termobj1} we can derive the following: integrating $\kernel{A}:X\to \Delta(\mathcal{Y})$ with respect to $\mu\in\Delta(\mathcal{X})$ and then discarding the output of $\kernel{A}$ leaves us with $\mu$:

\begin{align}
\begin{tikzpicture}
\path (0,0) node[dist] (mu) {$\mu$}
++ (1,0) coordinate (copy0)
+ (1.4,0.5) node (X) {}
++ (0.7,-0.5) node[kernel] (A) {$\kernel{A}$}
++(0.7,0) node (Y) {};
\draw (mu)--(copy0);
\draw (copy0) to [bend left] (X);
\draw[-{Rays [n=8]}] (copy0) to [bend right] (A) (A) -- (Y);
\end{tikzpicture}
= 
\begin{tikzpicture}
\path (0,0) node[dist] (mu) {$\mu$}
++ (1,0) coordinate (copy0)
+ (1.2,0.5) node (X) {}
++ (0.4,-0.3) coordinate (A)
++(0.1,0) node (Y) {};
\draw (mu)--(copy0);
\draw (copy0) to [bend left] (X);
\draw[-{Rays [n=8]}] (copy0) to [bend right] (A) (A) -- (Y);
\end{tikzpicture}
=
\begin{tikzpicture}
\path (0,0) node[dist] (mu) {$\mu$}
++ (1,0) node (X) {};
\draw (mu)--(X);
\end{tikzpicture}
\end{align}

In elementary notation, this is equivalent to the fact that, for all $B\in \mathcal{X}$, $\int_B \kernel{A}(x;B)d\mu(x) = \mu(B)$.

The following additional properties hold for $\stopper{0.2}$ and $\splitter{0.1}$:

\begin{align}
\begin{tikzpicture}
\path (0,0) node (XY) {$X\times Y$}
++ (1.5,0) node (Z) {};
\draw[-{Rays [n=8]}] (XY) -- (Z);
\end{tikzpicture} &=
\begin{tikzpicture}
\path (0,0) node (X) {$X$} 
++ (1,0) node (X1) {}
(0,-0.3) node (Y) {$Y$}
++ (1,0) node (Y1) {};
\draw[-{Rays [n=8]}] (X) -- (X1);
\draw[-{Rays [n=8]}] (Y) -- (Y1);
\end{tikzpicture}
\end{align}
\begin{align}
\begin{tikzpicture}
\path (0,0) node (XY) {$X\times Y$}
++ (1.2,0) coordinate (copy0)
++(1.2,0.3) node (XY1) {$X \times Y$}
++(0,-0.6) node (XY2) {$X\times Y$};
\draw (XY) -- (copy0) to [bend left] (XY1);
\draw (XY) -- (copy0) to [bend right] (XY2);
\end{tikzpicture} &=
\begin{tikzpicture}
\path (0,0) node (XY) {$X$}
++ (1.,0) coordinate (copy0)
++(1.,0.5) node (XY1) {$X$}
++(0,-1) node (XY2) {$X$}
(0,-0.3) node (F) {$Y$}
++(1.,0) coordinate (copy1)
++(1.,0.5) node (F1) {$Y$}
++(0,-1) node (F2) {$Y$};
\draw (XY) -- (copy0) to [bend left] (XY1);
\draw (copy0) to [bend right] (XY2);
\draw (F) -- (copy1) to [bend left] (F1);
\draw (copy1) to [bend right] (F2);
\end{tikzpicture}
\end{align}

A key fact that \emph{does not} hold in general is

\begin{align}
 \begin{tikzpicture}
\path (0,0) node (E) {}
++ (0.7,0) node[kernel] (A) {$\kernel{A}$}
++(0.7,0) coordinate (copy0)
++(0.5,0.3) node (F1) {}
+(0,-0.6) node (F2) {};
\draw (E) -- (A) -- (copy0) to [bend left] (F1);
\draw (copy0) to [bend right] (F2);
\end{tikzpicture} 
=
\begin{tikzpicture}
\path (0,0) node (E) {}
++(0.5,0) coordinate (copy0)
++(0.7,0.3) node[kernel] (A1) {$\kernel{A}$}
+(0,-0.6) node[kernel] (A2) {$\kernel{A}$}
++(0.75,0) node (F1) {}
+(0,-0.6) node (F2) {};
\draw (E) -- (copy0) to [bend left] (A1) (A1) -- (F1);
\draw (copy0) to [bend right] (A2) (A2) -- (F2);
\end{tikzpicture}
\label{eq:copy_commutes}
\end{align}

In fact, it holds only when $\kernel{A}$ is a \emph{deterministic} kernel.

\begin{definition}[Deterministic Markov kernel]
A \emph{deterministic} Markov kernel $\kernel{A}:E\to \Delta(\mathcal{F})$ is a kernel such that $\kernel{A}_x(B)\in\{0,1\}$ for all $x\in E$, $B\in\mathcal{F}$.
\end{definition}

\begin{theorem}[Copy map commutes for deterministic kernels \citep{fong_causal_2013}]
Equation \ref{eq:copy_commutes} holds iff $\kernel{A}$ is deterministic.
\end{theorem}

\subsubsection{Examples}

Given $\mu\in\Delta(X),\kernel{K}:X\to \Delta(Y)$, $A\in \mathcal{X}$ and $B\in\mathcal{Y}$:

\begin{align}
&A\times B\mapsto \int_A \kernel{K}(x;B)d\mu(x)\\ &\equiv \\\mu 
&\splitter{0.1}(\textbf{Id}_X\otimes \kernel{K})\\ &\equiv \\
&\begin{tikzpicture}
\path (0,0) node[dist] (mu) {$\mu$}
++ (1,0) coordinate (copy0)
+ (1.2,0.5) node (X) {$X$}
++ (0.5,-0.5) node[kernel] (A) {$\kernel{K}$}
++(0.7,0) node (Y) {$Y$};
\draw (mu)--(copy0);
\draw (copy0) to [bend left] (X);
\draw (copy0) to [bend right] (A) (A) -- (Y);
\end{tikzpicture}\label{eq:joint_measure}
\end{align}

\citet{cho_disintegration_2019} calls this operation ``integrating $\kernel{K}$ with respect to $\mu$''.

Given $\nu\in \Delta(\sigalg{X}\otimes\sigalg{Y})$, define the marginal $\nu^{\RV{Y}}\in \Delta(\mathcal{Y}):B\mapsto \mu(X\times B)$ for $B\in \mathcal{Y}$. Say that $\nu^{\RV{Y}}$ is obtained by marginalising over ``$X$'' (a notion that can be made more precise by assigning names to wires). Then

\begin{align}
	\nu(\stopper{0.25}\otimes \mathrm{Id}^Y) &= \begin{tikzpicture}
		\path (0,0) node[dist] (nu) {$\nu$}
		++ (0.7,-0.15) node (X) {$Y$}
		+(0,0.3) node (Y) {};
		\draw ($(nu.east)+(0,-0.15)$) -- (X);
		\draw[-{Rays[n=8]}] ($(nu.east)+(0,0.15)$) -- (Y);
	\end{tikzpicture}\\
	\nu(\stopper{0.25}\otimes \mathrm{Id}^Y)(B) &:= \nu(\stopper{0.25}\otimes \mathrm{Id}^Y)(B\times\{*\})\\
												&=\int_{X\times Y} \mathrm{Id}^Y_y(B) \stopper{0.2}_x(\{*\}) d\nu(x,y)\\
	&= \int_{X\times Y} \delta_y(B) \delta_*(\{*\}) d\nu(x,y)\\
	&= \int_{X\times B} d\nu(x,y)\\
	&= \nu(X\times B)\\
	&= \nu^{\RV{Y}}(B)
\end{align}

Thus the action of the erasing wire ``$X$'' is equivalent to marginalising over ``$X$''.

Consider the result of marginalising \ref{eq:joint_measure} over ``$X$'':
\begin{align}
  \nu^Y (B) &= \begin{tikzpicture}
\path (0,0) node[dist] (mu) {$\mu$}
++ (1,0) coordinate (copy0)
+ (1.2,0.5) node (X) {}
++ (0.5,-0.5) node[kernel] (A) {$\kernel{A}$}
++(0.7,0) node (Y) {$Y$};
\draw (mu)--(copy0);
\draw[-{Rays [n=8]}] (copy0) to [bend left] (X);
\draw (copy0) to [bend right] (A) (A) -- (Y);
\end{tikzpicture} \\
 &= \begin{tikzpicture}
\path (0,0) node[dist] (mu) {$\mu$} ++ (1,0) node[kernel] (A) {$\kernel{A}$} ++ (0.7,0) node (Y) {$Y$}; \draw (mu) -- (A) -- (Y);
\end{tikzpicture} \label{eq:marginalisation_graph}
\end{align}

\subsection{Random Variables}\label{ssec:random_variables}

The summary of this section is:
\begin{itemize}
\item Random variables are usually defined as measurable functions on a \emph{probability space}
\item It's possible to define them as measurable functions on a \emph{Markov kernel space} instead
\item It is useful to label wires with random variable names instead of names of spaces
\end{itemize}

Probability theory is primarily concerned with the behaviour of \emph{random variables}. This behaviour can be analysed via a collection of probability measures and Markov kernels representing joint, marginal and conditional distributions of random variables of interest. In the framework developed by Kolmogorov, this collection of joint, marginal and conditional distributions is modeled by a single underlying \emph{probability space}, and random variables by measurable functions on the probability space. 

We use the same approach here, with a couple of additions. We are interested in variables whose outcomes depend both on random processes and decisions. Suppose that given a particular distribution over decision variables, a probability distribution over the decision variables and random variables is obtained. Such a model is described by a Markov kernel rather than a probability distribution. We therefore investigate \emph{Markov kernel spaces}.

In the graphical notation that we are using, random variables can be thought of as a means of assigning unambiguous names to each wire in a set of diagrams. In order to do this, it is necessary to suppose that all diagrams in the set describe properties of an \emph{ambient Markov kernel} or \emph{ambient probability measure}. Consider the following example with the ambient probability measure $\mu\in\Delta(\mathcal{X}\otimes\mathcal{X})$. Suppose we have a Markov kernel $\kernel{K}:X\to \Delta(\mathcal{X})$ such that the following holds:

\begin{align}
\begin{tikzpicture}
\path (0,0) node[dist] (m) {$\mu$}
++ (0.7,0.15) node (E) {$X$}
++ (0,-0.3) node (F) {$X$};
\draw ($(m.east) + (0,0.15)$) -- (E);
\draw ($(m.east) + (0,-0.15)$) -- (F);
\end{tikzpicture} = \begin{tikzpicture}
\path (0,0) node[dist] (m) {$\mu$}
++ (0.7,0.15) coordinate (copy0)
+(0,-0.3) node (Fs) {}
++ (1.2,0) node (E) {$X$}
++(-0.7,-0.3) node[kernel] (K) {$\kernel{K}$}
++(0.7,0) node (F) {$X$};
\draw ($(m.east) + (0,0.15)$) -- (E);
\draw (copy0) to [bend right] (K) (K) -- (F);
\draw[-{Rays [n=8]}] ($(m.east) + (0,-0.15)$) -- (Fs);
\end{tikzpicture}\label{eq:disint_example}
\end{align}

Suppose that we also assign the names $\RV{X}_1$ to the upper output wire and $\RV{X}_2$ to the lower output wire in the diagram of $\mu$:

\begin{align}
\begin{tikzpicture}
\path (0,0) node[dist] (m) {$\mu$}
++ (0.7,0.15) node (E) {$\RV{X}_1$}
++ (0,-0.3) node (F) {$\RV{X}_2$};
\draw ($(m.east) + (0,0.15)$) -- (E);
\draw ($(m.east) + (0,-0.15)$) -- (F);
\end{tikzpicture}
\end{align}

Then it seems sensible to call $\kernel{K}$ ``the probability of $\RV{X}_2$ given $\RV{X}_1$''. We will make this precise, and it will match the usual notion of the probability of one variable given another (see \citet{cinlar_probability_2011} for a definition of this usual notion). 

\begin{definition}[Probability space, Markov kernel space]\label{def:kernel_space}
A \emph{Markov kernel space} $(\kernel{K},(D,\mathcal{D}),(\Omega,\mathcal{F}))$ is a Markov kernel $\kernel{K}:D\to \Delta(\mathcal{D}\otimes\mathcal{F})$, called the \emph{ambient kernel}, along with the sample space $(\Omega,\mathcal{F})$ and the domain $(D,\mathcal{D})$. We suppose that $\kernel{K}$ is such that there exists a \emph{fundamental kernel} $\kernel{K}_0$ satisfying

\begin{align}
\prob{K} := \begin{tikzpicture}
\path (0,0) node (O) {}
++(0.5,0) coordinate (copy0)
++ (0.5,0) node[kernel] (m) {$\kernel{K}_0$}
++ (0.7,0.) node (E) {}
++(0,-0.45) node (G) {};
\draw (O) -- (m) -- (E);
\draw (copy0) to [bend right] (G);
\end{tikzpicture}
\end{align}

For brevity, we will omit the $\sigma$-algebras in further definitions of Markov kernel spaces: $(\kernel{K},D,\Omega)$.

A \emph{probability space} $(\prob{P},\Omega,\mathcal{F})$ is a probability measure $\prob{P}:\Delta(\Omega)$, which we call the \emph{ambient measure}, along with the \emph{sample space} $\Omega$ and the \emph{events} $\mathcal{F}$. A probability space is equivalent to a Markov kernel space with domain $D=\{*\}$ - note that $\Omega\times \{*\}\cong \Omega$.
\end{definition}

\begin{definition}[Random variable]\label{def:random_variable}
Given a Markov kernel space $(\kernel{K},D,\Omega)$, a random variable $\RV{X}$ is a measurable function $\Omega\times D\to E$ for arbitrary measurable $E$.
\end{definition}

\begin{definition}[Domain variable]\label{def:domain_variable}
Given a Markov kernel space $(\kernel{K},D,\Omega)$, the \emph{domain variable} $\RV{D}:\Omega\times D\to D$ is the distinguished random variable $\RV{D}:(x,d)\mapsto d$.
\end{definition}

Unlike random variables on probability spaces, random variables on Markov kernel spaces do not generally have unique marginal distributions. An analogous operation of \emph{marginalisation} can be defined, but the result is generally a Markov kernel. We will define marginalisation via coupled tensor products.

\begin{definition}[Coupled tensor product $\utimes$]\label{def:ctensor}
Given two Markov kernels $\kernel{M}$ and $\kernel{N}$ or functions $f$ and $g$ with shared domain $E$, let $\kernel{M}\utimes\kernel{N}:=\splitter{0.1}(\kernel{M}\otimes\kernel{N})$ and $f\utimes g:=\splitter{0.1}(f\otimes g)$ where these expressions are interpreted using standard product notation. Graphically:

\begin{align}
\kernel{M}\utimes\kernel{N}&:=\begin{tikzpicture}
\path (0,0) node (E) {$E$}
++(0.5,0) coordinate (copy0)
+ (0.5,0.3) node[kernel] (M) {$\kernel{M}$}
+(1.2,0.3) node (X) {$\RV{X}$}
+ (0.5,-0.3) node[kernel] (N) {$\kernel{N}$}
+(1.2,-0.3) node (Y) {$\RV{Y}$};
\draw (E) -- (copy0) to [bend left] (M) (copy0) to [bend right] (N);
\draw (M) -- (X) (N) -- (Y);
\end{tikzpicture}\\
f\utimes g&:= \begin{tikzpicture}[scale=1.2]\path (0,0) node (E) {$E$}
++(0.5,0) coordinate (copy0)
+ (0.5,0.3) node[expectation] (M) {$f$}
+ (0.5,-0.3) node[expectation] (N) {$g$};
\draw (E) -- (copy0) to [bend left] (M) (copy0) to [bend right] (N);
\end{tikzpicture}
\end{align}
The operation denoted by $\utimes$ is associative (Lemma \ref{lem:utimes_assoc}), so we can without ambiguity write $f\utimes g\utimes h=(f\utimes g)\utimes h = f\utimes(g\utimes h)$ for finite groups of functions or Markov kernels sharing a domain. 

The notation $\utimes_{i\in [N]} f_i$ is taken to mean $f_1\utimes f_2\utimes ...\utimes f_N$.
\end{definition}

\begin{lemma}[$\utimes$ is associative]\label{lem:utimes_assoc}
For Markov kernels $\kernel{L}:E\to \delta(\mathcal{F})$, $\kernel{M}:E\to \delta(\mathcal{G})$ and $\kernel{N}:E\to \delta(\mathcal{H})$, $(\kernel{L}\utimes\kernel{M})\utimes\kernel{N}=\kernel{L}\utimes(\kernel{M}\utimes\kernel{N})$.
\end{lemma}

\begin{proof}

\begin{align}
	\kernel{L}\utimes(\kernel{M}\utimes\kernel{N}) &= 
	\begin{tikzpicture}[scale=0.8]
	\path (0,0) node (X) {$E$} 
	++ (0.8,0) coordinate (copy0)
	+ (1.5,0.5) node[kernel] (X1) {$\kernel{L}$} + (2.5,0.5) node (F) {$F$}
	++ (0.5,-0.5) coordinate (copy1)
	+(1,0.3) node[kernel] (X2) {$\kernel{M}$} + (2,0.3) node (G) {$G$}
	+(1,-0.5) node[kernel] (X3) {$\kernel{N}$} + (2,-0.5) node (H) {$H$};
	\draw (X) -- (copy0) to [bend left] (X1) (copy0) to [bend right] (copy1) to [bend left] (X2) (copy1) to [bend right] (X3);
	\draw (X1) -- (F) (X2) -- (G) (X3) -- (H);
	\end{tikzpicture}\\
	&=
	\begin{tikzpicture}[scale=0.8]
	\path (0,0) node (X) {$E$} 
	++ (0.8,0) coordinate (copy0)
	+ (1.5,0.7) node[kernel] (X1) {$\kernel{L}$} + (2.5,0.7) node (F) {$F$}
	+ (0.5,0.3) coordinate (copy1)
	++ (0.5,-0.5) coordinate (next)
	+(1,0.5) node[kernel] (X2) {$\kernel{M}$} + (2,0.5) node (G) {$G$}
	+(1,-0.5) node[kernel] (X3) {$\kernel{N}$} + (2,-0.5) node (H) {$H$};
	\draw (X) -- (copy0) to [bend left] (copy1) (copy0) to [bend right] (X3);
	\draw (copy1) to [bend left] (X1) (copy1) to [bend right] (X2);
	\draw (X1) -- (F) (X2) -- (G) (X3) -- (H);
	\end{tikzpicture}\\
	&= (\kernel{L}\utimes\kernel{M})\utimes\kernel{N}
\end{align}
This follows directly from Equation \ref{eq:ccom1}.
\end{proof}

\begin{definition}[Marginal distribution, marginal kernel]\label{def:marginal_distribution}
Given a probability space $(\prob{P},\Omega,\mathcal{F})$ and the random variable $\RV{X}:\Omega\to G$ the \emph{marginal distribution} of $\RV{X}$ is the probability measure $\prob{P}^{\RV{X}}:= \prob{P}\kernel{F}^{\RV{X}}$.

See Lemma \ref{lem:pushf_funk} for the proof that this matches the usual definition of marginal distribution.

Given a Markov kernel space $(\kernel{K},\Omega,\mathcal{F},D,\mathcal{D})$ and the random variable $\RV{X}:\Omega\to G$, the \emph{marginal kernel} is $\kernel{K}^{\RV{X}|\RV{D}}:=\kernel{K}\kernel{F}^{\RV{X}}$.
\end{definition}

\begin{definition}[Joint distribution, joint kernel]\label{def:joint_distribution}
Given a probability space $(\prob{P},\Omega,\mathcal{F})$ and the random variables $\RV{X}:\Omega\to G$ and $\RV{Y}:\Omega\to H$, the \emph{joint distribution} of $\RV{X}$ and $\RV{Y}$, $\prob{P}^{\RV{X}\RV{Y}}\in \Delta(\mathcal{G}\otimes\mathcal{H})$, is the marginal distribution of $\RV{X}\utimes\RV{Y}$. That is, $\prob{P}^{\RV{X}\RV{Y}}:=\prob{P} \kernel{F}^{\RV{X}\utimes\RV{Y}}$

This is identical to the definition in \citet{cinlar_probability_2011} if we note that the random variable $(\RV{X},\RV{Y}):\omega\mapsto (\RV{X}(\omega),\RV{Y}(\omega))$ (\c{C}inlar's definition) is precisely the same thing as $\RV{X}\utimes\RV{Y}$.

Analogously, the joint kernel $\kernel{K}^{\RV{X}\RV{Y}|\RV{D}}$ is the product $\kernel{K}\kernel{F}^{\RV{X}\utimes\RV{Y}}$.
\end{definition}

Joint distributions and kernels have a nice visual representation, as a result of Lemma \ref{lem:jdist_cprod} which follows.

\begin{lemma}[Product marginalisation interchange]\label{lem:jdist_cprod}
Given two functions, the kernel associated with their coupled product is equal to the coupled product of the kernels associated with each function.

Given $\RV{X}:\Omega\to G$ and $\RV{Y}:\Omega\to H$, $\kernel{F}^{\RV{X}\utimes\RV{Y}}=\kernel{F}^\RV{X}\utimes\kernel{F}^\RV{Y}$
\end{lemma}

\begin{proof}
For $a\in \Omega$, $B\in \mathcal{G}$, $C\in \mathcal{H}$,
\begin{align}
\kernel{F}^{\RV{X}\utimes\RV{Y}} (a;B\times C) &= \delta_{\RV{X}(a),\RV{Y}(a)}(B\times C)\\
									   &= \delta_{\RV{X}(a)}(B)\delta_{\RV{Y}(a)}(C)\\
									   &= (\delta_{\RV{X}(a)}\otimes\delta_{\RV{Y}(a)})(B\times C)\\
									   &= \kernel{F}^{\RV{X}}\utimes\kernel{F}^{\RV{Y}}
\end{align}
Equality follows from the monotone class theorem.
\end{proof}

\begin{corollary}\label{corr:rewrite_joint_dist}
Given a Markov kernel space $(\kernel{K}, \Omega, D)$ and random variables $\RV{X}:\Omega\times D\to X$, $\RV{Y}:\Omega\times D\to Y$, the following holds:

\begin{align}
\begin{tikzpicture}
\path (0,0) node (O) {$D$}
++(1,0) node[kernel] (K) {$\kernel{K}^{\RV{X}\RV{Y}|\RV{D}}$}
++ (1,0.15) node (X) {$X$}
+(0,-0.3) node (Y) {$Y$};
\draw (O) -- (K);
\draw ($(K.east) + (0,0.15)$) -- (X);
\draw ($(K.east) + (0,-0.15)$) -- (Y);
\end{tikzpicture}=
\begin{tikzpicture}
\path (0,0) node (O) {$D$}
++ (0.7, 0) node[kernel] (K) {$\kernel{K}$}
++ (0.6,0) coordinate (copy0)
++ (0.4,0.25) node[kernel] (X) {$\kernel{F}^{\RV{X}}$}
+(0,-0.5) node[kernel] (Y) {$\kernel{F}^{\RV{Y}}$}
++(0.7,0) node (Xo) {$X$}
+(0,-0.5) node (Yo) {$Y$};
\draw (O) -- (K) -- (copy0);
\draw (copy0) to [bend left] (X) (X) -- (Xo);
\draw (copy0) to [bend right] (Y) (Y) -- (Yo);
\end{tikzpicture}
\end{align}
\end{corollary}

We will now define wire labels for ``output'' wires.

\begin{definition}[Wire labels - joint kernels]\label{def:wl_jprob}
Suppose we have a Markov kernel space $(\kernel{K},D,\Omega)$, random variables $\RV{X}:\Omega\times D\to X$, $\RV{Y}:\Omega\times D\to Y$ and a Markov kernel $\kernel{L}:D\to \Delta(\mathcal{X}\times\mathcal{Y})$. The following \emph{output labelling} of $\mathbf{L}$:

\begin{align}
\begin{tikzpicture}
\path (0,0) node (A) {$D$}
++ (0.7,0) node[kernel] (m) {$\kernel{L}$}
++ (0.7,0.15) node (E) {\color{blue}$\RV{X}$}
++ (0,-0.3) node (F) {\color{blue}$\RV{Y}$};
\draw (A) -- (m);
\draw ($(m.east) + (0,0.15)$) -- (E);
\draw ($(m.east) + (0,-0.15)$) -- (F);
\end{tikzpicture}
\end{align}

is \emph{valid} iff

\begin{align}
\kernel{L} = \kernel{K}_{\RV{X}\RV{Y}|\RV{D}}\label{eq:labels_express_joint}
\end{align}

and

\begin{align}
\begin{tikzpicture}
\path (0,0) node (A) {$D$}
++ (1,0) node[kernel] (m) {$\kernel{L}$}
++ (1,0.15) node (E) {\color{blue}$\RV{X}$}
++ (0,-0.3) node (F) {};
\draw (A) -- (m) ($(m.east) + (0,0.15)$) -- (E);
\draw[-{Rays [n=8]}] ($(m.east) + (0,-0.15)$) -- (F);
\end{tikzpicture} = \kernel{K}^{\RV{X}|\RV{D}}\label{eq:labels_express_marginal_upper}
\end{align}

and

\begin{align}
\begin{tikzpicture}
\path (0,0) node (A) {$D$}
++ (1,0) node[kernel] (m) {$\kernel{L}$}
++ (1,0.15) node (E) {}
++ (0,-0.3) node (F) {\color{blue}$\RV{Y}$};
\draw (A) -- (m);
\draw[-{Rays [n=8]}] ($(m.east) + (0,0.15)$) -- (E);
\draw ($(m.east) + (0,-0.15)$) -- (F);
\end{tikzpicture} = \kernel{K}^{\RV{Y}|\RV{D}}\label{eq:labels_express_marginal_lower}
\end{align}

The second and third conditions are nontrivial: suppose $\RV{X}$ takes values in some product space $Range(\RV{X}) = W\times Z$, and $\RV{Y}$ takes values in $Y$. Then we could have $\kernel{L}=\kernel{K}^{\RV{X}\RV{Y}|\RV{D}}$ and draw the diagram

\begin{align}
\begin{tikzpicture}
\path (0,0) node (A) {$D$}
++ (0.7,0) node[kernel] (m) {$\kernel{L}$}
++ (1,0.15) node (E) {$W$}
++ (0.3,-0.3) node (F) {$Z\times Y$};
\draw (A) -- (m);
\draw ($(m.east) + (0,0.15)$) -- (E);
\draw ($(m.east) + (0,-0.15)$) -- (F);
\end{tikzpicture}\label{eq:cannot_marginalise}
\end{align}

For \emph{this} diagram, properties \ref{eq:labels_express_marginal_upper} and \ref{eq:labels_express_marginal_lower} do not hold, even though \ref{eq:labels_express_joint} does.

\end{definition}

\begin{lemma}[Output label assignments exist]
Given Markov kernel space $(\kernel{K},D,\Omega)$, random variables $\RV{X}:\Omega\times D\to X$ and $\RV{Y}:\Omega\times D\to Y$ then there exists a diagram of $\kernel{L}:=\kernel{K}^{\RV{X}\RV{Y}|\RV{D}}$ with a valid output labelling assigning ${\color{blue}\RV{X}}$ and ${\color{blue}\RV{Y}}$ to the output wires.
\end{lemma}

\begin{proof}
By definition, $\kernel{L}$ has signature $D\to \Delta(\mathcal{X}\otimes\mathcal{Y})$. Thus, by the rule that tensor product spaces can be represented by parallel wires, we can draw

\begin{align}
\begin{tikzpicture}
\path (0,0) node (A) {$D$}
++ (0.7,0) node[kernel] (m) {$\kernel{L}$}
++ (0.7,0.15) node (E) {$X$}
++ (0,-0.3) node (F) {$Y$};
\draw (A) -- (m);
\draw ($(m.east) + (0,0.15)$) -- (E);
\draw ($(m.east) + (0,-0.15)$) -- (F);
\end{tikzpicture}
\end{align}

By Corollary \ref{corr:rewrite_joint_dist}, we have

\begin{align}
\begin{tikzpicture}
\path (0,0) node (A) {$D$}
++ (0.7,0) node[kernel] (m) {$\kernel{L}$}
++ (0.7,0.15) node (E) {$X$}
++ (0,-0.3) node (F) {$Y$};
\draw (A) -- (m);
\draw ($(m.east) + (0,0.15)$) -- (E);
\draw ($(m.east) + (0,-0.15)$) -- (F);
\end{tikzpicture} = \begin{tikzpicture}
\path (0,0) node (O) {$D$}
++ (0.7, 0) node[kernel] (K) {$\kernel{K}$}
++ (0.6,0) coordinate (copy0)
++ (0.4,0.25) node[kernel] (X) {$\kernel{F}^{\RV{X}}$}
+(0,-0.5) node[kernel] (Y) {$\kernel{F}^{\RV{Y}}$}
++(0.7,0) node (Xo) {$X$}
+(0,-0.5) node (Yo) {$Y$};
\draw (O) -- (K) -- (copy0);
\draw (copy0) to [bend left] (X) (X) -- (Xo);
\draw (copy0) to [bend right] (Y) (Y) -- (Yo);
\end{tikzpicture}
\end{align}

Therefore 

\begin{align}
\begin{tikzpicture}
\path (0,0) node (O) {$D$}
++ (0.7, 0) node[kernel] (K) {$\kernel{K}$}
++ (0.6,0) coordinate (copy0)
++ (0.4,0.25) node[kernel] (X) {$\kernel{F}^{\RV{X}}$}
+(0,-0.5) node[kernel] (Y) {$\kernel{F}^{\RV{Y}}$}
++(0.7,0) node (Xo) {$X$}
+(0,-0.5) node (Yo) {};
\draw (O) -- (K) -- (copy0);
\draw (copy0) to [bend left] (X) (X) -- (Xo);
\draw[-{Rays[n=8]}] (copy0) to [bend right] (Y) (Y) -- (Yo);
\end{tikzpicture} &= \kernel{K}\kernel{F}^{\RV{X}}\\
				 &= \kernel{K}^{\RV{X}|\RV{D}}
\end{align}

\begin{align}
\begin{tikzpicture}
\path (0,0) node (O) {$D$}
++ (0.7, 0) node[kernel] (K) {$\kernel{K}$}
++ (0.6,0) coordinate (copy0)
++ (0.4,0.25) node[kernel] (X) {$\kernel{F}^{\RV{X}}$}
+(0,-0.5) node[kernel] (Y) {$\kernel{F}^{\RV{Y}}$}
++(0.7,0) node (Xo) {}
+(0,-0.5) node (Yo) {$Y$};
\draw (O) -- (K) -- (copy0);
\draw[-{Rays[n=8]}] (copy0) to [bend left] (X) (X) -- (Xo);
\draw (copy0) to [bend right] (Y) (Y) -- (Yo);
\end{tikzpicture} &= \kernel{K}\kernel{F}^{\RV{Y}}\\
				 &= \kernel{K}^{\RV{Y}|\RV{D}}
\end{align}
\end{proof}

In all further work, wire labels will be used without special colouring.

\begin{definition}[Disintegration]\label{def:disintegration}
Given a probability space $(\prob{P},\Omega,\mathcal{F})$, and random variables $\RV{X}$ and $\RV{Y}$, we say that $\kernel{M}:E\to \Delta(\mathcal{F})$ is a \emph{$\RV{Y}$ given $\RV{X}$ disintegration} of $\prob{P}$ iff
\begin{align}
\begin{tikzpicture}
\path (0,0) node[dist] (m) {$\prob{P}^{\RV{X}\RV{Y}}$}
++ (1,0.15) node (E) {$\RV{X}$}
++ (0,-0.3) node (F) {$\RV{Y}$};
\draw ($(m.east) + (0,0.15)$) -- (E);
\draw ($(m.east) + (0,-0.15)$) -- (F);
\end{tikzpicture} = \begin{tikzpicture}
\path (0,0) node[dist] (m) {$\prob{P}^{\RV{X}}$}
++ (0.7,0.15) coordinate (copy0)
+(0.2,-0.3) node (T) {}
++ (1.2,0) node (E) {$\RV{X}$}
++(-0.7,-0.3) node[kernel] (K) {$\kernel{M}$}
++(0.7,0) node (F) {$\RV{Y}$};
\draw ($(m.east) + (0,0.15)$) -- (E);
\draw (copy0) to [bend right] (K) (K) -- (F);
\draw[-{Rays [n=8]}] ($(m.east) + (0,-0.15)$) -- (T);
\end{tikzpicture}\label{eq:ordinary_disint}
\end{align}
$\kernel{M}$ is a version of $\prob{P}^{\RV{Y}|\RV{X}}$, ``the probability of $\RV{Y}$ given $\RV{X}$''. Let $\prob{P}^{\{\RV{Y}|\RV{X}\}}$ be the set of all kernels that satisfy \ref{eq:ordinary_disint} and $\prob{P}^{\RV{Y}|\RV{X}}$ an arbitrary member of $\prob{P}^{\RV{Y}|\RV{X}}$.

Given a Markov kernel space $(\kernel{K},D,\Omega)$ and random variables $\RV{X}:\Omega\times D\to X$, $\RV{Y}:\Omega\times D\to Y$, $\kernel{M}:D\times E\to \Delta(\mathcal{F})$ is a \emph{$\RV{Y}$ given $\RV{DX}$ disintegration} of $\kernel{K}^{\RV{YX}|\RV{D}}$ iff

\begin{align}
\begin{tikzpicture}
\path (0,0) node (O) {}
++ (1,0) node[kernel] (m) {$\kernel{K}^{\RV{YX}|\RV{D}}$}
++ (1,0.15) node (E) {$\RV{X}$}
++ (0,-0.3) node (F) {$\RV{Y}$};
\draw (O) -- (m) ($(m.east) + (0,0.15)$) -- (E);
\draw ($(m.east) + (0,-0.15)$) -- (F);
\end{tikzpicture} = \begin{tikzpicture}
\path (0,0) node (O) {}
++ (0.3,0) coordinate (copy1)
++ (1,0) node[kernel] (m) {$\kernel{K}^{\RV{YX}|\RV{D}}$}
++ (1,0.15) coordinate (copy0)
+(0.2,-0.3) node (T) {}
++ (1.2,0) node (E) {$\RV{X}$}
++(-0.7,-0.3) node[kernel] (K) {$\kernel{M}$}
++(0.7,0) node (F) {$\RV{Y}$};
\draw (O) -- (m) ($(m.east) + (0,0.15)$) -- (E);
\draw (copy0) to [bend right] (K) (K) -- (F);
\draw (copy1) to [out=290,in=180] ($(K.west) + (0,-0.15)$);
\draw[-{Rays [n=8]}] ($(m.east) + (0,-0.15)$) -- (T);
\end{tikzpicture}\label{eq:def_k_disint}
\end{align}

Write $\kernel{K}^{\{\RV{Y}|\RV{XD}\}}$ for the set of kernels satisfying \ref{eq:def_k_disint} and $\kernel{K}^{\RV{Y}|\RV{XD}}$ for an arbitrary member of $\kernel{K}^{\{\RV{Y}|\RV{XD}\}}$.
\end{definition}

\begin{definition}[Wire labels -- input]\label{def:wl_disint}

An input wire is \emph{connected} to an output wire if it is possible to trace a path from the start of the input wire to the end of the output wire without passing through any boxes, erase maps or right facing triangles.

If an input wire is connected to an output wire and that output wire has a valid label $\RV{X}$, then it is valid to label the input wire with $\RV{X}$.

For example, if the following are valid output labels with respect to $(\prob{P},\Omega)$:

\begin{align}
\begin{tikzpicture}
\path (0,0) node (A) {}
++ (0.7,0) coordinate (copy0)
++ (0.7,0) node[kernel] (m) {$\kernel{L}$}
++ (0.7,0) node (E) {\color{blue}$\RV{X}$}
++ (0,-0.3) node (F) {\color{blue}$\RV{Y}$};
\draw (A) -- (m) -- (E);
\draw (copy0) to [out=-60,in=180] (F);
\end{tikzpicture}\label{dia:kernel_l}
\end{align}

i.e. if $\kernel{L}\in \prob{P}^{\{\RV{X}\RV{Y}|\RV{Y}\}}$, then the following is a valid input label:


\begin{align}
\begin{tikzpicture}
\path (0,0) node (A) {\color{blue}$\RV{Y}$}
++ (0.7,0) coordinate (copy0)
++ (0.7,0) node[kernel] (m) {$\kernel{L}$}
++ (0.7,0) node (E) {\color{blue}$\RV{X}$}
++ (0,-0.3) node (F) {\color{blue}$\RV{Y}$};
\draw (A) -- (m) -- (E);
\draw (copy0) to [out=-60,in=180] (F);
\end{tikzpicture}
\end{align}

An input wire in a diagram for $\kernel{M}$ may be labeled $\RV{X}$ \emph{if and only if} copy and identity maps can be inserted to yield a diagram in which the input wire labeled $\RV{X}$ is connected to an output wire with valid label $\RV{X}$.

So, if $\kernel{M}\in \prob{P}^{\{\RV{X}|\RV{Y}\}}$, then it is straightforward to show that

\begin{align}
\begin{tikzpicture}
\path (0,0) node (A) {}
++ (0.7,0) coordinate (copy0)
++ (0.7,0) node[kernel] (m) {$\kernel{M}$}
++ (0.7,0) node (E) {\color{blue}$\RV{X}$}
++ (0,-0.3) node (F) {\color{blue}$\RV{Y}$};
\draw (A) -- (m) -- (E);
\draw (copy0) to [out=-60,in=180] (F);
\end{tikzpicture} \in \prob{P}^{\{\RV{X}\RV{Y}|\RV{Y}\}} \label{eq:const_from_m}
\end{align}

and hence the output labels are valid. Diagram \ref{eq:const_from_m} is constructed by taking the product of the copy map with $\kernel{M}\otimes\textbf{Id}$. Thus it is valid to label $\kernel{M}$ with

\begin{align}
\begin{tikzpicture}
\path (0,0) node (A) {\color{blue}$\RV{Y}$}
++ (0.7,0) node[kernel] (m) {$\kernel{M}$}
++ (0.7,0) node (E) {\color{blue}$\RV{X}$};
\draw (A) -- (m) -- (E);
\end{tikzpicture}
\end{align}
\end{definition}

\begin{lemma}[Labeling of disintegrations]
Given a kernel space $(\kernel{K},D,\Omega)$, random variables $\RV{X}$ and $\RV{Y}$, domain variable $\RV{D}$ and disintegration $\kernel{L}\in \kernel{K}^{\{\RV{Y}|\RV{X}\RV{D}\}}$, there is a diagram of $\kernel{L}$ with valid input labels ${\color{blue} \RV{X}}$ and ${\color{blue} \RV{D}}$ and valid output label ${\color{blue} \RV{Y}}$.
\end{lemma}

\begin{proof}
Note that for any variable $\RV{W}:\Omega\times D\to W$ and the domain variable $\RV{D}:\Omega\times D\to D$ we have by definition of $\kernel{K}$:
\begin{align}
\begin{tikzpicture}
\path (0,0) node (O) {}
++ (1,0) node[kernel] (m) {$\kernel{K}^{\RV{WD}|\RV{D}}$}
++ (1,0.15) node (E) {$\RV{W}$}
++ (0,-0.3) node (F) {$\RV{D}$};
\draw (O) -- (m) ($(m.east) + (0,0.15)$) -- (E);
\draw ($(m.east) + (0,-0.15)$) -- (F);
\end{tikzpicture} &= \begin{tikzpicture}
\path (0,0) node (O) {}
++ (0.3,0) coordinate (copy1)
++ (1,0) node[kernel] (m) {$\kernel{K}_{0}$}
++ (0.7,0) coordinate (copy0)
+ (0,-0.5) coordinate (copy2)
++ (0.7,0.3) node[kernel] (Fx) {$\kernel{F}^\RV{W}$}
++(0,-0.8) node[kernel] (Fd) {$\kernel{F}^{\RV{D}}$}
++(0.7,0) node (D) {$\RV{D}$}
++ (0,0.8) node (X) {$\RV{W}$};
\draw (O) -- (m) -- (copy0);
\draw (copy0) to [bend left] ($(Fx.west)+(0,0.1)$) (copy0) to [bend right] ($(Fd.west)+(0,0.1)$);
\draw (copy1) to [out=290,in=180] (copy2);
\draw (copy2) to [bend left] ($(Fx.west)+(0,-0.1)$) (copy2) to [bend right] ($(Fd.west)+(0,-0.1)$);
\draw (Fx) -- (X) (Fd) -- (D);
\end{tikzpicture}\\
&= \begin{tikzpicture}
\path (0,0) node (O) {}
++ (0.3,0) coordinate (copy1)
++ (1,0) node[kernel] (m) {$\kernel{K}_{0}$}
++ (0.7,0) coordinate (copy0)
+ (0,-0.5) coordinate (copy2)
++ (0.7,0.3) node[kernel] (Fx) {$\kernel{F}^\RV{W}$}
++(0,-0.8) coordinate (Fd)
++(0.7,0) node (D) {$\RV{D}$}
++ (0,0.8) node (X) {$\RV{W}$};
\draw (O) -- (m) -- (copy0);
\draw (copy0) to [bend left] ($(Fx.west)+(0,0.1)$);
\draw (copy1) to [out=290,in=180] (copy2) -- (D);
\draw (copy2) to [bend left] ($(Fx.west)+(0,-0.1)$);
\draw (Fx) -- (X) (Fd) -- (D);
\end{tikzpicture}\\
&= \begin{tikzpicture}
\path (0,0) node (O) {}
++ (0.3,0) coordinate (copy1)
+ (0.2,0) coordinate (copy3)
++ (1,0) node[kernel] (m) {$\kernel{K}_{0}$}
++ (0.7,0) coordinate (copy0)
+ (0,-0.5) coordinate (copy2)
++ (0.7,-0.1) node[kernel] (Fx) {$\kernel{F}^\RV{W}$}
++(0,-0.5) coordinate (Fd)
++(0.7,0) node (D) {$\RV{D}$}
++ (0,0.5) node (X) {$\RV{W}$};
\draw (O) -- (m);
\draw (m) to [out=0,in=180]  ($(Fx.west)+(0,0.1)$);
\draw (copy1) to [out=290,in=180] (D);
\draw (copy3) to [out=290,in=180] ($(Fx.west)+(0,-0.1)$);
\draw (Fx) -- (X);
\end{tikzpicture}\\
&= \begin{tikzpicture}
\path (0,0) node (O) {}
++ (0.3,0) coordinate (copy1)
++ (1,0) node[kernel] (m) {$\kernel{K}$}
++ (0.7,0) coordinate (copy0)
+ (0,-0.5) coordinate (copy2)
++ (0.7,-0.) node[kernel] (Fx) {$\kernel{F}^\RV{W}$}
++(0,-0.5) coordinate (Fd)
++(0.7,0) node (D) {$\RV{D}$}
++ (0,0.5) node (X) {$\RV{W}$};
\draw (O) -- (m);
\draw (m) to [out=0,in=180]  ($(Fx.west)+(0,0.0)$);
\draw (copy1) to [out=290,in=180] (D);
\draw (Fx) -- (X);
\end{tikzpicture}\\
&=\begin{tikzpicture}
\path (0,0) node (O) {}
++ (0.3,0) coordinate (copy1)
++ (1,0) node[kernel] (m) {$\kernel{K}^{\RV{W}|\RV{D}}$}
++(0,-0.5) coordinate (Fd)
++(1,0) node (D) {$\RV{D}$}
++ (0,0.5) node (X) {$\RV{W}$};
\draw (O) -- (m) -- (X);
\draw (copy1) to [out=290,in=180] (D);
\end{tikzpicture}
\end{align}
\end{proof}

We use the informal convention of labelling wires in quote marks $``\RV{X}''$ if that wire is ``supposed to'' carry the label $\RV{X}$ but the label may not be valid.

\begin{theorem}[Iterated disintegration]\label{th:iterated_disint}
Given a kernel space $(\kernel{K},D,\Omega)$, random variables $\RV{X}$, $\RV{Y}$ and $\RV{Z}$ and domain variable $\RV{D}$,
\begin{align}
\begin{tikzpicture}
	\path (0,0.15) node (D) {$``\RV{D}''$}
	+ (0,-0.3) node (X) {$``\RV{X}''$}
	++ (.7,-0.15) coordinate (copy0)
	++ (.7,0) node[kernel] (Yxd) {$\kernel{K}^{\RV{Y}|\RV{XD}}$}
	++ (0.7,0) coordinate (copy1)
	++(1.5,0) node[kernel] (Zxyd) {$\kernel{K}^{\RV{Z}|\RV{XYD}}$}
	++(1.5,0) node (Z) {$``\RV{Z}''$}
	+(0,-0.4) node (Y) {$``\RV{Y}''$};
	\draw (D) -- ($(Yxd.west) + (0,0.15)$) (X) -- ($(Yxd.west) + (0,-0.15)$);
	\draw (Yxd) -- (Zxyd);
	\draw ($(copy0) + (0,0.15)$) to [out=90,in=180] ($(Zxyd.west)+(0,0.15)$);
	\draw ($(copy0) + (0,-0.15)$) to [out=-90,in = 180] ($(Zxyd.west)+(0,-0.15)$);
	\draw (copy1) to [out=-90,in=180] (Y) (Zxyd) -- (Z);
\end{tikzpicture}\in \kernel{K}^{\{\RV{ZY}|\RV{XD}\}}
\end{align}

Equivalently, for $d\in D$ and $x\in X$, $A\in \sigalg{Y}$, $B\in\sigalg{Z}$,

\begin{align}
	(d,x;A,B)\mapsto \int_A \kernel{K}^{\RV{Z}|\RV{XYD}}_{(x,y,d)}(B) d\kernel{K}^{\RV{Y}|\RV{XD}}_{(x,d)}(y) \in \kernel{K}^{\{\RV{ZY}|\RV{XD}\}}
\end{align}
\end{theorem}

\begin{proof}
\todo[inline]{write this up}

\end{proof}

The existence of disintegrations of standard measurable probability spaces is well known.

\begin{theorem}[Disintegration existence - probability space]\label{th:disintegration_exist}
Given a probability measure $\prob{P}\in \Delta(\mathcal{X}\otimes \mathcal{Y})$, if $(F,\mathcal{F})$ is standard then a disintegration $\prob{P}^{\RV{Y}|\RV{X}}:X\to \Delta(\mathcal{Y})$ exists \citep{cinlar_probability_2011}.
\end{theorem}

In particular, if for all $x\in X$, $\prob{P}^{\RV{X}}(\RV{X}\in\{x\})>0$, then $\prob{P}^{\RV{Y}|\RV{X}}_x(A) = \frac{\prob{P}^{\RV{X}\RV{Y}}(\{x\}\times A)}{\prob{P}^{\RV{X}}(\{x\})}$.

For Markov kernel spaces, standard measurability is not known to guarantee that a disintegration exists. Given a kernel space $(\kernel{K},(D,\sigalg{D}),(\Omega,\sigalg{E}))$ with $D=[0,1]$ and $\Omega=[0,1]^2$, both Borel, let $\RV{X},\RV{Y},\RV{D}:\Omega\times D\to [0,1]$ project the first, second and third dimensions of $\Omega\times D$ respectively. Let $\kernel{K}_d(A) = \lambda(A)$, the Lebesgue measure of $A\in \sigalg{E}$ on $[0,1]^2$ for all $d\in D$. By Theorem \ref{th:disintegration_exist}, we have for each $d\in D$ a disintegration $Q(d):=(\kernel{K}_d)^{\RV{Y}|\RV{X}}$ of $(\kernel{K}_d)^{\RV{X}\RV{Y}}$, and it is fairly straightforward to show it must be the case that $Qd_x(A)=\lambda(A)$ for all $A\in \sigalg{B}([0,1])$ and $\lambda$-almost all $x\in [0,1]$. $Q(d)_x$ is a probability measure for every $(d,x)\in [0,1]^2$ because it is a disintegration, but $Q:D\times X\to \Delta(\sigalg{Y})$ given by $(d,x,A)\mapsto Q(d)_x(A)$ may fail to be a Markov kernel. Let $I:[0,1]\to \{0,1\}$ be the indicator function on a non-measurable set $C$, and define

\begin{align}
	Q(d)_{x}(A) = (1-I(d)\delta_{d}(\{x\}))\lambda(A) + I(d)\delta_{d}(\{x\})\delta_{0}(A)\label{eq:non_measurable_disint}
\end{align}

That is, $Q$ is the measure $\lambda{A}$ for all points $(x,d)$ except where $x=d$ and $d\in C$. Note that for each value of $d$, $Q$ differs from $\lambda(A)$ on at most a single point $x\in[0,1]$, which has measure $0$ under the Lebesgue measure $\lambda$. Thus $Q(d)$ so defined is indeed a disintegration in (\kernel{K}_d)^{\{\RV{Y}|\RV{X}\}}. Consider the function

\begin{align}
	Q^{\{0\}}:(d,x)&\mapsto Q(d)_x(\{0\})\\
	Q^{\{0\}-1}(\{1\})&=\{(d,x):Q(d)_x(\{0\})=1\}\\
	&= \{(d,x):d=x\And d\in C\}
\end{align}

Thus $Q^{\{0\}}$ is not measurable and consequently $Q$ fails to be a Markov kernel. The problem comes from the fact that $Q$ is defined by an uncountable collection of disintegrations $Q(d)$, each of which is individually measurable. In this case, the problem can be easily solved by defining $Q'$ without the non-measurable component in \ref{eq:non_measurable_disint}. What we would like are general conditions under which we know that we can choose an appropriate set of disintegrations $Q(d)$ in order for the resulting $Q$ to be a Markov kernel.

The following two theorems establish two separate sufficient conditions for the existence of disintegrations in a Markov kernel space. In the work that follows I will exclusively assume the first condition -- that all probability measures in the range of any kernel space under consideration are combinations of point measures and measures absolutely continuous with respect to the Lebesgue measure -- because I think it is unproblematic in this context to exclude continuous singular measures like the Cantor distribution. The second condition is included for completeness.

\begin{theorem}[Existence of disintegrations on kernel spaces: point and lebesgue continuous measures]
Given a kernel space $(\kernel{K},(D,\sigalg{D}),(\Omega,\sigalg{E}))$ with $D$, $\Omega$ standard measurable, let $f:\Omega\times D\to [0,1]$ be an isomorphism (such an isomorphism exists for all standard measurable spaces \citep{cinlar_probability_2011}. If for all $d\in D$, $\kernel{K}_d\kernel{F}_F = \mu_d^\lambda+\mu_d^p$ where $\mu_d^\lambda$ is absolutely continuous with respect to the Lebesgue measure and $\mu_d^p$ is a point measure, then for any $\RV{X},\RV{Y}\in \sigalg{E}\otimes\sigalg{D}$ and domain variable $\RV{D}:\Omega\times D\mapsto D$ a disintegration $\kernel{K}^{\RV{Y}|\RV{X}\RV{D}}$ exists.
\end{theorem}

\begin{proof}
We have a kernel space $(\kernel{K},(D),(\Omega),\sigalg{D}\otimes\sigalg{E}))$ with $D$, $\Omega$ standard measurable, an isomorphism $f:\Omega\times D\to [0,1]$ and for all $d\in D$, $\kernel{K}_d\kernel{F}_f = \mu_d^\lambda+\mu_d^p$ where $\mu_d^\lambda$ is absolutely continuous with respect to the Lebesgue measure and $\mu_d^p$ is a point measure. Let $\RV{D}:\Omega\times D \to D$ be a domain variable and $\RV{X}:\Omega\times D\to \mathbb{R}$ be a random variable. We will work in the image space $(\kernel{K}\kernel{F}_f,f(D),f(\Omega),\sigma(f))$ with random variables $\RV{D}f:=\RV{D}\circ f^{-1}:[0,1]\to D$ and $\RV{X}f:=\RV{X}\circ f^{-1}:[0,1]\to \mathbb{R}$. Let $\sigma(\RV{X}f):=\sigalg{F}$, and let $\RV{Y}:=f^{-1}$.

$\sigalg{F}$ is a sub-$\sigma$-algebra of $\sigalg{B}([0,1])$. 

\begin{align}
	\sigalg{F}_n &= \sigma(H_1,...,H_n)\\
	\sigalg{F} = \cup_{n\in\mathbb{N}}} \sigalg{F}_n
\end{align}

Each $\sigalg{F}_n$ has a finite partition $\sigalg{G}_n$. 

Therefore there exists 
\end{proof}

\begin{theorem}[Existence of disintegrations on kernel spaces: denumerable range]
Given a kernel space $(\kernel{K},(D,\sigalg{D}),(\Omega,\sigalg{E}))$ with $D$ denumerable and $\Omega$ standard measurable, then for any $\RV{X},\RV{Y}\in \sigalg{E}\otimes\sigalg{D}$ and domain variable $\RV{D}:\Omega\times D\mapsto D$ a disintegration $\kernel{K}^{\RV{Y}|\RV{X}\RV{D}}$ exists.
\end{theorem}


\begin{definition}[Relative probability space]

\todo[inline]{better name}

Given a Markov kernel space $(\kernel{K},D,\Omega)$ and a positive definite measure $\mu\in \Delta(\mathcal{D})$, $(\mu\kernel{K},\Omega\times D)$ is a \emph{relative} probability space.

For any random variable $\RV{X}:\Omega\times D\to X$ on $(\kernel{K},D,\Omega)$, its relative on $(\mu\kernel{K},\Omega\times D)$ is given by the same measurable function, and we give it the same name $\RV{X}$.
\end{definition}


\begin{lemma}[Agreement of disintegrations]\label{lem:agree_disint}
Given a Markov kernel space $(\kernel{K},D,\Omega)$, any relative probability space $(\mu\prob{K},\Omega\times D)$ and any random variables $\RV{X}:\Omega\times D\to X$, $\RV{Y}:\Omega\times D\to Y$, $\kernel{K}^{\{\RV{Y}|\RV{X}\RV{D}\}}=(\mu\prob{K})^{\{\RV{Y}|\RV{X}\RV{D}\}}$ (note that this set equality).
\end{lemma}

\begin{proof}
Define $\prob{P}:=\mu\kernel{K}$ and let $\kernel{M}$ be an arbitrary version of $\kernel{K}^{\{\RV{Y}|\RV{X}\RV{D}\}}$. Then
\begin{align}
\begin{tikzpicture}
\path (0,0) node[dist,inner sep=0 pt] (m) {$\prob{P}^{\RV{X}\RV{Y}\RV{D}}$}
++ (1,0.3) node (E) {$\RV{X}$}
++ (0,-0.3) node (F) {$\RV{Y}$}
++ (0,-0.3) node (D) {$\RV{D}$};
\draw ($(m.east) + (0,0.3)$) -- (E);
\draw ($(m.east) + (0,0)$) -- (F);
\draw ($(m.east) + (0,-0.3)$) -- (D);
\end{tikzpicture} &= \begin{tikzpicture}
\path (0,0) node[dist] (O) {$\mu$}
+ (0.75,0) coordinate (copy0)
++ (1.5,0) node[kernel] (m) {$\kernel{K}^{\RV{XY}|\RV{D}}$}
++ (1,0.15) node (E) {$\RV{X}$}
++ (0,-0.3) node (F) {$\RV{Y}$}
++ (0,-0.3) node (D) {$\RV{D}$};
\draw (O) -- (m) ($(m.east) + (0,0.15)$) -- (E);
\draw ($(m.east) + (0,-0.15)$) -- (F);
\draw (copy0) to [out=-60,in=180] (D);
\end{tikzpicture}\\
 &= \begin{tikzpicture}\path (0,0) node[dist] (O) {$\mu$}
++ (0.3,0) coordinate (copy1)
++ (1,0) node[kernel] (m) {$\kernel{K}^{\RV{X}|\RV{D}}$}
++ (1,0.15) coordinate (copy0)
++ (1.2,0) node (E) {$\RV{X}$}
++(-0.7,-0.3) node[kernel] (K) {$\kernel{M}$}
++(0.7,0) node (F) {$\RV{Y}$}
++(0,-0.3) node (D) {$\RV{D}$};
\draw (O) -- (m) ($(m.east) + (0,0.15)$) -- (E);
\draw (copy0) to [bend right] ($(K.west) + (0,0.1)$) (K) -- (F);
\draw (copy1) to [out=-45,in=180] ($(K.west) + (0,-0.1)$);
\draw (copy1) to [out=-90,in=180] (D);
\end{tikzpicture}\\
 &= \begin{tikzpicture}
\path (0,0) node[dist] (m) {$\prob{P}^{\RV{X}\RV{D}}$}
++ (0.7,0.15) coordinate (copy0)
+ (0,-0.3) coordinate (copy1)
+(0.2,-0.3) node (T) {}
++ (1.2,0) node (E) {$\RV{X}$}
++(-0.7,-0.3) node[kernel] (K) {$\kernel{M}$}
++(0.7,0) node (F) {$\RV{Y}$}
++ (0,-0.3) node (D) {$\RV{D}$};
\draw ($(m.east) + (0,0.15)$) -- (E);
\draw (copy0) to [bend right] ($(K.west) + (0,0.1)$) (K) -- (F);
\draw ($(m.east) + (0,-0.15)$) -- (copy1) -- ($(K.west) + (0,0)$);
\draw (copy1) to [out = -60, in=180] (D);
\end{tikzpicture}
\end{align}

Thus $\kernel{M}\in \prob{P}^{\{\RV{Y}|\RV{X}\RV{D}\}}$.

Let $\kernel{N}$ be an arbitrary version of $\prob{P}^{\{\RV{Y}|\RV{X}\RV{D}\}}$. To show that $\kernel{N}\in \kernel{K}^{\{\RV{Y}|\RV{X}\RV{D}\}}$, we will show for all $d\in D$

\begin{align}
	\prob{Q} &:= \begin{tikzpicture}
\path (0,0) node[dist] (D) {$\delta_{d}$}
++ (0.7,0) coordinate (copy0)
++(0.7,0) node[kernel] (K) {$\kernel{K}^{\RV{X}|\RV{D}}$}
++(0.5,0) coordinate (copy1)
++(0.8,0) node[kernel] (N) {$\kernel{N}$}
++(1,0) node (Y) {$\RV{Y}$}
++(0,-0.3) node (X) {$\RV{X}$}
++(0,-0.3) node (Do) {$\RV{D}$};
\draw (D) -- (K) -- (N) -- (Y);
\draw (copy0) to [out=-90,in=180] (Do);
\draw (copy1) to [out=-45,in=180] (X);
\draw (copy0) to [out=90,in=180] ($(N.west)+(0,0.15)$);
\end{tikzpicture}\\
 &= \kernel{K}^{\RV{X}\RV{Y}\RV{D}|\RV{D}}_d\label{eq:prob_disint_in_kernel_disint}
\end{align}



For $A\in\sigalg{X}$,$B\in\sigalg{Y}$, $d\in D$, we have $\prob{Q}(A\times B\times \emptyset)=0=\kernel{K}^{\RV{X}\RV{Y}\RV{D}|\RV{D}}_d(A\times B\times \emptyset$, and for $\{d\}\in\sigalg{D}$ we have $\mu(\{d\})>0$ so:

\begin{align}
\prob{Q}(A\times B\times \{d\}) &= \int_{X^2} \int_X \int_{D^3} \kernel{N}_{d'',x'}(A) \textbf{Id}_{x''}(B) \textbf{Id}_{d'''} (\{d\}) d\splitter{0.1}_d(d',d'',d''') d\kernel{K}^{\RV{X}|\RV{D}}_{d'}(x)d\splitter{0.1}_x(x',x'')\\
							&= \delta_d(\{d\}) \int_X \kernel{N}_{d,x}(A) \delta_x(B) d\kernel{K}^{\RV{X}|\RV{D}}_d(x)\\
							&= \frac{1}{\mu(\{d\})} \int_{\{d\}} d\mu(d') \int_X \kernel{N}_{d,x}(A) \delta_x(B) d\kernel{K}^{\RV{X}|\RV{D}}_d(x)\\
							&= \frac{1}{\mu(\{d\})} \int_D\int_X \kernel{N}_{d,x}(A) \delta_{d'}(\{d\}) \delta_x(B) d\kernel{K}^{\RV{X}|\RV{D}}_d(a) d\mu(d')\\
							&= \frac{1}{\mu(\{d\})} \int_D\int_X \kernel{N}_{d,x}(A) \delta_{d'}(\{d\}) \delta_x(B) d\kernel{K}^{\RV{X}|\RV{D}}_{d'}(a) d\mu(d')\\
							&= \frac{1}{\mu(\{d\})} \prob{P}^{\RV{X}\RV{Y}\RV{D}}(A\times B\times \{d\})\\
							&= \frac{1}{\mu(\{d\})} \int_D \kernel{K}_{d'}^{\RV{X}\RV{Y}\RV{D}|\RV{D}}(A\times B\times \{d\})d\mu(d')\\
							&= \frac{1}{\mu(\{d\})} \int_D \kernel{K}_{d'}{\RV{X}\RV{Y}|\RV{D}}(A\times B) \delta_{d'}(\{d\})d\mu(d')\\
							&= \kernel{K}_{d}^{\RV{X}\RV{Y}|\RV{D}}(A\times B)\\
							&= \kernel{K}_d^{\RV{X}\RV{Y}|\RV{D}}(A\times B) \delta_d(\{d\})\\
							&= \int_D \kernel{K}_{d'}^{\RV{X}\RV{Y}} (A\times B) \delta_{d''}(\{d\}) d\splitter{0.1}_d(d',d'')\\
							&= \kernel{K}_d^{\RV{X}\RV{Y}\RV{D}|\RV{D}}(A\times B\times \{d\})
\end{align}


Equality follows from the monotone class theorem. Thus $\kernel{N}\in \kernel{K}^{\{\RV{Y}|\RV{X}\RV{D}\}}$.
\end{proof}

Thus any kernel conditional probability $\kernel{K}^{\RV{Y}|\RV{X}\RV{D}}$ can equally well be considered a regular conditional probability $\prob{P}^{\RV{Y}|\RV{X}\RV{D}}$ for a related probability space $(\prob{P},\Omega\times D)$ under the obvious identification of random variables, provided $D$ is countable. Note that any conditional probability $\prob{P}^{\RV{Y}|\RV{X}}$ that is \emph{not} conditioned on $\RV{D}$ is undefined in the kernel space $(\kernel{K},D,\Omega)$.

\subsubsection{Conditional Independence}

\begin{definition}[Kernels constant in an argument]
	Given a kernel $(\kernel{K},D,\Omega)$ and random variables $\RV{Y}$ and $\RV{X}$, we say a verstion of the disintegration $\kernel{K}^{\RV{Y}|\RV{X}\RV{D}}$ is constant in $\RV{D}$ if for all $x\in X$, $d,d'\in D$, $\kernel{K}^{\RV{Y}|\RV{X}\RV{D}}_{(x,d)} = \kernel{K}^{\RV{Y}|\RV{X}\RV{D}}_{(x,d')}$.

\end{definition}

\begin{definition}[Domain Conditional Independence]
Given a kernel space $(\kernel{K},D,\Omega)$, relative probability space $(\prob{P},\Omega\times D)$, variables $\RV{X}$,$\RV{Y}$ and domain variable $\RV{D}$, $\RV{X}$ is \emph{conditionally independent} of $\RV{D}$ given $\RV{Y}$, written $\RV{X}\CI_{\kernel{K}} \RV{D}|\RV{Y}$ if any of the following equivalent conditions hold:

\todo[inline]{Almost sure equality}

\begin{enumerate}
	\item $\prob{P}^{\RV{X}\RV{D}|\RV{Y}} \sim \prob{P}^{\RV{X}|\RV{Y}}\utimes \prob{P}^{\RV{D}|\RV{Y}}$
	\item For any version of $\prob{P}^{\{\RV{X}|\RV{Y}\}}$, $\prob{P}^{\RV{X}|\RV{Y}}\otimes\stopper{0.1}_D$ is a version of  $\kernel{K}^{\{\RV{X}|\RV{Y}\RV{D}\}}$
	\item There exists a version of $\kernel{K}^{\{\RV{X}|\RV{Y}\RV{D}\}}\text{ constant in }\RV{D}$
\end{enumerate}
\end{definition}

\begin{theorem}[Definitions are equivalent]\label{th:ci_equivalence}
(1)$\implies$(2):
By Lemma \ref{lem:agree_disint}, $\prob{P}^{\{\RV{Y}|\RV{X}\RV{D}\}}=\kernel{K}^{\{\RV{Y}|\RV{X}\RV{D}\}}$. Thus it is sufficient to show that $\prob{P}^{\RV{X}|\RV{Y}}\otimes\stopper{0.1}$ is a version of $\prob{P}^{\{\RV{X}|\RV{Y}\RV{D}\}}$.

\begin{align}
\begin{tikzpicture}
	\path (0,0) node[dist] (Pxd) {$\prob{P}^{\RV{Y}\RV{D}}$}
	+ (0.7,0.1) coordinate (copy0)
	+ (0.7,-0.1) coordinate (copy1)
	++ (1.5,0) node[kernel] (Pyxd) {$\prob{P}^{\RV{X}|\RV{Y}}$}
	++(1,0) node (Y) {$``\RV{X}''$}
	+(0,0.3) node (D) {$``\RV{D}''$}
	+(0,0.6) node (X) {$``\RV{Y}''$};
	\draw ($(Pxd.east) + (0,0.1)$) -- ($(Pyxd.west)+(0,0.1)$);
	\draw ($(Pxd.east) + (0,-0.1)$) -- (copy1);
	\draw[-{Rays[n=8]}] (copy1) to [out=-80,in=180] ($(Pyxd.south)+(0,-0.3)$);
	\draw (copy0) to [out=80,in=180] (X);
	\draw (copy1) to [out=80,in=180] (D);
	\draw (Pyxd) -- (Y);
\end{tikzpicture} &= \begin{tikzpicture}
	\path (0,0) node[dist] (Pxd) {$\prob{P}^{\RV{Y}\RV{D}}$}
	+ (0.7,0.1) coordinate (copy0)
	+ (0.7,-0.1) coordinate (copy1)
	++ (1.5,0) node[kernel] (Pyxd) {$\prob{P}^{\RV{X}|\RV{Y}}$}
	++(1,0) node (Y) {$``\RV{X}''$}
	+(0,0.3) node (D) {$``\RV{D}''$}
	+(0,0.6) node (X) {$``\RV{Y}''$};
	\draw ($(Pxd.east) + (0,0.1)$) -- ($(Pyxd.west)+(0,0.1)$);
	\draw ($(Pxd.east) + (0,-0.1)$) -- (copy1);
	\draw (copy0) to [out=80,in=180] (X);
	\draw (copy1) to [out=80,in=180] (D);
	\draw (Pyxd) -- (Y);
\end{tikzpicture} \\
 &= \begin{tikzpicture}
	\path (0,0) node[dist] (Pxd) {$\prob{P}^{\RV{Y}}$}
	+ (0.7,-0.2) coordinate (copy1)
	++ (1.5,-0.2) node[kernel] (Pyxd) {$\prob{P}^{\RV{X}|\RV{Y}}$}
	+ (0,0.5) node[kernel] (Pdx) {$\prob{P}^{\RV{D}|\RV{Y}}$}
	++(1,0) node (Y) {$``\RV{X}''$}
	+(0,0.5) node (D) {$``\RV{D}''$}
	+(0,1.2) node (X) {$``\RV{Y}''$};
	\draw ($(Pxd.east) + (0,-0.2)$) -- ($(Pyxd.west)+(0,0)$);
	\draw (copy1) to [out=90,in=180] (X);
	\draw (copy1) to [out=80,in=180] (Pdx);
	\draw (Pdx) -- (D);
	\draw (Pyxd) -- (Y);
\end{tikzpicture} \\
&\overset{condition (1)}{=} \begin{tikzpicture}
	\path (0,0) node[dist] (Pxd) {$\prob{P}^{\RV{Y}}$}
	+ (0.7,0) coordinate (copy1)
	++ (1.5,0) node[kernel] (Pyxd) {$\prob{P}^{\RV{X}\RV{D}|\RV{Y}}$}
	++(1,-0.15) node (Y) {$``\RV{X}''$}
	+(0,0.3) node (D) {$``\RV{D}''$}
	+(0,0.6) node (X) {$``\RV{Y}''$};
	\draw (Pxd) -- (Pyxd);
	\draw (copy1) to [out=90,in=180] (X);
	\draw ($(Pyxd.east)+(0,0.15)$) -- (D);
	\draw ($(Pyxd.east)+(0,-0.15)$) -- (Y);
\end{tikzpicture}\\
&= \begin{tikzpicture}
	\path (0,0) node[dist] (Pxd) {$\prob{P}^{\RV{YDX}}$}
	++(1,-0.3) node (Y) {$\RV{X}$}
	+(0,0.3) node (D) {$\RV{D}$}
	+(0,0.6) node (X) {$\RV{Y}$};
	\draw ($(Pxd.east) + (0,0.3)$) -- (X) ($(Pxd.east) + (0,-0.3)$) -- (Y) (Pxd) -- (D);
\end{tikzpicture}
\end{align}

(2)$\implies$ (3)

$\prob{P}^{\RV{X}|\RV{Y}}\otimes\stopper{0.1}_D$ is a version of $\kernel{K}^{\{\RV{X}|\RV{Y}\RV{D}\}}$ by assumption, and is clearly constant in $\RV{D}$.

(3)$\implies$ (1)

By lemma \ref{lem:agree_disint}, there also exists a version of $\prob{P}^{\{\RV{X}|\RV{Y}\RV{D}\}}$ constant in $\RV{D}$. Let $\kernel{M}:Y\times D\to \Delta(\sigalg{X})$ be such a version. For arbitrary $d_0\in D$, let $\kernel{N}:=\kernel{M}_{(\cdot,d_0)}:Y\to \Delta(\sigalg{X})$ be the map $x\mapsto \kernel{M}_{(x,d_0)}$. By constancy in $\RV{D}$, $\kernel{M} = \stopper{0.1}\otimes \kernel{N}$. We wish to show $\prob{P}^{\RV{X}|\RV{Y}}\utimes \prob{P}^{\RV{D}|\RV{Y}}\in \prob{P}^{\{\RV{XD}|\RV{Y}\}}$. By Theorem \ref{th:iterated_disint}, we have 

\begin{align}
\begin{tikzpicture}
	\path (0,0) node[dist] (Pxd) {$\prob{P}^{\RV{Y}\RV{D}}$}
	+ (0.7,0.1) coordinate (copy0)
	+ (0.7,-0.1) coordinate (copy1)
	++ (1.5,0) node[kernel] (Pyxd) {$\kernel{N}$}
	++(1,0) node (Y) {$\RV{X}$}
	+(0,0.3) node (D) {$\RV{D}$}
	+(0,0.6) node (X) {$\RV{Y}$};
	\draw ($(Pxd.east) + (0,0.1)$) -- ($(Pyxd.west)+(0,0.1)$);
	\draw ($(Pxd.east) + (0,-0.1)$) -- (copy1);
	\draw[-{Rays[n=8]}] (copy1) to [out=-80,in=180] ($(Pyxd.south)+(0,-0.3)$);
	\draw (copy0) to [out=80,in=180] (X);
	\draw (copy1) to [out=80,in=180] (D);
	\draw (Pyxd) -- (Y);
\end{tikzpicture} &= \begin{tikzpicture}
	\path (0,0) node[dist] (Pxd) {$\prob{P}^{\RV{Y}}$}
	+ (0.7,0) coordinate (copy0)
	++ (1.5,0) node[kernel] (Dy) {$\prob{P}^{\RV{D}|\RV{Y}}$}
	++ (1.5,0) node[kernel] (Pyxd) {$\kernel{N}$}
	++(1,0) node (Y) {$\RV{X}$}
	+(0,0.3) node (D) {$\RV{D}$}
	+(0,0.6) node (X) {$\RV{Y}$};
	\draw ($(Pxd.east) + (0,0.1)$) -- ($(Pyxd.west)+(0,0.1)$);
	\draw ($(Pxd.east) + (0,-0.1)$) -- (copy1);
	\draw[-{Rays[n=8]}] (copy1) to [out=-80,in=180] ($(Pyxd.south)+(0,-0.3)$);
	\draw (copy0) to [out=80,in=180] (X);
	\draw (copy1) to [out=80,in=180] (D);
	\draw (Pyxd) -- (Y);
\end{tikzpicture}
\end{align}
\end{theorem}

\begin{definition}[Conditional probability existence]\label{def:conditional_probability_existence}
Given a kernel space $(\kernel{K},D,\Omega)$ and random variables $\RV{X}$, $\RV{Y}$, we say $\kernel{K}^{\{\RV{Y}|\RV{X}\}}$ \emph{exists} if $\RV{Y}\CI_{\kernel{K}} \RV{D}|\RV{X}$. If $\kernel{K}^{\{\RV{Y}|\RV{X}\}}$ exists then it is by definition equal to $\prob{P}^{\{\RV{Y}|\RV{X}\}}$ for any related probability space $(\prob{P},\Omega\times D)$.
\end{definition}

Note that $\kernel{K}^{\{\RV{Y}|\RV{X}\RV{D}\}}$ always exists.

\begin{definition}[Conditional Independence]\label{def:conditional_independence}
Given a kernel space $(\kernel{K},D,\Omega)$, some relative probability space $(\prob{P},\Omega\times D)$, variables $\RV{X}$,$\RV{Y}$ and $\RV{Z}$, $\RV{X}$ is \emph{conditionally independent} of $\RV{Z}$ given $\RV{Y}$, written $\RV{X}\CI_{\kernel{K}} \RV{Z}|\RV{Y}$ if $\kernel{K}^{\{\RV{XY}|\RV{Z}\}}$ exists and any of the following equivalent conditions hold:

\todo[inline]{Almost sure equality}

\begin{itemize}
	\item $\prob{P}^{\RV{X}\RV{Z}|\RV{Y}} \sim \prob{P}^{\RV{X}|\RV{Y}}\utimes \prob{P}^{\RV{Z}|\RV{Y}}$
	\item For any version of $\prob{P}^{\{\RV{X}|\RV{Y}\}}$, $\prob{P}^{\RV{X}|\RV{Y}}\otimes\stopper{0.1}_Z$ is a version of  $\kernel{K}^{\{\RV{X}|\RV{Y}\RV{Z}\}}$
	\item There exists a version of $\kernel{K}^{\{\RV{X}|\RV{Y}\RV{Z}\}}\text{ constant in }\RV{Z}$
\end{itemize}
\end{definition}

\begin{lemma}[Diagrammatic consequences of labels]

In general, diagram labels are ``well behaved'' with regard to the application of any of the special Markov kernels: identities \ref{eq:identity}, swaps \ref{eq:swap}, discards \ref{eq:discard} and copies \ref{eq:copy} as well as with respect to the coherence theorem of the CD category. They are not ``well behaved'' with respect to composition.

Fix some Markov kernel space $(\kernel{K},D,\Omega)$ and random variables $\RV{X}$, $\RV{Y}$, $\RV{Z}$ taking values in $X,Y,Z$ respectively. $\mathrm{Sat:}$ indicates that a labeled diagram satisfies definitions \ref{def:wl_jprob} and \ref{def:wl_disint} with respect to $(\mathscr{K},D,\Omega)$ and $\RV{X}$, $\RV{Y}$, $\RV{Z}$.  The following always holds:

\begin{align}
\mathrm{Sat:}
\begin{tikzpicture}
\path (0,0) node (A) {$\RV{X}$}
++(0.8,0) node (X) {$\RV{X}$};
\draw (A) -- (X);
\end{tikzpicture}
\end{align}

and the following implications hold:
\begin{align}
\mathrm{Sat:}\;\begin{tikzpicture}
\path (0,0) node (Z) {$\RV{Z}$} 
++ (0.7,0) node[kernel] (M) {$\kernel{K}$}
++ (0.7,0.15) node (X) {$\RV{X}$}
++(0,-0.3) node (Y) {$\RV{Y}$};
\draw (Z) -- (M) ($(M.east) + (0,0.15)$) -- (X);
\draw ($(M.east) + (0,-0.15)$) -- (Y);
\end{tikzpicture} &\implies \mathrm{Sat:}\; \begin{tikzpicture}
\path (0,0) node (Z) {$\RV{Z}$} 
++ (0.7,0) node[kernel] (M) {$\kernel{K}$}
++ (0.7,0.15) node (X) {$\RV{X}$}
++(0,-0.3) node (Y) {};
\draw (Z) -- (M) ($(M.east) + (0,0.15)$) -- (X);
\draw[-{Rays [n=8]}] ($(M.east) + (0,-0.15)$) -- (Y);
\end{tikzpicture}\\
\mathrm{Sat:}\;\begin{tikzpicture}
\path (0,0) node (Z) {$\RV{Z}$} 
++ (0.7,0) node[kernel] (M) {$\kernel{K}$}
++ (0.7,0.15) node (X) {$\RV{X}$}
++(0,-0.3) node (Y) {$\RV{Y}$};
\draw (Z) -- (M) ($(M.east) + (0,0.15)$) -- (X);
\draw ($(M.east) + (0,-0.15)$) -- (Y);
\end{tikzpicture} &\implies \mathrm{Sat:}\; \begin{tikzpicture}
\path (0,0) node (Z) {$\RV{Z}$} 
++ (0.7,0) node[kernel] (M) {$\kernel{K}$}
++ (0.7,0.15) node (X) {$\RV{Y}$}
++(0,-0.3) node (Y) {$\RV{X}$};
\draw (Z) -- (M) ($(M.east) + (0,0.15)$) to [out = 0, in = 180] (Y);
\draw ($(M.east) + (0,-0.15)$) to [out = 0, in = 180] (X);
\end{tikzpicture}\\
\mathrm{Sat:}\begin{tikzpicture}
\path (0,0) node (Z) {$\RV{Z}$} 
++ (0.7,0) node[kernel] (M) {$\mathrm{L}$}
++(0.6,0) node (X1) {$\RV{X}$};
\draw (Z) -- (M) (M)--(X1);
\end{tikzpicture}
&\implies \mathrm{Sat:}\begin{tikzpicture}
\path (0,0) node (Z) {$\RV{Z}$} 
++ (0.7,0) node[kernel] (M) {$\mathrm{L}$}
++ (0.7,0) coordinate (copy0)
++(0.5,0.2) node (X1) {$\RV{X}$}
++(0,-0.4) node (X2) {$\RV{X}$};
\draw (Z) -- (M) (M) -- (copy0) to [bend left] (X1);
\draw (copy0) to [bend right] (X2);
\end{tikzpicture}\\
\mathrm{Sat:}\begin{tikzpicture}
\path (0,0) node (X) {$\RV{Z}$}
++ (0.7,0) node[kernel] (K) {$\kernel{K}$}
++(0.7,0) node (Y) {$\RV{Y}$};
\draw (X) -- (K) -- (Y);
\end{tikzpicture} &\implies \mathrm{Sat:}
\begin{tikzpicture}
\path (0,0) node (A) {$\RV{Z}$}
++(0.5,0) coordinate (copy0)
+(1.2,0.3) node (X) {$\RV{Z}$}
++(0.5,-0.3) node[kernel] (K) {$\kernel{K}$}
+(0.7,0) node (Y) {$\RV{Y}$};
\draw (A) -- (copy0) to [bend left] (X);
\draw (copy0) to [bend right] (K) (K) -- (Y);
\end{tikzpicture}\label{eq:splitter_preserves_name}
\end{align}
\end{lemma}


\begin{proof}
\begin{itemize}
	\item $\mathrm{Id}_X$ is a version of $\prob{P}_{\RV{X}|\RV{X}}$ for all $\prob{P}$; $\prob{P}_{\RV{X}}\mathrm{Id}_X = \prob{P}_{\RV{X}}$
	\item $\kernel{K}\mathrm{Id}\otimes \stopper{0.2})(w;A) = \int_{X\times Y} \delta_x(A) \mathds{1}_Y(y) d\kernel{K}_w(x,y) = \kernel{K}_w(A\times Y) = \prob{P}_{\RV{X}|\RV{Z}}(w;A)$
	\item $\int_{X\times Y} \delta_{\mathrm{swap(x,y)}}(A\times B)d\kernel{K}_w(x,y) = \prob{P}_{\RV{Y}\RV{X}|\RV{Z}}(w;A\times B)$
	\item $\kernel{K}\splitter{0.1} (w;A\times B) = \int_{X} \delta_{x,x}(A\times B) d\kernel{K}_w(x) = \prob{P}_{\RV{X}\RV{X}|\RV{Z}} (w;A\times B)$
\end{itemize}
\ref{eq:splitter_preserves_name}: Suppose $\kernel{K}$ is a version of $\prob{P}_{\RV{Y}|\RV{Z}}$. Then
\begin{align}
\prob{P}_{\RV{Z}\RV{Y}} &= \begin{tikzpicture}
\path (0,0) node[dist] (m) {$\prob{P}_{\RV{Z}}$}
++ (0.7,0.15) coordinate (copy0)
++ (1.2,0) node (E) {$\RV{Z}$}
++(-0.7,-0.3) node[kernel] (K) {$\kernel{K}$}
++(0.7,0) node (F) {$\RV{Y}$};
\draw ($(m.east) + (0,0.15)$) -- (E);
\draw (copy0) to [bend right] (K) (K) -- (F);
\end{tikzpicture}\\
\prob{P}_{\RV{Z}\RV{Z}\RV{Y}} &= \begin{tikzpicture}
\path (0,0) node[dist] (m) {$\prob{P}_{\RV{Z}}$}
++ (0.7,0.15) coordinate (copy0)
+ (0.5,0) coordinate (copy1)
+ (1.2,0.3) node (Xm) {$\RV{Z}$}
++ (1.2,0) node (E) {$\RV{Z}$}
++(-0.7,-0.3) node[kernel] (K) {$\kernel{K}$}
++(0.7,0) node (F) {$\RV{Y}$};
\draw ($(m.east) + (0,0.15)$) -- (E);
\draw (copy0) to [bend right] (K) (K) -- (F);
\draw (copy1) to [bend left] (Xm);
\end{tikzpicture}\\
&= \begin{tikzpicture}
\path (0,0) node[dist] (m) {$\prob{P}_{\RV{Z}}$}
+ (0.5,0.15) coordinate (copy1)
++ (0.7,0.15) coordinate (copy0)
+ (1.2,0.3) node (Xm) {$\RV{Z}$}
++ (1.2,0) node (E) {$\RV{Z}$}
++(-0.7,-0.3) node[kernel] (K) {$\kernel{K}$}
++(0.7,0) node (F) {$\RV{Y}$};
\draw ($(m.east) + (0,0.15)$) -- (E);
\draw (copy0) to [bend right] (K) (K) -- (F);
\draw (copy1) to [bend left] (Xm);
\end{tikzpicture}
\end{align}
Therefore $\splitter{0.1}(\mathrm{Id}_X\otimes\kernel{K})$ is a version of $\prob{P}_{\RV{Z}\RV{Y}|\RV{Z}}$ by \ref{def:labeled_disint} 
\end{proof}

The following property, on the other hand, does \emph{not} generally hold:
\begin{align}
\mathrm{Sat:}\begin{tikzpicture}
\path (0,0) node (X) {$\RV{Z}$}
++ (0.7,0) node[kernel] (K) {$\kernel{K}$}
++(0.7,0) node (Y) {$\RV{Y}$};
\draw (X) -- (K) -- (Y);
\end{tikzpicture},
\begin{tikzpicture}
\path (0,0) node (X) {$\RV{Y}$}
++ (0.7,0) node[kernel] (K) {$\kernel{L}$}
++(0.7,0) node (Y) {$\RV{X}$};
\draw (X) -- (K) -- (Y);
\end{tikzpicture}
 &\implies \mathrm{Sat:}
\begin{tikzpicture}
\path (0,0) node (X) {$\RV{Z}$}
++ (0.7,0) node[kernel] (K) {$\kernel{K}$}
++(0.7,0) node[kernel] (L) {$\kernel{L}$}
++(0.7,0) node (X1) {$\RV{X}$};
\draw (X) -- (K) -- (Y) -- (L) -- (X1);
\end{tikzpicture}\label{eq:composition}
\end{align}

Consider some ambient measure $\prob{P}$ with $\RV{Z}=\RV{X}$ and $\prob{P}_{\RV{Y}|\RV{X}}=x\mapsto \mathrm{Bernoulli}(0.5)$ for all $z\in Z$. Then $\prob{P}_{\RV{Z}|\RV{Y}}=y\mapsto \prob{P}_{\RV{Z}}$, $\forall y\in Y$ and therefore $\prob{P}_{\RV{Y}|\RV{Z}}\prob{P}_{\RV{Z}|\RV{Y}}=x\mapsto \prob{P}_{\RV{Z}}$ but $\prob{P}_{\RV{Z}|\RV{X}} = x\mapsto \delta_x\neq \kernel{\prob{P}_{\RV{Y}|\RV{Z}}\prob{P}_{\RV{Z}|\RV{Y}}}$.






%!TEX root = main.tex

\chapter{Models with choices and consequences}\label{ch:2p_statmodels}

Probability sets, introduced in Chapter \ref{ch:tech_prereq}, will be used to model \emph{decision problems}, which are problems that involve choices and consequences. In such problems, three things are given: a set of options, one of which must be chosen, a set of consequences and a means of judging which consequences are more desirable than others. Such a problem requires an understanding of how each choice corresponds to consequences, as far as this is able to be understood. The fundamental type of problem studied in this thesis is how to map choices to consequences. 

In practice, causal inference is concerned with a wider variety of problems than this. A great deal of empirical causal analysis is concerned with problems a step removed from this: the purpose is to advise other decision makers on a course of action rather than to recommend an action directly. Nevertheless, a great deal of causal analysis is ultimately motivated by problems involving a choice among options, even if the analysis only addresses such problems indirectly. Section \ref{sec:whats_the_point} briefly reviews the attitude of prominent theorists of causal inference towards decision problems. Subsequently, it presents the basic definition of a decision problem, and two different kinds of models that can be used to represent the relationship between choices and consequences.

The reasons we provide for using probability sets to model decision problems are not rigorous. The strongest motivation for this choice is \emph{convention}: many varieties of decision theory induce probability set models, and Chapter \ref{ch:other_causal_frameworks} shows how many causal inference frameworks also induce probability set models. Some decision theories examined in this chapter justify their modelling choices by suggesting axioms for rational theories of decision under uncertainty. However, despite the various attempts at axiomatisation, the nature of theories of ``rational choice'' is contested -- there is no clear standard among the the theories surveyed here, or developed elsewhere. This work is not trying to resolve this dispute, yet modelling choices must still be made. Section \ref{sec:how_represent_conseqeunces} provides an overview of four major decision theories along with their axiomatisations (where applicable). These are \emph{Savage decision theory}, \emph{Jeffrey decision theory} (or evidential decision theory), Lewis' \emph{causal decision theory} and \emph{statistical decision theory}.

Section \ref{sec:how_represent_conseqeunces} describes in particular detail the connections between \emph{statistical decision theory} \citep{wald_statistical_1950} and probability set models of decision problems. We are able to demonstrate a close connection between probability set models of decision problems and the classical statistical notion of \emph{risk} of a decision rule, even though causal considerations are often not central to classical statistics. Secondly, the kind of probability set model -- which we call a \emph{see-do model} -- shows up again in Chapter \ref{ch:evaluating_decisions} where we consider the question of when a probability set model supports a certain notion of ``the causal effect of a variable'', and again in Chapter \ref{ch:other_causal_frameworks} where we consider the kinds of probability set models induced by other causal reasoning frameworks.

The formal definition of a variable in a probabilistic model is straightforward (Definition \ref{def:variable}). However, in practice the definitions of variables often includes informal content that enables the interpretation of a probabilistic model. In the field of causal models, one is likely to come across many different ``kinds'' of variables: for example, observed variables, unobserved variables, counterfactual variables and causal variables all play important roles in various causal inference frameworks. However, there is no formal distinction between these different kinds of variables -- Definition \ref{def:variable} applies to them all. Section \ref{sec:variable} is an attempt to clarify an understanding of the informal role of variables as ``pointing to the parts of the world that the model is about''. In comparison to the wide variety of variable types encountered in the causal literature, it offers a very limited theory of the informal semantics of variables. In short, observed variables correspond to a measurement procedure (in a sense that will be made precise), and unobserved variables do not.

\section{What is the point of causal inference?}\label{sec:whats_the_point}

\citet{pearl_book_2018} argues forcefully that causal reasoning frameworks should be understood by the questions that they answer. He also posits a ``ladder'' of types of causal question, where the $n$th level of question type also subsumes all lower levels. The question types are \citep{barenboim_foundations_2020}:
\begin{enumerate}
    \item \emph{Associational}: informally, ``questions about relationships and predictions''; formally defined as queries that can be answered by a single probability distribution
    \item \emph{Interventional}: informally, ``questions about the consequences of interventions''; formally defined as queries that can be answered by a causal Bayesian network (CBN)
    \item \emph{Counterfactual}: informally, ``questions concerning imagining alternate worlds''; formally defined as queries that can be answered by a structural causal model (SCM)
\end{enumerate}

Given that counterfactual questions are suggested to be the most general kind of causal question, one might ask why this work focuses on questions of an interventional nature. There are two reasons for this: Firstly, a class of informal questions is being used to motivate the theory of causal inference with probability sets. I have much stronger intuitions about informal decision problems than informal counterfactual queries. This does not seem to be a purely personal taste: questions about how decision problems should be represented have been studied much more than similar questions regarding counterfactual queries. Secondly, problems that involve comparing different choices on the basis of their consequences are an important class of problems on their own. Even within the causal inference literature, ``interventional'' questions and interpretations are much more prominent than counterfactual questions. For example, \citet{rubin_causal_2005} points out that causal inference often informs a decision maker by providing ``scientific knowledge'', but does not make recommendations by itself. \citep{imbens_causal_2015} introduces causal inference as the study of ``outcomes of manipulations'' and \citep{spirtes_causation_1993} highlights the universal relevance of understanding how to control certain outcomes, while further arguing that clarifying commonsense ideas of causation is also an important aim of causal inference. \citet{hernan_whatif_2020} present causal knowledge as critical for assessing the consequences of actions.

Speculatively, counterfactual queries may be able to be interpreted as decision problems with fanciful options. Consider an informal decision problem and a counterfactual query addressing similar material:
\begin{itemize}
    \item Decision problem: I want my headache to go away. If I take Aspirin, will it do so?
    \item Counterfactual query: I wish I didn't have headache. If I had taken the Aspirin, would I still have it?
\end{itemize}
If I haven't taken aspirin, then there's nothing I can actually choose to do to make it so that I had. However, if I imagine that I did have some option available that accomplished this, then the structure of the two questions seems rather similar. Both ask: if I take the option, what will the consequence be? Of course, it's hard to say what makes a correct answer to the second question, but this is a feature of counterfactual questions in general.

\subsection{Modelling decision problems}\label{sec:modelling_decision_problems}

People who need to make a decisions might (and often do) make them with no mathematical reasoning at all. However, this work is concerned with making decisions assisted by mathematical reasoning. In order to reason mathematically about a decision to be made, we assume that somehow, we have access to two sets:
\begin{enumerate}
    \item There is a set of choices $C$ that need to be compared
    \item There is a set of consequences $\Omega$ along with a utility function $u:\Omega\to \mathbb{R}$
\end{enumerate}

Given some means of relating between $C$ and $\Omega$, the order on $\Omega$ will induce some order on $C$. There are a great number of different ways that of relating elements of $C$ to $\Omega$. For example, a binary relation between the two sets will, given a total order on $\Omega$, induce a preorder on $C$. However, in this work the assumption is made that the relevant kinds of relations are either
\begin{itemize}
    \item A Markov kernel $C\kto \Omega$
    \item A Markov kernel $C\times H\kto \Omega$ for some set of hypotheses $H$
\end{itemize}
That is, for each choice $c\in C$ we have either a probability distribution in $\Delta(\Omega)$ or a set of probability distributions indexed by $h\in H$. Sections \ref{sec:cons_to_sdp} and \ref{sec:cc_theorem} discuss each choice in more detail. Where it is needed, we also assume that a utility function $\Omega\to \mathbb{R}$ is available and that choices are evaluated using the principle of expected utility.

Usually, someone confronted with a decision problem will not know for certain the consequences that arise from any given choice, and yet they may have some views about which consequences are more likely than others. Probability has a long and successful history of representing uncertain knowledge of this type. There are many works that aim to show that any method for representing uncertain knowledge that adheres to certain principles must be a probability distribution \citet{de_finetti_foresight_1992,horvitz_framework_1986}, along with criticism of these principles \citet{halpern_counter_1999}. A notable alternative to representing uncertainty with a single probability distribution represents uncertainty with a set of probability distributions, which is a type of \emph{vague probability} model \citep{walley_statistical_1991}. 

More relevant to the question of modelling decision problems are a number of works that establish conditions under which ``desirability'' or ``preference'' relations over sets of choices or propositions must be represented by a probability distribution along with a utility function. These works are surveyed in Section \ref{sec:how_represent_conseqeunces}. Ultimately, however, the question of whether probability is the right choice to represent uncertain knowledge in decision models is not a key focus of this work. It is a conventional choice, and one that is accepted here.

\subsection{Formal definitions}\label{sec:probability_sets}

We suppose that we are, at the outset, given a few basic ingredients: a set of choices $C$, a set of consequences $\Omega$ and a utility function $u:F\to \mathbb{R}$. We call these ingredients a ``decision problem''.

\begin{definition}[Decision problem]
A decision problem is a triple $(C,\Omega,u)$ consisting of a measurable set $(C,\sigalg{C})$ of choices, $(\Omega,\sigalg{F})$ consequences and a utility function $u:F\to \mathbb{R}$.
\end{definition}

Our task is to find a \emph{model} that relates $C$ to $\Omega$. We assume two forms of model -- a \emph{sharp model} associates each choice with a unique probability distribution, and a \emph{vague model} associates each choice with a set of probability distributions.

\begin{definition}[Choices only model]
Given a decision problem $(C,\Omega,u)$, a \emph{choices only model} is a function $C\kto \Omega$.
\end{definition}

\begin{definition}[Choices and hypotheses model]
Given a decision problem $(C,\Omega,u)$, a model with \emph{choices and hypotheses} is a function $C\times H\kto \Omega$ for some hypothesis set $H$.
\end{definition}

Both types of models induce probability sets.

\begin{definition}[Induced probability set]\label{def:induced_pset_1}
Given a decision problem $(C,\Omega,u)$ and a model $\prob{P}_\cdot:C\times H\kto \Omega$, the induced probability set is $\prob{P}_{C\times H}:=\{\prob{P}_\alpha|\alpha\in C\times H\}$.
\end{definition}


\section{Representation theorems for decision problems }\label{sec:how_represent_conseqeunces}

We assume decision models are probabilistic functions $C\to \Delta(\Omega)$ for some sample space $(\Omega,\sigalg{F})$ of ``consequences''. Probability distributions, and the principle of expected utility in particular, are common choices for evaluation under uncertainty. Representation theorems offer a more formal justification for this choice; they propose a collection of axioms regulating the sets of evaluations we want some decision evaluation model to admit, and then show that this model can be represented with (for example) a probability distribution along with a utility function. The desirability of some of the axioms in these theorems is not obvious. 

Here we review the representation theorems of \citet{savage_foundations_1954} and \citet{jeffrey_logic_1990}. We establish that both imply that choices are compared using a probabilistic function $C\to \Delta(\Omega)$ for a suitable selection of $C$ and $(\Omega,\sigalg{F})$, along with a ``desirability'' function which differs in type between the two theorems.

Lewis' \emph{causal decision theory} is also briefly reviewed. While the particular considerations that motivated this theory are not examined, this theory introduces \emph{dependency hypotheses}, which play a key role in the rest of this work.

The following discussion will often make reference to \emph{complete preference relations}. A complete preference relation is a relation $\succ,\prec,\sim$ on a set $A$ such that for any $a,b,c$ in $A$ we have:
\begin{itemize}
    \item Exactly one of $a\succ b$, $a\prec b$, $a\sim b$ holds
    \item $(a\succ b)\iff(b\prec a)$
    \item $a\succ b$ and $b\succ c$ implies $a\succ c$
\end{itemize}
In short, it is a total order without antisymmetry ($a$ and $b$ can be equally preferred even if they are not in fact equal).

This definition is meant to correspond to the common sense idea of having preferences over some set of things, where $\succ$ can be read as ``strictly better than'', $\prec$ read as ``strictly worse than'' and $\sim$ read as ``as good as''. Given any two things from the set, I can say which one I prefer, or if I prefer neither (and all of these are mutually exclusive). If I prefer $a$ to $a'$ then I think $a'$ is worse than $a$. Furthermore, if I prefer $a$ to $a'$ and $a'$ to $a''$ then I prefer $a$ to $a''$.

Define $a\preceq b$ to mean $a\prec b$ or $a \sim b$.

\subsection{von Neumann-Morgenstern utility}

\citet{von_neumann_theory_1944} proved that when the \emph{vNM axioms} hold (not defined here; see the original reference or \citet{steele_decision_2020}), an agent's preferences between ``lotteries'' (probability distributions in $\Delta(\Omega)$ for some $(\Omega,\sigalg{F})$) can be represented as the comparison of the expected value under each lottery of a utility function $u$ unique up to affine transformation. That is, for lotteries $\prob{P}_\alpha$ and $\prob{P}_{\alpha'}$, there exists some $u:\Omega\to \mathbb{R}$ unique up to affine transformation such that $\mathbb{E}_{\prob{P}_\alpha}[u]> \mathbb{E}_{\prob{P}_{\alpha'}}[u]$ if and only if $\prob{P}_{\alpha} \succ \prob{P}_{\alpha'}$.

In vNM theory, the set of lotteries is is the set of all probability measures on $(\Omega,\sigalg{F})$. Thus von Neumann-Morgenstern theorem gives conditions under which preferences \emph{over distributions of consequences} can be represented using expected utility. It is silent on the question of whether each choice should be mapped to a unique probability distribution over consequences.

\subsection{Savage decision theory}

Savage's decision theory establishes conditions under which, given \emph{acts} $C$, \emph{consequences} $\Omega$ and \emph{states} $(S,\sigalg{S})$ (which are ``possible mappings from acts to consequences''), the preference relation over acts can be represented with a probability distribution over states and a utility function $\Omega\to \mathbb{R}$. This is much closer to the subject of this work than the theorem of von Neumann and Morgernstern.

\begin{definition}[Elements of a Savage decision problem]
A \emph{Savage decision problem} features a measurable set of states $(S,\sigalg{S})$, a set of consequences $\Omega$ and a set of acts $C$ such that $|C|=\Omega^S$ and an evaluation function $T:S\times C\to F$ such that for any $f:S\to \Omega$ there exists $c\in C$ such that $T(\cdot,c)=f$.
\end{definition}

\begin{theorem}
Given any Savage decision problem $(S,\Omega,C,T)$ with a preference relation $(\prec,\sim)$ on $C$ that satisfies the \emph{Savage axioms} \ref{sec:savage_axioms}, there exists a unique probability distribution $\mu\in\Delta(\sigalg{S})$ and a utility $u:\Omega\to \mathbb{R}$ unique up to affine transformation such that
\begin{align}
    \alpha\preceq \alpha' &\iff \int_S u(T(s,\alpha))\mu(\mathrm{d}s) \leq \int_S u(T(s,\alpha'))\mu(\mathrm{d}s)&\forall \alpha,\alpha'\in C
\end{align}
\end{theorem}

\begin{proof}
\citet{savage_foundations_1954}
\end{proof}

If we equip consequences with a measures $(\Omega,\sigalg{F})$, Savage's setup implies the existence of a unique probabilistic function $C\to \Delta(\Omega)$ representing the ``probabilistic consequences'' of each choice.

\begin{theorem}
Given any Savage decision problem $(S,\Omega,C,T)$ with a preference relation $(\prec,\sim)$ on $C$ that satisfies the \emph{Savage axioms}, and a $\sigma$-algebra $\sigalg{F}$ on $\Omega$ such that $T$ is measurable, there is a probabilistic function $\prob{P}_{\cdot}:C\to \Delta(\Omega)$ and a utility $u:\Omega\to \mathbb{R}$ unique up to affine transformation such that
\begin{align}
    \alpha\preceq \alpha' &\iff \int_\Omega u(f)\prob{P}_\alpha(\mathrm{d}f) \leq \int_\Omega u(f)\prob{P}_{\alpha'}(\mathrm{d}f)&\forall \alpha,\alpha'\in C
\end{align}
\end{theorem}

\begin{proof}
Define $\prob{P}_\cdot:C\to \Delta(\Omega)$ by
\begin{align}
    \prob{P}_\alpha(A) &:= \mu (T_\alpha^{-1}(A))&\forall A\in \sigalg{F}
\end{align}
where $\RV{T}_\alpha:S\to F$ is the function $s\mapsto T(s,\alpha)$. $\prob{P}_\alpha$ is the pushforward of $T_\alpha$ under $\mu$.

Then 
\begin{align}
    \int_\Omega u(f)\prob{P}_\alpha(\mathrm{d}f) &= \int_S u \circ T_\alpha (s)\mu(\mathrm{d}s)\\
    &= \int_S u(T(s,\alpha))\mu(\mathrm{d}s)
\end{align}
\end{proof}

\subsubsection{Savage axioms}\label{sec:savage_axioms}

Careful analysis of Savage's theorem is outside the scope of this work, but given the relevant of Savage's representation theorem we will reproduce the axioms from \citet{savage_foundations_1954} with a small amount of commentary. Keep in mind that Savage's theorem establishes that the following are sufficient for representation with a probability set, not necessary, and furthermore the probability set representation of preferences satisfying these axioms is unique.

Given acts $C$, states $(S,\sigalg{S})$ and consequences $F$ and a map $T:S\times C\to F$, let all greek letters $\alpha,\beta$ etc. be elements of $C$. Savage's axioms are:
\begin{enumerate}[P1:]
    \item There is a complete preference relation $\preceq$ on $C$
    \begin{enumerate}[D1:]
        \item $\alpha\preceq \beta$ given $B\in \sigalg{S}$ if and only if $\alpha'\preceq \beta'$ for every $\alpha'$ and $\beta'$ such that $T(\alpha,s)=T(\alpha',s)$ for $s\in B$ and $T(\alpha',r)=T(\beta',r)$ for $r\not\in B$, and $\beta'\preceq \alpha'$ either for every such pair or for none.
    \end{enumerate}
    \item For every $\alpha,\beta$ and $B\in \sigalg{S}$, $\alpha\preceq \beta$ given $B$ or $\beta\preceq \alpha$ given $B$
    \begin{enumerate}[D2:]
        \item for $q,q'\in F$, $q\preceq q'$ if and only if $\alpha\preceq \alpha'$ where $T(\alpha,s)=q$ and $T(\alpha',s)=q'$ for all $s\in S$
        \item $B\in \sigalg{S}$ is null if and only if $\alpha\preceq \beta$ given B for every $\alpha,\beta\in C$
    \end{enumerate}
    \item If $T(\alpha,s)=q$ and $T(\alpha',s)=q'$ for every $s\in B$, $B\in \sigalg{S}$ non-null, then $\alpha\preceq \alpha'$ given $B$ if and only if $q\preceq q'$
    \begin{enumerate}[D4:]
        \item For $A,B\in \sigalg{S}$, $A\leqslant B$ if and only if $\alpha_A\preceq \alpha_B$ or $q\preceq q'$ for all $\alpha_A,\alpha_B\in C$, $q,q'\in F$ such that $T(\alpha_A,s) = q$ for $s\in A$, $T(\alpha_A,s')=q'$ for $s'\not\in A$, $T(\alpha_B,s)=q$ for $s\in B$, $T(\alpha_B,s')=q'$ for $s'\not\in B$. Read $\leqslant$ as ``is less probable than''
    \end{enumerate}
    \item For every $A,B\in\sigalg{S}$, $A\leqslant B$ or $B\leqslant A$
    \item For some $\alpha,\beta$, $\alpha\prec \beta$
    \item Suppose $\alpha\not\preceq \beta$. Then for every $\gamma$ there is a finite partition of $S$ such that if $\alpha'$ agrees with $\alpha$ and $\beta'$ agrees with $\beta$ except on some element $B$ of the partition, $\alpha'$ and $\beta'$ being equal to $\gamma$ on $B$, then $\alpha\not\preceq \beta'$ and $\alpha'\not\preceq \beta$
    \begin{enumerate}[D5:]
        \item $\alpha\preceq q$ for $q\in F$ given $B$ if and only if $\alpha\preceq \beta$ given $B$ where $T(\beta,s)=q$ for all $s\in S$
    \end{enumerate}
    \item If $\alpha\preceq T(\beta,s)$ given $B$ for every $s\in B$, then $\alpha\preceq \beta$ given $B$
    \begin{enumerate}[P7':]
        \item The proposition given by inverting every expression in D5 and P7
    \end{enumerate}
\end{enumerate}

Our initial view of decision problems was that the consequences $\Omega$ are a set of things we know how to rank and choices $C$ are the things we want to rank. This is not exactly Savage's setup -- he assumes a preference relation ranking ``acts'' $C$ to begin with. Furthermore, Savage also introduces a set of states $S$ and assumes that the set of acts corresponds to the set of all function $S\to \Omega$. Many decision problems might be able to be extended with states and the set of acts enriched so as to satisfy these requirements, but it is not obvious that this is always possible.

D1 formalises the idea of one act $\alpha$ being not preferred to another $\beta$ given the knowledge that the true state lies in the set $B$ (in short: ``given $B$'' or ``conditional on $B$''). P2 is sometimes called the ``sure thing principle'', as it implies the following: for any $\alpha, \beta$ if $\alpha$ is better than $\beta$ on some states and no worse on any other, then $\alpha\succ \beta$. In Savage's model, the ``likelihood'' that of any state cannot depend on the act chosen.

D4 + P4 defines the ``probability preorder'' $\leqslant$ on $(S,\sigalg{S})$ and assumes it is complete.

P5 is the requirement that the preference relation is non-trivial; not everything is equally desirable. This doesn't seem like it should be a practical requirement to me; we might hope that a model can distinguish between some of our options, but that doesn't mean we should assume it can. Savage claims that this requirement is ``innocuous'' because any exception must be trivial, but I'm not sure I agree.

P6 is a requirement of continuity; for any $\alpha\preceq \beta$, we can divide $S$ finely enough to squeeze a ``small slice'' of any third outcome $\gamma$ into the gap between the two.

P7 in combination with the other axioms forces preferences to be bounded.

\subsection{Jeffrey's decision theory}

Jeffrey's decision theory is an alternative to Savage's that starts from a different set of assumptions. One of the key differences is in what is assumed at the outset: where Savage assumes a set of states $S$, acts $C$ and consequences $\Omega$, Jeffrey's theory only considers a single space $\underline{\sigalg{F}}$, which is a complete atomless boolean algebra. Elements of $\underline{\sigalg{F}}$ are said to be propositions, although the structure of $\underline{\sigalg{F}}$ means we can't understand it as, for example, a set of propositions regarding the result of a particular measurement procedure (Section \ref{sec:variable}). The theory is set out in \citet{jeffrey_logic_1990}, and the key representation theorem proved in \citet{bolker_functions_1966}.

Recall that our fundamental problem is relating a set $C$ of things we can choose to a set $F$ of things we can compare. Jeffrey's theory uses a different strategy to accomplish this than Savages'; where identifies a set of acts $C$ with all functions $S\to F$ and proposes axioms that constrain a preference relation on $C$, Jeffrey assumes that choices are elements of the algebra $\underline{\sigalg{F}}$, along with propositions that do not correspond to choices. Jeffrey's axioms pertain to a preference relation on $\underline{\sigalg{F}}$. The ultimate result is, for our purposes, very similar.

\begin{theorem}\label{th:bolker_jeffrey}
Suppose there is a complete atomless Boolean algebra $\underline{\sigalg{F}}$ with a preference relation $\preceq$. If $\preceq$ satisfies the \emph{Bolker axioms} (Section \ref{sec:bolker_axioms}) then there exists a desirability function $\text{des}:\underline{\sigalg{F}}\to\mathbb{R}$ and a probability distribution $\mu\in \Delta(\underline{\sigalg{F}})$ such that for $A,B\in \underline{\sigalg{F}}$ and finite partition $D_1,...,D_n\in \underline{\sigalg{F}}$:
\begin{align}
    (A \preceq B) \iff \sum_{i}^n \text{des}(D_i) \mu(D_i|A) \leq \sum_{i}^n \text{des}(D_i) \mu(D_i|B) \label{eq:ev_dec_theory}
\end{align}
where $\mu(D_i|A):=\frac{\mu(A\cap D_i)}{\mu(A)}$ for $\mu(A)>0$, undefined otherwise.
\end{theorem}

\begin{proof}
\citet{bolker_functions_1966})
\end{proof}

As mentioned, in Jeffrey's theory the \emph{choices} $C$ are a subset of $\underline{\sigalg{F}}$. Thus we can deduce from a Jeffrey model a function $C\to \Delta(\underline{\sigalg{F}})$ that ``represents the consequences of choices'' in the sense of Theorem \ref{th:jeffrey_with_choices}.

\begin{theorem}\label{th:jeffrey_with_choices}
Suppose there is a complete atomless Boolean algebra $\underline{\sigalg{F}}$ with a preference relation $\preceq$ that satisfies the Bolker axioms, and a set of choices $C$ over which a preference relation is sought with $\mu(\alpha)>0$ for all $\alpha\in C$. Then there is a function $\prob{P}_\cdot:C\to \Delta(\underline{\sigalg{F}})$ such that for any $\alpha,\alpha'\in C$ and finite partition $D_1,...,D_n\in \underline{\sigalg{F}}$:
\begin{align}
    \alpha \preceq \alpha'\iff \sum_{i}^n \text{des}(D_i) \prob{P}_\alpha(D_i) \leq \sum_{i}^n \text{des}(D_i) \prob{P}_{\alpha'}(D_i)\label{eq:ev_with_choices}
\end{align}
Where $\mu$ and $\mathrm{des}$ are as in Theorem \ref{th:bolker_jeffrey}
\end{theorem}

\begin{proof}
Define $\prob{P}_\cdot$ by $\alpha\mapsto \mu(\cdot|\alpha)$. Then Equation \ref{eq:ev_with_choices} follows from Equation \ref{eq:ev_dec_theory}.
\end{proof}

\subsubsection{Bolker axioms}\label{sec:bolker_axioms}

$\underline{\sigalg{F}}$ a complete, atomless Boolean algebra with the impossible proposition. An example of such a set is constructed from the set of Lebesgue measurable sets on $[0,1]$ identifying any two sets that differ by a set of measure zero identified \citet{bolker_simultaneous_1967}. This is not a $\sigma$-algebra.
 
\begin{enumerate}[A1:]
    \item $\preceq$ is a complete preference relation
    \item $\underline{\sigalg{F}}$ is a complete, atomless Boolean algebra with the impossible proposition removed
    \item For $A,B\in \underline{\sigalg{F}}$, if $A\cap B=\emptyset$, then
    \begin{enumerate}[a)]
        \item If $A\succ B$ then $A\succ A\cup B \succ B$
        \item If $A\sim B$ then $A\sim A\cup B \sim B$
    \end{enumerate}
    \item Given $A\cap B=\emptyset$ and $A\sim B$, if $A\cup G\sim B\cup G$ for some $G$ where $A\cap G=B\cap G=\emptyset$ and $G\not\sim A$, then $A\cup G\sim B\cup G$ for every such $G$
    \begin{enumerate}[D1:]
        \item The supremum (infimum) of a subset $W\subset \underline{\sigalg{F}}$ is a set $G$ ($D$) such that for all $A\in W$, $G\subset A$ ($A\subset D$), and for any $E$ that also has this property, $G\subset E$ ($E\subset D$)
    \end{enumerate}
    \item Given $W:= \{W_i\}_{i\in M\subset \mathbb{N}}$ with $i<j\implies W_j\subset W_i$ and $W\subset \underline{\sigalg{F}}$ with supremum $G$ (infimum $D$), whenever $A\prec G \prec B$ ($A\prec D\prec B$) then there exists some $k\in M$ such that $i\geq k$ ($i\leq k$) implies $A\prec W_i \prec B$.
\end{enumerate}

Like Savage's theory, A1 requires the preference relation to be complete.

A3 is the assumption that the desirability of disjunctions of events lies between the desirability of each event; it is sometimes called ``averaging''. It notably rules out the following: if $A\succ B$ we cannot have $A\cup B\sim A$. In the Jeffrey-Bolker theory, propositions all have positive probabilities.

A4 allows a probability order to be defined on $\underline{\sigalg{F}}$. The conditions $A\cap B=\emptyset$, $A\sim B$, $A\cup G\sim B\cup G$ for some $G$ where $A\cap G=B\cap G=\emptyset$ and $G\not\sim A$ can be seen as a test for $A$ and $B$ being ``equally probable''. A4 requires that if $A$ and $B$ are rated as equally probable by one such test, then they are rated as equally probable by all such tests.

A5 is an axiom of continuity.

\subsection{Causal decision theory}

Causal decision theory was developed after both Jeffrey's and Savage's theory. A number of authors \citet{lewis_causal_1981,skyrms_causal_1982} felt that Jeffrey's theory erred by treating the consequences of a choice as an ``ordinary conditional probability''. \citet{lewis_causal_1981} suggested that causal decision theory can be used to evaluate choices when we are given a set $\Omega$ of consequences over which preferences are known, a set $C$ of choices and a set $H$ of dependency hypotheses (the letters have been changed to match usage in this work; in the original the consequences were called $S$, the choices $A$ and the dependency hypotheses $H$). Choices are then evaluated according to the causal decision rule. We have taken the liberty to state Lewis' rule in the language of the present work.

\begin{definition}[Causal decision rule]
Given a set $C$ of choices, sample space $(\Omega,\sigalg{F})$, variables $\RV{H}:\Omega\to H$ (the \emph{dependency hypothesis}) and $\RV{S}:\Omega\to S$ (the \emph{consequence}) and a utility $u:\Omega\to \mathbb{R}$, the \emph{causal utility} of a choice $\alpha\in C$ is given by
\begin{align}
    U(\alpha) := \int_S \int_H u(s) \prob{P}_\alpha^{\RV{S}|\RV{H}}(\mathrm{d}s|h) \prob{P}_C^{\RV{H}}(\mathrm{d}h)\label{eq:lewis_cdt}
\end{align}
For some probabilistic function $\prob{P}_\cdot:C\to \Delta(\Omega)$.
\end{definition}

The reasons why Lewis wanted to introduce dependency hypothesis and modify Jeffrey's rule to Equation \ref{eq:lewis_cdt} are controversial and do not come up in this work. However, causal decision theory is still relevant to this work in two ways: firstly, once again is a probabilistic function $\prob{P}_\cdot:C\to \Delta(\Omega)$. Secondly, causal decision theory introduces the notion of the dependency hypothesis $\RV{H}$. The dependency hypothesis is similar to the state in Savage's theory, however Lewis does not require a deterministic map from dependency hypotheses to consequences, nor does he require a choice to correspond to every possible function from dependency hypotheses to states.

Dependency hypotheses are quite an important idea in causal reasoning. Together Lewis' decision rule connect the theory of probability sets with \emph{statistical decision theory}, as Section \ref{sec:sdt} will show. Chapter \ref{ch:evaluating_decisions} goes into considerable detail concerning the question of when probability sets support certain types of dependency hypothesis. While they are typically not explicitly represented in common frameworks for causal inference, Chapter \ref{ch:other_causal_frameworks} discusses how dependency hypotheses are often implicit in these approaches, and shows how they can be made explicit.

\subsection{Statistical decision theory}\label{sec:sdt}

Statistical decision theory (SDT), created by \citet{wald_statistical_1950}, predates all of the decision theories discussed above. Savage's theory appears to have developed in part to explain some features of SDT \citet{savage_theory_1951}, and Jeffrey's theory and subsequent causal decision theories were in turn influenced by Savage's decision theory. While the later decision theories were concerned with articulating why their theory fit the role of a theory for rational decision under uncertainty, Wald focused much more on the mathematical formalism and solutions to statistical problems. Statistical decision theory introduced many fundamental ideas that have since entered the ``water supply'' of machine learning theory, such as \emph{decision rules} and \emph{risk} as a measure of the quality of a decision rule.

In contrast to the later decision theories, SDT has no explicit representation of the ``consequences'' of a decision. Rather, it is assumed that a loss function is given that maps decisions and hypotheses directly to a loss, which is a kind of desirability score similar to a utility (although it is minimised rather than maximised).

\begin{definition}[Statistical decision problem]
A statistical decision problem (SDP) is a tuple $(X, H, D, l, \prob{P}_\cdot)$ where $(X,\sigalg{X})$ is a set of outcomes, $(H,\sigalg{H})$ is a set of hypotheses, $(D,\sigalg{D})$ is a set of decisions, $l:D\times H\to \mathbb{R}$ is a loss function and $\prob{P}_\cdot:H\kto X$ is a Markov kernel from hypotheses to to outcomes.
\end{definition}

Statistical decision theory is concerned with the selection of \emph{decision rules}, rather than the selection of decisions directly. A decision rule maps observations to decisions, and may be deterministic or stochastic.

\begin{definition}[Decision rule]
Given a statistical decision problem $(X, H, D, l, \prob{P}_\cdot)$, a decision rule is a Markov kernel $\kernel{D}_\alpha:\Omega\kto D$.
\end{definition}

Because decision rules in SDT play the role of what we call \emph{choices}, we denote the set of all available decision rules by $C$. A further feature of SDT that is unlike the later decision theories is that SDT does not offer a single rule for assessing the desirability of any choice in $C$. Instead, it offers a definition of the risk, which assesses the desirability of a choice \emph{relative to a particular hypothesis}. The risk function completely characterises the problem of choosing a decision function. Two different rules are for turning this ``intermediate assessment'' into a final assessment of the available choices - Bayes optimality and minimax optimality. Bayes optimality reuquires a prior over hypotheses, while minimax optimality does not.

\begin{definition}[SDP Risk]\label{def:risk}
Given a statistical decision problem $(X, H, D, l, \prob{P}_\cdot)$ and decision functions $C$, the \emph{risk} functional $R:C\times H\to \mathbb{R}$ is defined by
\begin{align}
    R(\kernel{D}_\alpha,h):= \int_X \int_D l(d,h) \kernel{D}_\alpha(\mathrm{d}d|f)\kernel{P}_h(\mathrm{d}f)
\end{align}
\end{definition}

It is possible to find risk functions in problems that aren't SDPs. The definitions of Bayes and Minimax optimality still apply to risk functions obtained on other manners. Thus Bayes optimality and minimax optimality are defined in terms of risk functions in general, not SDP risk functions.

\begin{definition}[Bayes risk]
Given decision functions $C$, hypotheses $(H,\sigalg{H})$, risk $R:C\times H\to \mathbb{R}$ and prior $\mu\in \Delta(H)$, the $\mu$-\emph{Bayes risk} is
\begin{align}
    R_\mu(\kernel{D}_\alpha) &:= \int_{H} R(\kernel{D}_\alpha,h)\mu(\mathrm{d}h)
\end{align}
\end{definition}

\begin{definition}[Bayes optimal]
Given decision functions $C$, hypotheses $(H,\sigalg{H})$, risk $R:C\times H\to \mathbb{R}$ and prior $\mu\in \Delta(H)$, $\alpha\in C$ is $\mu$-Bayes optimal if
\begin{align}
    R_\mu(\kernel{D}_\alpha) &= \inf_{\alpha'\in C} R_{\mu}(\kernel{D}_{\alpha'})
\end{align}
\end{definition}

\begin{definition}[Minimax optimal]
Given decision functions $C$, hypotheses $(H,\sigalg{H})$, risk $R:C\times H\to \mathbb{R}$, a \emph{minimax decision function} is any decision function $\kernel{D}_\alpha$ satisfying
\begin{align}
    \sup_{h\in H}  R(\kernel{D}_\alpha,h) &= \inf_{\alpha' in C} \sup_{h\in H} R(\kernel{D}_{\alpha'},h)
\end{align}
\end{definition}

\subsubsection{From consequences to statistical decision problems}\label{sec:cons_to_sdp}

Statistical decision theory ignores the notion of general consequences of choices; the only ``consequence'' in the theory is the loss incurred by a particular decision under a particular hypothesis. The kinds of probability set models studied here probabilistically map decisions to consequences, and the set of consequences is understood to have a utility function to allow for assessment of the desirability of different choices via the principle of expected utility. Not every probability set model induces a statistical decision problem in this manner. A family of models that does are what we call \emph{conditionally independent see-do models}. These models feature observations (the ``see'' part) along with decisions and consequences (the ``do'' part), and the observations come ``before'' the decisions (hence see-do). Examples of this type of model will be encountered again in Chapters \ref{ch:evaluating_decisions} and \ref{ch:other_causal_frameworks}. Furthermore, there is a hypothesis such that consequences are assumed to be independent of observations conditional on the decision and the hypothesis. This is why they are qualified as ``conditionally independent'' see-do models.
\begin{definition}[See-do model]\label{def:see_do_model}
A probability set model of a statistical decision problem, or a \emph{see-do model} for short, is a tuple $(\prob{P}_{C\times H}, \RV{X},\RV{Y},\RV{D})$ where $\prob{P}_{C\times H}$ is a probability set indexed by elements of $C\times H$ on $(\Omega,\sigalg{F})$, $\RV{X}:\Omega\to X$ are the observations, $\RV{Y}:\Omega\to Y$ are the consequences and $\RV{D}:\Omega\to D$ are the decisions. $\prob{P}_{C\times H}$ must observe the following conditional independences:
\begin{align}
    \RV{X}&\CI^e_{\prob{P}_{C\times H}} \RV{C}|\RV{H}\\
    \RV{D}&\CI^e_{\prob{P}_{C\times H}} \RV{H}|\RV{C}
\end{align}
where $\RV{C}:C\times H\to C$ and $\RV{H}:C\times H\to H$ are the respective projections (see Definition \ref{def:eci_orig} for the definition of extended conditional independence).
\end{definition}
\begin{definition}[Conditionally independent see-do model]\label{def:ci_see_do_model}
A conditionally independent see-do model is a see do model $(\prob{P}_{C\times H}, \RV{X},\RV{Y},\RV{D})$ where the following additional conditional independence holds:
\begin{align}
    \RV{Y}&\CI^e_{\prob{P}_{C\times H}} (\RV{X,C})|(\RV{D},\RV{H})
\end{align}
\end{definition}
We assume that a utility function is available depending on the consequence $\RV{Y}$ only, and identify the loss with the negative expected utility, conditional on a particular decision and hypothesis.
\begin{definition}[Induced loss]
Given a see-do model $(\prob{P}_{C\times H}, \RV{X},\RV{Y},\RV{D})$ and a utility $u:Y\to \mathbb{R}$, the induced loss $l:D\times H\to \mathbb{R}$ is defined as
\begin{align}
    l(d,h)&:= -\int_Y u(y) \prob{P}_{C\times\{h\}}^{\RV{Y}|\RV{D}}(\mathrm{d}y|d)
\end{align}
\end{definition}
where the uniform conditional $\prob{P}_{C\times\{h\}}^{\RV{Y}|\RV{D}}$'s existence is guaranteed by $\RV{Y}\CI^e_{\prob{P}_{C\times H}} (\RV{X,C})|(\RV{D},\RV{H})$.

A see-do model induces a set of decision functions: for each $\alpha\in C$, there is an associated probability distribution $\prob{P}_\alpha^{\RV{D}|\RV{X}}$. Using the above definition of loss, the expected loss of a decision function in a conditionally independent see-do model induces a risk function identical to the SDP risk.
\begin{theorem}[Induced SDP risk]\label{th:ind_risk}
Given a conditionally independent see-do model $(\prob{P}_{C\times H}, \RV{X},\RV{Y},\RV{D})$ along with a utility $u:Y\to \mathbb{R}$, the expected utility for each choice $\alpha\in C$ and hypothesis $h\in H$ is equal to the negative SDP risk of the associated decision rule $\prob{P}_\alpha^{\RV{D}|\RV{X}}$ and hypothesis $h$.
\begin{align}
    \prob{P}_{\alpha,h}^{\RV{Y}}u &= -R(\prob{P}_{\{\alpha\}\times H}^{\RV{D}|\RV{X}},h)
\end{align}
\end{theorem}

\begin{proof}
The expected utility given $\alpha$ and $h$ is
\begin{align}
    \int_Y u(y)\prob{P}_{\alpha,h}^{\RV{Y}}(\mathrm{d}y) &= \int_Y  \int_D \int_{X} u(y)\prob{P}_{\alpha,h}^{\RV{Y}|\RV{DX}}(\mathrm{d}y|d,x)\prob{P}_{\alpha,h}^{\RV{D}|\RV{X}}(\mathrm{d}d|x)\prob{P}_{\alpha,h}^{\RV{X}}(\mathrm{d}x) \\
    &= \int_X  \int_D \int_{Y} u(y) \prob{P}_{\alpha,h}^{\RV{Y}|\RV{D}}(\mathrm{d}y|d)\prob{P}_{\alpha,h}^{\RV{D}|\RV{X}}(\mathrm{d}d|x)\prob{P}_{\alpha,h}^{\RV{X}}(\mathrm{d}x)\label{eq:because_of_ci}\\
    &= \int_{X} \int_D \int_Y u(y) \prob{P}_{C\times\{h\}}^{\RV{Y}|\RV{D}}(\mathrm{d}y|d)\prob{P}_{\{\alpha\}\times H}^{\RV{D}|\RV{X}}(\mathrm{d}d|x)\prob{P}_{C\times\{h\}}^{\RV{X}}(\mathrm{d}x)\\
     &= -\int_D\int_X l(d,h)\prob{P}_{\{\alpha\}\times H}^{\RV{D}|\RV{X}}(\mathrm{d}d|x)\prob{P}_{C\times\{h\}}^{\RV{X}}(\mathrm{d}x)\\
    &= -R(\prob{P}_{\{\alpha\}\times H}^{\RV{D}|\RV{X}},h)
\end{align}
where Equation \ref{eq:because_of_ci} follows from $\RV{Y}\CI^e_{\prob{P}_{C\times H}} (\RV{X,C})|(\RV{D},\RV{H})$, the uniform conditional $\prob{P}_{\{\alpha\}\times H}^{\RV{D}|\RV{X}}$ exists due to $\RV{D}\CI^e_{\prob{P}_{C\times H}} \RV{H}|\RV{C}$ and the uniform conditional $\prob{P}_{C\times\{h\}}^{\RV{X}}$ exists due to $\RV{X}\CI^e_{\prob{P}_{C\times H}} \RV{C}|\RV{H}$.
\end{proof}

Theorem \ref{th:ind_risk} does \emph{not} hold for general see-do models. General see-do models allow for the utility to depend on $\RV{X}$ even after conditioning on $\RV{D}$ and $\RV{H}$, while the form of the loss function in SDT forces no direct dependence on observations. The generic ``see-do risk'' (Definition \ref{def:see_do_risk}) provides a notion of risk for the more general case, while Theorem \ref{th:ind_risk} shows it reduces to SDP risk in the case of conditionally independent see-do models with a utility that depends only on the consequences $\RV{Y}$.
\begin{definition}[See-do risk]\label{def:see_do_risk}
Given a see-do model $(\prob{P}_{C\times H}, \RV{X},\RV{Y},\RV{D})$ along with a utility $u:X\times Y\to \mathbb{R}$, the \emph{see-do risk} $R:C\times H\to \mathbb{R}$ is given by
\begin{align}
    R(\alpha,h) &:= -\prob{P}_{\alpha,h}^{\RV{XY}}u&\forall \alpha\in C,h\in H
\end{align}
\end{definition}

Section \ref{sec:modelling_decision_problems} noted that two types of probability set model are considered: probability sets $\prob{P}_C$ indexed by choices alone, and probability sets $\prob{P}_{C\times H}$ jointly indexed by choices and hypotheses. See-do models are an instance of the second kind, jointly indexed by choices and hypotheses. Bayesian see-do models are of the former type, indexed by choices alone. A see-do model $(\prob{P}_{C\times H}, \RV{X},\RV{Y},\RV{D})$ and a prior over hypotheses $\mu\in \Delta(H)$ can by combined to form a Bayesian see-do model, and under the right conditions the risk of the Bayesian model reduces to the Bayes risk of the original see-do model.

\begin{definition}[Bayesian see-do model]
A Bayesian see-do model is a tuple $(\prob{P}_{C}, \RV{X},\RV{Y},\RV{D},\RV{H})$ where $\prob{P}_C$ is a probability set on $(\Omega,\sigalg{F})$, $\RV{X}:\Omega\to X$ are the observations, $\RV{Y}:\Omega\to Y$ are the consequences, $\RV{D}:\Omega\to D$ are the decisions and $\RV{H}:\Omega\to H$ is the hypothesis. $\prob{P}_C$ must observe the following conditional independences:
\begin{align}
    \RV{X}&\CI^e_{\prob{P}_{C}} \RV{C}|\RV{H}\\
    \RV{D}&\CI^e_{\prob{P}_{C}} \RV{H}|\RV{C}\\
    \RV{H}&\CI^e_{\prob{P}_C} \RV{C}
\end{align}
\end{definition}

\begin{definition}[Induced Bayesian see-do model]
Given a see-do model $(\prob{P}_{C\times H}, \RV{X},\RV{Y},\RV{D},\RV{H})$ on $(\Omega,\sigalg{F})$ and a prior $\mu\in \Delta(H)$, the induced Bayesian see-do model $\prob{P}_C$ on $(\Omega\times H,\sigalg{F}\otimes \sigalg{H})$ is
\begin{align}
    \prob{P}_C(A) &= \int_{\RV{H}^{-1}(A)} \prob{P}_{C\times \{h\}}(\Pi_{\Omega}^{-1}(A))\mu(\mathrm{d}h)&\forall A\in \sigalg{F}\otimes\sigalg{H}
\end{align}
Where $\Pi_\Omega:\Omega\times H\to \Omega$ is the projection onto $\Omega$.
\end{definition}

\begin{theorem}[Induced SDP Bayes risk]
Given a conditionally independent see-do model $(\prob{P}_{C}, \RV{X},\RV{Y},\RV{D},\RV{H})$ along with a utility $u:Y\to \mathbb{R}$ and a prior $\mu\in \Delta(H)$, the expected utility for each choice $\alpha\in C$ under the induced Bayesian see-do model is equal to the negative $\mu$-Bayes risk of that decision rule.
\end{theorem}

\begin{proof}
First, note that $h\mapsto \prob{P}_{C\times \{h\}}^{\RV{Y}|\RV{XD}}$ is a version of $\prob{P}_C^{\RV{Y}|\RV{XD}}$ and hence $\RV{Y}\CI^e_{\prob{P}_C} (\RV{X},\RV{C})|(\RV{H},\RV{D})$, a property it inherits from the underlying see-do model.

Also, note that $\prob{P}_C^{\RV{H}}=\mu$, by construction.

The expected utility of $\alpha\in C$ is 
\begin{align}
    \prob{P}_\alpha^{\RV{Y}} u &= \int_Y u(y) \prob{P}_{\alpha}^{\RV{Y}}(\mathrm{d}y) \\
    &= \int_Y  \int_D \int_{X} \int_H u(y)\prob{P}_{\alpha}^{\RV{Y}|\RV{DXH}}(\mathrm{d}y|d,x,h)\prob{P}_{\alpha}^{\RV{D}|\RV{XH}}(\mathrm{d}d|x,h)\prob{P}_{\alpha}^{\RV{X}|\RV{H}}(\mathrm{d}x|h)\prob{P}_\alpha^{\RV{H}}(\mathrm{d}h) \\
    &= \int_X  \int_D \int_{Y} \int_H u(y) \prob{P}_{\alpha}^{\RV{Y}|\RV{DH}}(\mathrm{d}y|d,h)\prob{P}_{\alpha}^{\RV{D}|\RV{X}}(\mathrm{d}d|x)\prob{P}_{\alpha}^{\RV{X}|\RV{H}}(\mathrm{d}x|h)\prob{P}_\alpha^{\RV{H}}(\mathrm{d}h)\label{eq:because_of_ci}\\
    &=  \int_X  \int_D \int_{Y} \int_H u(y) \prob{P}_{C}^{\RV{Y}|\RV{DH}}(\mathrm{d}y|d,h)\prob{P}_{\alpha}^{\RV{D}|\RV{X}}(\mathrm{d}d|x)\prob{P}_{C}^{\RV{X}|\RV{H}}(\mathrm{d}x|h)\mu(\mathrm{d}h)\\
     &= -\int_D\int_X\int_H l(d,h)\prob{P}_{\alpha}^{\RV{D}|\RV{X}}(\mathrm{d}d|x)\prob{P}_{C}^{\RV{X}|\RV{H}}(\mathrm{d}x|h)\mu(\mathrm{d}h)\\
    &= -\int_H R(\prob{P}_{\alpha}^{\RV{D}|\RV{X}},h)\mu(\mathrm{d}h)\\
    &= -R_{\mu}(\prob{P}_{\alpha}^{\RV{D}|\RV{X}})
\end{align}
\end{proof}

\subsubsection{Complete class theorem}\label{sec:cc_theorem}

The \emph{complete class theorem} establishes that, under certain conditions, any \emph{admissible} decision rule (Definition \ref{def:admissible_decision}) for a see-do model $\prob{P}_{C\times H}$ with a utility $u$ must minimise the Bayes risk for a Bayesian model constructed from $\prob{P}_{C\times H}$ and a prior over hypotheses $\mu\in \Delta(H)$. This can be interpreted in a similar way to the decision theoretic representation discussed above: if you accept that the relevant assumptions apply to the decision problem at hand, than there is a Bayesian see-do model along with $u$ that captures the important features of this problem. The assumptions are that a see-do model $\prob{P}_{C\times H}$ with a utility $u$ that satisfies the relevant conditions is available, and that the principle used to evaluate decision rules should yield an admissible decision rule (though it may also be desired to satisfy other properties as well).

If there are auxiliary requirements for choosing the decision rule, the complete class theorem does not prove that it is easy to find a Bayesian model that will yield rules satisfying these requirements.

See-do models (and statistical decision problems) have a lot of structure -- the loss, the assumption that consequences are conditionally independent of observations -- that is not actually critical to the complete class theorem. The complete class theorem is a theorem about risk function $R:C\times H\to \mathbb{R}$ that have certain properties. Theorem \ref{th:ind_risk} shows one way that a risk function can be derived from a see-do model along with a utility. However, it is also possible to derive risk functions from other classes of probability set models with utilities, and if the resulting risk function satisfies the appropriate conditions then the complete class theorem also applies to that class of model. For example, the complete class theorem also applies to see-do models without the assumption that consequences are conditionally independent of observations given the hypothesis and the decision, even though in this case the risk calculation is not the standard calculation for a statistical decision problem.

\begin{definition}[Risk function]
Given a set of choice $C$ and a set of hypotheses $H$, a risk function is a map $R:H\times C\to \mathbb{R}$.
\end{definition}

If the second set $H$ were, instead of hypotheses about nature, a set of options available to a second player playing a game, then a ``risk function'' defines a two-player zero-sum game \citet{toutenburg_ferguson_1967}.

\begin{definition}[Admissible choice]\label{def:admissible_decision}
Given a risk function $R:C\times H\to \mathbb{R}$, a choice r $\alpha\in C$ dominates a choice $\alpha'\in C$ if for all $h\in H$, $R(\alpha,h)\leq R(\alpha',h)$ and for at least on $h^*$, $R(\alpha,h)<R(\alpha,h^*)$. An \emph{admissible choice} is a choice $\alpha\in C$ such that there is no $\alpha'\in C$ dominating $\alpha$.
\end{definition}

\begin{definition}[Complete class]
A \emph{complete class} is any $B\subset C$ such that, for any $\alpha'\not in B$ there is some $\alpha\in B$ that dominates $\alpha'$. A \emph{minimal complete class} is a complete class $B$ such that no proper subset of $B$ is complete
\end{definition}

\begin{theorem}
If a minimal complete class $B\subset C$ exists then $B$ is the set consisting of all the admissible decision rules.
\end{theorem}

\begin{proof}
See \citet[Theorem 2.1]{toutenburg_ferguson_1967}
\end{proof}

\begin{definition}[Risk set]
Given a finite set of hypotheses $H$, a set of choices $C$ and a risk function $R:C\times H\to \mathbb{R}$, the risk set is the subset of $\mathbb{R}^{|H|}$ given by
\begin{align}
    S := \{(R(\alpha,h))_{h\in H}|\alpha\in C\}
\end{align}
\end{definition}

\begin{theorem}[Complete class theorem]
Given a risk function $R:C\times H\to \mathbb{R}$, if the risk set S is convex, bounded from below and closed downwards, and H is finite, then the set of Bayes optimal choices is a minimal complete class.
\end{theorem}

\begin{proof}
See \citet[~Theorem 2.10.2]{toutenburg_ferguson_1967}
\end{proof}

Two examples of the application of the complete class theorem will be presented (Examples \ref{ex:cc_sdt} and \ref{ex:cc_nonsdt}). In order to explain them, we need a few lemmas.

\begin{lemma}\label{lem:convex_closed}
Given $H$ and $C$ both finite and a risk function $R:C\times H\to \mathbb{R}$ and an associated probability set $\prob{P}_C$ on $(\Omega,\sigalg{F})$, $\Omega$ finite, if the function
\begin{align}
    \prob{P}_{\alpha,h}^{\RV{D}|\RV{X}}\mapsto R(\alpha,h)
\end{align}
is linear and 
\begin{align}
    Q:= ((\prob{P}_{\alpha,h}^{\RV{D}|\RV{X}})_{h\in H})_{\alpha\in C}
\end{align}
is convex closed, then the risk set $S$ is convex closed.
\end{lemma}

\begin{proof}
By linearity of 
\begin{align}
    \prob{P}_{\alpha,h}^{\RV{D}|\RV{X}}\mapsto R(\alpha,h)
\end{align}
we also have linearity of 
\begin{align}
    (\prob{P}_{\alpha,h}^{\RV{D}|\RV{X}})_{h\in H}\mapsto (R(\alpha,h))_{h\in H}
\end{align}
Furthermore, $Q$ is bounded when viewed as an element of $\mathbb{R}^{\Omega\times H\times C}$, and so $S$ is the linear image of a compact convex set, and is therefore also compact convex.
\end{proof}

\begin{lemma}\label{lem:linear}
For a see-do model $(\prob{P}_{C\times H}, \RV{X},\RV{Y},\RV{D},\RV{H})$ with utility $u:X\times Y\to \mathbb{R}$, the map
\begin{align}
    \prob{P}_{\alpha,h}^{\RV{D}|\RV{X}}\mapsto R(\alpha,h)
\end{align}
is linear.
\end{lemma}

\begin{proof}
By definition,
\begin{align}
    R(\alpha,h) &= - \prob{P}_{\alpha,h}^{\RV{XY}} u\\
    &= -\prob{P}_{C\times\{h\}}^{\RV{X}} \odot \prob{P}_{\alpha\times h}^{\RV{D}|\RV{X}} \odot \prob{P}_{C\times\{h\}}^{\RV{Y}|\RV{DX}} u
\end{align}
Which is a composition of kernel products involving $\prob{P}_{\alpha\times H}^{\RV{D}|\RV{X}}$, and kernel products are linear, hence this function is linear.
\end{proof}

The preceding theorem does \emph{not} hold for a utility defined on $\Omega$ rather than on $X\times Y$. In this case we have instead
\begin{align}
    -\prob{P}_{C\times\{h\}}^{\RV{X}} \odot \prob{P}_{\alpha\times h}^{\RV{D}|\RV{X}} \odot \prob{P}_{\alpha,h}^{\Omega|\RV{DX}} u
\end{align}
where $\alpha$ appears twice on the right hand side, rendering the map nonlinear.

\begin{lemma}\label{lem:all_kernels_is_convex_hull}
For finite $X$ and $D$, the set of all Markov kernels $X\kto D$ is convex closed.
\end{lemma}

\begin{proof}
From \citet{blackwell_theory_1979}, the set of all Markov kernels $X\kto D$ is the convex hull of the set of all deterministic Markov kernels $X\kto D$. There are a finite number of deterministic Markov kernels, and so the convex hull of this set is closed.
\end{proof}

\begin{example}\label{ex:cc_sdt}
Suppose we have a conditionally independent see-do model $(\prob{P}_{C}, \RV{X},\RV{Y},\RV{D},\RV{H})$ along with a bounded utility $u:Y\to \mathbb{R}$ where $H,D,X$ and $Y$ are all finite, and $\{\prob{P}_\alpha^{\RV{D}|\RV{X}}|\alpha\in C\}$ is the set of all Markov kernels $X\kto D$. Then the risk set is convex and closed downwards, and so the set of Bayes optimal choices is exactly the set of admissible choices.

The boundedness of the risk set $S$ follows from the boundedness of the utility $u$; if $u$ is bounded above by $k$, then $S$ is bounded below in every dimension by $-k$.

The fact that $S$ is convex and closed follows from Lemmas \ref{lem:convex_closed}, \ref{lem:linear} and \ref{lem:all_kernels_is_convex_hull}.
\end{example}

\begin{example}\label{ex:cc_nonsdt}
As before, but suppose we have the see-do model is not conditionally independent. Because none of the lemmas \ref{lem:convex_closed}, \ref{lem:linear} and \ref{lem:all_kernels_is_convex_hull} made use of the conditional independence assumption, the risk set is still convex and closed downwards and so the set of Bayes optimal choices is also exactly the set of admissible choices.
\end{example}

\section{Variables}\label{sec:variable}

In probability theory, it is standard to assume the existence of a probability space $(\mu,\Omega,\sigalg{F})$ and to define \emph{random variables} as measurable functions from $(\Omega,\sigalg{F})$ to $(\mathbb{R},\mathcal{B}(\mathbb{R}))$. However, variables aren't \emph{just} functions -- they're also typically understood to correspond to some measured aspect of the real world. For example, \citet{pearl_causality:_2009} offers the following two, purportedly equivalent, definitions of variables:
\begin{quote}
By a \emph{variable} we will mean an attribute, measurement or inquiry that may take on one of several possible outcomes, or values, from a specified domain. If we have beliefs (i.e., probabilities) attached to the possible values that a variable may attain, we will call that variable a random variable.
\end{quote}

\begin{quote}
This is a minor generalization of the textbook definition, according to which a random variable is a mapping from the sample space (e.g., the set of elementary events) to the real line. In our definition, the mapping is from the sample space to any set of objects called ``values,'' which may or may not be ordered.
\end{quote}

However, these are actually two different things. The first is a \emph{measurement}, which is something we can do in the real world that produces as a result an element of a mathematical set. The second is a \emph{function}, a purely mathematical object with a domain and a codomain and a mapping from the former into the latter. Measurement procedures play the extremely important role of ``pointing to the parts of the world'' that the model addresses.

The general scheme considered in this work is to assume that there is a collection of  ``complete measurement procedure'' $\proc{S}_\alpha$, one for each choice $\alpha\in C$. $\proc{S}_\alpha$ is considered to be the procedure that measures all quantities of interest, and any subprocedure corresponding to a particular quantity of interest reconstructed from the result of $\proc{S}$ by applying a function to its result. The function $\RV{X}$ that, when applied to the result of $\proc{S}$, yields the result of a measurement subprocedure $\proc{X}$ is the \emph{variable} associated with the measurement procedure $\proc{X}$. In this way, a variable $\RV{X}$ -- which is by itself just a mathematical function -- is associated with a measurement procedure in the real world.

\subsection{Variables and measurement procedures}

Consider Newton's second law in the form $\RV{F}=\RV{MA}$. This model relates ``variables'' $\RV{F}$, $\RV{M}$ and $\RV{A}$. As \citet{feynman_feynman_1979} noted, in order to understand this law, some pre-existing understanding of force, mass and acceleration is required. In order to offer a numerical value for the net force on a given object is, even the most knowledgeable physicist will have to go and do a measurement, which involves interacting with the object in some manner that cannot be completely mathematically specified, and which will return a numerical value that will be taken to be the net force.

In order to make sense of the equation $\RV{F}=\RV{MA}$, it must be understood relative to some measurement procedure $\proc{S}$ that simultaneously measures the force on an object, its mass and its acceleration, which can be recovered by the functions $\RV{F}$, $\RV{M}$ and $\RV{A}$ respectively. The equation then says that, whatever result $s$ this procedure yields, $\RV{F}(s)=\RV{M}(s)\RV{A}(s)$ will hold.

A measurement procedure $\proc{S}$ is akin to \citet{menger_random_2003}'s notion of variables as ``consistent classes of quantities'' that consist of pairing between real-world objects and quantities of some type. $\proc{S}$ itself is not a well-defined mathematical thing. At the same time, the set of values it may yield \emph{is} a well-defined mathematical set. No actual procedure can be guaranteed to return elements of a mathematical set known in advance -- anything can fail -- but we assume that we can study procedures reliable enough that we don't lose much by ignoring this possibility.

Note that, because $\proc{S}$ is not a purely mathematical thing, we cannot perform mathematical reasoning with $\proc{S}$ directly. It is much more practical to relegate $\proc{S}$ to the background, and reason in terms of the functions $\RV{F}$, $\RV{M}$ and $\RV{A}$. However, even if we don't talk about it much, $\proc{S}$ remains an important element of the law.

\subsection{Measurement procedures}\label{sec:mprocs}

\begin{definition}[Measurement procedure]
A \emph{measurement procedure} $\proc{B}$ is a procedure that involves interacting with the real world somehow and delivering an element of a mathematical set $X$ as a result. A procedure $\proc{B}$ is said to takes values in a set $B$.
\end{definition}

We adopt the convention that the procedure name $\proc{B}$ and the set of values $B$ share the same letter.

\begin{definition}[Values yielded by procedures]
$\proc{B}\yields x$ is the proposition that the the procedure $\proc{B}$ will yield the value $x\in X$. $\proc{B}\yields A$ for $A\subset X$ is the proposition $\lor_{x\in A} \proc{B}\yields x$.
\end{definition}

\begin{definition}[Equivalence of procedures]\label{def:equality}
Two procedures $\proc{B}$ and $\proc{C}$ are equal if they both take values in $X$ and $\proc{B}\yields x\iff \proc{C}\yields x$ for all $x\in X$.
\end{definition}

If two involve different measurement actions in the real world but necessarily yield the same result, we say they are equivalent.

It is worth noting that this notion of equivalence identifies procedures with different real-world actions. For example, ``measure the force'' and ``measure everything, then discard everything but the force'' are often different -- in particular, it might be possible to measure the force only before one has measured everything else. Thus the result yielded by the first procedure could be available before the result of the second. However, if the first is carried out in the course of carrying out the second, they both yield the same result in the end and so we treat them as equivalent. 

Measurement procedures are like functions without well-defined domains. Just like we can compose functions with other functions to create new functions, we can compose measurement procedures with functions to produce new measurement procedures.

\begin{definition}[Composition of functions with procedures]
Given a procedure $\proc{B}$ that takes values in some set $B$, and a function $f:B\to C$, define the ``composition'' $f\circ \proc{B}$ to be any procedure $\proc{C}$ that yields $f(x)$ whenever $\proc{B}$ yields $x$. We can construct such a procedure by describing the steps: first, do $\proc{B}$ and secondly, apply $f$ to the value yielded by $\proc{B}$.
\end{definition}

For example, $\proc{MA}$ is the composition of $h:(x,y)\mapsto xy$ with the procedure $(\proc{M},\proc{A})$ that yields the mass and acceleration of the same object. Measurement procedure composition is associative:

\begin{align}
    (g\circ f)\circ\proc{B}\text{ yields } x &\iff B\text{ yields } (g\circ f)^{-1}(x) \\
    &\iff B\text{ yields } f^{-1}(g^{-1}(x))\\
    &\iff f\circ B \text{ yields } g^{-1}(x)\\
    &\iff g\circ(f\circ B)\text{ yields } x
\end{align}


One might wonder whether there is also some kind of ``tensor product'' operation that takes a standalone $\proc{M}$ and a standalone $\proc{A}$ and returns a procedure $(\proc{M},\proc{A})$. Unlike function composition, this would be an operation that acts on two procedures rather than a procedure and a function. Thus this ``append'' combines real-world operations somehow, which might introduce additional requirements (we can't just measure mass and acceleration; we need to measure the mass and acceleration of the same object at the same time), and may be under-specified. For example, measuring a subatomic particle's position and momentum can be done separately, but if we wish to combine the two procedures then we can get different results depending on the order in which we combine them.

Our approach here is to suppose that there is some complete measurement procedure $\proc{S}$ to be modeled, which takes values in the observable sample space $(\Psi,\sigalg{E})$ and for all measurement procedures of interest there is some $f$ such that the procedure is equivalent to $f\circ \proc{S}$ for some $f$. In this manner, we assume that any problems that arise from a need to combine real world actions have already been solved in the course of defining $\proc{S}$.

Given that measurement processes are in practice finite precision and with finite range, $\Psi$ will generally be a finite set. We can therefore equip $\Psi$ with the collection of measurable sets given by the power set $\sigalg{E}:=\mathscr{P}(\Psi)$, and $(\Psi,\sigalg{E})$ is a standard measurable space. $\sigalg{E}$ stands for a complete collection of logical propositions we can generate that depend on the results yielded by the measurement procedure $\proc{S}$.

One could also consider measurement procedures to produce results in $(\mathbb{R},\mathcal{B}(\mathbb{R}))$ (i.e. the reals with the Borel sigma-algebra) or a set isomorphic to it. This choice is often made in practice, and following standard practice we also often consider variables to take values in sets isomorphic to $(\mathbb{R},\mathcal{B}(\mathbb{R}))$. However, for measurement in particular this seems to be a choice of convenience rather than necessity -- for any measurement with finite precision and range, it is possible to specify a finite set of possible results.

\subsection{Observable variables}

Our \emph{complete} procedure $\proc{S}$ represents a large collection of subprocedures of interest, each of which can be obtained by composition of some function with $\proc{S}$. We call the pair consisting of a subprocedure of interest $\proc{X}$ along with the variable $\RV{X}$ used to obtain it from $\proc{S}$ an \emph{observable variable}.

\begin{definition}[Observable variable]
Given a measurement procedure $\proc{S}$ taking values in $(\Psi,\sigalg{E})$, an observable variable is a pair $(\RV{X}\circ \proc{S},\RV{X})$ where $\RV{X}:(\Psi,\sigalg{E})\to (X,\sigalg{X})$ is a measurable function and $\proc{X}:=\RV{X}\circ \proc{S}$ is the measurement procedure induced by $\RV{X}$ and $\proc{S}$.
\end{definition}

For the model $\RV{F}=\RV{MA}$, for example, suppose we have a complete measurement procedure $\proc{S}$ that yields a triple (force, mass, acceleration) taking values in the sets $X$, $Y$, $Z$ respectively. Then we can define the ``force'' variable $(\proc{F},\RV{F})$ where $\proc{F}:=\RV{F}\circ \proc{S}$ and $\RV{F}:X\times Y\times Z\to X$ is the projection function onto $X$.

A measurement procedure yields a particular value when it is completed. We will call a proposition of the form ``$\proc{X}$ yields $x$'' an \emph{observation}. Note that $\proc{X}$ need not be a complete procedure here. Given the complete procedure $\proc{S}$, a variable $\RV{X}:\Psi\to X$ and the corresponding procedure $\proc{X}=\RV{X}\circ\proc{S}$, the proposition ``$\proc{X}$ yields $x$'' is equivalent to the proposition ``$\proc{S}$ yields a value in $\RV{X}^{-1}(x)$''. Because of this, we define the \emph{event} $\RV{X}\yields x$ to be the set $\RV{X}^{-1}(x)$.

\begin{definition}[Event]
Given the complete procedure $\proc{S}$ taking values in $\Psi$ and an observable variable $(\RV{X}\circ \proc{S},\RV{X})$ for $\RV{X}:\Psi\to X$, the \emph{event} $\RV{X}\yields x$ is the set $\RV{X}^{-1}(x)$ for any $x\in X$.
\end{definition}

If we are given an observation ``$\proc{X}$ yields $x$'', then the corresponding event $\RV{X}\yields x$ is \emph{compatible with this observation}.

It is common to use the symbol $=$ instead of $\bowtie$ to stand for ``yields'', but we want to avoid this because $\RV{Y}=y$ already has a meaning, namely that $\RV{Y}$ is a constant function everywhere equal to $y$.

An \emph{impossible event} is the empty set. If $\RV{X}\yields x=\emptyset$ this means that we have identified no possible outcomes of the measurement process $\proc{S}$ compatible with the observation ``$\proc{X}$ yields $x$''. 

\subsection{Model variables}

Observable variables are special in the sense that they are tied to a particular measurement procedure $\proc{S}$. However, the measurement procedure $\proc{S}$ does not enter into our mathematical reasoning; it guides our construction of a mathematical model, but once this is done mathematical reasoning proceeds entirely with mathematical objects like sets and functions, with no further reference to the measurement procedure.

A \emph{model variable} is simply a measurable function with domain $(\Psi,\sigalg{E})$.

Model variables do not have to be derived from observable variables. We may instead choose a sample space for our model $(\Omega,\sigalg{F})$ that does not correspond to the possible values that $\proc{S}$ might yield. In that case, we require a surjective model variable $\RV{S}:\Omega\to \Psi$ called the complete observable variable, and every observable variable $(\RV{X}'\circ \proc{S},\RV{X}')$ is associated with the model variable $\RV{X}:=\RV{X}'\circ \RV{S}$.

An \emph{unobserved variable} is a variable whose set of possible values is not constrained by the results of the measurement procedure.

\begin{definition}[Unobserved variable]\label{def:unobserved_variable}
Given a sample space $(\Omega,\sigalg{F})$ and a complete observable variable $\RV{S}:\Omega\to\Psi$, a model variable $\RV{Y}:\Omega\to Y$ is \emph{unobserved} if $\RV{Y}(\RV{S}\yields s)=Y$ for all $s\in \Psi$.
\end{definition}

\subsection{Variable sequences and partial order}

Given $\RV{Y}:\Omega\to X$, we can define a sequence of variables: $(\RV{X},\RV{Y}):=\omega\mapsto (\RV{X}(\omega),\RV{Y}(\omega))$. $(\RV{X},\RV{Y})$ has the property that $(\RV{X},\RV{Y})\yields (x,y)= \RV{X}\yields x\cap \RV{Y}\yields y$, which supports the interpretation of $(\RV{X},\RV{Y})$ as the values yielded by $\RV{X}$ and $\RV{Y}$ together.

Define the partial order on variables $\varlessthan$ where $\RV{X}\varlessthan \RV{Y}$ can be read ``$\RV{X}$ is completely determined by $\RV{Y}$''.

\begin{definition}[Variables determined by another variable]\label{def:variable_po}
Given a sample space $(\Omega,\sigalg{F})$ and variables $\RV{X}:\Omega\to X$, $\RV{Y}:\Omega\to Y$, $\RV{X}\varlessthan \RV{Y}$ if there is some $f:Y\to X$ such that $\RV{X}=f\circ \RV{Y}$.
\end{definition}

Clearly, $\RV{X}\varlessthan(\RV{X},\RV{Y})$ for any $\RV{X}$ and $\RV{Y}$.

\subsection{Decision procedures}\label{sec:actions}

The kind of problem we want to solve requires us to compare the consequences of different choices from a set of possibilities $C$. We take the \emph{consequences of} $\alpha\in C$ to refer to the values obtained by some measurement procedure $\proc{S}_\alpha$ associated with the choice $\alpha$.

As we have said, what exactly a ``measurement procedure'' is is a bit vague -- it's ``what we actually do to get the numbers we associate with variables''. It seems we could describe the above in terms of a single measurement procedure $\proc{S}$, which involves:

\begin{enumerate}
    \item Choose $\alpha$
    \item Proceed according to $\proc{S}_\alpha$
\end{enumerate}

However, $\proc{S}$ is problematic to model. The model is often part of the process of choosing $\alpha$, and so a model of $\proc{S}$ that involves the step ``choose $\alpha$'' will be self-referential. Because of this, we don't try to model $\proc{S}$, and whether this changes anything is an open question.

\begin{definition}[Decision procedure]
A decision procedure is a collection $\{\proc{S}_\alpha\}_{\alpha\in C}$ of measurement procedures.
\end{definition}

Like measurement procedures, a decision procedure $\{\proc{S}_\alpha\}_{\alpha\in A}$ isn't a well-defined mathematical object; it's not really a ``set'', because the contents are real-world actions.
%!TEX root = main.tex

\chapter{Decision problems with repeatable phenomena}\label{ch:evaluating_decisions}

Chapter \ref{ch:tech_prereq} introduced probability sets as generic tools for causal modelling, while Chapter \ref{ch:2p_statmodels} examined how probability set models can be used to construct mathematical models of decision problems. In general, few assumptions were made about the structure of the models in question. Section \ref{sec:cons_to_sdp} added some structure with \emph{see-do} models, which featured variables representing observations and consequences, and non-stochastic variables representing decisions and hypotheses. Extended conditional independence properties defined the roles of these different variables in the model. However, observations and consequences in see-do models could be just about anything -- they need not take values in the same set, nor be related to one another in any particular way, nor do observations need to form a sequence. Causal inference in practice often concerns the question of making choices in a context where observations and consequences are repeatable, and the subject of this chaper is to examine probability sets that model decision problems with this kind of repeatability.

Repeatability in classical statistical models is often expressed by the assumption of \emph{exchangeability of observations}. This is the assumption that the measurement procedure produces a sequence of values that are ``all alike'' in the particular sense that any rearrangement of the sequence should be modeled with the same model (although note that exchangeability is only implied by this assumption if observations are modeled with a unique probability model, see \citet[pg. 463]{walley_statistical_1991}). An exchangeable probability distribution over a sequence of variables is a mixture of \emph{independent and identically distributed} distributions. Models with choices generally cannot be represented by a mixture of identically distributed sequences, because different choices will usually mean different actions will be taken and different distributions will result. The appropriate generalisation of independent and identical distributions seems to be independent and identical response functions -- that is, two elements of the sequence will have the same distribution over \emph{outputs} given identical \emph{inputs}. Chapter \ref{ch:other_causal_frameworks} reviews how this is a typical assumption of causal modelling frameworks, and this chapter investigates when models of sequences with choices will features independent and identical response functions.

The key result of this chapter is that a model of choices that results in a sequence of variables is representable as a mixture of independent and identical response functions is equivalent to the assumption of \emph{causal contractibility}. Causal contractibility is defined in Section \ref{sec:ccontracibility} and compared to prior work on similar questions. A representation theorem for causally contractible Markov kernels is proved in Section \ref{sec:rep_theorem}. It is applied to probability sets modelling ``one-shot'' choices with no dependence on prior data in Section \ref{sec:data_independent_actions}, and generalised to ``adaptive'' choices where actions may depend on prior data in Section \ref{sec:data_dependent}; the latter generalisation requires the notion of \emph{combs}, introduced in Section \ref{sec:def_combs}.

Causal contractibility is a more complicated assumption than exchangeability. An alternaive to directly arguing for causal contractibility is given by Theorem \ref{th:cc_ind_treat}, which shows that if there are ``experimental identifiers'' that are exchangeable in the appropriate way, and if, given either the input variables or the choice nothing relevant can be gained by learning the other, then causal contractibility holds.  To my knowledge, this is the only example of a theorem of its type that proves the existence of ``causal effects'' in models featuring choices. Somewhat similar theorems exist that address the identification of potential outcomes, but they prove different things (the identifiability of a certain kind of latent variable, which is not identical to the consequences of a choice), and provide different conditions. A particular difference is that the conditions for Theorem \ref{th:cc_ind_treat} concern only \emph{observable} variables, and while randomisation of actions is sometimes an admissible condition, it is not a necessary one.

The study of causal inference is often concerned with situations where Theorem \ref{th:cc_ind_treat} does not apply. A problem of particular interest involves going from ``passive observations'' to ``active interventions''. Section \ref{sec:weaker_assumptions} considers how this problem can be represented using probability set models, and why simple assumptions that would render it solvable are rarely acceptable.

\section{Explanation of the main theorems}

It is commonly accepted that a properly executed randomised experiment can enable the identification of causal effects, and this idea is supported by a handful of results in the potential outcomes tradition, such as those appearing in \citet{rubin_causal_2005,imbens_causal_2015}. It is also commonly accepted that the causal effects identified by randmised experiments are the appropriate thing to use to evaluate different choices, but this is based on tacit understanding rather than mathematical results. 

Similarly, in the graphical models tradition there is a tacit understanding that randomised experiments and ``interventions'' should lead to the same model (see, for example, \citet[Chap. 4]{pearl_book_2018}).

The fact is, however, that it's possible to have a precise understanding of ``causal effect'' along with a precise understanding of ``consequences of choices'' at the same time. Theorems \ref{th:data_ind_CC} and \ref{th:response_hdep} provide necessary and sufficient conditions for models with choices to contain independent and identical reponse functions. Theorem \ref{th:cc_ind_treat} provides a sufficient condition that might, in some situations, be easier to evaluate. These theorems establish that the existence of independent and identical response functions is a question that depends on which variables are identified as ``inputs'' and which we identify as ``outputs''.
That is, a modeler who identifies some sequence of variables $\RV{D}_i$ as presumptive inputs and $\RV{Y}_i$ as presumptive outputs to independent and identical response functions must have substantial prior knowledge about the distribution of these variables under different choices.

\section{Relevance to previous work}

This chapter draws on three different lines of work. The first is the study of representations of symmetric of probability models. The equivalence between infinite exchangeable probability models and mixtures of independent and identically distributed models was shown by \cite{de_finetti_foresight_1992}. This result has been extended in many ways, including to finite sequences \citet{kerns_definettis_2006,diaconis_finite_1980} and for partially exchangeable arrays \citet{aldous_representations_1981}. A comprehensive overview of results is presented in \citet{kallenberg_probabilistic_2005}. Particularly similar to our result is the notion of ``partial exchangeability'' from \citet{diaconis_recent_1988}.

The second line of work is the study of exchangeability-like assumptions in causal models. \citet{lindley_role_1981} discuss models consisting of a sequence of exchangeabile observations along with ``one more observation'', a structure that is similar to the models with observations and consequences discussed in section \ref{sec:weaker_assumptions}. Lindley discusses the application of this model to questions of causation, but does not explore this deeply due to the perceived difficulty of finding a satisfactory definition of causation. \citet{rubin_causal_2005}'s discussion of causal inference with potential outcomes discusses the assumption of exchangeable potential outcomes, and \citet{imbens_causal_2015} uses this assumption to prove several identification results. However, as Lindley observed, this is an assumption of exchangeability of unobservable variables, and it is difficult to see how judgements about the measurement procedures being modeled do or do not imply exchangeability amongst such variables. \citet{saarela_role_2020}, using structural causal models, proposes \emph{conditional exchangeability}, which refers to the invariance of a joint distribution over outcomes under ``surgical switches'' of the values of causal variables of interest. This definition depends on having a structural model, a property not shared by the current work.

Exchangeability in the setting of causal models is often discussed in terms of the exchangeability of \emph{people} (or more generically, \emph{experimental setups}). This is in a sense the inverse of Rubin's position above -- a symmetry of a measurement procedure is given, but the implications regarding the corresponding mathematical model are unclear. \citet{hernan_beyond_2012,greenland_identifiability_1986,banerjee_chapter_2017,dawid_decision-theoretic_2020} all discuss assumptions along these lines.

The other component of causal contractibility is \emph{locality} (Definition \ref{def:caus_cont}). This is also suggested by existing work -- in particular, the stable unit treatment distribution assumption (SUTDA) in \citet{dawid_decision-theoretic_2020}, and the stable unit treatment value assumption (SUTVA) in \citep{rubin_causal_2005}:
\begin{blockquote}
(SUTVA) comprises two sub-assumptions. First, it assumes that \emph{there is no interference between units (Cox 1958)}; that is, neither $Y_i(1)$ nor $Y_i(0)$ is affected by what action any other unit received. Second, it assumes that \emph{there are no hidden versions of treatments}; no matter how unit $i$ received treatment $1$, the outcome that would be observed would be $Y_i(1)$ and similarly for treatment $0$.
\end{blockquote}

Finally, the idea of \emph{combs} in probabilistic models was first proposed by \citet{chiribella_quantum_2008} and an application to causal models was developed by \citet{jacobs_causal_2019}.

\section{Independent and Identical Response Functions}\label{sec:response_functions}

Start with a sequence of variable pairs $(\RV{X}_i,\RV{Y}_i)_{i\in \mathbb{N}}$ where $\RV{X}_i$ is the $i$th ``input'' and $\RV{Y}_i$ is the corresponding ``output'', each taking values in $X$ and $Y$ respectively. $(\RV{X}_i,\RV{Y}_i)_{i\in \mathbb{N}}$ are related by a sequence of ``independent and identical response functions'' (``IIRs'') if there is some variable $\RV{H}$ (generally a latent variable) such that $\prob{P}_C^{\RV{Y}_i|\RV{X}_i\RV{H}}=\prob_C^{\RV{Y}_j|\RV{X}_j\RV{H}}$ for all $i,j$ and for all $i$ $\RV{Y}_i\CI_{\prob{P}_C}^e (\RV{X}_{<i},\RV{Y}_{<i},\RV{C})|(\RV{H},\RV{X}_i)$.

The result of this chapter is that the pairs in a sequence $(\RV{X},\RV{Y}):=(\RV{X}_i,\RV{Y}_i)_{i\in \mathbb{N}}$ modeled by $\prob{P}_C$ are related by IIRs if and only if there is a causally contractible \emph{uniform comb} $\prob{P}_C^{\RV{Y}\combbreak\RV{X}}$. Combs are a generalisation of conditional probabilities and, and if we assume that actions are data-independent ($\RV{X}_i\CI_{\prob{P}_C}^e \RV{Y}_{<i} C|\RV{X}_{<i}$), this reduces to the assumption that the uniform conditional distribution $\prob{P}_C^{\RV{Y}|\RV{X}}$ is causally contractible. Many of the results will focus on the simpler case where we deal with the conditional probability instead of the comb -. Specifically, a representation theorem (Theorem \ref{th:ciid_rep_kernel}) is proven for general Markov kernels in Section \ref{sec:ccontracibility}, and this applied to models $\prob{P}_C$ with data-independent actions in Section \ref{sec:data_independent_actions}. Section \ref{sec:assessing}, discusses questions related to when the assumption of causal contractibility might be held to apply to a particular problem. Combs are introduced in Section \ref{sec:data_dependent}, and Theorem \ref{th:ciid_rep_kernel} is applied to models with data-dependent actions.

\subsection{Causally contractible Markov kernels}\label{sec:ccontracibility}

Causal contractibility is the conjunction of two properties of a Markov kernel $X^{\mathbb{N}}\kto Y^{\mathbb{N}}$: such a Markov kernel is causally contractible if it is \emph{local} and \emph{commutes with exchange}. 

Graphical notation can offer an intuitive picture of these two sub-assumptions. In the simplified case where we have $\kernel{K}:X^2\kto Y^2$, exchange commutativity for two inputs and outputs is given by the following equality:
\begin{align}
    \tikzfig{commutativity_of_exchange}
\end{align}
It expresses the idea that swapping the inputs is equivalent to swapping the outputs. Locality is given by the following pair of equalities:
\begin{align}
    \tikzfig{cons_locality_1}\\
    \tikzfig{cons_locality_2}
\end{align}
and expresses the notion that the outputs are independent of the non-corresponding input, conditional on the corresponding input.

\subsubsection{Definition of causal contractibility}

The general definitions follow.

\begin{definition}[Locality]\label{def:caus_cont}
A Markov kernel $\kernel{K}:X^{\mathbb{N}}\kto Y^{\mathbb{N}}$ is \emph{local} if for all $n\in \mathbb{N}$, $A_i\in \sigalg{Y}$, $(x_{[n]},x_{[n]^C})\in\mathbb{N}$ there exists $\kernel{L}:X^n\kto Y^n$ such that
\begin{align}
    \tikzfig{local_lhs} &= \tikzfig{local_rhs}\\
    &\iff\\
    \kernel{K}(\bigtimes_{i\in [n]} A_i\times Y^{\mathbb{N}}|x_{[n]},x_{[n]^C}) &= \kernel{L}(\bigtimes_{i\in [n]} A_i|x_{[n]})
\end{align}
\end{definition}

\begin{definition}[Exchange commutativity]\label{def:caus_exch}
A Markov kernel $\kernel{K}:X^{\mathbb{N}}\kto Y^{\mathbb{N}}$ \emph{commutes with exchange} if for all finite permutations $\rho:\mathbb{N}\to\mathbb{N}$, $A_i\in \sigalg{Y}$, $(x_{[n]},x_{[n]^C})\in\mathbb{N}$
\begin{align}
    \kernel{K}\mathrm{swap}_{\rho,Y} &=  \mathrm{swap}_{\rho,X} \kernel{K}\\
    &\iff\\
    \kernel{K}(\bigtimes_{i\in\mathbb{N}} A_{\rho(i)}|(x_i)_{i\in {\mathbb{N}}}) &= \kernel{K}(\bigtimes_{i\in\mathbb{N}} A_{i}|(x_{\rho(i)})_{i\in {\mathbb{N}}})
\end{align}
\end{definition}

Causal contractibility is the conjunction of both assumptions.
\begin{definition}[Causal contractibility]
A Markov kernel $\kernel{K}:X^{\mathbb{N}}\kto Y^{\mathbb{N}}$ is \emph{causally contractible} if it is local and commutes with exchange.
\end{definition}

\subsubsection{Properties of causally contractible Markov kernels}

A causally contractible Markov kernel $\kernel{K}:X^{\mathbb{N}}\kto Y^{\mathbb{N}}$ can be mapped to a kernel over a ``contracted'' domain $Y^n$, $\kernel{K}_n:X^{\mathbb{N}}\kto Y^n$, using the appropriate $\text{del}$ map. Theorem \ref{th:equal_of_condits} establishes that any contraction of $\kernel{K}$ to the same domain $Y^n$ yields the same result (see also Theorem \ref{th:equal_of_reduced_condits}). This feature is the motivation for the name \emph{causal contractibility}. 

Theorem \ref{th:no_implication} shows that exchange commutativity and locality are independent assumptions by constructing counterexamples.

Before these theorems are proved, the following definition and Lemma will prove helpful.

All swaps can be written as a product of transpositions, so proving that a property holds for all finite transpositions is enough to show it holds for all finite swaps. It's useful to define a notation for transpositions.
\begin{definition}[Finite transposition]
Given two equally sized sequences $A=(a_i)_{i\in [n]}$, $B=(b_i)_{i\in [n]}$, ${A\leftrightarrow B}:\mathbb{N}\to \mathbb{N}$ is the permutation that sends the $i$th element of $A$ to the $i$th element of $B$ and vise versa. Note that $A\leftrightarrow B$ is its own inverse.
\end{definition}

Lemma \ref{lem:infinitely_extended_kernels} is used to extend finite sequences to infinite ones, and is used in a number of upcoming theorems.

\begin{lemma}[Infinitely extended kernels]\label{lem:infinitely_extended_kernels}
Given a collection of Markov kernels $\kernel{K}_i:X^i\kto Y^i$ for all $i\in \mathbb{N}$, if we have for every $j>i$
\begin{align}
    \kernel{K}_j(\text{id}_{X_i}\otimes \text{del}_{X_{j-i}}) &= \kernel{K}_i\otimes \text{del}_{X_{j-i}}\label{eq:marginalise_comb}
\end{align} 
then there is a unique Markov kernel $\kernel{K}:X^{\mathbb{N}}\kto Y^{\mathbb{N}}$ such that for all $i,j\in \mathbb{N}$,$j>i$
\begin{align}
    \kernel{K}(\text{id}_{X_i}\otimes \text{del}_{X_{j-i}})&= \kernel{K}_i\otimes \text{del}_{X_{j-i}}
\end{align}
\end{lemma}

\begin{proof}
Take any $x\in X^{\mathbb{N}}$ and let $x_{|m}\in X^n$ be the first $n$ elements of $x$. By Equation \ref{eq:marginalise_comb}, for any $A_i\in \sigalg{Y}$, $i\in [m]$
\begin{align}
    \kernel{K}_n(\bigtimes_{i\in [m]}A_i\times Y^{n-m}|x_{|n}) &= \kernel{K}_m(\bigtimes_{i\in [m]}A_i|x_{|m})
\end{align}

Furthermore, by the definition of the $\mathrm{swap}$ map for any permutation $\rho:[n]\to[n]$
\begin{align}
    \kernel{K}_n\mathrm{swap}_{\rho}(\bigtimes_{i\in [m]}A_{\rho(i)}\times Y^{n-m}|x_{|n}) &= \kernel{K}_n(\bigtimes_{i\in [m]}A_{i}\times Y^{n-m}|x_{|n})
\end{align}
thus by the Kolmogorov Extension Theorem \citep{cinlar_probability_2011}, for each $x\in X^{\mathbb{N}}$ there is a unique probability measure $\prob{Q}_x\in \Delta(Y^{\mathbb{N}}$ satisfying
\begin{align}
    \prob{Q}_d(\bigtimes_{i\in [n]}A_i\times Y^{\mathbb{N}}) &= \kernel{K}_n(\bigtimes_{i\in [n]}A_{\rho(i)}|d_{|n})\label{eq:q_is_Markov}
\end{align}

Furthermore, for each $\{A_i\in\sigalg{Y}|i\in \mathbb{N}\}$, $n\in \mathbb{N}$ note that for $p>n$
\begin{align}
\prob{Q}_d(\bigtimes_{i\in[n]} A_i \times Y^{\mathbb{N}})&\geq \prob{Q}_d(\bigtimes_{i\in [p]} A_i\times Y^{\mathbb{N}})\\
&\geq \prob{Q}_d(\bigtimes_{i\in \mathbb{N}} A_i)
\end{align}
so by the Monotone convergence theorem, the sequence $\prob{Q}_d(\bigtimes_{i\in[n]} A_i)$ converges as $n\to \infty$ to $\prob{Q}_d(\bigtimes_{i\in\mathbb{N}} A_i)$. $d\mapsto \prob{Q}_d^{\RV{Z}_n}(\bigtimes_{i\in[n]} A_i)$ is measurable for all $n$, $\{A_i\in\sigalg{Y}|i\in \mathbb{N}\}$ by Equation \ref{eq:q_is_Markov}, and so $d\mapsto Q_d$ is also measurable.

Thus $d\mapsto Q_d$ is the desired $\prob{P}_C^{\RV{Y}_{\mathbb{N}}\combbreak \RV{D}_{\mathbb{N}}}:D^\mathbb{N}\kto Y^\mathbb{N}$.
\end{proof}

Theorem \ref{th:equal_of_condits} shows that, given a causally contractible kernel, the following operations yield equivalent results:
\begin{itemize}
    \item Marginalising all but the first $n$ outputs
    \item Marginalising all outputs except for the positions $A\subset\mathbb{N}$ where $|A|=n$, and swapping the first $n$ inputs with the elements of $A$
\end{itemize}

\begin{definition}[Marginalising kernel]
Given $(X,\sigalg{X})$ and $A\subset\mathbb{N}$, $\mathrm{marg}_A:X^\mathbb{N}\kto X^A$ is the Markov kernel given by
\begin{align}
    \bigotimes_{i\in \mathbb{N}} \text{switch}_{A,i}
\end{align}
where
\begin{align}
    \text{switch}_A &= \begin{cases}
                        \text{id}_X&i\in A\\
                        \text{del}_X&i\not\in A
                        \end{cases}
\end{align} 
\end{definition}

\begin{theorem}[Equality of equally sized contractions]\label{th:equal_of_condits}
A Markov kernel $\kernel{K}:X^{\mathbb{N}}\kto Y^{\mathbb{N}}$ is \emph{causally contractible} if and only if for every $n\in \mathbb{N}$ and every $A\subset\mathbb{N}$ there exists some $\kernel{L}:X^n\kto Y^n$ such that
\begin{align}
    \kernel{K} \text{marg}_A &= \text{swap}_{[n]\leftrightarrow A} \kernel{L}\otimes \text{del}_{X^{\mathbb{N}}}
\end{align}
\end{theorem}

\begin{proof}
Only if:
By exchange commutativity
\begin{align}
    \text{swap}_{[n]\leftrightarrow A} \kernel{K} &= \kernel{K} \text{swap}_{[n]\leftrightarrow A}
\end{align}
multiply both sides by $\text{swap}_{[n]\leftrightarrow A}$ on the right and, because $\text{swap}_{[n]\leftrightarrow A}$ is its own inverse,
\begin{align}
        \text{swap}_{[n]\leftrightarrow A} \kernel{K}\text{swap}_{[n]\leftrightarrow A} &= \kernel{K}
\end{align}
so
\begin{align}
    \kernel{K}\text{marg}_A &= \text{swap}_{[n]\leftrightarrow A} \kernel{K}\text{swap}_{[n]\leftrightarrow A}\text{marg}_A\\
    &= \text{swap}_{[n]\leftrightarrow A} \kernel{K}\text{marg}_{[n]}
\end{align}
By locality, there exists some $\kernel{L}:X^n\kto Y^n$ such that
\begin{align}
    \kernel{K} \text{marg}_{[n]} &= \kernel{K}(\text{id}_{[n]}\otimes \text{del}_{X^{\mathbb{N}}})\\
     &= \kernel{L}\otimes \mathrm{del}_{X^{\mathbb{N}}}
\end{align}
If:
Taking $A=[n]$ for all $n$ establishes locality.

For exchange commutativity, note that for all $x\in X^{\mathbb{N}}$, $n\in\mathbb{N}$, we have
\begin{align}
    \text{swap}_{A\leftrightarrow [n]} \kernel{K} \text{marg}_A &= \text{swap}_{A\leftrightarrow [n]} \kernel{K} \text{swap}_{A\leftrightarrow [n]} (\text{id}_{[n]}\otimes \text{del}_{X^{\mathbb{N}}})\\
     &= \kernel{K} \text{marg}_{[n]}\\
     &= \kernel{K}(\text{id}_{[n]}\otimes \text{del}_{X^{\mathbb{N}}})
\end{align}
Then by Lemma \ref{lem:infinitely_extended_kernels}
\begin{align}
    \text{swap}_{A\leftrightarrow [n]} \kernel{K} \text{swap}_{A\leftrightarrow [n]} &= \kernel{K}
\end{align}
Consider an arbitrary finite permutation $\rho:\mathbb{N}\to \mathbb{N}$. $\rho$ can be decomposed into a finite set of cyclic permutations on disjoint orbits. Each cyclic permutation is simply the composition of some set of transpositions, and so $\rho$ itself can be written as a composition of a sequence of transpositions. Thus for any finite $\rho:\mathbb{N}\to\mathbb{N}$
\begin{align}
    \text{swap}_{\rho} \kernel{K} \text{swap}_{\rho} &= \kernel{K}
\end{align}
\end{proof}

Theorem \ref{th:no_implication} shows that neither locality nor exchange commutativity is implied by the other.

\begin{theorem}\label{th:no_implication}
Exchange commutativity does not imply locality or vise versa.
\end{theorem}

\begin{proof}
First, a Markov kernel that exhibits exchange commutativity but not locality. Suppose $D=Y=\{0,1\}$ and $\kernel{K}:D^2\kto Y^2$ is given by
\begin{align}
    \kernel{K}(y_1,y_2|d_1,d_2) &= \llbracket (y_1,y_2)= (d_1+d_2,d_1+d_2) \rrbracket
\end{align}
then 
\begin{align}
    \kernel{K}(y_1,Y|d_1,d_2) &= \llbracket y_1 = d_1+d_2 \rrbracket
\end{align}
and there is no function depending on $y_1$ and $d_1$ only that is equal to this. Thus $\kernel{K}$ does not satisfy locality. 

However, taking $\rho$ to be the unique nontrivial swap $\{0,1\}\to \{0,1\}$
\begin{align}
    \text{swap}_{\rho,D}\kernel{K}(y_1,y_2|d_1,d_2) &= \kernel{K}(y_1,y_2|d_2,d_1)\\
    &= \llbracket (y_1,y_2)= (d_2+d_1,d_2+d_1) \rrbracket\\
    &= \llbracket (y_1,y_2)= (d_1+d_2,d_1+d_2) \rrbracket\\
    &= \llbracket (y_2,y_1)= (d_1+d_2,d_1+d_2) \rrbracket\\
    &= \kernel{K}\text{swap}_{\rho,Y}(y_1,y_2|d_1,d_2)
\end{align}
so $\kernel{K}$ commutes with exchange.

Next, a Markov kernel that satisfies locality but does not commute with exchange. Suppose again $D=Y=\{0,1\}$ and $\kernel{K}:D^2\kto Y^2$ is given by
\begin{align}
    \kernel{K}(y_1,y_2|d_1,d_2) &= \llbracket (y_1,y_2)= (0,1) \rrbracket
\end{align}

Then:
\begin{align}
    \kernel{K}(y_1|d_1,d_2) &= \llbracket y_1= 0 \rrbracket\\
    &= \kernel{K}(y_1|d_1)\\
    \kernel{K}(y_2|d_1,d_2)&= \llbracket y_2= 1 \rrbracket\\
    &= \kernel{K}(y_2|d_2)
\end{align}
so $\kernel{K}$ satisfies locality.

However, $\kernel{K}$ does not commute with exchange.
\begin{align}
    \text{swap}_{\rho(\RV{D})} \kernel{K}(y_1,y_2|d_1,d_2) &= \kernel{K}(y_1,y_2|d_2,d_1)\\
    &=\llbracket (y_1,y_2)= (0,1) \rrbracket\\
    &\neq \llbracket (y_2,y_1)= (0,1) \rrbracket\\
    &= \kernel{K}\text{swap}_{\rho(\RV{D})}(y_1,y_2|d_1,d_2)
\end{align}
\end{proof}

Theorem \ref{th:no_implication} presents abstract counterexamples to show that the assumptions of exchange commutativity and locality are indpedent. For some more practical examples, a model of the treatment of several patients who have already been examined might satisfy consequence locality but not exchange commutativity. Patient B's treatment could be assumed not to affect patient A, but the same results would not be expected from giving patient A's treatment to patient B as from giving patient A's treatment to patient A. 

A model of economic interventions might satisfy exchange commutativity but not locality. If a government prints money to make exactly $n$ payments of \$10 000 are made to a number of undistinguished recipients, the government cannot say much about the impact of who exactly receives the payment. However, the amount of inflation created by the payments depends on the number of payments made; making 100 such payments will have a negligible effect on inflation, while making payments to everyone in the country will have a substantial effect, and this will in turn affect the outcomes of the people who did or did not receive payment. \citet{dawid_causal_2000} offers the alternative example of herd immunity in vaccination campaigns as a situation where commutativity of exchange holds but locality does not.

Although locality seems to imply a lack of interference between inputs and outputs of different indices, it actually allows for some models with certain kinds of interference between actions and outcomes of different indices. For example: consider an experiment where I first flip a coin and record the results of this flip as the outcome of the ``step 1''. Subsequently, I can choose either to copy the outcome from step 1 to be the input for ``step 2'' (this is the choice $\RV{D}_1=0$), or flip a second coin use this as the input for step 2 (this is the choice $\RV{D}_1=1$). At the second step, I may further choose to copy the provisional results ($\RV{D}_2=0$) or invert them ($\RV{D}_2=1$). Then
\begin{align}
    \prob{P}_S^{\RV{Y}_1|\RV{D}}(y_1|d_1,d_2) &= 0.5\\
    \prob{P}_S^{\RV{Y}_2|\RV{D}}(y_2|d_1,d_2) &= 0.5
\end{align}
\begin{itemize}
    \item The marginal distribution of both experiments in isolation is $\text{Bernoulli}(0.5)$ no matter what choices I make, so a model of this experiment would satisfy Definition \ref{def:caus_cont}
    \item Nevertheless, the choice at step 1 affects the result of step 2
\end{itemize}

\subsubsection{Representation theorems for causally contractible Markov kernels}\label{sec:rep_theorem}

Theorem \ref{th:table_rep_kernel} shows that a causally contractible Markov kernel can be represented as the product of a column exchangeable probability distribution and a ``lookup function''. This representation is identical to the representation of potential outcomes models (see, for example, \citet{rubin_causal_2005}), but Theorem \ref{th:table_rep_kernel} applies to arbitrary kernels and the resulting representation will usually not be interpretable as a potential outcomes models. This theorem allows De Finetti's theorem to be applied to the column exchangeable probability distribution, which is a key step in proving the main result (Theorem \ref{th:ciid_rep_kernel}).

\begin{theorem}\label{th:table_rep_kernel}
A Markov kernel $\kernel{K}:X^{\mathbb{N}}\kto Y^{\mathbb{N}}$ is causally contractible if and only if there exists a column exchangeable probability distribution $\mu \Delta(Y^{|X|\times \mathbb{N}})$ such that
\begin{align}
    \kernel{K} &= \tikzfig{lookup_representation_kernel}\label{eq:lup_rep_kernel}\\
    &\iff\\
    \kernel{K}(A|(x_i)_{i\in \mathbb{N}}) &= \mu \Pi_{(x_i i)_{i\in\mathbb{N}}}(A)\forall A\in \sigalg{Y}^{\mathbb{N}}
\end{align}
Where $\Pi_{(d_i i)_{i\in\mathbb{N}}}:Y^{|X|\times \mathbb{N}}\to Y^{\mathbb{N}}$ is the function 
\begin{align}
    (y_{j i})_{j,i \in X\times  \mathbb{N}}\mapsto (y_{d_i i})_{i\in \mathbb{N}}
\end{align}
that projects the $(x_i,i)$ indices of $y$ for all $i\in \mathbb{N}$, and $\prob{F}_{\text{ev}}$ is the Markov kernel associated with the evaluation map
\begin{align}
    \text{ev}:X^\mathbb{N}\times Y^{X\times \mathbb{N}}&\to Y\\
    ((x_i)_\mathbb{N},(y_{ji})_{j,i\in X\times \mathbb{N}})&\mapsto (y_{x_i i})_{i\in \mathbb{N}}
\end{align}
\end{theorem}

\begin{proof}
Only if:
Choose $e:=(e_i)_{i\in\mathbb{N}}$ such that $e_{i+|X|j}$ is the $i$th element of $X$ for all $i,j\in \mathbb{N}$.

Define
\begin{align}
    \mu(\bigtimes_{(i,j)\in X\times \mathbb{N}} A_{ij}):=\kernel{K}(\bigtimes_{(i,j)\in X\times \mathbb{N}} A_{ij}|e)& \forall A_{ij}\in \sigalg{Y}
\end{align}

Now consider any $x:=(x_i)_{i\in \mathbb{N}}\in X^{\mathbb{N}}$. By definition of $e$, $e_{x_i i}=x_i$ for any $i,j\in \mathbb{N}$.

Define
\begin{align}
    \prob{Q}:X^{\mathbb{N}}\kto Y^{\mathbb{N}}\\
    \prob{Q}:= \tikzfig{lookup_representation_kernel}
\end{align}
and consider some $A\subset \mathbb{N}$, $|A|=n$ and $B:= (x_i,i))_{i\in A}$. Note that the subsequence of $e$ indexed by $B$, $e_B:=(e_{x_i i})_{i\in A}=x_A$. Thus given the swap map $\mathrm{swap}_{A\leftrightarrow B}:\mathbb{N}\to\mathbb{N}$ that sends the first element of $A$ to the first element of $B$ and so forth, $\mathrm{swap}_{A\leftrightarrow B} (e_B) = x_A$. For arbitrary $\{C_i\in \sigalg{Y}|i\in A\}$, define $C_A:=\mathrm{swap}_{[n]\leftrightarrow A} (\times_{i\in [n]} C_i\times Y^{\mathbb{N}})$. Then, for arbitrary $x\in X^{\mathbb{N}}$
\begin{align}
    \prob{Q}(C_A|x) &= \mu (\mathrm{ev}_x^{-1}(C_A))\label{eq:q_mu_rel}
\end{align}

The argument of $\mu$ is
\begin{align}
    \mathrm{ev}_x^{-1}(C_A)&=\{(y_{ji})_{j,i\in X\times\mathbb{N}}|(y_{x_i i})_{i\in\mathbb{N}}\in C_A\}\\
    &= \bigtimes_{i\in \mathbb{N}} \bigtimes_{j\in X} D_{ji}
\end{align}
where
\begin{align}
    D_{ji} = \begin{cases}
        C_i & (j,i)\in B\\
        Y & \text{otherwise}
    \end{cases}
\end{align}
and so
\begin{align}
    \text{swap}_{A\leftrightarrow B} (\mathrm{ev}_x^{-1}(C_A)) &= C_A\label{eq:swap_select_relation}
\end{align}

Substituting Equation \ref{eq:swap_select_relation} into \ref{eq:q_mu_rel}
\begin{align}
    \prob{Q}(C_A|x) &= \mu \text{swap}_{A\leftrightarrow B} (C_A)\\
    &= \kernel{K} \text{swap}_{A\leftrightarrow B} (C_A|e)\\
    &= \kernel{K}\text{swap}_{A\leftrightarrow B} (C_A|e_B,\text{swap}_{B\leftrightarrow A}(x)_B^C)&\text{by locality}\\
    &= \kernel{K}\text{swap}_{A\leftrightarrow B} (C_A|\text{swap}_{B\leftrightarrow A}(x))\\
    &= \text{swap}_{B\leftrightarrow A} \kernel{K}\text{swap}_{A\leftrightarrow B} (C_A|x)\\
    &= \kernel{K}(C_A|x)&\text{by commutativity of exchange}
\end{align}

Because this holds for all $x$, $A\subset\mathbb{N}$, by Lemma \ref{lem:infinitely_extended_kernels}

\begin{align}
    \prob{Q} &= \kernel{K}
\end{align}

Next we will show $\mu$ is column exchangeable. Consider any column swap $\text{swap}_{c}:X\times \mathbb{N}\to X\times \mathbb{N}$ that acts as the identity on the $X$ component and a finite permutation on the $\mathbb{N}$ component. From the definition of $e$, $\text{swap}_c(e)=e$. Thus by commutativity of exchange, for any $A\in \sigalg{Y}^{\mathbb{N}}$
\begin{align}
 \kernel{K}(A|e) &= \text{swap}_c\kernel{K}\text{swap}_c(A|e)\\
 &= \kernel{K}\text{swap}_c(A|\text{swap}_c(e))\\
 &= \kernel{K}\text{swap}_c(A|e)
\end{align}


If:
Suppose 
\begin{align}
    \kernel{K} &= \tikzfig{lookup_representation_kernel}
\end{align}
where $\mu$ is column exchangeable, and consider any two $x,x'\in X^{\mathbb{N}}$ such that some subsequences are equal $x_S=x'_T$ with $S,T\subset \mathbb{N}$ and $|S|=|T|=[n]$.

For any $\{A_i\in\sigalg{Y}|i\in S\}$, let $A_S = \text{swap}_{[n]\leftrightarrow S} \times_{i\in [n]} A_i\times Y^{\mathbb{N}}$, $A_T = \text{swap}_{S\leftrightarrow T} (A_S)$, $B=(x_i i)_{i\in S}$ and $C=(x_i i)_{i\in T}=(x_{\text{swap}_{S\leftrightarrow T}}(i) i)_{i\in S}$. By Equations \ref{eq:q_mu_rel} and \ref{eq:swap_select_relation}
\begin{align}
    \kernel{K}(A_S|x) &= \mu \text{swap}_{S\leftrightarrow B} (A_S)\\
    &= \mu \text{swap}_{T\leftrightarrow C} (A_T)&\text{ by column exchangeability of }\mu\\
    &= \kernel{K}(A_T|\text{swap}_{S\leftrightarrow T}(x))\\
    &=  \text{swap}_{S\leftrightarrow T}\kernel{K}(A_T| x)\\
    &= \text{swap}_{S\leftrightarrow T} \kernel{K} \text{swap}_{S\leftrightarrow T} (A_S| x)
\end{align}
so $\kernel{K}$ is causally contractible by Theorem \ref{th:equal_of_condits}.
\end{proof}

Theorem \ref{th:ciid_rep_kernel} is the main result of this section. It shows that a causally contractible Markov kernel $X^{\mathbb{N}}\kto Y^{\mathbb{N}}$ is representable as a ``prior'' $\mu\in \Delta(H)$ and a ``parallel product'' of Markov kernels $H\times X\kto Y$. These will be the response conditionals when Theorem \ref{th:ciid_rep_kernel} is applied to probability set models.

\begin{definition}[Measurable set of probability distributions]
Given a measurable set $(\Omega,\sigalg{F})$, the measurable set of distributions on $\Omega$, $\mathcal{M}_1(\Omega)$, is the set of all probability distributions on $\Omega$ equipped with the coarsest $\sigma$-algebra such that the evaulation maps $\eta_B:\nu\mapsto \nu(B)$ are measurable for all $B\in \sigalg{F}$.
\end{definition}

\begin{theorem}\label{th:ciid_rep_kernel}
Given a kernel $\kernel{K}:X^{\mathbb{N}}\kto Y^{\mathbb{N}}$, let $H:=\mathcal{M}_1(Y^X)$ be the measurable set of probability distributions on $(Y^X,\sigalg{Y}^X)$. $\kernel{K}$ is causally contractible if and only if there is some $\RV{H}:Y^{X\times\mathbb{N}}\to H$ and $\kernel{L}:H\times X\kto Y$ such that
\begin{align}
    \kernel{K} &= \tikzfig{do_model_representation_kernel}\\
    &\iff\\
    \kernel{K}(\bigtimes_{i\in\mathbb{N}}A_i|(x_i)_{i\in\mathbb{N}}) &= \int_H \prod_{i\in\mathbb{N}} \kernel{L}(A_i|h,x_i)\mu\kernel{F}_{\RV{H}}(\mathrm{d}h)
\end{align}
\end{theorem}

\begin{proof}
By Theorem \ref{th:table_rep_kernel}, we can represent the conditional probability $\kernel{K}$ as
\begin{align}
        \kernel{K} &= \tikzfig{lookup_representation_kernel}\label{eq:lookup_representation}
\end{align}
where $\mu$ is column exchangeable.

As a preliminary, we will show
\begin{align}
    \kernel{F}_{\mathrm{ev}} &= \tikzfig{lookup_rep_intermediate_kernel}\label{eq:ev_alternate_rep}
\end{align}
where  $\mathrm{evs}_{Y^D\times D}:Y^D\times D\to Y$ is the single-shot evaluation function
\begin{align}
    (x,(y_i)_{i\in X})\mapsto y_x
\end{align}

Recall that $\mathrm{ev}$ is the function
\begin{align}
    ((x_i)_\mathbb{N},(y_{ji})_{j,i\in X\times \mathbb{N}})&\mapsto (y_{x_i i})_{i\in \mathbb{N}}
\end{align}
By definition, for any $\{A_i\in\sigalg{Y}|i\in \mathbb{N}\}$
\begin{align}
    \kernel{F}_{\mathrm{ev}}(\bigtimes_{i\in \mathbb{N}}A_i|(x_i)_\mathbb{N},(y_{ji})_{i\in X\times \mathbb{N}}) &= \delta_{(y_{x_i i})_{i\in \mathbb{N}}}(\bigtimes_{i\in \mathbb{N}}A_i)\\
        &= \prod_{i\in \mathbb{N}} \delta_{y_{x_i i}} (A_i)\\
        &= \prod_{i\in \mathbb{N}} \kernel{F}_{\text{evs}} (A_i|x_i,(y_{ji})_{j\in X})\\
        &= \left(\bigotimes_{i\in\mathbb{N}} \kernel{F}_{\mathrm{evs}} \right)(\bigtimes_{i\in \mathbb{N}}A_i|(x_i)_\mathbb{N},(y_{ji})_{j\in X\times \mathbb{N}})
\end{align}
which is what we wanted to show.

Only if:
Define $\kernel{M}:H\kto Y^D$ by $\kernel{M}(A|h)=h(A)$ for all $A\in\sigalg{Y}^X$, $h\in H$. By the column exchangeability of $\mu$, from \citet[Prop. 1.4]{kallenberg_basic_2005} there is a directing random measure $\RV{H}:Y^{X\times\mathbb{N}}\to H$ such that
\begin{align}
    \mu &= \tikzfig{de_finetti_representation_kernel}\label{eq:df_rep_mu}\\
    &\iff\\
    \mu(\bigtimes_{i\in \mathbb{N}} A_i) &= \int_H \prod_{i\in \mathbb{N}} \kernel{M}(A_i|h) \mu\kernel{F}_{\RV{H}}(\mathrm{d}h)&\forall A_i\in\sigalg{Y}^X
\end{align}

By Equations \ref{eq:lookup_representation} and \ref{eq:ev_alternate_rep}
\begin{align}
    \kernel{K} &= \tikzfig{do_model_representation_kernel_pre}\\
    &:= \tikzfig{do_model_representation_kernel}\label{eq:lup_rep_combined}
\end{align}
Where we can connect the copied outputs of $\mu\kernel{F}_{\RV{H}}$ to the inputs of each $\kernel{M}$ ``inside the plate'' as the plates in Equations \ref{eq:ev_alternate_rep} and \ref{eq:df_rep_mu} are equal in number and each connected wire represents a single copy of $Y^D$.

If:
By assumption, for any $\{A_i\in \sigalg{Y}|i\in\mathbb{N}\}$, $x:=(x_i)_{i\in\mathbb{N}}\in X^{\mathbb{N}}$
\begin{align}
    \kernel{K}(\bigtimes_{i\in \mathbb{N}} A_i|x) &= \int_H \prod_{i\in \mathbb{N}}\kernel{L}(A_i|h,x_i)\mu(\mathrm{d}h)
\end{align}

Consider any $S,T\subset\mathbb{N}$ with $|S|=|T|$, and define $A_S:=\times_{i\in\mathbb{N}} B_i$ where $B_i=Y$ if $i\not\in S$, otherwise $A_i$ is an arbitrary element of $\sigalg{Y}$. Define $A_T:=\times_{i\in\mathbb{N}} B_{\mathrm{swap}_{S\leftrightarrow T}(i)}$.

\begin{align}
    \kernel{K}(A_S|x) &= \int_H \prod_{i\in S}\kernel{L}(A_i|h,x_i)\mu(\mathrm{d}h)\\
                      &= \int_H\prod_{i\in T}\kernel{L}(A_i|h,x_{\mathrm{swap}_{S\leftrightarrow T}(i)})\mu(\mathrm{d}h)\\
                      &= \mathrm{swap}_{S\leftrightarrow T}\kernel{K}(A_T|x)\\
                      &= \mathrm{swap}_{S\leftrightarrow T}\kernel{K}\mathrm{swap}_{S\leftrightarrow T}(A_S|x)
\end{align}
So by Theorem \ref{th:equal_of_condits}, $\kernel{K}$ is causally contractible.
\end{proof}

\subsection{Causal contractibility with data-independent actions}\label{sec:data_independent_actions}

Theorem \ref{th:ciid_rep_kernel} is proven for ``some causally contractible Markov kernel'', without a concrete role in a decision model. The simplest way to apply this to a decision model $\prob{P}_C$ is to assume that inputs are independent of previous observations -- that is, $\RV{D}_i\CI^e_{\prob{P}_C} (\RV{Y}_{<i},C)|\RV{D}_{<i}$. With this assumption, decision models with independent and identical response functions are those with \emph{causally contractible conditional probabilities} $\prob{P}_C^{\RV{Y}|\RV{D}}$. The assumption of data independence is a very limiting assumption -- see-do models, for example (Definition \ref{def:see_do_model}) do not satisfy it. The price to pay for the more general case is that we need to consider causally contractible \emph{combs} rather than causally contractible conditional probabilities, and this is addressed in Section \ref{sec:data_dependent}.

Given a sequence of variables $(\RV{D}_i,\RV{Y}_i)_{i\in \mathbb{N}}$ where the ``inputs'' are $\RV{D}:=(\RV{D}_i)_{i\in\mathbb{N}}$ and the ``outputs'' are $\RV{Y}=(\RV{Y}_i)_{i\in\mathbb{N}}$, say the inputs are independent of previous observations if $\RV{D}_i\CI^e_{\prob{P}_C} (\RV{Y}_{<i},C)|\RV{D}_{<i}$ for all $i\in\mathbb{N}$. This models an experiment where it may be possible to choose different inputs $\RV{D}$, but all the inputs are determined before the outputs $\RV{Y}$ are known. If a model satisfies this, then the dependence of $\RV{Y}$ on $\RV{D}$ is a sequence of repeatable response functions if and only if the uniform conditional $\prob{P}_C^{\RV{Y}|\RV{D}}$ exists (Definition \ref{def:cprob_pset}) and is causally contractible.

We call a model $\prob{P}_C$ with sequential outputs $\RV{Y}$ and a corresponding sequence of data-independent inputs $\RV{D}$ a ``sequential just-do model''.

\begin{definition}[Sequential just-do model]
A \emph{sequential just-do model} is a triple $(\prob{P}_C,\RV{D},\RV{Y})$ where $\prob{P}_C$ is a probability set on $(\Omega,\sigalg{F})$, $\RV{D}$ is a sequence of ``inputs'' $\RV{D}:=(\RV{D}_i)_{i\in\mathbb{N}}$ and $\RV{Y}$ is a corresponding sequence of ``outputs'' $\RV{Y}=(\RV{Y}_i)_{i\in\mathbb{N}}$ where $\RV{D}_i:\Omega\to D$ and $\RV{Y}_i:\Omega\to Y$. Furthermore, it is required that $\RV{X}_i\CI^e_{\prob{P}_C} (\RV{Y}_{<i},C)|\RV{X}_{<i}$ for all $i\in \mathbb{N}$, and $\RV{Y}\CI^e_{\prob{P}_C} C|\RV{D}$.
\end{definition}

To apply Theorem \ref{th:ciid_rep_kernel} to a sequential just-do model, it is necessary to extend the model to a larger sample space including ``latent variables'' taking values in $Y^D$. A latent extension is a model over a larger collection of variables that reduces to the original model when we restrict our attention to the original collection of variables.

\begin{definition}[Latent extension]
Given a probability set $\prob{P}_C'$ on $(\Omega,\sigalg{F})$ and some measurable set $(G,\sigalg{G})$, a probability set $\prob{P}_C$ is a \emph{latent extension} of $\prob{P}_C'$ to $(\Omega\times G,\sigalg{F}\otimes \sigalg{G})$ if $\prob{P}_C \kernel{F}_{\Pi_{\Omega}} = \prob{P}_C'$.
\end{definition}

\begin{notation}[Variables on a latent extension]
Given a probability set $\prob{P}_C'$ on $(\Omega,\sigalg{F})$ and a latent extension $\prob{P}_C$ on $(\Omega\times G,\sigalg{F}\otimes \sigalg{G})$, every variable on the original sample space is given a primed name $\RV{X}'$, $\RV{Y}'$ etc., and corresponds to an unprimed variable on the larger space $\RV{X}:=\Pi_\Omega\circ \RV{X}'$.
\end{notation}

Theorem \ref{th:data_ind_CC} applies Theorem \ref{th:ciid_rep_kernel} to the case of a model with data-independent actions and derives the required conditional independences and equalities to show that a sequential just-do model $(\prob{P}_C,\RV{D},\RV{Y})$ with causally contractible $\prob{P}_C^{\RV{Y}|\RV{X}}$ satisfies the required conditional independences and equalities of conditional distributions for $\RV{X}$ and $\RV{Y}$ to be related by repeatable response functions.

\begin{theorem}[Data-independent causal contractibility]\label{th:data_ind_CC}
Given a sequential just-do model $(\prob{P}_C',\RV{D}',\RV{Y}')$ on $(\Omega,\sigalg{F})$, then $\prob{P}_C^{\prime \RV{Y}'|\RV{D}'}$ is causally contractible if and only if there is a latent extension $\prob{P}_C$ of $\prob{P}_C'$ to $(\Omega\times Y^{D\times\mathbb{N}},\sigalg{F}\otimes\sigalg{Y}^{D\times\mathbb{N}})$ with some hypothesis $\RV{H}:\Omega\times Y^{D\times\mathbb{N}}\to H$ such that $\RV{Y}_i\CI^e_{\prob{P}_C'} (\RV{Y}_{<i},\RV{X}_{<i},C)|(\RV{X}_i,\RV{H})$ and $\prob{P}_C^{\RV{Y}_i|\RV{X}_i\RV{H}}=\prob{P}_C^{\RV{Y}_j|\RV{X}_j\RV{H}}$ for all $i,j\in \mathbb{N}$ and $\RV{H}\CI_{\prob{P}_C} (\RV{X},\RV{C})$.
\end{theorem}

\begin{proof}
If:
First, define the extension $\prob{P}_C$. From Theorem \ref{th:table_rep_kernel} and causal contractibility of $\prob{P}_C^{\prime \RV{Y}'|\RV{D}'}$ there is some $\mu\in \Delta(Y^{D\times\mathbb{N}})$ such that
\begin{align}
    \prob{P}_C^{\prime \RV{Y}'|\RV{D}'} &= \tikzfig{lookup_representation_variablised}\label{eq:lup_rep_varb}
\end{align}
Let $\prob{P}_C^{\RV{Y}^{D}|\RV{D}}=\mu\otimes \text{del}_{D^{\mathbb{N}}}$, $\prob{P}_C^{\RV{Y}|\RV{Y}^D\RV{D}}=\kernel{F}_{\mathrm{ev}}$ and $\RV{Y}^D:=\Pi_{Y^{D\times\mathbb{N}}}$, the projection $\Omega\times Y^{D\times\mathbb{N}}\to \Omega$. Let $\RV{W}=\Pi_{\Omega}$ and for each $\alpha\in C$, set 
\begin{align}
    \prob{P}_\alpha^{\RV{W}|\RV{Y}^D\RV{D}} &= \tikzfig{augmented_ccontracible}
\end{align}
Then 
\begin{align}
    \prob{P}_\alpha^{\RV{W}} &= \tikzfig{augmented_ccon2}\\
    &= \tikzfig{augmented_ccon3}\\
    &= \prob{P}_\alpha^{\prime \text{id}_\Omega}
    &= \prob{P}_\alpha
\end{align}

Before going further, it's necessary to check that there is some nonempty probability set $\prob{P}_C$ with these conditionals. By Theorem \ref{lem:valid_extendability}, because $\prob{P}_\alpha^{\RV{W}}=\prob{P}_{\alpha}'$ it is sufficient to show that $\prob{P}_\alpha^{\RV{Y}^D|\RV{W}}$ is valid. Because 
\begin{align}
    (\RV{W},\RV{Y}^D)(\Omega\times Y^{D\times\mathbb{N}})&=\Omega\times Y^{D\times\mathbb{N}}\\
    &=\RV{W}(\Omega\times Y^{D\times\mathbb{N}})\times \RV{Y}^D(\Omega\times Y^{D\times\mathbb{N}})
\end{align}
there are no impossible events, and so validity is guaranteed for any $\prob{P}_\alpha^{\RV{Y}^D|\RV{W}}$.

Thus $\prob{P}_C$ is a latent extension of $\prob{P}_C'$, and so $\prob{P}_C^{\RV{Y}|\RV{D}}$ is also causally contractible.

From Theorem \ref{th:ciid_rep_kernel} and by construction of $\prob{P}_C$, there exists a directing random measure $\RV{H}^*:Y^{D\times\mathbb{N}}\to H$ such that, defining $\RV{H}=\RV{H}^*\circ\Pi_{D\times\mathbb{N}}$
\begin{align}
    \prob{P}_C^{\RV{Y}|\RV{D}} &= \tikzfig{do_model_representation}
\end{align}
it remains to be shown that $\kernel{L}$ is a version of $\prob{P}^{\RV{Y}_i|\RV{D}_i\RV{H}}$ for all $i\in \mathbb{N}$ and $\RV{Y}_i\CI^e_{\prob{P}_C'} (\RV{Y}_{<i},\RV{D}_{<i},C)|(\RV{D}_i,\RV{H})$.

To show $\kernel{L}$ is a version of $\prob{P}^{\RV{Y}_i|\RV{H}\RV{D}_i}$ for all $i\in \mathbb{N}$:
\begin{align}
    \kernel{L}&=\tikzfig{kernel_l_broken_down}\\
              &=\tikzfig{kernel_l_broken_down2}\\
              &=\tikzfig{kernel_l_broken_down3}\label{eq:h_ci_x}\\
              &=\tikzfig{kernel_l_broken_down4}\label{eq:Y_ci_h_yd}\\
              &=\prob{P}_C^{\RV{Y}_i|\RV{H}\RV{D}_i}
\end{align}
Where \ref{eq:h_ci_x} follows from $\RV{H}\CI^e_{\prob{P}_C} (\RV{D}_i,\RV{C})$, which itself follows from $\RV{Y}^D_i\CI^e_{\prob{P}_C} (\RV{D}_i,\RV{C})$ which holds by construction. \ref{eq:Y_ci_h_yd} follows from $\RV{Y}_i\CI^e_{\prob{P}_C} (\RV{H},\RV{C})|(\RV{Y}^D_i,\RV{D}_i)$, which follows from $\RV{Y}_i$ being a deterministic function of $(\RV{Y}^D_i,\RV{D}_i)$.

For independence, note that
\begin{align}
     \prob{P}_C^{\RV{Y}_{<i}|\RV{H}\RV{X}_{<i}\RV{X}_i} &= \tikzfig{independence_inductive_base_0}\\
     &= \tikzfig{independence_inductive_base}
\end{align}
hence $\RV{Y}_{<i}\CI^e_{\prob{P}_C} (\RV{X}_i,\RV{C})|(\RV{H},\RV{X}_{<i})$

Then
\begin{align}
    \prob{P}_C^{\RV{Y}_i\RV{Y}_{<i}|\RV{H}\RV{D}_i\RV{D}_{<i}} &= \tikzfig{independence_inductive}\\
    \implies \prob{P}_C^{\RV{Y}_i|\RV{H}\RV{D}_i\RV{D}_{<i}} &\overset{\prob{P}_C}{\cong} \tikzfig{independence_inductive_last}
\end{align}
by Theorem \ref{th:higher_order_conditionals}. Hence $\RV{Y}_i\CI^e_{\prob{P}_C} (\RV{X}_{<i},\RV{Y}_{<i},\RV{C})|(\RV{H},\RV{X}_i)$.

Only if:
If $\prob{P}_C$ is a latent extension of $\prob{P}_C'$, then $\prob{P}_C^{\RV{Y}|\RV{D}}$ is causally contractible if and only if $\prob{P}_C^{\prime \RV{Y}'|\RV{D}'}$ is causally contractible. Thus it is sufficient to show $\prob{P}_C^{\RV{Y}|\RV{D}}$ is causally contractible.

By assumption, for all $i\in \mathbb{N}$
\begin{align}
    \prob{P}_C^{\RV{Y}_i|\RV{H}\RV{X}_{[i]}\RV{Y}_{<i}} &\overset{\prob{P}_C}{\cong} \text{del}_{X^{i-1}\times Y^{i-1}}\otimes \prob{P}_C^{\RV{Y}_1|\RV{H}\RV{X}_1}
\end{align}

Thus for all $n\in \mathbb{N}$ by repeated application of Theorem \ref{th:higher_order_conditionals}
\begin{align}
    \prob{P}_C^{\RV{Y}_{[n]}|\RV{H}\RV{X}_{[n]}} &\overset{\prob{P}_C}{\cong} \tikzfig{do_model_representation_finite}
\end{align}
thus by Lemma \ref{lem:infinitely_extended_kernels}
\begin{align}
    \prob{P}_C^{\RV{Y}_{\mathbb{N}}|\RV{H}\RV{X}_{\mathbb{N}}} &\overset{\prob{P}_C}{\cong} \tikzfig{do_model_representation_noprior}
\end{align}
and, because $\RV{H}\CI_{\prob{P}_C}^e(\RV{X},\RV{C})$
\begin{align}
    \prob{P}_C^{\RV{Y}_{\mathbb{N}}|\RV{X}_{\mathbb{N}}} &\overset{\prob{P}_C}{\cong} \tikzfig{do_model_representation}
\end{align}
causal contractibility follows from Theorem \ref{th:ciid_rep_kernel}.
\end{proof}

A consequence of Theorem \ref{th:equal_of_condits} applied to a just-do models $\prob{P}_C$ with causally contractible $\prob{P}_C^{\RV{Y}|\RV{D}}$ is that, for any $A,B\subset\mathbb{N}$ with $|A|=|B|$, $\prob{P}_C^{\RV{Y}_A|\RV{D}_A}=\prob{P}_C^{\RV{Y}_B|\RV{D}_B}$. A further consequence is the interchangeability of conditioning data -- for any $i\in \mathbb{N}$, $\prob{P}_C^{\RV{Y}_i|\RV{D}_i\RV{Y}_A\RV{D}_A}=\prob{P}_C^{\RV{Y}_i|\RV{D}_i\RV{Y}_B\RV{D}_B}$.

\begin{theorem}[Equality of subsequence conditionals]\label{th:equal_of_reduced_condits}
A sequential just-do model $(\prob{P}_C,\RV{D},\RV{Y})$ with $\prob{P}_C^{\RV{Y}|\RV{D}}$ causally contractible satisfies, for any $A,B\subset \mathbb{N}$ with $|A|=|B|$
\begin{align}
    \prob{P}_C^{\RV{Y}_A|\RV{D}_A} \overset{\prob{P}_C}{\cong} \prob{P}_C^{\RV{Y}_B|\RV{D}_B}
\end{align}
\end{theorem}

\begin{proof}
Only if:
For any $A,B\subset \mathbb{N}$, let $\text{swap}_{B\leftrightarrow A,D}:D^{\mathbb{N}}\kto D^{\mathbb{N}}$ be the transposiiton of $B$ with $A$ indices and $\text{swap}_{B\leftrightarrow A,Y}:Y^{\mathbb{N}}\kto Y^{\mathbb{N}}$ be the same defined on $Y$. By Theorem \ref{th:equal_of_condits}
\begin{align}
    \prob{P}_C^{\RV{Y}_A|\RV{D}_A}\otimes \text{del}_{D^{\mathbb{N}}} &=  \prob{P}_C^{\RV{Y}|\RV{D}} \text{marg}_A\\
     &= \text{swap}_{A\leftrightarrow[n],D} \prob{P}_C^{\RV{Y}_{[n]}|\RV{D}}\\
    &= \prob{P}_C^{\RV{Y}_{[n]}|\RV{D}_{[n]}}\otimes \text{del}_{D^{\mathbb{N}}}\\
    &= \text{swap}_{N\leftrightarrow[n],D} \prob{P}_C^{\RV{Y}_{[n]}|\RV{D}}\\
    &= \prob{P}_C^{\RV{Y}_B|\RV{D}_B}\otimes \text{del}_{D^{\mathbb{N}}}
\end{align}
\end{proof}

\subsubsection{Examples}

Purely passive observations can be modeled with a probability set $\prob{P}_C$ where $|\prob{P}_C|=1$. In this case, a model that is exchangeable over the sequence of pairs $(\RV{D}_i,\RV{Y}_i)_{i\in \mathbb{N}}$ has $\prob{P}_C^{\RV{Y}|\RV{D}}$ causally contractible. This follows from the fact that

\begin{align}
    \prob{P}_C^{\RV{YD}} &= \tikzfig{do_model_rep_onechoice_combined}
\end{align}
and so
\begin{align}
    \tikzfig{do_model_rep_onechoice}
\end{align}
is a version of $\prob{P}_C^{\RV{Y}|\RV{D}}$.

Instead of passive observations only, a model might feature a subsequence of passive observations and a subsequence of active interventions. Say the passive observations are $(\RV{D},\RV{Y})_{i\in\mathbb{N}}$ and the active interventions are $(\RV{E},\RV{Z})_{i\in \mathbb{N}}$. By the previous argument, $\prob{P}_C^{\RV{Y}|\RV{D}}$ is causally contractible. We might further assume that $\prob{P}_C^{\RV{YZ}|\RV{DE}}$ is causally contractible -- that is, there is a repeatable response function $\prob{P}_C^{\RV{Z}_i|\RV{E}_i\RV{H}}$ equal to $\prob{P}_C^{\RV{Y}_i|\RV{D}_i\RV{H}}$.

One conseqeuence of this is ``observational imitation'': any choice $\alpha$ that makes $\prob{P}_\alpha^{\RV{D}\RV{E}}$ exchangeable also makes $\prob{P}_\alpha^{\RV{YZ}}$ exchangeable. That is, if for some permutation $\mathrm{swap}_\rho$
\begin{align}
    \prob{P}_\alpha^{\RV{DE}}\mathrm{swap}_\rho &= \prob{P}_\alpha^{\RV{DE}}
\end{align}
then by commutativity of exchange
\begin{align}
    \prob{P}_\alpha^{\RV{YZ}} &= \prob{P}_\alpha^{\RV{DE}} \prob{P}_C^{\RV{YZ}|\RV{DE}}\\
    &=  \prob{P}_\alpha^{\RV{DE}}\mathrm{swap}_\rho \prob{P}_C^{\RV{YZ}|\RV{DE}}\\
    &= \prob{P}_\alpha^{\RV{DE}} \prob{P}_C^{\RV{YZ}|\RV{DE}}\mathrm{swap}_\rho\\
    &= \prob{P}_C^{\RV{YZ}|\RV{DE}}\mathrm{swap}_\rho
\end{align}

However, the assumption that $\prob{P}_C^{\RV{YZ}|\RV{DE}}$ is causally contractible seems unreasonable in most situations. One implication of this assumption is (by Theorem \ref{th:equal_of_condits}):
\begin{align}
    \prob{P}^{\RV{Y}\RV{Z}_i|\RV{D}\RV{E}_i}_C &= \prob{P}^{\RV{Z}|\RV{E}}\\
    \implies \prob{P}^{\RV{Z}_i|\RV{E}_i\RV{D}\RV{Y}}_C &= \prob{P}^{\RV{Z}_{i}|\RV{E}_i\RV{E}_{\{i\}^C}\RV{Z}_{\{i\}^C}}
\end{align}
That is, the model must yield the same result when conditioned on either the observational results, or the results of other active interventions. It is rare to assume \emph{a priori} that observational and experimental data are equally informative. Such a conclusion could be drawn \emph{after} reviewing both sequences of data, see for example \citet{eckles_bias_2021}, or it might be rejected \citet{gordon_comparison_2018,gordon_close_2022}.

\begin{example}[Backdoor adjustment]\label{ex:backdoor}
If a sequential just-do model $(\prob{P}_C,(\RV{D},\RV{X}),\RV{Y})$ has $\prob{P}_C^{\RV{Y}|\RV{DX}}$ causally contractible as well as:
\begin{itemize}
    \item $\RV{X}_{i}\CI^e_{\prob{P}_C}\RV{D}_{i}C|\RV{H}$ ($\RV{X}_i$ is extended independent of $\RV{D}_i$ conditional on $\RV{H}$)
    \item $\prob{P}_C^{\RV{X}_{i}|\RV{H}}\cong \prob{P}_C^{\RV{X}_{1}|\RV{H}}$ (the distribution of $\RV{X}$ is exchangeable)
 \end{itemize}
Then the model exhibits a kind of ``backdoor adjustment'' \citet[Chap. 1]{pearl_causality:_2009}. Specifically
\begin{align}
    \prob{P}_\alpha^{\RV{Y}_{i}|\RV{D}_{i}\RV{H}}(A|d,h) &= \int_X \prob{P}_\alpha^{\RV{Y}_{i}|\RV{X}_{i}\RV{D}_{i}\RV{H}}(A|d,x,h)\prob{P}_\alpha^{\RV{X}_{i}|\RV{D}_{i}\RV{H}}(\mathrm{d}x|d,h)\\
    &= \int_X \prob{P}_C^{\RV{Y}_{1}|\RV{X}_{1}\RV{D}_{1}\RV{H}}(A|d,x,h)\prob{P}_C^{\RV{X}_{i}|\RV{H}}(\mathrm{d}x|h)\\
    &= \int_X \prob{P}_C^{\RV{Y}_{1}|\RV{X}_{1}\RV{D}_{1}\RV{H}}(A|d,x,h)\prob{P}_C^{\RV{X}_{1}|\RV{H}}(\mathrm{d}x|h)\label{eq:backdoor}
\end{align}
\end{example}


Equation \ref{eq:backdoor} is identical to the backdoor adjustment formula for an intervention on $\RV{D}_1$ targeting $\RV{Y}_1$ where $\RV{X}_1$ is a common cause of both.

\section{Causal contractibility in sequences of active choices}\label{sec:assessing}

Assessing when a particular sequence of experiments should be modeled with a causally contractible model can be difficult. As noted, a purely observational sequence is causally contractible if it is exchangeable. However, the point of all this theory is to study models that offer different choices. The assumption of causal contractibility can be justified for a sequence of active interventions if the following are satisfied:
\begin{enumerate}
    \item There exist variables $\RV{I}$ representing ``unique experiment identifiers'' which satisfy the assumption that $\prob{P}_C^{\RV{Y}|\RV{DI}}$ is causally contractible (informally: it doesn't matter which order the experiments are conducted in, and treatments in each experiment do not affect any other experiments)
    \item Given a permutation $\rho$ of identifiers, $\prob{P}_\alpha^{\RV{YD}\rho(\RV{I})}=\prob{P}_\alpha^{\RV{YDI}}$ (informally: unique identifiers are not themselves informative)
    \item The map $\alpha\to \prob{P}_\alpha^{\RV{D}}$ is deterministic Markov kernel associated with an invertible function $f:C\to D^{\mathbb{I}}$
\end{enumerate}

Theorem \ref{th:cc_ind_treat} shows that, under these assumptions, $\prob{P}_C^{\RV{Y}|\RV{D}}$ is also causally contractible.

The first assumption -- that causal contractibility is satisfied jointly conditioning on decisions $\RV{D}$ and identifers $\RV{I}$ -- seems to often be a background assumption in the literature, while assumptions similar to the second two are discussed explicitly. For example, \citet{greenland_identifiability_1986} explain
\begin{quote}
    Equivalence of response type may be thought of in terms of exchangeability of individuals: if the exposure states of the two individuals had been exchanged, the same data distribution would have resulted.
\end{quote}
Note that exchanging individuals involved in an experiment and exchanging the individuals' exposure states are two different things, and the former doesn't imply the latter -- for example, there might be some background trend such that individuals treated later experience different outcomes to individuals treated at the start. Assumptions 1 and 2 \emph{together} imply that permuting identifiers or permuting decisions both lead to the same distribution.

\citet{dawid_decision-theoretic_2020} suggests (with some qualifications) that ``post-treatment exchangeability'' for a decision problem regarding taking aspirin to treat a headache may be acceptable if the data are from
\begin{quote}
    A group of individuals whom I can regard, in an intuitive sense, as similar to myself, with headaches similar to my own.
\end{quote}
This seems on the face of it similar to assumption 2: that I can permute the identifies ``me'' and ``someone else'' without changing the model.

Finally, \citet{rubin_causal_2005} discusses two separate assumptions to justify causal identifiability:
\begin{quote}
    indexing of the units is, by definition, a random permutation of $1,..., N$, and thus any distribution on the science must be row-exchangeable [...] The second critical fact is that if the treatment assignment mechanism is ignorable (e.g., randomized), then when the expression for the assignment mechanism (2) is evaluated at the observed data, it is free of dependence on $Y_{mis}$
\end{quote}
Here ``the science'' means (roughly) \emph{the response function of each individual}, and exchangeability of these response functions is a similar assumption to permutability of individual identifiers (though we don't derive the exact correspondence here). Rubin's second condition is that treatment assignment is ignorable. Like Assumption 3, this assumption limits ``how much we can learn from the treatment assignment'', but again we don't derive the exact correspondence. Note that in Rubin's scheme the preliminary assumption of \emph{stable unit-treatment values} is made in order to establish the existence of individual response functions, which plays a similar role to Assumption 1. 

Rubin's result also differs substantially from this one in that it applies to \emph{identification of potential outcomes in randomised experiments}, while this result is about the existence of response conditionals \emph{when there are different choices that can be made}.

As an example of the application of Theorem \ref{th:cc_ind_treat}, consider an experiment where $n$ patients, each with an individual identifier $\RV{I}_i$, receive treatment $\RV{D}_i$ and experience outcome $\RV{Y}_i$. $\prob{P}_C^{\RV{Y}_{[n]}|\RV{D}_{[n]}\RV{I}_{[n]}}$ can be extended to an infinite sequence $\prob{P}_C^{\RV{Y}|\RV{DI}}$ that is causally contractible (see Assumption 1), no matter which choice $\alpha\in C$ is decided on, all identifiers can be swapped without altering the distribution over consequences (see Assumption 2), and finally that the treatment vector $\RV{D}$ is a deterministic and invertible function of the choice $\alpha\in C$ then $\prob{P}_C^{\RV{Y}|\RV{D}}$ is causally contractible, and hence there are response functions $\prob{P}_C^{\RV{Y}_i|\RV{D}_i\RV{H}}$.

Theorem \ref{th:cc_ind_treat} can also be extended to the case where $\RV{D}$ is a function of the choice $\alpha$ and a ``random signal'' $\RV{R}$.

\begin{lemma}\label{lem:ind_to_cc}
Given sequential just-do model $(\prob{P}_C,(\RV{D},\RV{I}),\RV{Y})$ with $\prob{P}_C^{\RV{Y}|\RV{DI}}$ causally contractible, if $\RV{Y}\CI_{\prob{P}_C}^e (\RV{I},\RV{C})|\RV{D}$ then $\prob{P}_C^{\RV{Y}|\RV{D}}$ is also causally contractible.
\end{lemma}

\begin{proof}
For arbitrary $\nu\in \Delta(I^{\mathbb{N}})$, by assumption of causal contractibility of $\prob{P}_C^{\RV{Y}|\RV{DI}}$ and Theorem \ref{th:ciid_rep_kernel}
\begin{align}
    \prob{P}_C^{\RV{Y}|\RV{DI}} &= \tikzfig{index_independence_1}\\
    &= \tikzfig{index_independence_2}\\
    &= \tikzfig{index_independence_3}\\
    \implies \prob{P}_C^{\RV{Y}|\RV{D}} &= \tikzfig{index_independence_4}
\end{align}
Applying Theorem \ref{th:ciid_rep_kernel}, $\prob{P}_C^{\RV{Y}|\RV{D}}$ is causally contractible.
\end{proof}

An \emph{identifier variable} is a variable $\RV{I}$ that takes values in the set of finite permutations of $\mathbb{N}$. It is associated with a sequence $(\RV{I}_i)_{i\in \mathbb{N}}$ where $\RV{I}_i=\RV{I}(i)$. Each $\RV{I}_i$ takes values in $\mathbb{N}$ and $\RV{I}_i\neq \RV{I}_j$ for all $j\neq i$.

\begin{definition}[Identifier variable]
Given a probability set $\prob{P}_C$ on $(\Omega,\sigalg{F})$, let $I$ be the set of finite permutations $\mathbb{N}\to \mathbb{N}$. A variable $\RV{I}:\Omega\to I$ be a variable taking values in $I$ is an \emph{identifier variable}.
\end{definition}

If a uniform conditional probability is invariant to permutations of an index variable, then it is independent of that index variable.

\begin{lemma}\label{lem:ind}
Given a probability set $\prob{P}_C$ where $\RV{Y}\CI_{\prob{P}_C}^e \RV{C}|(\RV{D},\RV{I})$ and $\RV{I}:\Omega\to I$ is an identifier variable, if for each finite permutation $\rho:\mathbb{N}\to \mathbb{N}$
\begin{align}
    \prob{P}_\alpha^{\RV{Y}|\RV{ID}} &= (\text{swap}_{\rho(I)}\otimes \text{Id}_X )\prob{P}_\alpha^{\RV{Y}|\RV{ID}}
\end{align}
then $\RV{Y}\CI_{\prob{P}_C}^e (\RV{I},\RV{C})|\RV{D}$.
\end{lemma}

\begin{proof}
By definition of the set $I$ of finite permutations, for every $\rho\in I$, $B\in\sigalg{Y}^{\mathbb{N}}$, $d\in D^{\mathbb{N}}$ there is a finite permutation $\rho^{-1}\in I$ such that $\rho\circ\rho^{-1}=\text{id}_{\mathbb{N}}$. Then
\begin{align}
    \prob{P}_C^{\RV{Y}|\RV{ID}}(B|i,d) &= (\kernel{F}_{\rho^{-1}}\otimes \text{Id}_X )\prob{P}_C^{\RV{Y}|\RV{ID}}(B|\rho,d)\\
    &= \prob{P}_C^{\RV{Y}|\RV{ID}}(B|\text{id}_{\mathbb{N}},d)
\end{align}
Therefore
\begin{align}
    \prob{P}_C^{\RV{Y}|\RV{ID}} &\overset{\prob{P}_C}{\cong} \text{erase}_{I}\otimes \kernel{K} 
\end{align}
where $\kernel{K}:D^{\mathbb{N}}\kto Y^{\mathbb{N}}$ is the kernel
\begin{align}
    (B|d)\mapsto \prob{P}_C^{\RV{Y}|\RV{ID}}(B|\text{id}_{\mathbb{N}},d)
\end{align}
\end{proof}

The following theorem assumes that the set of choices $C$ is countable and there is a one-to-one function $f:C\to D^{\mathbb{N}}$. Thus, if $|D|>1$, $f$ cannot be surjective.

\begin{theorem}\label{th:cc_ind_treat}
Given a sequential just-do model $(\prob{P}_C,(\RV{D},\RV{I}),\RV{Y})$ on $(\Omega,\sigalg{F})$ with $C$ countable, $\prob{P}_C^{\RV{Y}|\RV{DI}}$ causally contractible $\RV{I}:\Omega\to I$ an identifier variable, if for each $\alpha\in C$, $\rho\in I$
\begin{align}
    \prob{P}_\alpha^{\RV{Y}|\RV{I}} &= \kernel{F}_{\rho}\prob{P}_\alpha^{\RV{Y}|\RV{I}}
\end{align}
and furthermore
\begin{align}
    &\RV{YI}\CI^e_{\prob{P}_C} \RV{C}|\RV{D}\\
    &\RV{YI}\CI^e_{\prob{P}_C} \RV{D}|\RV{C}
\end{align}
then $\prob{P}_C^{\RV{Y}|\RV{D}}$ is causally contractible.
\end{theorem}

\begin{proof}
For any $\alpha\in C$ by Theorem \ref{th:higher_order_conditionals}
\begin{align}
    \prob{P}_\alpha^{\RV{Y}|\RV{I}} &= \tikzfig{kernel_fac_with_idents}\\
    &= \tikzfig{kernel_fac_with_idents_indepped}
\end{align}

Define $\kernel{Q}$ by $\alpha\mapsto \prob{P}_\alpha$ and $\kernel{Q}^{\cdot|\cdot\RV{C}}$ by $\alpha\mapsto \prob{P}_\alpha^{*}$ and $\kernel{Q}^{\RV{C}}$ is an arbitrary distribution in $\Delta(C)$ with full support. Note that the support of $\kernel{Q}^{\RV{IDY}}$ is the union of the support of $\prob{P}^{\RV{IDY}}_\alpha$ for all $\alpha$. Then
\begin{align}
    \kernel{Q}^{\RV{Y}|\RV{IC}} &\overset{\prob{Q}}{\cong} \tikzfig{kernel_fac_with_idents_kernelised}
\end{align}

By assumption $\RV{YI}\CI^e_{\prob{P}_C} \RV{D}|\RV{C}$, it is also the case that
\begin{align}
    \kernel{Q}^{\RV{Y}|\RV{ID}} &\overset{\prob{Q}}{\cong} \tikzfig{kernel_Q_fac_with_idents}\\
    &\overset{\prob{Q}}{\cong} \tikzfig{kernel_Q_fac_with_idents_indepped}\\
    &\overset{\prob{Q}}{\cong} \tikzfig{kernel_Q_fac_with_idents_subbed}
\end{align}
But
\begin{align}
    \kernel{Q}^{\RV{Y}|\RV{ID}}=\sum_{\alpha\in C} \prob{P}_\alpha^{\RV{Y}|\RV{ID}}\kernel{Q}^{\RV{C}}(\alpha)\\
    &= \prob{P}_C^{\RV{Y}|\RV{ID}}
    \implies \tikzfig{kernel_Q_fac_with_idents_subbed} &= \prob{P}_C^{\RV{Y}|\RV{ID}}
\end{align}

Furthermore, by assumption, for any permutation $\rho:\mathbb{N}\to\mathbb{N}$
\begin{align}
    \kernel{Q}^{\RV{Y}|\RV{IC}} &= \tikzfig{kernel_Q_with_swap}\\
    \implies \prob{P}_C^{\RV{Y}|\RV{ID}} &= \tikzfig{kernel_Q_fac_with_idents_swapped}\\
    &= \tikzfig{kernel_Q_fac_with_idents_swapped_exch}\\
    &= \tikzfig{kernel_P_with_swap}
\end{align}
Then by Lemma \ref{lem:ind} the independence $\RV{Y}\CI_{\prob{P}_C}^e \RV{I}C|\RV{D}$ holds, and by Lemma \ref{lem:ind_to_cc} $\prob{P}_C^{\RV{Y}|\RV{D}}$ is causally contractible.
\end{proof}

Theorem \ref{th:cc_ind_treat} can be extended to the case where decisions $\RV{D}$ are one-to-one deterministic, or random mixtures of one-to-one deterministic.

\begin{corollary}
Given a sequential just-do model $(\prob{P}_{C'},(\RV{D},\RV{I}),\RV{Y})$ satisfying the conditions of Theorem \ref{th:cc_ind_treat} and a second model $(\prob{P}_{C},(\RV{D},\RV{I}),\RV{Y})$ such that for all $\alpha\in C$ there is some set of coefficients $k_i$ such that
\begin{align}
    \prob{P}_\alpha &= \sum_{c\in C'} k_c \prob{P}_c 
\end{align}
then $\prob{P}_C^{\RV{Y}|\RV{D}}$ is also causally contractible.
\end{corollary}

\begin{proof}
For all $\alpha\in C$
\begin{align}
    \prob{P}_\alpha^{\RV{Y}|\RV{D}} &= \sum_{c\in C'} k_c \prob{P}_c^{\RV{Y}|\RV{D}}\\
    &= \prob{P}_{C'}^{\RV{Y}|\RV{D}}
\end{align}
\end{proof}

Dropping the assumption $\RV{YI}\CI^e_{\prob{P}_C} \RV{C}|\RV{D}$ means that, in general, one or both of $\prob{P}_C^{\RV{Y}|\RV{D}}$ or $\prob{P}_C^{\RV{Y}|\RV{ID}}$ may be ill-defined (note that the independence is merely a sufficient condition, not a necessary condition for these uniform conditional probabilities). The condition $\RV{YI}\CI^e_{\prob{P}_C} \RV{C}|\RV{D}$ alone also does \emph{not} imply the conclusion of Theorem \ref{th:cc_ind_treat}. 

Constructing the following example requires the hypotheses that any given identifier $i\in\mathbb{N}$ could be associated with one of two input-output maps $D\kto Y$. This generates space of hypotheses $H=\{0,1\}^{\mathbb{N}}$, which needs to be equipped with an algebra of measurable sets. Equipped with the product topology, $H$ is a countable product of separable, completely metrizable spaces and is therefore also separable and completely metrizable \citep[Thm. 16.4,Thm. 24.11]{willard_general_1970}. Thus $(H,\mathcal{B}(H))$ is a standard measurable space and because it is uncountable, it is isomorphic to $([0,1],\mathcal{B}([0,1]))$.

\begin{example}
Take $Y=C=D=\{0,1\}$ and take $(H,\sigalg{H})$ to be $\{0,1\}^{\mathbb{N}}$ equipped with the product topology. For any $i\neq 1$, $\RV{Y}_i\RV{I}_i\RV{D}_i\CI^e_{\prob{P}_C} \RV{C}$, while $\prob{P}_\alpha^{\RV{D}_1}=\delta_\alpha$ and $\RV{I}_i\CI^e_{\prob{P}_C} \RV{C}$.

$\RV{YI}\CI^e_{\prob{P}_C} \RV{C}|\RV{D}$ follows from the fact that $\RV{C}$ can be (almost surely) written as a function of $\RV{D}$.

For all $i,\in \mathbb{N}$, $y,d\in \{0,1\}$, $h\in H$ set
\begin{align}
    \prob{P}_C^{\RV{Y}_i|\RV{H}\RV{I}_i\RV{D}_i}(y|h,j,d) &= \delta_1(p(j,h))\delta_d(y) + \delta_0(p(j,h))\delta_{1-d}(y)
\end{align}
where $p(j,h)$ projects the $j$-th component of $h$. That is, if $h$ maps $j$ to 1, $\RV{Y}$ goes with $\RV{D}$ while if $h$ maps $j$ to $0$, $\RV{Y}$ goes opposite $\RV{D}$. Suppose also 
\begin{align}
    \RV{Y}_i\CI_{\prob{P}_C}^e (\RV{X}_{<i},\RV{Y}_{<i},\RV{I}_{<i},\RV{C})|(\RV{X}_i,\RV{Y}_i,\RV{H})
\end{align}
Then $\prob{P}_C^{\RV{Y}|\RV{DI}}$ is causally contractible. Set $\prob{P}_{C}^{\RV{H}}$ to be the uniform measure on $(H,\sigalg{H})$ and for $i>1$
\begin{align}
    \prob{P}_C^{\RV{D}_i|\RV{I}_i\RV{H}}(d|j,h) &= \delta_{p(j,h)}(d)
\end{align}
that is, if $h$ maps $j$ to 1, $\RV{D}$ is 1 while if $h$ maps $j$ to $0$, $\RV{D}$ is 0. This also implies
\begin{align}
    \prob{P}_C^{\RV{I}_i|\RV{D}_i\RV{H}}(p(\cdot,h)^{-1}(d)|d,h) &= 1\label{eq:all_eq_d}
\end{align}

Then, for $i>1$
\begin{align}
    \prob{P}_\alpha^{\RV{Y}_i|\RV{H}\RV{D}_i}(y|h,d) &= \sum_{j\in \mathbb{N}} \delta_1(p(j,h))\delta_d(y)\prob{P}_C^{\RV{I}_i|\RV{D}_i\RV{H}}(j|d,h) + \delta_0(p(j,h))\delta_{1-d}(y)\prob{P}_C^{\RV{I}_i|\RV{D}_i\RV{H}}(j|d,h)\\
    &= \sum_{j\in \mathbb{N}} \delta_1(d)\delta_d(y)\prob{P}_C^{\RV{I}_i|\RV{D}_i\RV{H}}(j|d,h) + \delta_0(d)\delta_{1-d}(y)\prob{P}_C^{\RV{I}_i|\RV{D}_i\RV{H}}(j|d,h)&\text{by Eq \ref{eq:all_eq_d}}\\
    &= \delta_1(y)\\
    \implies \prob{P}_\alpha^{\RV{Y}_i|\RV{D}_i}(y|d) &= \delta_1(y)
\end{align}

For $q\in I$, set
\begin{align}
    \prob{P}_C^{\RV{I}|\RV{H}}(q|h)&= \begin{cases}
        0.5 & q=(1,2,3,4,...) \text{ or } (1,3,2,4,...)\\
        0&\text{otherwise}
    \end{cases}
\end{align}
and set
\begin{align}
    \prob{P}_C^{\RV{H}|\RV{D}}(h) &= \begin{cases}
        0.5 & h=(0,1,0,1,1,...)\text{ or }h=(0,0,1,1,1,...)\\
        0 &\text{otherwise}
    \end{cases}
\end{align}
Let $\overline{H}$ be the support of $\prob{P}_C^{\RV{H}|\RV{D}}(h)$.

Then for $i=1$
\begin{align}
    \prob{P}_\alpha^{\RV{Y}_1|\RV{D}_1}(y|h,d) &= \sum_{h\in H} \sum_{j\in \mathbb{N}} \prob{P}_\alpha^{\RV{I}_1|\RV{D}_1\RV{H}}(j|d,h)\prob{P}_C^{\RV{H}|\RV{D}_1}(h|d)\left(\delta_1(p(j,h))\delta_d(y) + \delta_0(p(j,h))\delta_{1-d}(y)\right)\\
    &= \sum_{h\in \overline{H}} 0.5( \delta_1(p(1,h))\delta_d(y) + \delta_0(p(1,h))\delta_{1-d}(y))\\
    &= \delta_{1-d}(y))\\
    &\neq  \prob{P}_\alpha^{\RV{Y}_i|\RV{D}_i}(y|h,d) & i\neq 1
\end{align}
Thus $\prob{P}_C^{\RV{Y}|\RV{D}}$ is not causally contractible by Theorem \ref{th:equal_of_condits}. 

However, given any finite permutation $\rho:\mathbb{N}\to\mathbb{N}$
\begin{align}
    \prob{P}_\alpha^{\RV{Y}|\RV{I}}(y|q) &= \sum_{h\in \overline{H}}\sum_{d\in\{0,1\}^{\mathbb{N}}} \prod_{i\in \mathbb{N}} \prob{P}_C^{\RV{Y}_i|\RV{I}_i\RV{D}_i\RV{H}}(y_i|q_i,d_i,h) \prob{P}_\alpha^{\RV{D}_i|\RV{I}_i\RV{H}}(d_i|q_i,h)\prob{P}_C^{\RV{H}}(h)\\
    &= \delta_{1-\alpha}(y_1)\delta_{(1)_{i\in\mathbb{N}}}(y_{>1})\\
    &= \prob{P}_\alpha^{\RV{Y}|\RV{I}}(y|\rho^{-1}(q))\\
    &= \kernel{F}_{\rho}\prob{P}_\alpha^{\RV{Y}|\RV{I}}(y|q)
\end{align}
\end{example}

\subsubsection{Example: body mass index}

Given a sequential just-do model $(\prob{P}_C,(\RV{B},\RV{I}),\RV{Y})$ with $\RV{B}:=(\RV{B}_i)_{i\in M}$ representing body mass index of individual $\RV{I}_i$ and $\RV{Y}:=(\RV{Y}_i)_{i\in M}$ representing health outcomes of interest for the same individual, \citet{hernan_does_2008} noted that there are multiple different choices that can influence an individual's body mass index $\RV{B}_i$ in the same way. Thus $\RV{YI}\CI^e_{\prob{P}_C} \RV{C}|\RV{B}$ might generally be rejected, and so there may be no uniform conditional $\prob{P}_C^{\RV{Y}|\RV{IB}}$. In this case, $\prob{P}_C^{\RV{Y}|\RV{IB}}$ cannot be causally contractible because it doesn't exist.

Suppose instead a model $(\prob{P}_C,(\RV{D},\RV{I}),(\RV{B},\RV{Y}))$ is given, with $\RV{D}=(\RV{D}_i)_{i\in M}$ representing ``decisions'', appropriately fine-grained to satisfy
\begin{align}
    &\RV{YBI}\CI^e_{\prob{P}_C} \RV{C}|\RV{D}\\
    &\RV{YBI}\CI^e_{\prob{P}_C} \RV{D}|\RV{C}
\end{align}
and $\prob{P}_C^{\RV{YB}|\RV{ID}}$ causally contractible. Then by Theorem \ref{th:cc_ind_treat} $\prob{P}_C^{\RV{Y}|\RV{BD}}$ is also causally contractible. In general, there may be some $U\subset H$ such that for any $h\in U$ 
\begin{align}
    \prob{P}_C^{\RV{Y}_i|\RV{B}_i\RV{D}_i\RV{H}}(y|b,d,h) &= \prob{P}_C^{\RV{Y}_i|\RV{B}_i\RV{H}}(y|b,h)\label{eq:conditional_conditional_independence}
\end{align}
then, \emph{conditioning on }$\RV{H}\in U$, the resulting $\prob{P}_{C,\RV{H}\in U}^{\RV{Y}|\RV{B}}$ is causally contractible.
\todo[inline]{Defining conditioning}
So it may be possible to derive the fact that there is a repeatable response conditional $\prob{P}_{C,\RV{H}\in U}^{\RV{Y}_i|\RV{H}\RV{B}_i}$ if $\RV{H}\in U$ is implied by available data, even if it is not assumed outright.

\section{Response conditionals with data-dependent actions}\label{sec:data_dependent}

The results of the previous section concern ``just-do'' models where actions have not dependence on previous data. Decision problems of interest actually have actions that depend on data -- what's really wanted are ``see-do'' models of some variety (see Definition \ref{def:see_do_model}). Here, Theorem \ref{th:data_ind_CC} is generalised to sequential see-do models with the use of \emph{probability combs}.

To begin with an example, consider a probability set $(\prob{P}_C,\RV{D},\RV{Y})$ with $\RV{D}:=(\RV{D}_i)_{i\in\mathbb{N}}$ and $\RV{Y}:=(\RV{Y}_i)_{i\in\mathbb{N}}$ as usual, and take a subsequence $(\RV{D}_i,\RV{Y}_i)_{i\in [2]}$ of length 2. Suppose $\prob{P}_C$ features repeatable response conditionals in the sense that the following holds
\begin{align}
    \RV{Y}_i&\CI^e_{\prob{P}_C} (\RV{Y}_{<i},\RV{D}_{<i},\RV{C})|\RV{H}\RV{D}_i&\forall i\in \mathbb{N}\\
    \land \RV{H} &\CI^e_{\prob{P}_C} \RV{D} C\\
    \land \prob{P}_C^{\RV{Y}_i|\RV{H}\RV{D}_i} &= \prob{P}_C^{\RV{Y}_0|\RV{H}\RV{D}_0} & \forall i\in \mathbb{N}
\end{align}

Then, for arbitrary $\alpha\in C$
\begin{align}
    \prob{P}_\alpha^{\RV{Y}_{[2]}} &= \tikzfig{response_conditional_comb}
\end{align}
note that $\RV{D}_2$ depends on $\RV{Y}_1$ and $\RV{D}_1$. Instead of multiplying by a distribution over $(\RV{D}_1,\RV{D}_2)$, $\prob{P}_\alpha^{\RV{D}_2|\RV{Y}_1\RV{D}_1}$ has been ``inserted'' between the response conditionals $\prob{P}_C^{\RV{Y}_1|\RV{D}_1\RV{H}}$ and $\prob{P}_C^{\RV{Y}_2|\RV{D}_2\RV{H}}$. A comb is a Markov kernel that yields a probability distribution when another Markov kernel of appropriate type is inserted in this manner.

Given $\prob{P}_C^{\RV{Y}_1|\RV{D}_1\RV{H}}$ and $\prob{P}_C^{\RV{Y}_2|\RV{D}_2\RV{H}}$, define the comb
\begin{align}
    \prob{P}_C^{\RV{Y}_{[2]}\combbreak \RV{D}_{[2]}} := \tikzfig{causally_contractible_comb}
\end{align}
then $\prob{P}_C^{\RV{Y}_{[2]}\combbreak \RV{D}_{[2]}}$ is causally contractible. $\prob{P}_C^{\RV{Y}_{[2]}\combbreak \RV{D}_{[2]}}$ is \emph{not} a uniform conditional probability; in general 
\begin{align}
    \prob{P}_\alpha^{\RV{D}_1\RV{D}_2} \prob{P}_C^{\RV{Y}_{[2]}\combbreak \RV{D}_{[2]}}\neq \prob{P}_\alpha^{\RV{Y}_1\RV{Y}_2}
\end{align}

\subsection{Combs}\label{sec:def_combs}

Combs generalise conditional probabilities in this sense: given a conditional distribution and a marginal distribution of the right type, joining them together (with the semidirect product\ref{def:copyproduct}) I get a marginal distribution of a different type. Define ``1-combs'' as conditional probabilities and ``0-combs'' as conditional distributions. Then the previous observation can be restated as: given a 1-comb and a 0-comb of the right type,  joining them together yields a 0-comb of a different type. Higher order combs generalise this: given an $n$-comb and an $n-1$-comb of the right type, joining them yields an $n-1$ comb.

Joining combs uses an ``insert'' operation (Definition \ref{def:insert_discrete}).  A graphical depiction of this operation gives some intuition for why it is called ``insert'':
% \begin{align}
%     \prob{P}_\alpha^{\RV{XY}}&=\prob{P}_\alpha^{\RV{X}}\cprod\prob{P}_C^{\RV{Y}|\RV{X}}\\
%     &= \tikzfig{conditional_semidirect_product}
% \end{align}
% and the insert operation looks like
\begin{align}
    \prob{P}_\alpha^{\RV{Y}_{1}\RV{D}_2\RV{Y}_2|\RV{D}_1}&=\text{insert}(\prob{P}_\alpha^{\RV{D}_2|\RV{D}_1\RV{Y}_1},\prob{P}_C^{\RV{Y_{[2]}}\combbreak\RV{D}_{[2]}})\\
    &= \tikzfig{comb_insert_complicated}\label{eq:comb_insert_complicated}\\
    &= \tikzfig{comb_insert_gettingsimpler}\\
    &= \tikzfig{comb_insert_simple}\label{eq:comb_insert_simple}
\end{align}
While Equation \ref{eq:comb_insert_complicated} is a well-formed string diagram in the category of Markov kernels, Equation \ref{eq:comb_insert_simple} is not. In the case that all the underlying sets are discrete, Equation \ref{eq:comb_insert_simple} can be defined using an extended string diagram notation appropriate for the category of real-valued matrices \citep{jacobs_causal_2019}, though we do not introduce this extension here.

Formal definitions of finite and infinite combs follow, which will be used in Section \ref{sec:data_dependent_representation} to generalise Theorem \ref{th:data_ind_CC} to the data-dependent case.

\begin{definition}[Uniform $n$-Comb]
Given a probability set $\prob{P}_C$ with variables $\RV{Y}_i:\Omega\to Y$, $\RV{D}_i:\Omega\to D$ for $i\in [n]$ and uniform conditional probabilities $\{\RV{P}_C^{\RV{Y}_i|\RV{D}_{[i]}\RV{Y}_{[i-1]}}|i\in [n]\}$, the uniform $n$-comb $\prob{P}_C^{\RV{Y}_{[n]}\combbreak \RV{D}_{[n]}}:D^n\kto Y^n$ is the Markov kernel given by the recursive definition
\begin{align}
    \prob{P}_C^{\RV{Y}_{1}\combbreak \RV{D}_{1}} &= \prob{P}_C^{\RV{Y}_1|\RV{D}_1}\\
    \prob{P}_C^{\RV{Y}_{[m]}\combbreak \RV{D}_{[m]}} &= \tikzfig{comb_inductive}
\end{align}
\end{definition}

\begin{definition}[Uniform $\mathbb{N}$-comb]
Given a probability set $\prob{P}_C$ with variables $\RV{Y}_i:\Omega\to Y$ and $\RV{D}_i:\Omega\to D$ for $i\in \mathbb{N}$ and uniform conditional probabilities $\{\RV{P}_C^{\RV{Y}_i|\RV{D}_{[i]}\RV{Y}_{[i-1]}}|i\in \mathbb{N}\}$, the uniform $\mathbb{N}$-comb $\prob{P}_C^{\RV{Y}_{\mathbb{N}}\combbreak \RV{D}_{\mathbb{N}}}:D^\mathbb{N}\kto Y^\mathbb{N}$ is the Markov kernel such that for all $n\in \mathbb{N}$
\begin{align}
    \prob{P}_C^{\RV{Y}_{\mathbb{N}}\combbreak \RV{D}_{\mathbb{N}}}(\mathrm{id}_{Y^{n}}\otimes \mathrm{del}_{Y^{\mathbb{N}}}) &= \prob{P}_C^{\RV{Y}_{[n]}\combbreak \RV{D}_{[n]}}\otimes \mathrm{del}_{Y^{\mathbb{N}}}
\end{align}
\end{definition}

\begin{theorem}[Existence of $\mathbb{N}$-combs]
Given a probability set $\prob{P}_C$ with variables $\RV{Y}_i:\Omega\to Y$ and $\RV{D}_i:\Omega\to D$ for $i\in \mathbb{N}$ and uniform conditional probabilities $\{\RV{P}_C^{\RV{Y}_i|\RV{D}_{[i]}\RV{Y}_{[i-1]}}|i\in \mathbb{N}\}$, a uniform $\mathbb{N}$-comb $\prob{P}_C^{\RV{Y}_{\mathbb{N}}\combbreak \RV{D}_{\mathbb{N}}}:D^\mathbb{N}\kto Y^\mathbb{N}$ exists.
\end{theorem}

\begin{proof}
For each $n\in \mathbb{N}$ $m<n$, we have
\begin{align}
    \prob{P}_C^{\RV{Y}_{[n]}\combbreak \RV{D}_{[n]}}(\mathrm{id}_{Y^{n-m}})\otimes \mathrm{del}_{Y^m}) &= \prob{P}_C^{\RV{Y}_{[n-m]}\combbreak \RV{D}_{[n-m]}}\otimes \mathrm{del}_{Y^m}
\end{align}

Therefore the existence of $\prob{P}_C^{\RV{Y}_{\mathbb{N}}\combbreak \RV{D}_{\mathbb{N}}}$ is a consequence of Lemma \ref{lem:infinitely_extended_kernels}.
\end{proof}

For discrete sets, the insert operation has a compact definition

\begin{definition}[Comb insert - discrete]\label{def:insert_discrete}
Given an $n$-comb $\prob{P}_\alpha^{\RV{Y}_{[n]}\combbreak \RV{D}_{[n]}}$ and an $n-1$ comb $\prob{P}_\alpha^{\RV{D}_{[2,n]}|\RV{Y}_{[n-1]}}$, $(D,\sigalg{D})$ and $(Y,\sigalg{Y})$ discrete, for all $y_i\in Y$ and $d_i\in D$
\begin{align}
    \mathrm{insert}(\prob{P}_\alpha^{\RV{D}_{[2,n]}\combbreak \RV{Y}_{[n-1]}},\prob{P}_\alpha^{\RV{Y}_{[n]}\combbreak \RV{D}_{[n]}})(y_{[n]},d_{[2,n]}|d_1) &= \prob{P}_\alpha^{\RV{Y}_{[n]}\combbreak \RV{D}_{[n]}}(y_n|d_{[2,n]},d_1)\prob{P}_\alpha^{\RV{D}_{[1,n]}\combbreak\RV{Y}_{[n-1]}}(d_{[2,n]}|y_{[n-1]})
\end{align}
\end{definition}

\subsection{Response conditionals in models with data dependent actions}\label{sec:data_dependent_representation}

Theorem \ref{th:response_hdep} generalises Theorem \ref{th:data_ind_CC} to models $(\prob{P}_C,\RV{D},\RV{Y})$ with data-dependent actions, where instead the assumption that the uniform comb $\prob{P}_C^{\RV{Y}\combbreak \RV{D}}$ is causally contractible replaces the assumption that the conditional probability $\prob{P}_C^{\RV{Y}| \RV{D}}$ is causally contractible.

\begin{definition}[Sequential see-do model]
A \emph{sequential see-do model} is a triple $(\prob{P}_C,\RV{D},\RV{Y})$ where $\prob{P}_C$ is a probability set on $(\Omega,\sigalg{F})$, $\RV{D}$ is a sequence of ``inputs'' $\RV{D}:=(\RV{D}_i)_{i\in\mathbb{N}}$ and $\RV{Y}$ is a corresponding sequence of ``outputs'' $\RV{Y}=(\RV{Y}_i)_{i\in\mathbb{N}}$ where $\RV{D}_i:\Omega\to D$ and $\RV{Y}_i:\Omega\to Y$ and $\RV{Y}_i\CI^e_{\prob{P}_C} C|(\RV{D}_{[i]},\RV{Y}_{<i})$.
\end{definition}

\begin{theorem}[]\label{th:response_hdep}
Given a sequential see-do model $(\prob{P}_C',\RV{D}',\RV{Y}')$ on $(\Omega,\sigalg{F})$, then $\prob{P}_C^{\prime \RV{Y}'\combbreak \RV{D}'}$ is causally contractible if and only if there is a latent extension $\prob{P}_C$ of $\prob{P}_C'$ to $(\Omega\times H,\sigalg{F}\otimes\sigalg{Y}^{D\times\mathbb{N}})$ with hypothesis $\RV{H}:\Omega\times H\to H$ such that $\RV{Y}_i\CI^e_{\prob{P}_C'} (\RV{Y}_{<i},\RV{X}_{<i},C)|(\RV{X}_i,\RV{H})$ and $\prob{P}_C^{\RV{Y}_i|\RV{X}_i\RV{H}}=\prob{P}_C^{\RV{Y}_j|\RV{X}_j\RV{H}}$ for all $i,j\in \mathbb{N}$ and $\RV{H}\CI_{\prob{P}_C} (\RV{X},\RV{C})$.
\end{theorem}

\begin{proof}
If:
By assumption, there is some $\kernel{L}:H\times D\kto Y$ such that
\begin{align}
    \prob{P}_C^{\RV{Y}_i|\RV{H}\RV{D}_i} &= \kernel{L}
\end{align}
and $\RV{Y}_i\CI^e_{\prob{P}_C} (\RV{Y}_{<i},\RV{D}_{<i})|(\RV{D}_i,\RV{H})$. Thus
\begin{align}
    \prob{P}_C^{\RV{Y}_i|\RV{H}\RV{D}_i\RV{Y}_{<i}\RV{D}_{<i}} &= \kernel{L}\otimes \text{erase}_{Y^{i-1}\times D^{i-1}}
\end{align}
and so
\begin{align}
    \prob{P}_C^{\RV{Y}\combbreak \RV{D}} &= \tikzfig{do_model_representation}\label{eq:comb_representation_w_CI}
\end{align}
and so by Theorem \ref{th:ciid_rep_kernel}, $\prob{P}_C^{\RV{Y}\combbreak \RV{D}}$ is causally contractible.

Only if:
First, define the extension $\prob{P}_C$. From Theorem \ref{th:ciid_rep_kernel} and causal contractibility of $\prob{P}_C^{\prime \RV{Y}'\combbreak \RV{D}'}$ there is some $H$, $\mu\in \Delta(H)$ and $\kernel{L}:H\times D\kto Y$ such that
\begin{align}
    \prob{P}_C^{\prime \RV{Y}'\combbreak\RV{D}'} &= \tikzfig{do_model_representation_mu}
\end{align}
thus, by the definition of the comb insert operation
\begin{align}
    \prob{P}_\alpha^{\prime\RV{D}'_{[n]} \RV{Y}'_{[n]}} &= \prob{P}_\alpha^{\RV{D}_1}\odot \text{insert}(\prob{P}_\alpha^{\prime \RV{D}'_{[2,n]}\combbreak\RV{Y}'_{[n-1]}}, \prob{P}_C^{\prime \RV{Y}'_{[n]}\combbreak\RV{D}'_{[n]}}) 
\end{align}
Let
\begin{align}
    \prob{P}_C^{\RV{Y}_i|\RV{H}\RV{D}_i} &= \kernel{L}\label{eq:identical_response_assumption}
\end{align}
and let $\RV{Y}_i\CI^e_{\prob{P}_C} (\RV{Y}_{<i},\RV{D}_{<i})|(\RV{D}_i,\RV{H})$, and for all $\alpha$ set $\prob{P}_\alpha^{\RV{W}|\RV{DY}}=\prob{P}_\alpha^{\prime \RV{W}'|\RV{D'Y'}}$ for all $\RV{W}':\Omega\to W$ and $\prob{P}_\alpha^{\RV{D}_i|\RV{Y}_{<i}\RV{D}_{<i}}=\prob{P}_\alpha^{\prime \RV{D}_i'|\RV{Y}_{<i}'\RV{D}_{<i}''}$.

It remains to be shown that $\prob{P}_\alpha^{\RV{DY}}=\prob{P}_\alpha^{\prime \RV{DY}}$.

By Equation \ref{eq:identical_response_assumption} and $\RV{Y}_i\CI^e_{\prob{P}_C} (\RV{Y}_{<i},\RV{D}_{<i})|(\RV{D}_i,\RV{H})$, it follows (for identical reasons as Equation \ref{eq:comb_representation_w_CI}) that
\begin{align}
    \prob{P}_C^{\RV{Y}\combbreak \RV{D}} &= \tikzfig{do_model_representation}\\
    &= \tikzfig{do_model_representation_mu}\\
    &= \prob{P}_C^{\prime \RV{Y}'\combbreak\RV{D}'}
\end{align}

And so for all $n\in \mathbb{N}$
\begin{align}
    \prob{P}_\alpha^{\RV{D}_{[n]} \RV{Y}_{[n]}} &=  \prob{P}_\alpha^{\RV{D}_1}\odot \text{insert}(\prob{P}_\alpha^{ \RV{D}_{[2,n]}\combbreak\RV{Y}_{[n-1]}}, \prob{P}_C^{\prime \RV{Y}_{[n]}\combbreak\RV{D}_{[n]}}) \\
    &= \prob{P}_\alpha^{\RV{D}_1}\odot \text{insert}(\prob{P}_\alpha^{\prime \RV{D}'_{[2,n]}\combbreak\RV{Y}'_{[n-1]}}, \prob{P}_C^{\prime \RV{Y}'_{[n]}\combbreak\RV{D}'_{[n]}}) \\
    &= \prob{P}_\alpha^{\prime\RV{D}'_{[n]} \RV{Y}'_{[n]}}
\end{align}
\end{proof}

In contrast to the data-independent case where causal contractibility of $\prob{P}_C^{\RV{Y}|\RV{X}}$ implies the equivalence of all subsequence conditionals $\prob{P}_C^{\RV{Y}_A|\RV{X}_A}$ for all equally sized $A\subset\mathbb{N}$, a causally contractible comb $\prob{P}_C^{\RV{Y}\combbreak \RV{D}}$ does not generally imply that subsequence combs $\prob{P}_C^{\RV{Y}_A\combbreak \RV{D}_A}$ and $\prob{P}_C^{\RV{Y}_B\combbreak \RV{D}_B}$ are equivalent with $|A|=|B|$.


\subsection{Combs are the output of the ``fix'' operation}

There is a relationship between combs and the ``fix'' operation defined in \citet{richardson_nested_2017}. In particular, suppose we have a probability $\prob{P}_\alpha$ and a comb $\prob{P}_\alpha^{\RV{Y}_{[2]}|\RV{D}_{[2]}}$. Then (assuming discrete sets)
\begin{align}
    \prob{P}_\alpha^{\RV{Y}_{[2]}\combbreak \RV{D}_{[2]}}(y_1,y_2|d_1,d_2) &= \prob{P}_\alpha^{\RV{Y}_1|\RV{D}_1}(y_1|d_1)\prob{P}_\alpha^{\RV{Y}_2|\RV{D}_2}(y_2|d_2)\\
    &= \frac{\prob{P}_\alpha^{\RV{Y}_1|\RV{D}_1}(y_1|d_1)\prob{P}_\alpha^{\RV{D}_2|\RV{Y}_1\RV{D}_1}(d_2|y_1,d_1)\prob{P}_\alpha^{\RV{Y}_2|\RV{D}_2}(y_2|d_2)}{\prob{P}_\alpha^{\RV{D}_2|\RV{Y}_1\RV{D}_1}(d_2|y_1,d_1)}\\
    &= \frac{\prob{P}_\alpha^{\RV{Y}_{[2]}\RV{D}_2|\RV{D}_1}(y_1,y_2,d_2|d_1)}{\prob{P}_\alpha^{\RV{D}_2|\RV{Y}_1\RV{D}_1}(d_2|y_1,d_1)}
\end{align}
That is (at least in this case), the result of ``division by a conditional probability'' used in the fix operation is a comb. We speculate that the output of the fix operation is, in general, an $n$-comb, but we have not proven this.


\section{Weaker assumptions than causal contractibility}\label{sec:weaker_assumptions}

The results so far apply to purely observational models or to models where every ``input'' in the sequence is fixed at the point of choosing $\alpha$ (or a fixed random function is chosen at this point). Most of the interest in causal inference is how to use observational data -- which is plentiful -- to deduce consequences of choices. Suppose in the following that superscripter ``$o$'' refers to observational variables (obtained by some measurement procedure not responsive to choices) and ``$v$'' refers to interventional variables (obtained by some measurement procedure responsive to choices). That is $\RV{Y}^o:=(\RV{Y}_i^o)_{i\in \mathbb{N}}$ is a sequence of observantional variables, $\RV{Y}^v$ a sequence of interventional variables and $\RV{Y}^{o,v}:=(\RV{Y}^o_i,\RV{Y}^v_i)_{i\in\mathbb{N}}$ is a mixed sequence of both observational and interventional variables. $\RV{Y}_i^o$ and $\RV{Y}_i^v$ are assumed to take values in the same set $Y$.

One approach to bridging the gap between observations and interventions is to assume ``causal sufficiency'', which is tantamount (in the data-independent case) to assuming causal contractibility of $\prob{P}_C^{\RV{Y}^{o,v}|\RV{X}^{o,v}\RV{D}^{o,v}}$ with $\RV{D}^v$ responsive to choices and $\RV{X}^v$ unresponsive (see Example \ref{ex:backdoor}). As discussed, this is rarely a reasonable assumption -- it implies interchangeability between observational and interventional samples.

A weaker assumption that is often adopted is to consider models satisfying causal contractibility with respect to $\prob{P}_C^{\RV{Y}^{o,v}|\RV{U}^{o,v}\RV{D}^{o,v}}$, where $\RV{U}^{o,v}$ is unobserved. That is, while $\RV{U}^{o,v}$ appears in the model, it is not associted with any measurement procedure. This model still asserts that $(\RV{U}^o_i,\RV{X}^o_i,\RV{Y}^o_i)$ triples are interchangeable with $(\RV{U}^v_i,\RV{X}^v_i,\RV{Y}^v_i)$ triples, but neither of these are measurement outcomes. On the other hand, $(\RV{D}_i^o,\RV{Y}_i^o)$ pairs are not generally interchangeable with $(\RV{D}_i^v,\RV{Y}_i^v)$.

Consider models that satisfy causal contractibility with respect to $\prob{P}_C^{\RV{Y}^{o,v}|\RV{W}^{o,v}}$, where no comment is made about whether $\RV{W}^{o,v}$ is observed, unobseved or some function of observed and unobserved variables. This is a generalisation of the class of models discussed in the previous paragraph.  In isolation, this assumption is not especially interesting -- for example, the support of $\RV{W}^{o}_i$ and $\RV{W}^v_i$ might be disjoint. Suppose also, then, that $W$ is finite and $\RV{W}^o_i$ has full support. This assumption amounts to the assumption that, no matter what choice is made, ``nothing truly new can be done'' (which we call ``Ecclesiastes' assumption''\footnote{Ecclesiastes 1:9 reads ``Everything that happens has happened before; nothing is new, nothing under the sun.''\citep{noauthor_holy_1995}}). More precisely, for any choice $\alpha\in C$ and any consequence $\RV{Y}_i^v$, there is a random subsequence $\RV{Q}$ of indices $(1,2,3,....)$ such that the distribution $\prob{P}_\alpha^{\RV{Y}^{o,v}}$ is unchanged by permutations that only swap elements in the sequence $(RV{Y}^o_\RV{Q},\RV{Y}^v_i)$.

\begin{theorem}\label{th:condit_exchange}
Given just-do model $\prob{P}_C$ with $\prob{P}_C^{\RV{Y}^{o,v}|\RV{W}^{o,v}}$ causally contractible, $W$ finite and $\prob{P}_C^{\RV{W}^o|\RV{H}}(w|h)>0$ for all $w,h$, define $q:W^{\mathbb{N}}\times W\to (\{*\}\cup \mathscr{P}(\mathbb{N})$ by 
\begin{align}
    q:((w^o_j)_{\mathbb{N}},w^v_i)&\mapsto \{j|w^o_j=w^v_i\}
\end{align}
and take $\RV{Q}:=q\circ(\RV{W}^o,\RV{W}^v_i)$ for arbitrary $i\in \mathbb{N}$. For an index set $U\in\mathbb{N}$ Take $\text{swap}_{\cdot}:Y^{\mathbb{N}}\times Y^{\mathbb{N}}\to Y^{\mathbb{N}}\times Y^{\mathbb{N}}$ to be an arbitrary finite swap that acts as the identity on all indices $(j,x)\not\in \RV{Q}\times \{o\}\cup\{(i,v)\}$. Then $\prob{P}^{\RV{Y}^{o}\RV{Y}^v_i}\text{swap}_{\RV{Q}} = \prob{P}^{\RV{Y}^{o}\RV{Y}^v_i}$.
\end{theorem}

\begin{proof}
Note that for $B_j\in \sigalg{W}$, where $\rho_q:\mathbb{N}\times\{i,v\}\to \mathbb{N}\times\{i,v\}$ is the permutation function associated with $\text{swap}_{q}$
\begin{align}
    \prob{P}_\alpha^{\RV{W}^o\RV{W}^v_i}\text{swap}_{\RV{Q}} (\bigtimes_{j\in\mathbb{N}} B_j) &= \int_{W^{\mathbb{N}}}\int_{\mathscr{P}(\mathbb{N})} \prod_{k\not\in q\times\{o\}\cup\{(i,v)\}} \delta_{w_k}(B_k) \prod_{l\in q\times\{o\}\cup\{(i,v)\}} \delta_{\rho_q(w_l)} (B_l) \prob{P}_\alpha^{\RV{Q}|\RV{W}^o\RV{W}^v_i}(\mathrm{d}q|w)\prob{P}_\alpha^{}(\mathrm{d}w)\\
    &= \int_{W^{\mathbb{N}}}\int_{\mathscr{P}(\mathbb{N})} \prod_{k\not\in q\times\{o\}\cup\{(i,v)\}} \delta_{w_k}(B_k) \prod_{l\in q\times\{o\}\cup\{(i,v)\}} \delta_{w_l} (B_l) \prob{P}_\alpha^{\RV{Q}|\RV{W}^o\RV{W}^v_i}(\mathrm{d}q|w)\prob{P}_\alpha^{}(\mathrm{d}w)\label{eq:all_the_same}\\
    &= \prob{P}_\alpha^{\RV{W}^o\RV{W}^v_i}
\end{align}
where Eq. \ref{eq:all_the_same} follows from the fact that for every $k,l\in q\times\{o\}\cup\{(i,v)\}$, $w_k=w_l$.

Thus for $A\in \sigalg{Y}^{\mathbb{N}}$
\begin{align}
    \prob{P}_\alpha^{\RV{Y}^{o}\RV{Y}^v_i}\text{swap}_{\RV{Q}}(A) &= [\prob{P}_\alpha^{\RV{W}^o\RV{W}^v_i} \prob{P}_\alpha^{\RV{Y}^{o}\RV{Y}^v_i|\RV{Q}\RV{W}^o\RV{W}^v_i}\text{swap}_{\RV{Q}}](A)\\
    &= [\prob{P}_\alpha^{\RV{W}^o\RV{W}^v_i} \text{swap}_{\RV{Q}^{-1}} \prob{P}_\alpha^{\RV{Y}^{o}\RV{Y}^v_i|\RV{W}^o\RV{W}^v_i}\text{swap}_{\RV{Q}}](A)\\
    &= \prob{P}_\alpha^{\RV{Y}^{o}\RV{Y}^v_i}\label{eq:by_cc1}
\end{align}
Where Eq. \ref{eq:by_cc1} follows from causal contractibility of $\prob{P}_\alpha^{\RV{Y}^{o}\RV{Y}^v_i|\RV{W}^o\RV{W}^v_i}$.
\end{proof}

It also follows from Ecclesiastes' assumption and finite $W$ that if some $\RV{X}_i^o$, $\RV{Z}_i^o$ are \emph{deterministically} related given $\RV{W}$, then $\prob{P}_C^{\RV{Z}|\RV{X}}$ is causally contractible.

\begin{theorem}\label{th:det_obs_to_cons}
Given just-do model $\prob{P}_C$ with $\prob{P}_C^{\RV{X}^{o,v}\RV{Z}^{o,v}|\RV{W}^{o,v}}$ causally contractible, $W$ finite and $\prob{P}_C^{\RV{W}^o_i|\RV{H}}(w|h)>0$ for all $w,h$, if $\prob{P}_C^{\RV{Z}^o_0|\RV{X}^o_0\RV{H}}$ is deterministic then $\prob{P}_C^{\RV{Z}^{o,v}|\RV{X}^{o,v}}$ is causally contractible.
\end{theorem}

\begin{proof}
Because $\prob{P}_\alpha^{\RV{W}_0^o}\prob{P}_C^{\RV{Z}^o_0|\RV{X}^o_0\RV{W}^o_0\RV{H}}$ is deterministic, so is $\prob{P}_C^{\RV{Z}_0|\RV{X}_0\RV{W}_0\RV{H}}$.

Fix $h\in H$.  Suppose there is some $w,w'\in W$ such that
\begin{align}
    \prob{P}_C^{\RV{Z}_0|\RV{X}_0\RV{W}_0\RV{H}}(A|x,w,h) &\neq \prob{P}_C^{\RV{Z}_0|\RV{X}_0\RV{W}_0\RV{H}}(A|x,w',h)
\end{align}
then, by determinism, we can assume without loss of generality
\begin{align}
    \prob{P}_C^{\RV{Z}_0|\RV{X}_0\RV{W}_0\RV{H}}(A|x,w,h) = 1\\
    \prob{P}_C^{\RV{Z}_0|\RV{X}_0\RV{W}_0\RV{H}}(A|x,w',h) = 0
\end{align}
but $W$ is finite and $\prob{P}_C^{\RV{W}^o_i|\RV{H}}(w|h)>0$ for all $w$, so there is some $a>0$ such that $\prob{P}_C^{\RV{W}^o_i|\RV{H}}(w|h)\geq a$ for all $w$, and so
\begin{align}
    a \leq \sum_{w\in W} \prob{P}_C^{\RV{W}^o_0|\RV{X}^o_0,\RV{H}}(w|x,h)\prob{P}_C^{\RV{Z}_0|\RV{X}_0\RV{W}_0\RV{H}}(A|x,w,h)\leq 1-a
\end{align}
contradicting determinism of $\prob{P}_C^{\RV{Z}^o_0|\RV{X}^o_0\RV{H}}$.

Thus for all $w,w'$
\begin{align}
    \prob{P}_C^{\RV{Z}_0|\RV{X}_0\RV{W}_0\RV{H}}(A|x,w,h) &= \prob{P}_C^{\RV{Z}_0|\RV{X}_0\RV{W}_0\RV{H}}(A|x,w',h)
\end{align}
i.e. $\RV{Z}_0\CI^e_{\prob{P}_C} (\RV{W}_0,\RV{C})|(\RV{X}_0,\RV{H})$. But then there is some $\prob{P}_C^{\RV{Z}_0|\RV{X}_0\RV{H}}$ such that
\begin{align}
    \prob{P}_C^{\RV{Z}_0|\RV{X}_0\RV{W}_0\RV{H}} &= \prob{P}_C^{\RV{Z}_0|\RV{X}_0\RV{H}}\otimes \text{erase}_W\\
    \implies \prob{P}_\alpha^{\RV{Z}^v_i|\RV{X}^v_i\RV{H}} &= \prob{P}_C^{\RV{Z}_0|\RV{X}_0\RV{H}}
\end{align}
\end{proof}

Theorem \ref{th:det_obs_to_cons} doesn't hold in the case of approximate determinism, however. Intuitively, approximate determinism can hold if there is some value of $\RV{W}$ for which $\RV{Z}$ is not conditionally independent given $\RV{H}$ and $\RV{X}$, but it only ocurrs very rarely in observations. On the other hand, values of $\RV{W}$ rare in observations might, under some choices, become common. 

\begin{example}
Say $\prob{P}_C^{\RV{Z}^o_i|\RV{X}^o_i\RV{H}}$ is \emph{approximately deterministic} if $\prob{P}_C^{\RV{Z}^o_i|\RV{X}^o_i\RV{H}}(A|x,h)\in [0,\epsilon]\cup[1-\epsilon,1]$ for all $A\in\sigalg{Z}$, $x,h\in X\times H$.

Take $Z=X=W=H=\{0,1\}$. Set
\begin{align}
    \prob{P}_C^{\RV{Z}_0|\RV{X}_0\RV{W}_0\RV{H}}(1|1,1,1) = 1\\
    \prob{P}_C^{\RV{Z}_0|\RV{X}_0\RV{W}_0\RV{H}}(1|1,0,1) = 0
\end{align}
and
\begin{align}
    \prob{P}_C^{\RV{W}^o_0|\RV{H}}(1|1)=1-\epsilon
\end{align}
then
\begin{align}
    \prob{P}_C^{\RV{Z}_0|\RV{X}_0\RV{H}}(1|1,1) = 1-\epsilon
\end{align}
however, suppose there is some $\alpha$ such that
\begin{align}
    \prob{P}_\alpha^{\RV{W}^v_i|\RV{H}}(1|1)=0
\end{align}
then
\begin{align}
    \prob{P}_\alpha^{\RV{Z}_0|\RV{X}_0\RV{H}}(1|1,1) = 0\\
    &\neq \prob{P}_C^{\RV{Z}_0|\RV{X}_0\RV{H}}(1|1,1)
\end{align}
\end{example}
%!TEX root = main.tex










\begin{lemma}[Infinitely extended kernels]\label{lem:infinitely_extended_kernels}
Given a collection of Markov kernels $\kernel{K}_i:X^i\kto Y^i$ for all $i\in \mathbb{N}$, if we have for every $j>i$
\begin{align}
	\kernel{K}_j(\text{id}_{X_i}\otimes \text{del}_{X_{j-i}}) &= \kernel{K}_i\otimes \text{del}_{X_{j-i}}\label{eq:marginalise_comb}
\end{align} 
then there is a unique Markov kernel $\kernel{K}:X^{\mathbb{N}}\kto Y^{\mathbb{N}}$ such that for all $i,j\in \mathbb{N}$,$j>i$
\begin{align}
	\kernel{K}(\text{id}_{X_i}\otimes \text{del}_{X_{j-i}})&= \kernel{K}_i\otimes \text{del}_{X_{j-i}}
\end{align}
\end{lemma}

\begin{proof}
Take any $x\in X^{\mathbb{N}}$ and let $x_{|m}\in X^n$ be the first $n$ elements of $x$. By Equation \ref{eq:marginalise_comb}, for any $A_i\in \sigalg{Y}$, $i\in [m]$

\begin{align}
    \kernel{K}_n(\bigtimes_{i\in [m]}A_i\times Y^{n-m}|x_{|n}) &= \kernel{K}_m(\bigtimes_{i\in [m]}A_i|x_{|m})
\end{align}

Furthermore, by the definition of the $\mathrm{swap}$ map for any permutation $\rho:[n]\to[n]$
\begin{align}
    \kernel{K}_n\mathrm{swap}_{\rho}(\bigtimes_{i\in [m]}A_{\rho(i)}\times Y^{n-m}|x_{|n}) &= \kernel{K}_n(\bigtimes_{i\in [m]}A_{i}\times Y^{n-m}|x_{|n})
\end{align}

Thus by the Kolmogorov Extension Theorem \citep{cinlar_probability_2011}, for each $x\in X^{\mathbb{N}}$ there is a unique probability measure $\prob{Q}_x\in \Delta(Y^{\mathbb{N}}$ satisfying
\begin{align}
    \prob{Q}_d(\bigtimes_{i\in [n]}A_i\times Y^{\mathbb{N}}) &= \kernel{K}_n(\bigtimes_{i\in [n]}A_{\rho(i)}|d_{|n})\label{eq:q_is_Markov}
\end{align}

Furthermore, for each $\{A_i\in\sigalg{Y}|i\in \mathbb{N}\}$, $n\in \mathbb{N}$ note that for $p>n$

\begin{align}
\prob{Q}_d(\bigtimes_{i\in[n]} A_i \times Y^{\mathbb{N}})&\geq \prob{Q}_d(\bigtimes_{i\in [p]} A_i\times Y^{\mathbb{N}})\\
&\geq \prob{Q}_d(\bigtimes_{i\in \mathbb{N}} A_i)
\end{align}

so by the Monotone convergence theorem, the sequence $\prob{Q}_d(\bigtimes_{i\in[n]} A_i)$ converges as $n\to \infty$ to $\prob{Q}_d(\bigtimes_{i\in\mathbb{N}} A_i)$. $d\mapsto \prob{Q}_d^{\RV{Z}_n}(\bigtimes_{i\in[n]} A_i)$ is measurable for all $n$, $\{A_i\in\sigalg{Y}|i\in \mathbb{N}\}$ by Equation \ref{eq:q_is_Markov}, and so $d\mapsto Q_d$ is also measurable.

Thus $d\mapsto Q_d$ is the desired $\prob{P}_C^{\RV{Y}_{\mathbb{N}}\combbreak \RV{D}_{\mathbb{N}}}:D^\mathbb{N}\kto Y^\mathbb{N}$.
\end{proof}


\begin{theorem}[Conditional independence in augmented causally contractible model]
Suppose we have a probability set $\prob{P}_C$ on $\Omega$ with causally contractible $\mathbb{N}$-comb $\prob{P}_C^{\RV{Y}\combbreak \RV{D}}$, $\RV{Y}=(\RV{Y}_i)_{i\in\mathbb{N}}$, $\RV{D}=(\RV{D}_i)_{i\in\mathbb{N}}$, augmented with the hypothesis variable $\RV{H}$ as in Lemma \ref{lem:aug_cc}. Then for all $n$, $\RV{Y}_n\CI^e_{\prob{P}_C}(\RV{Y}_{<n},\RV{D}_{<n})|(\RV{H},\RV{D}_n)$.
\end{theorem}

\begin{proof}
For all $\alpha\in C$, $n\in \mathbb{N}$
\begin{align}
	\prob{P}_\alpha^{\RV{Y}_{[n]}\RV{D}_{[n]}|\RV{H}} &= \tikzfig{cc_comb_factorisation}
\end{align}
Thus for all $\alpha$, by Theorem \ref{th:higher_order_conditionals}
\begin{align}
	\prob{P}_C^{\RV{Y}_n|\RV{D}_n\RV{H}}\otimes \mathrm{del}_{D^{n-1}\times Y^{n-1}} \overset{\prob{P}_\alpha}{\cong} \prob{P}_C^{\RV{Y}_n|\RV{D}_{[n]}\RV{Y}_{<n}\RV{H}}
\end{align}
which implies $\RV{Y}_n\CI^e_{\prob{P}_C}(\RV{Y}_{<n},\RV{D}_{<n})|(\RV{H},\RV{D}_n)$.
\end{proof}


\begin{theorem}[Conditional independence in non-data-dependant causally contractible model]
Suppose we have a probability set $\prob{P}_C$ on $\Omega$ with causally contractible $\mathbb{N}$-comb $\prob{P}_C^{\RV{Y}\combbreak \RV{D}}$, $\RV{Y}=(\RV{Y}_i)_{i\in\mathbb{N}}$, $\RV{D}=(\RV{D}_i)_{i\in\mathbb{N}}$, augmented with the hypothesis variable $\RV{H}$ as in Lemma \ref{lem:aug_cc}. Then for all $n$, $\RV{Y}_n\CI^e_{\prob{P}_C}(\RV{Y}_{<n},\RV{D}_{<n})|(\RV{H},\RV{D}_n)$.
\end{theorem}

\begin{lemma}\label{lem:aug_cc}
Suppose we have a probability set $\prob{P}_C'$ on $(\Omega',\sigalg{F}')$ with causally contractible $\mathbb{N}$-comb $\prob{P}_C^{\prime \RV{Y}'\combbreak \RV{D}''}$, $\RV{Y}'=(\RV{Y}'_i)_{i\in\mathbb{N}}$, $\RV{D}'=(\RV{D}'_i)_{i\in\mathbb{N}}$. Then there exists $H$ and an \emph{augmented} model $\prob{P}_C$ on $(\Omega,\sigalg{F}):=((\Omega'\times H,\sigalg{F}'\otimes\sigalg{H})$ such that $\prob{P}_C\Pi_{\Omega'}=\prob{P}_C'$ and, defining $\RV{H}:\Omega'\times H\to H$ as the projection onto $H$,
\begin{align}
    \prob{P}_C^{\RV{Y}\combbreak \RV{D}} &= \tikzfig{do_model_representation}\label{eq:do_model_rep}
\end{align}
\end{lemma}

\begin{proof}
Let $H$ be the hypothesis space from Theorem \ref{th:ciid_rep_kernel} and $\prob{P}_C^{\RV{H}}$ be some directing random measure for $\prob{P}_C^{\prime \RV{Y}\combbreak \RV{D}}$. Then by Theorem \ref{eq:do_model_rep}, Equation \ref{eq:do_model_rep} holds. Furthermore, by equality of combs, we have for all $\alpha\in C$
\begin{align}
	\prob{P}_\alpha^{\RV{YD}} &= \prob{P}_\alpha^{\prime \RV{Y'D'}}
\end{align}

Define $\RV{W}':\Omega'\to \Omega'$ as the identity function on $\Omega'$, $\RV{W}:\Omega'\times H\to \Omega'$ as the projection to $\Omega'$. For each $\alpha\in C$, define $\prob{P}_\alpha$ by

\begin{align}
	\prob{P}_\alpha^{\RV{W}} &= \prob{P}_\alpha^{\RV{YD}}\odot \prob{P}_\alpha^{\prime \RV{W}'|\RV{Y'D'}} (\mathrm{del}_{Y^{\mathbb{N}D^{\mathbb{N}}H}}\otimes \mathrm{id}_{\Omega'})
\end{align}

Then

\begin{align}
	\prob{P}_\alpha^{\RV{W}} &= \prob{P}_\alpha\Pi_{\Omega'}\\
	&= \prob{P}_\alpha^{\prime \RV{Y'D'}}\odot \prob{P}_\alpha^{\prime \RV{W}'|\RV{Y'D'}}(\mathrm{del}_{Y^{\mathbb{N}D^{\mathbb{N}}H}}\otimes \mathrm{id}_{\Omega'})\\
	&= \prob{P}_\alpha^{\prime \RV{W}'}
\end{align}
\end{proof}
%!TEX root = main.tex

\chapter{Statistical Decision Theory}\label{ch:sdt}

\section{Summary}

Statistical models are ubiquitous in the analysis of inference problems. A statistical model features a set of \emph{states}, and each state is mapped to a probability distribution over \emph{outcomes}. If we want to model problems involving \emph{decisions} and \emph{consequences}, we need to consider different kinds of statistical models. We introduce two types of model for this purpose: \emph{two player statistical models} which differ from classical statistical model in that the state is assumed to consist of a decision and a \emph{hypothesis} (the two players are the decision maker ``player D'' and the hypothesis selector ``player H''). They model the consequences of decisions under various hypotheses. A \emph{see-do model} is a special case of two player statistical model that can be used in situtations where some \emph{observations} are available for review prior to selecting a decision. See-do models are the main focus of work here and problems involving observations, decisions and consequences will be discussed at length in Chapter \ref{ch:evaluating_decisions}.

A common simplifying assumption made when using classical statistical models is that they are \emph{conditionally independent and identically distributed} (conditinally IID); this means that the model maps each state to an independent and identically distributed (IID) sequence of observations. This just a common choice, it is not a strict requirement. A similar assumption is likely to be useful for see-do models. If consequences depend on choices, then it does not make sense to assert that observations and consequences together form a single IID sequence of random variables, so we need to consider alternatives. We propose that models where observervations are an IID sequence and choices and consequences together are \emph{independent and functionally identical} (IFI; defined later in this chapter) are similar to conditionally IID statistical models. 

Instead of directly assuming that a conditionally IID model is appropriate, \emph{De Finetti's representation theorem} shows that probability models where the sequence of observations is \emph{exchangeable} induce conditionally IID statistical models. The assumption of exchangeabile observations is preferable to the assumption of IID observations if a probability model is being used to represent subjective uncertainty. We investigate whether there is an analogous result relating ``exchangeability-like'' assumptions for see-do models to ``IID-like'' assumptions. We show that there is: in particular, a see-do ``forecast'' with exchangeable observations and \emph{functionally exchangeable} decision to consequence maps induces a see-do model with IID observations and IFI consequences.

The assumption of functional exchangeability will appear again in Chapter \ref{ch:ints_counterfactuals} as part of the definition of \emph{counterfactual models}, and the joint assumptions of exchangeable observations and functionally exchangeable consequences to motivate the assumption of \emph{imitability} in Chapter \ref{ch:inferring_causes}, an assumption that in combination with a number of other assumptions allows for inference of consequences from data.

\section{Modelling observations, choices and consequences}

\subsection{Modelling observations with statistical models}

Statistical models are a ubiquitous type of model in statistics and machine learning. They consist of a set of \emph{states} $(S,\sigalg{S})$, and for each state the model prescribes a single probability distribution on a given measurable set of \emph{outcomes} $(O,\sigalg{O})$.

\begin{definition}[Statistical model]\label{def:statistical model}
A statistical model is a set of states $(S,\sigalg{S})$, a set of outcomes $(O,\sigalg{O})$ and a stochastic map $\kernel{T}:S\to \Delta(\sigalg{O})$.
\end{definition}

\begin{definition}[State and outcome variables]\label{def:state_outcome}
Given a statistical model $(\kernel{T},(O,\sigalg{O}),(S,\sigalg{S}))$, define the \emph{state variable} $\RV{S}:S\times O\to S$ as the projection from $S\times O\to S$ and define the \emph{outcome variable} $\RV{O}:S\times O\to O$ as the projection onto $O$.
\end{definition}

The common example of a potentially biased coin is modelled with a statistical model. We suppose our coin has some rate of heads $\theta\in [0,1]$, and we furthermore suppose that for each $\theta$ the result of flipping the coin can be modeled (in some sense) by the probability distribution $\text{Bernoulli}(\theta)$. The statistical model here is the set of states $S=[0,1]$ (corresponding to \emph{rates of heads}), the observation space $O=\{0,1\}^n$ with the discrete sigma-algebra (where $n$ is the number of flips observed) and the stochastic map $\kernel{B}:[0,1]\to \Delta(\mathscr{P}(0,1))$ which is given by $\kernel{B}:\theta\to \text{Bernoulli}(\theta)$.

Almost any theoretical treatment of statistics or machine learning will at some point make use of statistical models to describe the problem they are addressing -- for a collection of examples from the last 100 years, see \cite{Goodfellow-et-al-2016,vapnik_nature_2013,bishop_pattern_2006,le_cam_comparison_1996,freedman_asymptotic_1963,wald_statistical_1950,de_finetti_foresight_1992,fisher_statistical_1992}. They are often simply assumed without a great deal of discussion of why this type of model is chosen, or what role they play.

If we want to reason about how well some learning algorithm performs in some context, we typically require a reasonable model of the context in which the learning algorithm operates. The algorithms themselves may not give us such a model. Because learning almost always operates in a context with noise an uncertainty, we need models that can handle noise and uncertainty. Probability models are a very common choice for this. In addition, it is often assumed that we do not know with certainty the exact probability model that should be used to model a context. A statistical model assumes a certain number of states may prevail -- reflecting uncertainty in the ``mechanics of the world'' -- and given any state it gives us a probability distribution -- reflecting uncertainty remaining after the mechanics of the world are well-understood.

Learning algorithms don't necessarily implement reasonable models of the world. For example, consider a linear regressor that takes a set of predictors $x\in X$ and targets $y\in Y$ and returns some $\beta\in B$ such that $(y-x^T\beta)^2$ is as small as possible. It is possible to interpret $B$ as a set of states, and consider the learner to be implementing the statistical model $(\kernel{T},B,\mathcal{L}_{X\to Y})$ where $\mathcal{L}_{X\to Y}$ is the set of liner function $X\to Y$ given by $\{x\mapsto x^T\beta|\beta\in B\}$ and $\kernel{T}$ maps a state $\beta\in B$ deterministically to the function $x\mapsto x^T\beta$. This is, formally, a statistical model, but it is not one that would typically be considered a good model of the world in the kinds of problems that a linear regressor is used to solve. One problem with this model is that it is deterministic - the outcome for any $\beta\in B$ will be a particular function $X\to Y$. However, it will almost never be the case that some set of targets $y$ will be an exact function of some set of predictors $x$, and insisting on an exact functional relationship will typically give very poor generalisation results if this demand can be satisfied at all.

Suppose we want to ask whether the function $f$ given by a linear regressor is useful for some purpose. In order to address this question, we want to consider a more appropriate model of the world than the crude statistical model given above, and consider what behaviour we will see from the regressor under different assumptions imposed on this model. Statistical models typically are used to serve the purpose of a ``more appropriate model of the world''. In this example, we might consider a statistical model $(\kernel{T},H,O)$ where for each $h\in H$, $\kernel{T}_h\in \Delta(\sigalg{Y}\otimes\sigalg{X})$ such that $\kernel{T}_h^{\RV{Y}|\RV{X}}=\text{Normal}(\RV{X}^T\beta_h,\sigma_h)$. If we assume the data generating process is described by such a probability distribution for some $h$, we can ask questions like ``does the linear regressor output a $\beta$ such that $\RV{Y}-\RV{X}^T\beta<\epsilon$ with high probability for all $h\in H$?''

\subsection{Modelling choices and consequences with two-player statistical models}

The states in a statistical model are usually considered to be ``under the control of nature''. In the possibly biased coin example above, if we were to consider some ``player D'' acutally flipping the coin and trying to infer the bias, we would typically assume that their opinion about the coin's bias does not affect the coin's actual bias; they could decide it is biased towards heads when in fact it is completely unbiased. In some cases player D can make choices that affect the outcomes. Suppose player D has the option to choose how high to toss the coin -- perhaps they can aim for a toss height anywhere from 10 to 50cm. This plausibly affects the outcomes of their coin toss and, unlike the coin's bias, they gets to choose the height they intend to toss it. If they decide to toss it to a height of 15cm then 15cm is the height they have chosen to toss it to. Unlike the state, which can differ from whatever player D ultimately decides on, the choice made by player D is the same thing as whatever they ultimately decide on. We call features of the state that are not under player D's control \emph{hypotheses} and features that are under player D's control \emph{decisions}, and statistical models in which the state is the Cartesian product of a set of hypotheses and a set of decisions ``two player statistical models'' (the two players being nature or ``player H'' and the decision maker or ``player D'').

\begin{definition}[Two player statistical model]\label{def:2p_stat}
A \emph{two-player statistical model} is a tuple $(\kernel{T},\RV{H},\RV{D},\RV{O})$ where $(\kernel{T},(H\times D,\sigalg{H}\otimes\sigalg{D}), (O,\sigalg{O}))$ is a statistical model and $\RV{H}:H\times D\times O\to H$, $\RV{D}:H\times D\times O\to D$ and $\RV{O}:H\times D\times O\to O$ are measurable functions that project elements of $H\times D\times O$ to their respective codomains. $\RV{H}$ is called the \emph{hypothesis}, $\RV{D}$ the \emph{decision} and $\RV{O}$ the \emph{outcome}.
\end{definition}

Whenever we propose a two player statistical model, we will also assume for any random variables $\RV{X}: H\times D\times O\to X$ and $\RV{Y}:H\times D\times O\to Y$, a disintegration $\kernel{K}^{\RV{Y}|\RV{XDH}}:X\times D\times H\to \Delta(\sigalg{Y})$ exists (see Theorem \ref{th:existence_continous} for a sufficient condition).

The problems that we will mostly study in this work, in addition to having a second player (``player D''), will often involve some data $\RV{X}$ that is observed before the second player is able to make a choice. Two player statistical models with \emph{observations} are called \emph{see-do models}.

\begin{definition}[See-Do model]\label{def:seedo}
A \emph{see-do model} $(\kernel{T},\RV{H},\RV{D},\RV{X},\RV{Y})$ is a two-player statistical model along with two additional random variables: the \emph{observation} $\RV{X}: H\times D\times O\to X$ and the \emph{consequence} $\RV{Y}:H\times D\times O\to Y$. The outcome variable is defined to be the coupled product of the observation and the consequence $\RV{O}=(\RV{X},\RV{Y})$, and we will leave this implicit when specifying a see-do model. A see-do model must observe the conditional independence:
\begin{align}
\RV{X}\CI_\kernel{T} \RV{D}|\RV{H} \label{eq:see_do_independence_requirement}
\end{align}
\end{definition}

We can informally read the independence requirement as saying ``the observations are independent of the decision given the hypothesis''. This does not imply that probability models we construct from $\kernel{T}$ will necessarily have the property that $\RV{D}$ and $\RV{X}$ will be independent conditional on $\RV{H}$, and in fact this will often not be the case. In Chapter \ref{ch:ints_counterfactuals} we will argue that this requirement captures the intuition that observations are not ``affected'' by decisions. For now, we will observe that this independence requirement means that $\kernel{T}$ can be drawn with no path from $\RV{D}$ to $\RV{X}$.

Explicitly, the independence on line \ref{eq:see_do_independence_requirement} implies that the kernel $\kernel{T}$ can be drawn as follows:

\begin{align}
\kernel{T}:= \begin{tikzpicture}
                 \path (0,0) node (H) {$\RV{H}$}
                 + (0,-1) node (D) {$\RV{D}$}
                 ++ (0.5,0) node[copymap] (copy0) {}
                 ++ (0.9,0) node[kernel] (XH) {$\kernel{T}^{\RV{X}|\RV{H}}$}
                 + (0,-0.85)  node[kernel] (YHD) {$\kernel{T}^{\RV{Y}|\RV{HD}}$}
                 ++ (1,0) node (X) {$\RV{X}$}
                 + (0,-0.85) node (Y) {$\RV{Y}$};
                 \draw (H) -- (XH) -- (X);
                 \draw (copy0) to [out=-45,in=180] ($(YHD.west) + (0,0.15)$);
                 \draw (D) to [out=0,in=180] ($(YHD.west) + (0,-0.15)$) (YHD) -- (Y);
             \end{tikzpicture}
\end{align}

In this picture, again informally, $\RV{Y}$ takes input from $\RV{D}$ but $\RV{X}$ does not.    
%!TEX root = main.tex

\chapter{See-do models, interventions and counterfactuals}\label{ch:4}


\section{How do see-do models relate to other approaches to causal inference?}

\begin{itemize}
    \item Review of approaches: CBN, CBN soft intervention, CBN fat-hand intervention, CBN noise intervention, SEM (Pearl/Heckman), PO unit model, PO population model, SWIG, Dawid decision theoretic model, Heckerman decision theoretic model, Rohde/Lattimore Bayesian model
    \item Focus on CBN, PO unit model, PO population model
\end{itemize}


\section{Interpretations of the choice set}

\begin{itemize}
    \item Decisions or actions we could actually make - decision problem
    \item Idealised/hypothetical choices constrained by a set of causal relationships - interventions
    \item Suppositions - counterfactuals
    \item Further possibility - intervention $\to$ decisions might be actuator randomisation
\end{itemize}

\section{Causal Bayesian Networks as see-do models}

\begin{itemize}
    \item Definition of CBN, intervention set (recall: existence of disintegrations, decomposability)
    \item How interventions differ from decisions: no effect strength uncertainty, side effects, may be more interventions than what we actually know how to do
    \item Example: sets of CBNs and d-separation
\end{itemize}

\section{Unit Potential Outcomes models}

\begin{itemize}
    \item Counterfactual random variables Yx answer a question: "what would Y be supposing X was x?"
    \item Proposed formalisation of suppositions: (....)
    \item Implies existence of counterfactual random variables
    \item Difference between suppositions and decisions: determinism, other conditions
    \item "3-player models": hypotheses, suppositions and interventions/decisions
    \item Error in key theorem of Ruben, Imbens (ignorability does not imply functional exchangeability)
    \item What can be represented by a 3 player model?
    \begin{itemize}
        \item ``1 of 2 counterfactuals'': anything
        \item ``3 of 2 counterfactuals'': very restrictive
        \item ``2 of 3 counterfactuals'': Bell's theorem, counterfactual definiteness
    \end{itemize}
\end{itemize}



\todo[inline]{This chapter is currently a disorganised cut and paste}

The field of causal inference is additionally concerned with types of questions called ``counterfactual'' by Pearl. There is substantial theoretical interest in counterfactual questions, but counterfactual questions are much more rarely found in applications than interventional questions. Even though see-do models are motivated by the need to answer interventional questions, the theory develope here is surprisingly appliccable to counterfactuals as well. In particular, the theory of see-do models offers explanations for three key features of counterfactual models:
\begin{itemize}
    \item \textbf{Apparent absence of choices}: \emph{Potential outcomes} models, which purportedly answer counterfactual questions, are standard statistical models \emph{without choices} \citep{rubin_causal_2005}
    \item \textbf{Deterministic dependence on unobserved variables}: Counterfactual models involve \emph{deterministic} dependence on unobserved variables \citep{pearl_causality:_2009,rubin_causal_2005,richardson2013single}
    \item \textbf{Residual dependence on observations}: Counterfactual questions depend on the given data \emph{even if the joint distribution of this data is known}. For example, \citet{pearl_causality:_2009} introduces a particular method for conditioning a known joint distribution on observations that he calls \emph{abduction}
\end{itemize}

Potential outcomes models lack a notion of ``choices'' because there is a generic method to ``add choices'' to a potential outcomes model, which is implicitly used whenever potential outcomes models are used. Furthermore, we show that a see-do model induces a potential outcome model if and only if it is a model of \emph{parallel choices}, and in this case the observed consequences depend deterministically on the unobserved potential outcomes in precisely the manner as given in \citet{rubin_causal_2005}. Parallel choices can be roughly understood as models of sequences of experiments where an action can be chosen for each experiment, and with the special properties that repeating the same action deterministically yields the same consequence, and the consequences of a sequence of actions doesn't depend on the order in which the actions are taken. That is, we show that the fundamental property of any ``counterfactual'' model is \emph{deterministic reproducibility} and \emph{action exchangeability}, and while these models may admit a ``counterfactual'' interpretation, they are fundamentally just a special class of see-do models.

\todo[inline]{But the proof is still in my notebook}

\todo[inline]{Interestingly, it seems to be possible to construct a see-do model where the ``hypothesis'' is a quantum state, and quantum mechanics + locality seems to rule out parallel choices in such models in a manner similar to Bell's theorem. ``Seems to'' because I haven't actually proven any of these things.}

The residual dependence on observations exhibited by counterfactual questions is a generic property of see-do models, and it is a particular property of \emph{decision problems} are notable in that it is often

\todo[inline]{Where to discuss the connections to statistical decision theory?}

See-do models are closely related to \emph{statistical decision theory} introduced by \citet{wald_statistical_1950} and elaborated by \citet{savage_foundations_1972} after Wald's death. See-do models equipped with a \emph{utility function} induce a slightly generalised form of statistical decision problems, and the complete class theorm is appliccable to these models.

A stylistic difference between see-do models and most other causal models is that see-do models explicitly represent both the observation model and the consequence model and their coupling, making them ``two picture'' causal models. Causal Bayesian Networks and Single World Interention Graphs \citep{richardson2013single} use ``one picture'' to represent the observation model and the consequence model. However, both of these approaches employ ``graph mutilation'', so one picture on the page actually corresponds to many pictures when combined with the mutilation rules. For more on how these different types of models relate, see Section \ref{sec:single_double_representation}. \citet{lattimore_replacing_2019}'s Bayesian causal inference employs two-picture causal models, as do ``twin networks'' \citep{pearl_causality:_2009}.

Sometimes we are interested in modelling situations where we can also make some choices that also affect the eventual consequences. For example, I might hypothesise $\RV{H}_1$: the switch on the wall controls my light, $\RV{H}_2$: the switch on the wall does not control my light. Then, given $\RV{H}_1$ I can choose to toggle the switch, and I will see my light turn on, or I can choose not to toggle the switch and I will not see my light turn on. Given $\RV{H}_2$, neither choice will result in a light turned on. Choices are clearly different to hypotheses: the choice I make depends on what I want to happen, while whether or not a hypothesis is true has no regard for my ambitions.

A ``statistical model with choices'' is simply a map $\prob{T}:D\times \RV{H}\to \Delta(\sigalg{E})$ for some set of choices $D$, hypotheses $\RV{H}$ and outcome space $(E,\sigalg{E})$. We can also distinguish two types of outcomes: \emph{observations} which are given prior to a choice being made and \emph{consequences} which happen after a choice is made. Observations cannot be affected by the choices made, while consequences are not subject to this restriction. That is, observations are what we might \emph{see} before making a choice, which depends on the hypothesis alone, and if we are lucky we may be able to invert this dependence to learn something about the hypothesis from observations. On the other hand, the consequences of what we \emph{do} depends jointly on the hypothesis and the choice we make and we judge which choices are more desirable on the basis of which consequences we expect them to produce. 

What we are studying is a family of models that generalises of statistical models to include hypotheses, choices, observations and consequences. These models are referred to as \emph{see-do models}. Hypotheses, observations, consequences and choices are not individually new ideas. \emph{Statistical decision problems} \citep{wald_statistical_1950,savage_foundations_1972} extend statistical models with decisions and \emph{losses}. Like consequences, losses depend on which choices are made. However, unlike consequences, losses must be ordered and reflect the preferences of a decision maker. \emph{Influence diagrams} are directed graphs created to represent decision problems that feature ``choice nodes'', ``chance nodes'' and ``utility nodes''. An influence diagram may be associated with a particular probability distribution \cite{nilsson_evaluating_2013} or with a set of probability distributions \cite{dawid_influence_2002}.

See-do models have deep roots in decision theory. Decision theory asks, out of a set of available acts, which ones ought to be chosen. See-do models answer an intermediate question: out of a set of available acts, what are the consequences of each? This question is described by \citet{pearl_causality:_2009} as an ``interventional'' question.

See-do models depend cruicially on a set of choices $D$. While these models can obviously answer questions like ``what is likely to happen if I choose $d\in D$?'', this construction appears to rule out ``causal'' questions like ``Does rain cause wet roads?''. We define a restricted idea of causation called $D$-\emph{causation}. Roughly, if the roads get wet when it rains regardless of my choice of $d\in D$, then rain ``$D$-causes'' wet roads. $D$-causation is closely related to the idea \emph{limited invariance} put forward by \citet{heckerman_decision-theoretic_1995}.


\subsection{D-causation}

The choice set $D$ is a primitive element of a see-do model. However, while we claim that see-do models are the basic objects studied in causal inference, so far we have no notion of ``causation''. What we call $D$-\emph{causation} is one such notion. It is called $D$-causation because it is a notion of causation that depends on the set of choices available. A similar idea, called \emph{limited unresponsiveness}, is discussed extensively in the decision theoretic account of causation found in \citet{heckerman_decision-theoretic_1995}. The main difference is that see-do maps are fundamentally stochastic while Heckerman and Shachter work with ``states'' (approximately hypotheses in our terminology) that map decisions deterministically to consequences. In addition, while we define $D$-causation relative to a see-do map $\kernel{T}$, Heckerman and Shachter define limited unresponsiveness with respect to \emph{sets} of states.

Section \ref{sec:cbns_without_d} explores the difficulty of defining ``objective causation'' without reference to a set of choices. $D$ need not be interpreted as the set of choices available to an agent, but however we want to interpret it, all existing examples of causal models seem to require this set.

See Section \ref{ssec:random_variables} for the definition of random variables in Kernel spaces.

One way to motivate the notion of $D$-causation is to observe that for many decision problems, I may wish to include a very large set of choices $D$. Suppose I aim to have my light switched on, and there is a switch that controls the light. Often, the relevant choices for such a problem would appear to be $D_0=\{\text{flip the switch},\text{don't flip the switch}\}$. However, this doesn't come close to exhausting the set of things I might choose to do, and I might wish to consider a larger set of possibilities. For simplicity's sake, suppose I have instead the following set of options:

\begin{align*}
D_1:=&\{``\text{walk to the switch and press it with my thumb}'', \\
    &``\text{trip over the lego on the floor, hop to the light switch and stab my finger at it}'',\\
    &``\text{stay in bed}''\}
\end{align*}

If having the light turned on is all that matters, I could consider any acts in $D_1$ to be equivalent if, in the end, the light switch ends up in the same position. In this case, I could say that the light switch position $D_1$-causes the state of the light. Subject to the assumption that the light switch position $D_1$-causes the state of the light, I can reduce my problem to one of choosing from $D_0$ (noting that some choices correspond to mixtures of elements of $D_0$).

If I consider an even larger set of possible acts $D_2$, I might not accept that the switch position $D_2$-causes the state of the light. Let $D_2$ be the following acts:

\begin{align*}
D_2:=&\{``\text{walk to the switch and press it with my thumb}'', \\
    &``\text{trip over the lego on the floor, hop to the light switch and stab my finger at it}'',\\
    &``\text{stay in bed}'',\\
    &``\text{toggle the mains power, then flip the light switch}''\}
\end{align*}

In this case, it would be unreasonable to suppose that all acts that left the light switch in the ``on'' position would also result in the light being ``on''. Thus the switch does not $D_2$-cause the light to be on.

Formally, $D$-causation is defined in terms of conditional independence. Given a see-do model $\kernel{T}:H\times D\to \Delta(\sigalg{X}\otimes\sigalg{Y})$, define the \emph{consequence model} $\kernel{C}:H\times D\to \Delta(\sigalg{Y})$ as $\kernel{C}:=\kernel{T}^{\RV{Y}|\RV{H}\RV{D}}$.

\begin{definition}[$D$-causation]\label{def:d_cause}
Given a hypothesis $h\in H$ and a consequence model $\kernel{C}:H\times D\to \Delta(\mathcal{Y})$, random variables $\RV{Y}_1:Y\times D\to Y_1$, $\RV{Y}_2:Y\times D\to Y_2$ and $\RV{D}:Y\times D\to D$ (defined the usual way), $\RV{Y}_1$ $D$-causes $\RV{Y}_2$ iff $\RV{Y}_2\CI_{\kernel{C}} \RV{D}|\RV{Y}_1\RV{H}$.
\end{definition}

\subsection{D-causation vs Limited Unresponsiveness}

Heckerman and Shachter study deterministic ``consequence models''. Furthermore, what we call hypotheses $h\in H$, Heckerman and Schachter call states $s\in S$. Heckerman and Shachter's notion of causation is defined by \emph{limited unresponsiveness} rather than \emph{conditional independence}, which depends on a partition of states rather than a particular hypothesis.

\begin{definition}[Limited unresponsiveness]
    Given states $S$, deterministic consequence models $\kernel{C}_s:D\to \Delta(F)$ for each $s\in A$ and a random variables $\RV{Y}_1:F\to Y_1$, $\RV{Y}_2:F\to Y_2$, $\RV{Y}_1$ is unresponsive to $\RV{D}$ in states limited by $\RV{Y}_2$ if $\kernel{C}_{(s,d)}^{\RV{Y}_2|\RV{S}\RV{D}}=\kernel{C}_{(s,d')}^{\RV{Y}_2|\RV{S}\RV{D}}\implies \kernel{C}_{(s,d)}^{\RV{Y}_1|\RV{S}\RV{D}}=\kernel{C}_{(s,d')}^{\RV{Y}_1|\RV{S}\RV{D}}$ for all $d,d'\in D$, $s\in S$. Write $\RV{Y}_1\not\hookleftarrow_{\RV{Y}_2} \RV{D}$
\end{definition}

\begin{lemma}[Limited unresponsiveness implies $D$-causation]
For deterministic consequence models, $\RV{Y}_1\not\hookleftarrow_{\RV{Y}_2} \RV{D} $ implies $\RV{Y}_2$ $D$-causes $\RV{Y}_1$.
\end{lemma}

\begin{proof}
By the assumption of determinism, for each $s\in S$ and $d\in D$ there exists $y_1(s,d)$ and $y_2(s,d)$ such that $\kernel{C}^{\RV{Y}_1\RV{Y}_2|\RV{S}\RV{D}}_{s,d} = \delta_{y_1(s,d)}\otimes\delta_{y_2(s,d)}$.

By the assumption of limited unresponsiveness, for all $d,d'$ such that $y_2(s,d)=y_2(s,d')$, $y_1(s,d)=y_1(s,d')$ also. Define $f:Y_2\times S\to Y_1$ by $(s,y_1)\mapsto y(s,[y_1(s,\cdot)]^{-1}(y_1(s,d)))$ where $[y_1(s,\cdot)]^{-1}(a)$ is an arbitrary element of $\{d|y_1(s,d)=a\}$. For all $s,d$, $f(y_1(s,d),s)=y_2(s,d)$. Define $\kernel{M}:Y_2\times S\times D\to \Delta(\mathcal{Y}_1)$ by $(y_2,s,d)\mapsto \delta_{f(y_2,s)}$. $\kernel{M}$ is a version of $\kernel{C}^{\RV{Y}_1|\RV{Y}_2,\RV{S},\RV{D}}$ because, for all $A\in \mathcal{Y}_2$, $B\in \mathcal{Y}_1$, $s\in S$, $d\in D$:

\begin{align}
    \kernel{C}^{\RV{Y}_2|\RV{S}\RV{D}}_{(s,d)}\splitter{0.1}(\kernel{M}\otimes\mathrm{Id}) &= \int_A \kernel{M}(y_2',d,s;B) d\delta_{y_2(s,d)}(y_2') \\
                                                                                        &= \int_A \delta_{f(y_2',s)}(B) d\delta_{y_2(s,d)}(y_2') \\
                                                                                        &= \delta_{f(y_2(s,d),s)}(B)\delta_{y_2(s,d)}(A) \\
                                                                                        &= \delta_{y_1(s,d)}(B)\delta_{y_2(s,d)}(A)\\
                                                                                        &= \delta_{y_2(s,d)}\otimes\delta_{y_1(s,d)}(A\times B)
\end{align}

$\kernel{M}$ is clearly constant in $\RV{D}$. Therefore $\RV{Y}_1\CI_{\kernel{C}}\RV{D}|\RV{Y}_2\RV{S}$.
\end{proof}

However, despite limited unresponsiveness implying $D$-causation, it does not imply $D$-causation in mixtures of states\todo{define this}. Suppose $D=\{0,1\}$ where $1$ stands for ``toggle light switch'' and $0$ stands for ``do nothing''. Suppose $S=\{[0,0],[0,1],[1,0],[1,1]\}$ where $[0,0]$ represents ``switch initially off, mains off'' the other states generalise this in the obvious way. Finally, $\RV{F}\in\{0,1\}$ is the final position of the switch and $\RV{L}\in\{0,1\}$ is the final state of the light. We have

\begin{align}
    \kernel{C}^{\RV{L}\RV{F}|\RV{D}\RV{S}}_{d,[i,m]} = \delta_{(d\text{ XOR }i)\text{ AND }m}\otimes \delta_{(d\text{ XOR }i)\text{ AND }m}
\end{align}

Within states $[0,0]$ and $[1,0]$, the light is always off, so $\RV{F}=a\implies \RV{L}=0$ for any $a$. In states $[0,1]$ and $[1,1]$, $\RV{F}=1\implies \RV{L}=1$ and $\RV{F}=0\implies \RV{L}=0$. Thus $\RV{L}\not\hookleftarrow_{\RV{F}} \RV{D}$. However, suppose we take a mixture of consequence models:
\begin{align}
    \kernel{C}_\gamma &= \frac{1}{4}\kernel{C}_{\cdot,[0,0]} + \frac{1}{4}\kernel{C}_{\cdot,[0,1]} + \frac{1}{2}\kernel{C}_{\cdot,[1,1]}\\
    \kernel{C}^{\RV{F}\RV{L}|\RV{D}}_\gamma &= \frac{1}{4} \left[\begin{matrix}
                        1 & 0\\ 0 & 1
                      \end{matrix}\right]\otimes \left[\begin{matrix}
                        1 & 0\\ 1 & 0
                      \end{matrix}\right] + \frac{1}{4} \left[\begin{matrix}
                        1 & 0\\ 0 & 1
                      \end{matrix}\right]\otimes \left[\begin{matrix}
                        1 & 0\\ 0 & 1
                      \end{matrix}\right] + \frac{1}{2}\left[\begin{matrix}
                        0 & 1\\ 1 & 0
                      \end{matrix}\right]\otimes \left[\begin{matrix}
                        0 & 1\\ 1 & 0
                      \end{matrix}\right]
\end{align}

Then

\begin{align}
    [1,0]\kernel{C}^{\RV{F}\RV{L}|\RV{D}}_{\gamma} &= \frac{1}{4}[0,1]\otimes[1,0]+\frac{1}{4}[0,1]\otimes[0,1]+\frac{1}{2}[1,0]\otimes[1,0]\\
    [1,0]\splitter{0.1}(\kernel{C}^{\RV{F}|\RV{D}}_\gamma\otimes \kernel{C}^{\RV{L}|\RV{D}}_\gamma) &= (\frac{1}{2}[0,1]+\frac{1}{2}[1,0])\otimes(\frac{1}{4}[0,1]+\frac{3}{4}[1,0])\\
    \implies [1,0]\kernel{C}^{\RV{F}\RV{L}|\RV{D}}_{\gamma} &\neq [1,0] \splitter{0.1} (\kernel{C}^{\RV{F}|\RV{D}}_\gamma\otimes \kernel{C}^{\RV{L}|\RV{D}}_\gamma)
\end{align}

Thus under the prior $\gamma$, $\RV{F}$ does not $D$-cause\todo{define this} $\RV{L}$ even though $\RV{F}$ $D$-causes $\RV{L}$ in all states $S$. The definition of $D$-causation was motivated by the idea that we could reduce a difficult decision problem with a large set $D$ to a simpler problem with a smaller ``effective'' set of decisions by exploiting conditional independence. Even if $\RV{X}$ $D$-causes $\RV{Y}$ in every $\RV{H}\in S$, $\RV{X}$ does not necessarily $D$-cause $\RV{Y}$ in mixtures of states in $S$. For this reason, we do not say that $\RV{X}$ $D$-causes $\RV{Y}$ in $S$ if $\RV{X}$ $D$-causes $\RV{Y}$ in every $\RV{H}\in S$, and in this way we differ substantially from \citet{heckerman_decision-theoretic_1995}.

Instead, we simply extend the definition of $D$-causation to mixtures of hypotheses: if $\gamma\in \Delta(\RV{H})$ is a mixture of hypotheses, define $\kernel{C}_\gamma:= (\gamma\otimes\textbf{Id})\kernel{C}$. Then $\RV{X}$ $D$-causes $\RV{Y}$ relative to $\gamma$ iff $\RV{Y}\CI_{\kernel{C}_\gamma} \RV{D}|\RV{X}$.

Theorem \ref{th:univ_d_causation} shows that under some conditions, $D$-causation can hold for arbitrary mixtures over subsets of the hypothesis class $\RV{H}$.

\begin{theorem}[Universal $D$-causation]\label{th:univ_d_causation}
If $\RV{X}\CI\RV{H}|\RV{D}$ for all $\RV{H},\RV{H}'\in S\subset \RV{H}$ and $\RV{X}$ $D$-causes $\RV{Y}$ in all $\RV{H}\in S$, then $\RV{X}$ $D$-causes $\RV{Y}$ with respect to all mixed consequence models $\kernel{C}_\gamma$ for all $\gamma\in \Delta(\RV{H})$ with $\gamma(S)=1$.
\end{theorem}

\begin{proof}

For $\gamma\in \Delta(\RV{H})$, define the mixture

\begin{align}
\kernel{C}_\gamma := \begin{tikzpicture}
    \path (0,0) node[dist] (g) {$\gamma$}
    + (0,-0.45) node (D) {$\RV{D}$}
    ++ (1,-0.3) node[kernel] (C) {$\kernel{C}$}
    ++ (1,0) node (F) {$\RV{F}$};
    \draw (g) to [out=0,in=180] ($(C.west) + (0,0.15)$) (D) -- ($(C.west) + (0,-0.15)$) (C) -- (F);
\end{tikzpicture}
\end{align}

Because $\kernel{C}_\RV{H}^{\RV{X}|\RV{D}} = \kernel{C}_{\RV{H}'}^{\RV{X}|\RV{D}}$ for all $\RV{H},\RV{H}'\in \RV{H}$, we have

\begin{align}
\begin{tikzpicture}
    \path (0,0) node[dist] (g) {$\gamma$}
    + (0.7,-0.15) node[copymap] (copy0) {}
    + (0,-0.45) node (D) {$\RV{D}$}
    ++ (1.5,-0.3) node[kernel] (C) {$\kernel{C}^{\RV{X}|\RV{D}\RV{H}}$}
    ++ (1,0) node (X) {$\RV{X}$}
    + (0,0.5) node (T) {$\RV{H}$};
    \draw (g) to [out=0,in=180] (copy0) -- ($(C.west) + (0,0.15)$) (D) -- ($(C.west) + (0,-0.15)$);
    \draw (C) -- (X);
    \draw (copy0) to [out=90,in=180] (T);
\end{tikzpicture} &= \begin{tikzpicture}
    \path (0,0) node[dist] (g) {$\gamma$}
    + (0,0.5) node[dist] (g2) {$\gamma$}
    + (0.7,-0.15) node[copymap] (copy0) {}
    + (0,-0.45) node (D) {$\RV{D}$}
    ++ (1.5,-0.3) node[kernel] (C) {$\kernel{C}^{\RV{X}|\RV{D}\RV{H}}$}
    ++ (1,0) node (X) {$\RV{X}$}
    + (0,0.3) node (T) {$\RV{H}$};
    \draw (g) to [out=0,in=180] (copy0) -- ($(C.west) + (0,0.15)$) (D) -- ($(C.west) + (0,-0.15)$);
    \draw (C) -- (X);
    \draw (g2) to [out=0,in=180] (T);
\end{tikzpicture} \label{eq:decompose_condi_x}
\end{align}

Also

\begin{align}
    \kernel{C}_\gamma^{\RV{XY}|\RV{D}} &= \begin{tikzpicture}
    \path (0,0) node[dist] (g) {$\gamma$}
    + (0,-0.45) node (D) {$\RV{D}$}
    ++ (1,-0.3) node[kernel] (C) {$\kernel{C}$}
    ++ (1,0) node[kernel] (F) {$\kernel{F}^{\RV{X}\utimes\RV{Y}}$}
    ++ (1,0.15) node (X) {$\RV{X}$}
    + (0,-0.3) node (Y) {$\RV{Y}$};
    \draw (g) to [out=0,in=180] ($(C.west) + (0,0.15)$) (D) -- ($(C.west) + (0,-0.15)$) (C) -- (F);
    \draw ($(F.east) + (0,0.15)$) -- (X) ($(F.east) + (0,-0.15)$) -- (Y);
\end{tikzpicture}\\
    &= \begin{tikzpicture}
    \path (0,0) node[dist] (g) {$\gamma$}
    + (0,-0.45) node (D) {$\RV{D}$}
    ++ (1,-0.3) node[kernel] (C) {$\kernel{C}^{\RV{XY}|\RV{D}\RV{H}}$}
    ++ (1,0.15) node (X) {$\RV{X}$}
    + (0,-0.3) node (Y) {$\RV{Y}$};
    \draw (g) to [out=0,in=180] ($(C.west) + (0,0.15)$) (D) -- ($(C.west) + (0,-0.15)$);
    \draw ($(C.east) + (0,0.15)$) -- (X) ($(C.east) + (0,-0.15)$) -- (Y);
\end{tikzpicture}\\
 &= \begin{tikzpicture}
    \path (0,0) node[dist] (g) {$\gamma$}
    + (0,-0.45) node (D) {$\RV{D}$}
    + (0.7,-0.45) node[copymap] (copy0) {}
    + (0.7,-0.15) node[copymap] (copy1) {}
    ++ (1.4,-0.3) node[kernel] (C) {$\kernel{C}^{\RV{X}|\RV{D}\RV{H}}$}
    + (0,0.6) coordinate (via0)
    + (0,-0.6) coordinate (via1)
    ++ (0.9,0) node[copymap] (copy2) {}
    ++ (0.7,0) node[kernel] (Yx) {$\kernel{C}^{\RV{Y}|\RV{X}\RV{D}\RV{H}}$}
    ++ (1.2,0.15) node (X) {$\RV{Y}$}
    + (0,-0.5) node (Y) {$\RV{X}$};
    \draw (g) to [out=0,in=180] (copy1) -- ($(C.west) + (0,0.15)$) (D) -- ($(C.west) + (0,-0.15)$) (C)--(Yx);
    \draw (copy0) to [out=-90,in=180] (via1) to [out=0,in=180] ($(Yx.west) + (0,-0.15)$) (copy1) to [out=90,in=180] (via0) to [out=0,in=180] ($(Yx.west) + (0,0.15)$);
    \draw ($(Yx.east) + (0,0.15)$) -- (X) (copy2) to [out=-90,in=180] (Y);
 \end{tikzpicture}\\
 &\overset{\RV{Y}\CI \RV{D}|\RV{X}\RV{H}}{=} \begin{tikzpicture}
    \path (0,0) node[dist] (g) {$\gamma$}
    + (0,-0.45) node (D) {$\RV{D}$}
    + (0.7,-0.15) node[copymap] (copy1) {}
    ++ (1.4,-0.3) node[kernel] (C) {$\kernel{C}^{\RV{X}|\RV{D}\RV{H}}$}
    ++ (0.9,0.1) node[copymap] (copy2) {}
    ++ (0.7,0.3) node[kernel] (Yx) {$\kernel{C}^{\RV{Y}|\RV{X}\RV{H}}$}
    ++ (1.2,0.15) node (X) {$\RV{Y}$}
    + (0,-0.5) node (Y) {$\RV{X}$};
    \draw (g) to [out=0,in=180] (copy1) -- ($(C.west) + (0,0.15)$) (D) -- ($(C.west) + (0,-0.15)$) (C) to [out=0,in=180] (copy2) to [out=0,in=180] (Yx);
    \draw (copy1) to [out=90,in=180] ($(Yx.west) + (0,0.15)$);
    \draw ($(Yx.east) + (0,0.15)$) -- (X) (copy2) to [out=-90,in=180] (Y);
 \end{tikzpicture} \\
 &\overset{\ref{eq:decompose_condi_x}}{=} \begin{tikzpicture}
    \path (0,0) node[dist] (g) {$\gamma$}
    + (0,-0.45) node (D) {$\RV{D}$}
    + (0.7,-0.15) node[copymap] (copy1) {}
    ++ (1.4,-0.3) node[kernel] (C) {$\kernel{C}^{\RV{X}|\RV{D}\RV{H}}$}
    + (1,0.6) node[dist] (g2) {$\gamma$}
    ++ (0.9,0.1) node[copymap] (copy2) {}
    ++ (1,0.3) node[kernel] (Yx) {$\kernel{C}^{\RV{Y}|\RV{X}\RV{H}}$}
    ++ (1.2,0.15) node (X) {$\RV{Y}$}
    + (0,-0.5) node (Y) {$\RV{X}$};
    \draw (g) to [out=0,in=180] (copy1) -- ($(C.west) + (0,0.15)$) (D) -- ($(C.west) + (0,-0.15)$) (C) to [out=0,in=180] (copy2) to [out=0,in=180] (Yx);
    \draw (g2) to [out=0,in=180] ($(Yx.west) + (0,0.15)$);
    \draw ($(Yx.east) + (0,0.15)$) -- (X) (copy2) to [out=-90,in=180] (Y);
 \end{tikzpicture}\\
 &= \overset{\ref{eq:decompose_condi_x}}{=} \begin{tikzpicture}
    \path (0,0) node (g) {}
    + (0,-0.45) node (D) {$\RV{D}$}
    + (0.7,-0.45) node[copymap] (copy1) {}
    ++ (1.4,-0.3) node[kernel] (C) {$\kernel{C}_\gamma^{\RV{X}|\RV{D}\RV{H}}$}
    + (1,0.6) node[dist] (g2) {$\gamma$}
    ++ (0.9,0.1) node[copymap] (copy2) {}
    ++ (1,0.3) node[kernel] (Yx) {$\kernel{C}^{\RV{Y}|\RV{X}\RV{H}}$}
    + (-0.5,0.6) coordinate (stop0)
    ++ (1.2,0.15) node (X) {$\RV{Y}$}
    + (0,-0.5) node (Y) {$\RV{X}$};
    \draw (D) -- ($(C.west) + (0,-0.15)$) (C) to [out=0,in=180] (copy2) to [out=0,in=180] (Yx);
    \draw (g2) to [out=0,in=180] ($(Yx.west) + (0,0.15)$);
    \draw ($(Yx.east) + (0,0.15)$) -- (X) (copy2) to [out=-90,in=180] (Y);
    \draw[-{Rays[n=8]}] (copy1) to [out=90,in=180] (stop0);
 \end{tikzpicture}\label{eq:is_conditional}
\end{align}
Equation \ref{eq:is_conditional} establishes that $(\gamma\otimes\textbf{Id}_X\otimes\stopper{0.3}_D)\kernel{C}^{\RV{Y}|\RV{X}\RV{H}}$ is a version of $\kernel{C}_\gamma^{\RV{Y}|\RV{X}\RV{D}}$, and thus $\RV{Y}\CI_{\kernel{C}_\gamma} \RV{D}|\RV{X}$.

This can also be derived from the semi-graphoid rules:

\begin{align}
    \RV{H}\CI \RV{D} \land \RV{H}\CI \RV{X} | \RV{D} &\implies \RV{H}\CI \RV{XD}\\
    &\implies \RV{H}\CI \RV{D}|\RV{X}\\
    \RV{D} \CI \RV{H}|\RV{X} \land \RV{D}\CI \RV{Y}|\RV{X}\RV{H} &\implies \RV{D}\CI \RV{Y}|\RV{X}\\
    &\implies \RV{Y}\CI\RV{D}|\RV{X}
\end{align}
\end{proof}

\subsection{Properties of D-causation}

If $\RV{X}$ D-causes $\RV{Y}$ relative to $\kernel{C}_\RV{H}$, then the following holds:

\begin{align}
    \kernel{C}_{\RV{H}}^{\RV{X}|\RV{D}} &= \begin{tikzpicture}
    \path (0,0) node (D) {$\RV{D}$}
    ++ (0.9,0) node[kernel] (Xd) {$\kernel{C}^{\RV{X}|\RV{D}}$}
    ++ (1.3,0) node[kernel] (Yd) {$\kernel{C}^{\RV{Y}|\RV{X}}$}
    ++ (0.9,0) node (Y) {$\RV{Y}$};
    \draw (D) -- (Xd) -- (Yd) -- (Y); 
    \end{tikzpicture}
\end{align}

This follows from version (2) of Definition \ref{def:conditional_independence}:

\begin{align}
    \kernel{C}_\RV{H}^{\RV{X}|\RV{D}} &= \begin{tikzpicture}
    \path (0,0) node (D) {$\RV{D}$}
    ++ (0.7,0) node[copymap] (copy0) {}
    ++ (0.7,0) node[kernel] (Xd) {$\kernel{C}^{\RV{X}|\RV{D}}$}
    + (0,0.5) coordinate (via1)
    ++ (1.3,0) node[kernel] (Yd) {$\kernel{C}^{\RV{Y}|\RV{X}\RV{D}}$}
    ++ (0.9,0) node (Y) {$\RV{Y}$};
    \draw (D) -- (Xd) -- (Yd) -- (Y);
    \draw (copy0) to [out=90,in=180] (via1) to [out=0,in=180] ($(Yd.west)+(0,0.15)$); 
    \end{tikzpicture}\\
     &= \begin{tikzpicture}
    \path (0,0) node (D) {$\RV{D}$}
    ++ (0.7,0) node[copymap] (copy0) {}
    ++ (0.7,0) node[kernel] (Xd) {$\kernel{C}^{\RV{X}|\RV{D}}$}
    + (1.3,0.5) coordinate (via1)
    ++ (1.3,0) node[kernel] (Yd) {$\kernel{C}^{\RV{Y}|\RV{X}}$}
    ++ (0.9,0) node (Y) {$\RV{Y}$};
    \draw (D) -- (Xd) -- (Yd) -- (Y);
    \draw[-{Rays[n=8]}] (copy0) to [out=90,in=180] (via1); 
    \end{tikzpicture}\\
    &= \begin{tikzpicture}
    \path (0,0) node (D) {$\RV{D}$}
    ++ (0.9,0) node[kernel] (Xd) {$\kernel{C}^{\RV{X}|\RV{D}}$}
    ++ (1.3,0) node[kernel] (Yd) {$\kernel{C}^{\RV{Y}|\RV{X}}$}
    ++ (0.9,0) node (Y) {$\RV{Y}$};
    \draw (D) -- (Xd) -- (Yd) -- (Y); 
    \end{tikzpicture}
\end{align}

D-causation is not transitive: if $\RV{X}$ D-causes $\RV{Y}$ and $\RV{Y}$ D-causes $\RV{Z}$ then $\RV{X}$ doesn't necessarily D-cause $\RV{Z}$.

\todo[inline]{Pearl's ``front door adjustment'' and general identification results make use of composing ``sub-consequence-kernels'' like this. Show, if possible, that Pearl's ``sub-consequence-kernels'' obey $D$-causation like relations}

\todo[inline]{Does this ``weak D-causation'' respect mixing under the same conditions as regular D-causation?}

\subsection{Decision sequences and parallel decisions}

Just as observations $\RV{X}$ can be a sequence of random variables $\RV{X}_1$, $\RV{X}_2$, ..., $\RV{D}$ can be a sequence of ``sub-choices'' $\RV{D}_1$, $\RV{D}_2$, ... . Note that by positing such a sequence there is no requirement that $\RV{D}_1$ comes ``before'' $\RV{D}_2$ in any particular sense.




\section{Existence of counterfactuals}

\todo[inline]{I'm struggling with how to explain this well.}

``Counterfactual'' or ``potential outcomes'' models in the causal inference literature are consequence models where choices can be considered in \emph{parallel}. 

Before defining parallel choices, we will consider a ``counterfactual model'' without parallel choices. Consider the following definitions, first from \citet{pearl_causality:_2009} pg. 203-204. I have preserved his notation, including not using any special fonts for things called ``variables'' because this term is used interchangeably with ``sets of variables'' and using special fonts for variables might give the impression that these should be treated as different things while using special fonts for sets of variables is inconsistent with my usual notation.

\todo[inline]{The real solution here is that Pearl's ``variable sets'' are actually ``coupled variables'', see Definition \ref{def:ctensor}, but I'd rather not change his definitions if I can avoid it}

\todo[inline]{put the following inside a quote environment somehow, the regular quote environment fails due to too much markup}
\vspace{1cm}

```
\paragraph{Definition 7.1.1 (Causal Model)}
A causal model is a triple
$M = \langle U, V, F\rangle$,
where:
\begin{enumerate}[label=(\roman*)]
    \item $U$ is a set of \emph{background} variables, (also called \emph{exogenous}), that are determined by factors outside the model;
    \item $V$ is a set $\{V_1 , V_2 ,..., V_n\}$ of variables, called \emph{endogenous}, that are determined by variables in the model -- that is, variables in $U\cup V$;
    \item $F$ is a set of functions $\{f_1 , f_2 ,..., f_n\}$ such that each $f_i$ is a mapping from (the respective domains of) $U_i \cup PA_i$ to $V_i$, where $U i \subseteq U$ and $PA_i \subseteq V \setminus V_i$ and the entire set $F$ forms a mapping from $U$ to $V$. In other words, each $f_i$ in $$v_i = f_i (pa_i , u_i ),\qquad  i\in 1, ... n,$$ assigns a value to $V_i$ that depends on (the values of) a select set of variables in $V \cup U$, and the entire set $F$ has a unique solution $V(u)$.
\end{enumerate}

\paragraph{Definition 7.1.2 (Submodel)}
Let $M$ be a causal model, $X$ a set of variables in $V$, and $x$ a particular realization of $X$. A submodel $M_x$ of $M$ is the causal model $$M_x =\{U, V, F_x\},$$ where $$F_x = \{ f_i : V_i \notin X\}\cup\{ X = x\}.$$

\paragraph{Definition 7.1.3 (Effect of Action)}
Let $M$ be a causal model, $X$ a set of variables in $V$, and $x$ a particular realization of $X$. The effect of action $do(X=x)$ on $M$ is given by the submodel $M_x$

\paragraph{Definition 7.1.4 (Potential Response)}
Let $X$ and $Y$ be two subsets of variables in $V$. The potential response of $Y$ to action $do(X = x)$, denoted $Y_x(u)$, is the solution for $Y$ of the set of equations $F_x$, that is, $Y_x(u) = Y_{M_x}(u)$.

\paragraph{Definition 7.1.6 (Probabilistic Causal Model)}
A probabilistic causal model is a pair $\langle M, P(u)\rangle$, where $M$ is a causal model and $P(u)$ is a probability function defined over the domain of U.
'''


\vspace{1cm}

Implicitly, Definition 7.1.3 proposes a set of ``actions'' that have ``effects'' given by $M_x$. It's not entirely clear what this set of actions should be -- the definition seems to suggest that there is an action for each ``realization'' of each variable in $V$, which would imply that the set of actions corresponds to the range of $V$. For the following discussion, we will call the set of actions $D$, whatever it actually contains (we have deliberately chosen to use the same letter as we use to represent choices or actions in see-do models).

Given $D$, Definition 7.1.3 appears to define a function $h:\mathscr{M}\times D\to \mathscr{M}$, where $\mathscr{M}$ is the space of causal models with background variables $U$ and endogenous variables $V$, such that for $M\in \mathscr{M}$, $do(X=x)\in D$, $h(M,do(X=x))=M_x$.

Definition 7.1.4 then appears to define a function $Y_\cdot(\cdot):D\times U\to Y$ (distinct from $Y$, which appears to be a function $U\to\text{something}$) and calls $Y_\cdot(\cdot)$ the ``potential response''. We could always consider the variable $\RV{V}:=\utimes_{i\in [n]} \RV{V}_i$ and define the ``total potential response'' $\mathbf{g}:=\RV{V}_\cdot(\cdot)$, which captures the potential responses of any subset of variables in $V$.

From this, we might surmise that in the Pearlean view, it is necessary that a ``counterfactual'' or ``potential response'' model has a probability measure $P$ on background variables $U$, a set of actions $D$ and a \emph{deterministic} potential response function $\mathbf{g}:D\times U\to V$.

Pearl's model also features a second deterministic function $\mathbf{f}:U\to Y$, and $G$ is derived from $F$ via the equation modifications permitted by $D$. It is straightforward to show that an arbitrary function $\mathbf{f}:U\to Y$ can be constructed from Pearl's set of functions $f_i$, and if $D$ may modify the set $F$ arbitrarily, then it appears that $\mathbf{g}$ can in principle be an arbitrary function $D\times U\to Y$ (though many possible choices would be quite unusual).

Pearl's counterfactual model seems to essentially be a deterministic map $\mathbf{g}:D\times U\to V$ along with a probability measure $P$ on $U$. Putting these together and marginalising over $U$ (as we might expect we want to do with ``background variables'') simply yields a consequence map $D\to \Delta(\sigalg{V})$, which doesn't seem to have any special counterfactual properties.

In order to pose counterfactual questions, Pearl introduces the idea of holding $U$ fixed:
\\
````
\paragraph{Definition 7.1.5 (Counterfactual)}
Let $X$ and $Y$ be two subsets of variables in $V$. The counterfactual sentence ``$Y$ would be $y$ (in situation $u$), had $X$ been $x$'' is interpreted as the equality $Y_x(u) = y$, with $Y_x(u)$
being the potential response of $Y$ to $X = x$.'
'''
\\

Holding $U$ fixed allows SCM counterfactual models to answer questions about what would have happened if we had taken different actions given the same background context. For example, we can compare $Y_x(u)$ with $Y_{x'}(u)$ and interpret the comparison as telling us what would have happened in the same situation $u$ if we did $x$ and, at the same time, what would happen if we did $x'$. It is the ability to consider different actions ``in exactly the same situation'' that makes these models ``counterfactual''.

One obvious question is: does $\mathbf{g}$ have to be deterministic? While SCMs are defined in terms of deterministic functions with noise arguments, it's not clear that this is a necessary feature of counterfactual models. If $\mathbf{g}$ were properly stochastic, what is the problem with considering $\mathbf{g}(x,u)$ and $\mathbf{g}(x',u)$ to represents what would happen in a fixed situation $u$ if I did $x$ and if I did $x'$ respectively? In fact, a nondeterministic $\mathbf{g}$  arguably fails to capture a key intuition of taking actions ``in exactly the same situation''. If I want to know the result of doing action $x$ and, in exactly the same situation, the result of doing action $x$, then one might intuitively think that the result should always be \emph{deterministically the same}. This property, which we call \emph{deterministic reproducibility}, does not hold if we consider a nondeterministic potential response map $\mathbf{g}$.

This idea of doing $x$ and, in the same situation, doing $x$ doesn't render very well in English. Furthermore, even though deterministic reproducibility seems to be an important property of counterfactual SCMs, they don't help very much to elucidate the idea. ``If I take action $x$ in situation $U$ I get $V_x(u)$ and if I take action $x$ in situation $U$ I get $V_x(u)$'' is just a redundant repetition. It seems that we want some way to express the idea of having two copies of $V_x(u)$ or, more generally, having multiple copies of a potential response function in such a way that we can make comparisons between their results.

The idea that we need \emph{can} be clearly expressed with a see-do model. 

\bibliographystyle{plainnat}
\bibliography{references}

\appendix
\newpage
\section*{Appendix:}

% \input{appendix_AIstats}

\end{document}
