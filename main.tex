\documentclass{book}

% If your paper is accepted, change the options for the package
% aistats2020 as follows:
%
% \usepackage[accepted]{aistats2020}
%
% This option will print headings for the title of your paper and
% headings for the authors names, plus a copyright note at the end of
% the first column of the first page.

% If you set papersize explicitly, activate the following three lines:
%\special{papersize = 8.5in, 11in}
%\setlength{\pdfpageheight}{11in}
%\setlength{\pdfpagewidth}{8.5in}

% If you use natbib package, activate the following three lines:
\usepackage[round]{natbib}
\renewcommand{\bibname}{References}
\renewcommand{\bibsection}{\subsubsection*{\bibname}}

% If you use BibTeX in apalike style, activate the following line:
%\bibliographystyle{apalike}

\usepackage[T1]{fontenc}    % use 8-bit T1 fonts
\usepackage{hyperref}       % hyperlinks
\usepackage{url}            % simple URL typesetting
\usepackage{booktabs}       % professional-quality tables
\usepackage{amsfonts}       % blackboard math symbols
\usepackage{nicefrac}       % compact symbols for 1/2, etc.
\usepackage{microtype}      % microtypography

% My packages

\usepackage[mathscr]{euscript}
\usepackage{graphicx}
\usepackage {tikz}
\usetikzlibrary {positioning}
\usetikzlibrary{shapes.misc}
\usetikzlibrary{shapes.geometric}
\usetikzlibrary{calc}
\usetikzlibrary{arrows.meta}
\usepackage{amsthm}
\usepackage{amsmath}
\usepackage{amssymb}
\usepackage{dsfont}
\usepackage{stmaryrd }
\usepackage{csquotes}
\usepackage{wasysym}
\usepackage[]{todonotes}
\usepackage[shortlabels]{enumitem}
\usepackage{bm}
\usepackage{isomath}

\theoremstyle{plain}
\newtheorem{theorem}{Theorem}[section]
\newtheorem{corollary}[theorem]{Corollary}
\newtheorem{lemma}[theorem]{Lemma}
\newtheorem{proposition}[theorem]{Proposition}


\newtheorem{innercustomthm}{Theorem}
\newenvironment{customthm}[1]
  {\renewcommand\theinnercustomthm{#1}\innercustomthm}
  {\endinnercustomthm}

\theoremstyle{definition}
\newtheorem{definition}[theorem]{Definition}
\newtheorem{example}[theorem]{Example}

\DeclareMathAlphabet{\mathsfit}{T1}{\sfdefault}{\mddefault}{\sldefault}

\newcommand{\CI}{\mathrel{\text{\scalebox{1.07}{$\perp\mkern-10mu\perp$}}}}
\newcommand{\CII}{\mathrel{\text{\scalebox{1.07}{$\perp\mkern-10mu\perp\mkern-10mu\perp$}}}}
\newcommand{\RV}[1]{\ensuremath{\mathsf{#1}}}
\newcommand{\URV}[1]{\ensuremath{\underline{\RV{#1}}}}
\newcommand{\PA}[2]{\ensuremath{\text{Pa}_{#1}(#2)}}
\newcommand{\ND}[2]{\ensuremath{\text{ND}_{#1}(#2)}}
\newcommand{\CH}[2]{\ensuremath{\text{Ch}_{#1}(#2)}}
\newcommand{\DE}[2]{\ensuremath{\text{De}_{#1}(#2)}}
\newcommand{\ID}[1]{\ensuremath{\text{Id}_{#1}}}
\newcommand{\utimes}{\ensuremath{\underline{\otimes}}}
\newcommand{\prob}[1]{\ensuremath{\mathbb{#1}}}
\newcommand{\kernel}[1]{\ensuremath{\mathbb{#1}}}
\newcommand{\seedo}{\ensuremath{\mathbb{T}}}
\newcommand{\diagram}[1]{\ensuremath{\mathscr{#1}}}
\newcommand{\sigalg}[1]{\ensuremath{\mathcal{#1}}}
\newcommand{\vecRV}[1]{\ensuremath{\mathsfbfit{#1}}}
\newcommand{\vecVal}[1]{\ensuremath{\mathbf{#1}}}
\newcommand{\prodSet}[1]{\ensuremath{\mathbf{#1}}}
\newcommand{\indx}[1]{\ensuremath{\mathcal{#1}}}
\newcommand{\nod}[1]{\ensuremath{\mathsfit{#1}}}

\makeatletter
\newcommand*\bigcdot{\mathpalette\bigcdot@{.5}}
\newcommand*\bigcdot@[2]{\mathbin{\vcenter{\hbox{\scalebox{#2}{$\m@th#1\bullet$}}}}}
\makeatother

\tikzset{
	triangle/.style = {regular polygon, regular polygon sides=3 },
    node rotated/.style = {rotate=90},
    border rotated/.style = {shape border rotate=90},
    dist/.style = {triangle,draw,border rotated, inner sep=0pt},
    smalldist/.style = {triangle,draw,border rotated},
    kernel/.style={rectangle,draw,inner sep = 2pt},
    expectation/.style = {triangle,draw,inner sep=0pt,shape border rotate=270}}

\newcommand\DCI{
	\begin{tikzpicture}[scale=0.35]
	\draw[->] (1,0) -- (0,0);
	\draw (0.6,0) -- (0.6,0.75);
	\draw (0.4,0) -- (0.4,0.75);
	\end{tikzpicture}
}

\newcommand\splitter[1]{%
\begin{tikzpicture}[scale=#1]
\draw (0,-1) -- (0,0);
\draw (0,0) to [bend right] (1,1);
\draw (0,0) to [bend left] (-1,1);
\end{tikzpicture}
}

\newcommand\stopper[1]{%
\begin{tikzpicture}[scale=#1]
\draw[-{Rays [n=8]}] (0,-1) -- (0,0);
\end{tikzpicture}
}

\newcommand\source[1]{%
\begin{tikzpicture}[scale=#1]
\path (0,0) node[prob,fill=gray] (P) {};
\draw (P) -- ($(P.east) + (1,0)$);
\end{tikzpicture}
}

\DeclareMathOperator*{\argmax}{arg\,max}
\DeclareMathOperator*{\argmin}{arg\,min}
\DeclareMathOperator*{\arginf}{arg\,inf}
\DeclareMathOperator*{\argsup}{arg\,sup}

\newcommand{\cheng}[1]{ {\color{purple}[{\bf Cheng:~{#1}}]} }

\title{Causal Statistical Decision Theory|What are interventions?}
\date{\today}

\author{ David Johnston }

\begin{document}

\maketitle


% \begin{abstract}
% We develop \emph{causal statistical decison theory} (CSDT) a novel theory of causal inference which we derive by introducing the idea that ``decisions have consequences'' to statistical decision theory. CSDT features \emph{causal theories} as the central object of study. We show that causal Bayesian networks have a natural representation as a causal theory and that potential outcomes models may arguably be represented as causal theories as well. In both cases the resulting theories feature unreasonably rich sets of decisions, which we suggest is because both approaches aim to produce reusable causal models. Using causal theories, we investigate reusability -- when can knowledge gained using one causal theory be applied to another -- and show that this is possible when the theories are related by a \emph{coarsening}.
% \end{abstract}
\tableofcontents



%!TEX root = main.tex


\chapter{Introduction}\label{ch:introduction}

\todo[inline]{I'm thinking of classification problems as types of prediction problems, though it's not really what ``prediction'' means. The key feature is that there is a ground truth that is not known at the time the prediction or class is offered, but will become fully known at some later point.}

Data driven prediction problems and data driven decision making problems have a lot in common. The outcomes some people are interested in predicting are often outcomes other people want to influence. A forecaster might want to predict the winner of the next election, while a party strategist is interested in maximising their party's chance of victory. A product manager may be simultaneously interested in accurately inferring the sentiment expressed in reviews of their product, and in making product changes that increase the frequency that this sentiment is positive. Furthermore, data relevant to prediction is often relevant to decision making and vise-versa. Political parties often reason that electorates in which their predicted chance of victory is very low are not worth investing campaign resources in, and if a forecaster learns of evidence that one party had adopted a particularly effective election strategy they might want to revisit their prediction of the eventual election winner. The overlap is not perfect: comprehensive electorate level polls are probably more useful to the forecaster while small-scale controlled experiments are probably more useful to the strategist.

A key difference between prediction and influence problems is the ``multiplicity of futures'' that each problem asks us to consider. A forecaster wants to identify -- loosely speaking -- the single most likely outcome, while a strategist must consider multiple options and identify the likely outcomes associated with each of these. As a consequence of this difference, the forecaster receives more complete feedback about the quality of their forecast than the strategist. Unlike the forecaster, all but one of the options that the strategist considers are never realised, and the world never offers feedback on these alternative options.

This difference suggests that it might be easier to assess the reliability of a predictive algorithm than the reliability of a decision-making algorithm, and this is be borne out in practice. Validating a predictive algorithm using data split into training and holdout sets is a ubiquitous in machine learning. For many data generating processes, appropriately conducted validation is widely considered to be a reliable indicator of an algorithm's performance for sufficiently similar data generating processes. In contrast, the most well-known condition that is widely accepted to yield reliable decision making algorithms is that the data used to draw inferences comes from a well-conducted controlled experiment. Data that satisfies this is much rarer than data that standard machine learning validation approaches can be applied to. There are approaches to causal inference that don't depend on experimental data, but they depend on other assumptions which are similarly applicable to a limited fraction of datasets. Alternatives to controlled experiments often come with the additional headache of being difficult to assess for a given dataset.

Some of the most far-reaching recent development in algorithmic decision making have involved only the elementary theory of randomised experiments. Operational advances that enable controlled experiments to be conducted at large scales have driven substantial changes in the operations of many online businesses \citep{kohavi_surprising_2017}, and Abhijit Banerjee and Esther Duflo were recently awarded a Nobel prize in part for their pioneering role in the use of large numbers of randomised controlled trials (RCTs) to assess the effectiveness of different kinds of development interventions \citep{zhang_abdul_2014}. Some fields of science have also been significantly affected by ``negative progress'' in the science of assessing experimental results. For example, in psychology, strong evidence has emerged that experimental findings from this field provide weaker evidence to a reader of the findings about the consequences of the reader's actions than many had believed \citet{open_science_collaboration_estimating_2015,stroebe_what_2019}. In a similar time frame, standards for what constitutes a ``well-conducted'' experiment have risen across many fields \citep{nosek_preregistration_2018,liberati_prisma_2009}.

An individual who wants to use data to make better decisions can consider running a controlled experiment of their own. This may not be possible, and even if it is, there may be large amounts of apparently relevant data available that seems wasteful to ignore on the basis of its non-experimental provenance. This individual might therefore be motivated to make some additional assumptions which allow them to draw conclusions about how to act from non-experimental data.

Some examples of assumptions this person could consider are (where * indicates that causal conclusions follow only when the data displays some key features):
\begin{itemize}
    \item There is an input variable independent of the \emph{potential outcomes} conditional on some covariates \citep[Chap. ~12]{imbens_causal_2015}, \citep[Chaps ~2, 3, 5]{angrist_mastering_2014}
    \item [*] There is a variable closely correlated with the \emph{potential outcomes} for each observation \citep[Chap. ~21]{imbens_causal_2015}
    \item \emph{Potential outcomes} are partially observed and vary in a simple way with some variable
    \item [*] The set of observed and unobserved variables have a known \emph{causal structure}
    \item The set of observed variables is \emph{causally sufficient} and the \emph{causal structure} is \emph{faithful} to the conditional independence structure of the observed variables

    \item DID (assumption: initial level accounts for confounders)
    \item Causal sufficiency + faithfulness
    \item Detailed causal structure w/identifiable substructure
\end{itemize}

A common feature of all of these: they're kinda confusing



 - Predictive machine learning has advanced tremendously w/combination of experimentation + theory
 - Causal inference has stuck close to established theory
 - Deriving assumptions is an exact science, making assumptions is not
 - Assumptions are so important in causal inference, and coming at it from a slightly different (decision theoretic) direction can a) expand the space of available assumptions and b) situating common assumptions in a different context


 Further reasons for alternative foundations
 - CGM theory has been undeniably productive -- mediation, confounding, ``m-structures'', causal discovery
 - Some outstanding questions:
 - Not always obvious how to map pre-formal knowledge to interventions
 - Not all variables are ``causally compatible''



\section{Theories of causal inference}

Beginning in the 1930s, a number of associations between cigarette smoking and lung cancer were established: on a population level, lung cancer rates rose rapidly alongside the prevalence of cigarette smoking. Lung cancer patients were far more likely to have a smoking history than demographically similar individuals without cancer and smokers were around 40 times as likely as demographically similar non-smokers to go on to develop lung cancer. In laborotory experiments, cells which were introduced to tobacco smoke developed \emph{ciliastasis}, and mice exposed to cigarette smoke tars developed tumors\citep{proctor_history_2012}. Nevertheless, until the late 1950s, substantial controversy persisted over the question of whether the available data was sufficient to establish that smoking cigarettes \emph{caused} lung cancer. Cigarette manufacturers famously argued against any possible connection \citep{oreskes_merchants_2011} and Roland Fisher in particular argued that the available data was not enough to establish that smoking actually caused lung cancer \citep{fisher_cancer_1958}. Today, it is widely accepted that cigarettes do cause lung cancer, along with other serious conditions such as vascular disease and chronic respiratory disease \citep{world_health_organisation_tobacco_nodate,wiblin_why_2016}.

The question of a causal link between smoking and cancer is a very important one to many different people. Individuals who enjoy smoking (or think they might) may wish to avoid smoking if cigarettes pose a severe health risk, so they are interested in knowing whether or not it is so. Additionally, some may desire reassurance that their habit is not too risky, whether or not this is true. Potential and actual investors in cigarette manufacturers may see health concerns as a barrier to adoption, and also may personally want to avoid supporting products that harm many people. Like smokers, such people might have some interest in knowing the truth of this question, and a separate interest in hearing that cigarettes are not too risky, whether or not this is true. Governments and organisations with a responsibility for public health may see themselves as having responsibility to discourage smoking as much as possible if smoking is severely detrimental to health. The costs and benefits of poor decisions about smoking are large: 8 million annual deaths are attributed to cigarette-caused cancer and vascular disease in 2018\citep{world_health_organisation_tobacco_nodate} while  global cigarette sales were estimated at US\$711 billion in 2020 \citep{noauthor_cigarettes_nodate} (a figure which might be substantially larger if cigarettes were not widely believed to be harmful).

The question of whether or not cigarette smoking causes cancer illustrates two key facts about causal questions: First, having the right answers to causal questions is of tremendous importance to huge numbers of people. Second, confusion over causal questions can persist even when a great deal of data and facts relevant to the question are agreed upon.

Causal conclusions are often justified on the basis of ad-hoc reasoning. For example \citet{krittanawong_association_2020} state:

\begin{quote}
[...] the potential benefit of increased chocolate consumption, reducing coronary artery disease (CAD) risk is not known. We aimed to explore the association between chocolate consumption and CAD.
\end{quote}

It is not clear whether Krittanawong et. al. mean that a negative association between chocolate consumption and CAD implies that increased chocolate consumption is likely to reduce coronary artery disease (which is suggested by the word ``benefit''), or that an association may be relevant to the question and the reader should draw their own conclusions. Whether the implication is being suggested by Krittanawong et. al. or merely imputed by na\"ive readers, it is being drawn on an ad-hoc basis -- no argument for the implication can be found in this paper. As \citet{pearl_causality:_2009} has forcefully argued, additional assumptions are always required to answer causal questions from associational facts, and stating these assumptions explicitly allows those assumptions to be productively scrutinised.

For causal questions that are controversial or difficult, it is tremendously advantageous to be able to address them transparently. Theories of causation enable this; given a theory of causation and a set of assumptions, if anyone claims that some conclusion follows it is publicly verifiable whether or not it actually does so. If the deduction is correct, then any remaining disagreement must be in the assumptions or in the theory. For people who are interested in understanding what is true, pinpointing disagreement can be enlightening. Someone could learn, for example, that there are assumptions that they find plausible that permit conclusions they did not initially believe. Alternatively, if a motivated conclusion follows only from implausible assumptions, hearing these assumptions explicitly might make the conclusion less attractive. 

Theories of causation help us to answer causal questions, which means that before we have any theory, we already have causal questions we want to answer. If potential outcomes notation and causal graphical models had never been invented there would still be just as many people who want to the answer to questions something like ``does smoking causes cancer?'', even if on-one could say what exactly they meant by ``causes'' and even if many people actually want answers to slightly different questions. Theories exist to serve our need for transparent answers to causal questions.

Potential outcomes and causal graphical models are prominent examples of ``practical theories'' of causation. I call them ``practical theories'' because most of the time we encounter them they are being used to answer ``practical'' questions like ``Does smoking cause cancer?'', or ``In general, when does data allow us to conclude that $X$ causes $Y$?'' It is less common to see the ``fundamental questions'' addressed, like ``Does the theory of causal graphical models offer an adequate account of what `cause' means?'', which is more often found in the field of philosophy. \citet{spirtes_causation_1993} explain their motivation to study what I call ``practical theories of causation'' as follows:

\begin{quote}
One approach to clarifying the notion of causation -- the philosophers’ approach ever since Plato -- is to try to define ``causation'' in other terms, to provide necessary and sufficient and noncircular conditions for one thing, or feature or event or circumstance, to cause another, the way one can define ``bachelor'' as ``unmarried adult male human.'' Another approach to the same problem -- the mathematician’s approach ever since Euclid -- is to provide axioms that use the notion of causation without defining it, and to investigate the necessary consequences of those assumptions. We have few fruitful examples of the first sort of clarification, but many of the second [...]
\end{quote}

I think what Spirtes, Glymour and Scheines (henceforth: SGS) mean here is that they \emph{define} a notion of causation -- because causal graphical models do define a notion of causation -- without interrogating whether it means the same thing as the word ``causation''. Incidentally, since publication of this paragraph, the notion of causation defined by causal graphical models has been subject to substantial interrogation by philosophers \citep{woodward_causation_2016}.

I am sympathetic to the argument that it does not matter a great deal whether ``causal-graphical-models-causation'' and ``causation'' mean the same thing in everyday language. It is common for words to have somewhat different meanings when used by specialists to when they are used by laypeople, and this isn't because the specialists are ignorant or confused about their subject. However, I think it matters a lot which causal questions can be transparently answered by ``causal-graphical-models-causation'', and so I believe that the notions of causation adopted by practical theories do warrant scrutiny.

I think one reason that SGS are keen to avoid dwelling on the definition of causation is that satisfactory definitions of causation are difficult. For example, causal graphical models depend on the notion of \emph{causal relationships} between variables. These may be defined as follows:

\begin{quote}
$\RV{X}_i$ is a \emph{cause} of $\RV{X}_j$ if there is an \emph{ideal intervention} on $\RV{X}_i$ that changes the value $\RV{X}_j$
\end{quote}

This definition is incomplete without a definition of ``ideal interventions''. Ideal interventions may be defined by their action in ``causally sufficient models'':
\begin{itemize}
    \item An $[\RV{X}_i,\RV{X}_j]$-ideal intervention is an operation whose result is determined by applying the \emph{do-calculus} to a \emph{causally sufficient} model $((\Omega,\mathcal{F},\prob{P}),\diagram{G},\vecRV{U})$
    \item A model $((\Omega,\mathcal{F},\prob{P}),\diagram{G},\vecRV{U})$ is $[\RV{X}_i,\RV{X}_j]$-causally sufficient if $\RV{U}$ contains $\RV{X}_i$, $\RV{X}_j$ and ``all intervenable variables that \emph{cause}'' both $\RV{X}_i$ and $\RV{X}_j$ \footnote{Weaker conditions for causal sufficiency are possible, but they don't avoid circularity \citep{shpitser_complete_2008}}
\end{itemize}

While I don't offer a definition of the \emph{do-calculus} in this introduction, it can be rigorously defined, see for example \citet{pearl_causality:_2009}. The problem is that the definition of a \emph{causally sufficient} model itself invokes the word \emph{cause}, which is what the original definition was trying to address. Circularity is a recognised problem with interventional definitions of causation \citep{woodward_causation_2016}. In Section \ref{sec:cbns_without_d}, I further show models with ideal interventions generally have counterintuitive properties. The purpose of a theory of causation like causal graphical models is to support transparent reasoning about causal questions, and a circular definition that leads to counterintuitive conclusions undermines this purpose.

As with Euclid's parallel postulate, I think it is reasonable to ask if the notion of ideal interventions and other causal definitions can be modified or avoided. Causal statistical decision theory (CSDT) is a theory of causation that is motivated by the problem of \emph{what is generally needed to answer causal questions} rather than \emph{what does ``causation'' mean?} Along similar lines to CSDT, \citet{dawid_decision-theoretic_2020} has observed that the problem of deciding how to act in light of data can be formalised without appeal to theories of causation. We develop this in substantial detail, showing how both \emph{interventional models} and \emph{counterfactual models} arise as special cases of CSDT.\todo{I want to revisit the claims about what I actually show, hopefully to add to it}

A key feature of CSDT is what I call the \emph{option set}. This is the set of decisions, acts or counterfactual propositions under consideration in a given problem. A causal graphical model and a potential outcomes model will both implicitly define an option set as a result of their basic definitions of causation, but CSDT demands that this is done explicitly. I argue that this is a key strength of CSDT, on the basis of the following claims which I defend in the following chapters:

\begin{itemize}
    \item Causal questions are not well-posed without an option set in the same way a function is not well-defined without its domain
    \item The option set need not correspond in any fixed manner to the set of observed variables
    \item The nature of the option set can affect the difficulty of causal inference questions
\end{itemize}


\todo[inline]{I commented out an additional section about potential outcomes and closest world counterfactuals, which is a second example of ``opaque causal definitions''. I'm interested if any readers think it would be good to have a second example}


% Potential outcomes basic assumptions

% \begin{itemize}
%     \item Potential outcomes defines ``the treatment effect of $\RV{X}_i$ on $\RV{X}_j$'' in terms of the value of $\RV{X}_j$ under the \emph{counterfacutal supposition} that $\RV{X}_i$ had taken a different value
% \end{itemize}

% In fact, the notion of ``ideal intervention'' often seems to underpin potential outcomes models as well. Work in the potential outcomes theory often uses the idea of the value of $\RV{X}_j$ under a counterfactual supposition concerning $\RV{X}_i$ interchangeably with the idea the response of $\RV{X}_j$ to an idealised intervention on $\RV{X}_i$ \citep{morgan_counterfactuals_2014,rubin_causal_2005,richardson2013single}. \cite{lewis_causation_1986} offered a definition of the value $\RV{X}_j$ under counterfactual suppositions in terms of the value it would take in the world that was ``closest'' to the real world but in which the value of $\RV{X}_i$ was altered. There are many ways that we could use to measure how close one world is to another, many of which need not invoke any notion of ``ideal intervention'', but I have never encountered practical work on causal inference that was based on considerations of such similarity measures.


\section{Causally compatible variables}\label{sec:cc_vars}


%!TEX root = main.tex

\todo[inline][Todo: I need the following theorem in this chapter]
\begin{theorem}[Representation]\label{th:representaiton}
\todo[inline]{Representation theorem: can uniquely define kernel $P^{\RV{X}|\RV{Y}}$ with $P^{\RV{Z}|\RV{Y}}$ and $P^{\RV{X}|\RV{Z}\RV{Y}}$ }
\end{theorem}


\subsection{Probability Theory}

Given a set $A$, a $\sigma$-algebra $\mathcal{A}$ is a collection of subsets of $A$ where
\begin{itemize}
	\item $A\in \mathcal{A}$ and $\emptyset\in \mathcal{A}$
	\item $B\in \mathcal{A}\implies B^C\in\mathcal{A}$
	\item $\mathcal{A}$ is closed under countable unions: For any countable collection $\{B_i|i\in Z\subset \mathbb{N}\}$ of elements of $\mathcal{A}$, $\cup_{i\in Z}B_i\in \mathcal{A}$ 
\end{itemize}

A measurable space $(A,\mathcal{A})$ is a set $A$ along with a $\sigma$-algebra $\mathcal{A}$. Sometimes the sigma algebra will be left implicit, in which case $A$ will just be introduced as a measurable space.

\paragraph{Common $\sigma$ algebras}

For any $A$, $\{\emptyset,A\}$ is a $\sigma$-algebra. In particular, it is the only sigma algebra for any one element set $\{*\}$.

For countable $A$, the power set $\mathscr{P}(A)$ is known as the discrete $\sigma$-algebra.

Given $A$ and a collection of subsets of $B\subset\mathscr{P}(A)$, $\sigma(B)$ is the smallest $\sigma$-algebra containing all the elements of $B$. 

Let $T$ be all the open subsets of $\mathbb{R}$. Then $\mathcal{B}(\mathbb{R}):=\sigma(T)$ is the \emph{Borel $\sigma$-algebra} on the reals. This definition extends to an arbitrary topological space $A$ with topology $T$.

A \emph{standard measurable set} is a measurable set $A$ that is isomorphic either to a discrete measurable space $A$ or $(\mathbb{R}, \mathcal{B}(\mathbb{R}))$. For any $A$ that is a complete separable metric space, $(A,\mathcal{B}(A))$ is standard measurable. 

Given a measurable space $(E,\mathcal{E})$, a map $\mu:\mathcal{E}\to [0,1]$ is a \emph{probability measure} if
\begin{itemize}
	\item $\mu(E)=1$, $\mu(\emptyset)=0$
	\item Given countable collection $\{A_i\}\subset\mathscr{E}$, $\mu(\cup_{i} A_i) = \sum_i \mu(A_i)$
\end{itemize}

Write by $\Delta(\mathcal{E})$ the set of all probability measures on $\mathcal{E}$.

A particular probability measure we will often discuss is the \emph{Dirac measure}. For any $x\in X$, the Dirac measure $\delta_x\in \Delta(\sigalg{X})$ is the probability measure where $\delta_x(A)=0$ if $x\not\in A$ and $\delta_x(A)=1$ if $x\in A$.

Given another measurable space $(F,\mathcal{F})$, a \emph{stochastic map} or \emph{Markov kernel} is a map $\kernel{M}:E\times\mathcal{F}\to [0,1]$ such that
\begin{itemize}
	\item The map $\kernel{M}(\cdot;A):x\mapsto \kernel{M}(x;A)$ is $\mathcal{E}$-measurable for all $A\in \mathcal{F}$
	\item The map $\kernel{M}_x:A\mapsto \kernel{M}(x;A)$ is a probability measure on $F$ for all $x\in E$
\end{itemize}

Extending the subscript notation, for $\kernel{C}:X\times Y\to \Delta(\mathcal{Z})$  and $x\in X$ we will write $\kernel{C}_{x,\cdot}$ for the ``curried'' map $y\mapsto \kernel{C}_{x,y}$. If $\kernel{C}$ is a Markov kernel with respect to $(X\times Y, \sigalg{X}\otimes\sigalg{Y}),(Z,\sigalg{Z})$ then it is straightforward to show that $\kernel{C}_{x,\cdot}$ is a Markov kernel with respect to $(Y,\sigalg{Y}),(Z,\sigalg{Z})$.

This yields the notational conventions for arbitrary kernel $\kernel{C}$:

\begin{itemize}
	\item $\kernel{C}$ with no subscripts is a Markov kernel
	\item $\kernel{C}_{\cdot,a,b}$ with at least one $\cdot$ subscript is a Markov kernel
	\item $\kernel{C}_y$ with no $\cdot$ subscripts is a probability measure
\end{itemize}

The map $x\mapsto \kernel{M}_x$ is of type $E\to \Delta(\mathcal{F})$. We will abuse notation somewhat to write $\kernel{M}:E\to \Delta(\mathcal{F})$. In this sense, we view Markov kernels as maps from elements of $E$ to probability measures on $\mathcal{F}$. This is simply a convention that helps us to think about constructions involving Markov kernels, and it is equally valid to view Markov kernels as maps from elements of $\mathcal{F}$ to measurable functions $E\to[0,1]$, a view found in \citet{clerc_pointless_2017}, or simply in terms of their definition above.

Given an indiscrete measurable space $(\{*\},\{\{*\},\emptyset\})$, we identify Markov kernels $\kernel{N}:\{*\}\to \Delta(\mathcal{E})$ with the probability measure $\kernel{N}_*$. In addition, there is a unique Markov kernel $\stopper{0.2}:E\to \Delta(\{\{*\},\emptyset\})$ given by $x\mapsto \delta_*$ for all $x\in E$ which we will call the ``discard'' map.

Two Markov kernels $\kernel{M}X\to \Delta(\sigalg{Y})$ and $\kernel{N}:X\to \Delta(\sigalg{Y})$ are equal iff for all $x\in X$, $A\in \sigalg{Y}$
\begin{align}
	\kernel{M}_x(A) = \kernel{N}_x(A)
\end{align}

We will typically be more concerned with ``almost sure'' equality than exact equality, which will be defined later.

\subsection{Product Notation}\label{ssec:product_notation}

Probability measures, Markov kernels and measurable functions can be combined to yield new probability measures, Markov kernels or measurable functions. Given $\mu\in \Delta(\mathcal{X})$, $\RV{T}:Y\to T$, $\kernel{M}:X\to \Delta(\sigalg{Y})$ and $\kernel{N}:Y\to \Delta(\sigalg{Z})$ define:

The \textbf{measure-kernel} product $\mu \kernel{M}:\sigalg{Y}\to [0,1]$ where for all $A\in\sigalg{Y}$,

\begin{align}
\mu\kernel{M} (A) := \int_X \kernel{M}_x (A) d\mu(x)
\end{align}

The \textbf{kernel-function} product $\kernel{M} \RV{T}:X\to T$ where for all $x\in X$:

\begin{align}
\kernel{M}\RV{T}(x) := \int_Y T(y) d\kernel{M}_x(y)
\end{align}


The \textbf{kernel-kernel} product $\kernel{M}\kernel{N}:X\to \Delta(\sigalg{Z})$ where for all $x\in X$, $A\in \sigalg{Z}$:

\begin{align}
(\kernel{M}\kernel{N})_x(A) &:= \int_Y \kernel{N}_y(A) d\kernel{M}_x(y)
 \end{align} 

All kernel products are associative \citep{cinlar_probability_2011}. An intuition for this notation can be gained from thinking of probability measures $\mu\in \Delta(\mathcal{X})$ as row vectors, Markov kernels $\kernel{M},\kernel{N}$ as matrices and measurable functions $\RV{T}:Y\to T$ as column vectors and kernel products are vector-matrix and matrix-matrix products. If the $X,Y,Z$ and $T$ are discrete spaces then this analogy is precise.

Finally, the \textbf{tensor product} $\kernel{M}\otimes \kernel{N}:X\times Y\to \Delta(\sigalg{Y}\otimes\sigalg{Z})$ is yields the kernel that applies $\kernel{M}$ and $\kernel{N}$ ``in parallel''. For all $x\in X$, $y\in Y$, $G\in \sigalg{Y}$ and $H\in \sigalg{Z}$:

\begin{align}
(\kernel{M}\otimes \kernel{N})_{x,y}(G\times H) := \kernel{M}_x(G)\kernel{N}_y(H)
\end{align}

\subsection{String Diagrams}\label{ssec:mken_diagrams}

Some constructions are unwieldly in product notation; for example, given $\mu\in \Delta(\mathcal{E})$ and $\kernel{M}:E\to (\mathcal{F})$, it is not straightforward to write an expression using kernel products and tensor products that represents the ``joint distribution'' given by $A\times B\mapsto \int_A \kernel{M}(x;B)d\mu$.

An alternative notation known as \emph{string diagrams} provides greater expressive capability than product notation while being more visually clear than integral notation. \citet{cho_disintegration_2019} provides an extensive introduction to string diagram notation for probability theory.

Key features of string diagrams include:
\begin{itemize}
	\item String diagrams as they are used in this work can always be interpreted as a mixture of kernel-kernel products and tensor products of Markov kernels
	\item String diagrams are the subject of a coherence theorem: two string diagrams that differ only by planar deformation are always equal \citep{selinger_survey_2010}. This also holds for a number of additional transformations detailed below
	\begin{itemize}
		\item Informally, diagrams that look like they should be the same are in fact the same
	\end{itemize}
\end{itemize}

\subsubsection{Elements of string diagrams}

The basic elements of a string diagram are Markov kernels. Diagrams representing Markov kernels can be assembled into larger diagrams by taking regular products or tensor products.

Indiscrete spaces play a key role in string diagrams. An indiscrete space is any one element measurable space $(\{*\},\{\emptyset,\{*\}\})$ which admits the unique probability measure $\mu:\{\emptyset,\{*\}\}\to(0,1)$ given by $\mu(\emptyset)=0$, $\mu(\{*\})=1$. Any probability measure $\mu\in \Delta(\sigalg{X})$ can be interpreted as a Markov kernel $\mu':\{*\}\to \Delta(\mathcal{X})$ where $\mu'_*=\mu$ (note that $*$ is the \emph{only} argument $\mu'$ can be given).


A Markov kernel $\kernel{M}:X\to \Delta(\mathcal{Y})$ can always be represented as a rectangular box with input and output wires labeled with the relevant spaces:

\begin{align}
\begin{tikzpicture}
\path (0,0) node (A) {$X$}
++(0.75,0) node[kernel] (B) {$\kernel{M}$}
++(0.75,0) node (C) {$Y$};
\draw (A) -- (B) -- (C);
\end{tikzpicture}
\end{align}

Note that we will later substitute labelling wires with spaces for labelling them with random variable names.

Probability measures are kernels with an indiscrete domain $\mu \in \Delta(\mathcal{X})$ can be written as triangles:
\begin{align}
\begin{tikzpicture}
\path (0,0) node[dist] (B) {$\mu$}
++(0.75,0) node (C) {$X$};
\draw (B) -- (C);
\end{tikzpicture}\label{eq:prob_meas_sd}
\end{align}

Note that Eq \ref{eq:prob_meas_sd} technically represents a Markov kernel $\mu':\{*\}\to\Delta(\mathcal{X})$, but for our purposes this distinction isn't practically important.

We do \emph{not} define kernel-function products for string diagrams. While kernel-kernel products always yield Markov kernels as a result, and measure-kernel products can be reinterpreted as kernel-kernel products, kernel-function products do not admit such a reinterpretation. \citet{cho_disintegration_2019} defines the operation of \emph{conditioning} using kernel-function products, but this will take extra work to incorporate into our notation which hasn't yet been done.

\paragraph{Elementary operations}

Kernel-kernel products have a visually similar representations in string diagram notation to the previously introduced product notation. Given $\kernel{M}:X\to\Delta(\mathcal{Y})$ and $\kernel{N}:Y\to \Delta(\mathcal{Z})$, we have 

\begin{align}
\kernel{M}\kernel{N} := \begin{tikzpicture}
 \path (0,0) node (E) {$X$}
 ++ (1,0) node[kernel] (M) {$\kernel{M}$}
 ++ (1,0) node[kernel] (N) {$\kernel{N}$}
 ++(1,0) node (G) {$Z$};
 \draw (E) -- (M) -- (N) -- (G);
\end{tikzpicture}\label{eq:sd_composition}
\end{align}

For $\mu\in \Delta(\mathcal{E})$,

\begin{align}
\mu\kernel{M} &:= \begin{tikzpicture}
 \path (0,0) node[dist] (M) {$\mu$}
 ++ (1,0) node[kernel] (N) {$\kernel{M}$}
 ++(1,0) node (G) {$Z$};
 \draw (M) -- (N) -- (G);
\end{tikzpicture}
\end{align}

Tensor products in string diagram notation are represented by vertical juxtaposition. For $\kernel{O}:Z\to \Delta(\mathcal{W})$:

\begin{align}
\kernel{M}\otimes\kernel{O}&:= \begin{tikzpicture}
\path (0,0) node (E) {$X$}
++(1,0) node[kernel] (M) {$\kernel{M}$}
++(1,0) node (F) {$Y$}
(0,-0.5) node (F1) {$Z$}
++(1,0) node[kernel] (N) {$\kernel{O}$}
+(1,0) node (G) {$W$};
\draw (E) -- (M) -- (F);
\draw (F1) -- (N) -- (G);
\end{tikzpicture}
\end{align}

A space $X$ is identified with the identity kernel $\mathrm{Id}^X:X\to \Delta(\sigalg{X})$, $x\mapsto \delta_x$. A bare wire represents an identity kernel or, equivalently, the space given by its labels:

\begin{align}
\mathrm{Id}^X:=\begin{tikzpicture}
\path (0,0) node (X) {$X$}
++(2,0) node (Y) {$X$};
\draw (X) -- (Y);
\end{tikzpicture}
\end{align}

Product spaces $X\times Y$ are identified with tensor products of identity kernels $X\times Y \cong \kernel{I}^X\otimes \kernel{I}^Y$. These can be represented either by two parallel wires or by a single wire equipped with appropriate labels:
\begin{align}
X\times Y \cong \mathrm{Id}^X\otimes \mathrm{Id}^Y &:= \begin{tikzpicture}
\path (0,0) node (E) {$X$}
++(1,0) node (F) {$X$}
(0,-0.5) node (F1) {$Y$}
+(1,0) node (G) {$Y$};
\draw (E) -- (F);
\draw (F1) -- (G);
\end{tikzpicture}\\
&= \begin{tikzpicture}
\path (0,0) node (X) {$X\times Y$}
++(2,0) node (Y) {$X\times Y$};
\draw (X) -- (Y);
\end{tikzpicture}
\end{align}

A kernel $\kernel{L}:X\to \Delta(\mathcal{Y}\otimes\mathcal{Z})$ can be written using either two parallel output wires or a single output wire, appropriately labeled:

\begin{align}
&\begin{tikzpicture}
\path (0,0) node (E) {$X$}
++ (1,0) node[kernel] (L) {$\kernel{L}$}
++ (1,0.15) node (F) {$Y$}
+(0,-0.3) node (G) {$Z$};
\draw (E) -- (L);
\draw ($(L.east) + (0,0.15)$) -- (F);
\draw ($(L.east)+ (0,-0.15)$) -- (G);
\end{tikzpicture}\\
&\equiv\\
&\begin{tikzpicture}
\path (0,0) node (E) {$X$}
++ (1,0) node[kernel] (L) {$\kernel{L}$}
++ (1.5,0) node (F) {$Y\times Z$};
\draw (E) -- (L) -- (F);
\end{tikzpicture}
\end{align}

\paragraph{Markov kernels with special notation}

A number of Markov kernels are given special notation distinct from the generic ``box'' above. This notation facilitates intuitive visual representation.

As has already been noted, the identity kernel $\textbf{Id}:X\to \Delta(X)$ maps a point $x$ to the measure $\delta_x$ that places all mass on the same point:

\begin{align}
\textbf{Id} : x\mapsto \delta_x \equiv \begin{tikzpicture}\path (0,0) node (X) {$X$} + (1,0) node (X1) {$X$}; \draw (X)--(X1); \end{tikzpicture}\label{eq:identity}
\end{align}

The identity kernel is an identity under left and right products:

\begin{align}
	(\kernel{K}\textbf{Id})_w(A) &= \int_X \textbf{Id}_x(A) d\kernel{K}_w (x) \\
							 	 &= \int_X \delta_x(A) d\kernel{K}_w(x)\\
							 	 &= \int_A d\kernel{K}_w(x)\\
							 	 &= \kernel{K}_w(A)\\
	(\textbf{Id}\kernel{K})_w(A) &= \int_X \kernel{K}_x (A) d\textbf{Id}_w(x)\\
								 &= \int_X  \kernel{K}_x(A) d\delta_w(x)\\
								 &= \kernel{K}_w(A)								  
\end{align}

The copy map $\splitter{0.1}:X\to \Delta(\mathcal{X}\times \mathcal{X})$ maps a point $x$ to two identical copies of x:
\begin{align}
 \splitter{0.1}: x\mapsto \delta_{(x,x)} \equiv \begin{tikzpicture}
 \path (0,0) node (X) {$X$} ++ (0.5,0) coordinate (copy0) ++ (0.5,0.25) node (X1) {$X$} ++(0,-0.5) node (X2) {$X$};\draw (X)--(copy0) to [bend left] (X1) (copy0) to [bend right] (X2);
 \end{tikzpicture}\label{eq:copy}
 \end{align} 

The copy map ``copies'' its arguments to kernels or under the right product:

\begin{align}
	\int_(X\times X) \kernel{K}_{x',x''}(A) d\splitter{0.1}_x (x',x'') &= \int_(X\times X) \kernel{K}_{x',x''}(A) d\delta_{(x,x)}(x',x'')\\
															&= \kernel{K}_{x,x}(A)
\end{align}

The swap map $\sigma:X\times Y\to \Delta(\mathcal{Y}\otimes\mathcal{X})$ swaps its inputs:

\begin{align}
\sigma := (x,y)\to \delta_{(y,x)} \equiv \begin{tikzpicture}
\path (0,0) node (X) {$X$}
+(1,0.3) node (X1) {$X$}
(0,0.3) node (Y) {$Y$}
+(1,-0.3) node (Y1) {$Y$};
\draw (X)--(X1) (Y) -- (Y1);
\end{tikzpicture}\label{eq:swap}
\end{align}

Under products are taken with the swap map, arguments are interchanged. For $\kernel{K}:X\times Y\to \Delta(\sigalg{Z})$ and $\kernel{L}:Z\to \Delta(\sigalg{X}\otimes\sigalg{Y})$, $A\in \sigalg{X}$, $B\in\sigalg{Y}$:

\begin{align}
	(\sigma\kernel{K})_{y,x}(A) &= \int_(X\times Y) \kernel{K}_{x',y'}(A) d\sigma_{(y,x)}(x',y') &= \int_(X\times Y) \kernel{K}_{x',y'}(A) d\delta_{(x,y)}(x',y')\\
													   &= \kernel{K}_{x,y}(A)\\
	(\kernel{L}\sigma)_{z}(B\times A) &= \int_{X\times Y} \sigma_{x',y'}(B\times A) d\kernel{L}_z(x',y')\\
	&= \int_{X\times Y} \delta_{(y',x')} (B\times A) d\kernel{L}_z(x',y')\\
	&= \kernel{L}_z(A\times B)
\end{align}

The discard map $\stopper{0.2}:X\to \Delta(\{*\})$ maps every input to $\delta_{*}$, the unique probability measure on the indiscrete set $\{\emptyset,\{*\}\}$.
\begin{align}
\stopper{0.2}: x\mapsto \delta_{*} \equiv \begin{tikzpicture}
 \draw[-{Rays [n=8]}] (0,0) node (X) {$X$} (X) -- (1,0);
\end{tikzpicture}\label{eq:discard}
\end{align}

Any measurable function $g:W\to X$ has an associated Markov kernel $\kernel{F}^g:W\to \Delta(\mathcal{X})$ given by $\kernel{F}^g:w\mapsto \delta_{g(w)}$. Given a probability measue $\mu\in \Delta(\sigalg{W})$, $\mu g$ is a measure-function product while $\mu \kernel{F}^g$ is commonly called the pushforward measure $g_\# \mu$. We will generalise this slightly to the notion of \emph{pushforward kernels}.

\begin{definition}[Kernel associated with a function]\label{def:functional_kernel}
Given a measurable function $g:W\to X$, define the function induced kernel $\kernel{F}^{g}:W\to \Delta(\mathcal{X})$ to be the the Markov kernel $w\mapsto \delta_{g(w)}$ for all $w\in W$.
\end{definition}

\begin{definition}[Pushforward kernel]
Given a kernel $\kernel{M}:V\to \Delta(\mathcal{W})$ and a measurable function $g:W\to X$, the \emph{pushforward kernel} $g_\# \kernel{M}:V\to \Delta(\mathcal{X})$ is the kernel $g_\# \kernel{M}$ such that $(g_\# \kernel{M})_a(B) = \kernel{M}_a(g^{-1}(B))$ for all $a\in V$, $B\in \sigalg{X}$.
\end{definition}

\begin{lemma}[Pushforward kernels are functional kernel products]\label{lem:pushf_funk}
Given a kernel $\kernel{M}:V\to \Delta(\mathcal{W})$ and a measurable function $g:W\to X$, $g_\# \kernel{M} = \kernel{M} \kernel{F}^{g}$.
\end{lemma}

\begin{proof}
for any $a\in V$, $B\in \sigalg{X}$:
\begin{align}
	(\kernel{M}\kernel{F}^g)_a(B) &= \int_W \delta_{g(y)}(B) d\kernel{M}_a(y)\\
								&= \int_W \delta_{y}(g^{-1}(B)) d\kernel{M}_a(y)\\
								&= \int_{g^{-1}(B)} d\kernel{M}_a(y)\\
								&= (g_{\#} \kernel{M})_a (B)
\end{align}
\end{proof}

\subsubsection{Working With String Diagrams}\label{sssec:string_diagram_manipulation}

todo:
\begin{itemize}
\item Infinite copy map
\item De Finetti's representation theorem
\end{itemize}

There are a relatively small number of manipulation rules that are useful for string diagrams. In addition, we will define graphically analogues of the standard notions of \emph{conditional probability}, \emph{conditioning}, and infinite sequences of exchangeable random variables.

\paragraph{Axioms of Symmetric Monoidal Categories}

For the following, we either omit labels or label diagrams with their domain and codomain spaces, as we are discussing identities of kernels rather than identities of components of a condtional probability space. Recalling the unique Markov kernels defined above, the following equivalences, known as the \emph{commutative comonoid axioms}, hold among string diagrams:

\begin{align}
	\begin{tikzpicture}[scale=0.8]
	\path (0,0) node (X) {} 
	++ (0.5,0) coordinate (copy0)
	+ (1.5,0.5) node (X1) {}
	++ (0.5,-0.5) coordinate (copy1)
	+(1,0.5) node (X2) {}
	+(1,-0.5) node (X3) {};
	\draw (X) -- (copy0) to [bend left] (X1) (copy0) to [bend right] (copy1) to [bend left] (X2) (copy1) to [bend right] (X3);
	\end{tikzpicture}
	=
	\begin{tikzpicture}[scale=0.8]
	\path (0,0) node (X) {} 
	++ (0.5,0) coordinate (copy0)
	+ (1.5,-0.5) node (X1) {}
	++ (0.5,0.5) coordinate (copy1)
	+(1,0.5) node (X2) {}
	+(1,-0.5) node (X3) {};
	\draw (X) -- (copy0) to [bend right] (X1) (copy0) to [bend left] (copy1) to [bend left] (X2) (copy1) to [bend right] (X3);
	\end{tikzpicture}
	:=
	\begin{tikzpicture}[scale=0.8]
	\path (0,0) node (X) {} 
	++ (0.5,0) coordinate (copy0)
	+ (1,0.5) node (X1) {}
	+(1,0) node (X2) {}
	+(1,-0.5) node (X3) {};
	\draw (X) -- (copy0) to [bend left] (X1) (copy0) to (X2) (copy0) to [bend right] (X3);
	\end{tikzpicture}\label{eq:ccom1}
\end{align}

\begin{align}
	\begin{tikzpicture}[scale=0.8]
	\path (0,0) node (X) {}
	++(0.5,0) coordinate (copy0)
	+ (1,0.5) node (S) {}
	+(1,-0.5) node (X1) {};
	\draw (X) -- (copy0) to [bend right] (X1);
	\draw[-{Rays [n=8]}] (copy0) to [bend left] (S);
	\end{tikzpicture}
	= 
	\begin{tikzpicture}[scale=0.8]
	\path (0,0) node (X) {}
	++(0.5,0) coordinate (copy0)
	+ (1,-0.5) node (S) {}
	+(1,0.5) node (X1) {};
	\draw (X) -- (copy0) to [bend left] (X1);
	\draw[-{Rays [n=8]}] (copy0) to [bend right] (S);
	\end{tikzpicture}
	=
	\begin{tikzpicture}[scale=0.8]
	\path (0,0) node (X) {}
	++ (1,0) node (X1) {};
	\draw (X) -- (X1);
	\end{tikzpicture}\label{eq:ccom2}
\end{align}

\begin{align}
	\begin{tikzpicture}[scale=0.8]
	\path (0,0) node (X) {$\RV{X}$}
	++(0.5,0) coordinate (copy0)
	+ (1,0.5) node (X2) {$\RV{X}$}
	+(1,-0.5) node (X1) {$\RV{X}$};
	\draw (X) -- (copy0) to [bend right] (X1);
	\draw (copy0) to [bend left] (X2);
	\end{tikzpicture}
=
	\begin{tikzpicture}[scale=0.8]
	\path (0,0) node (X) {}
	++(0.5,0) coordinate (copy0)
	+ (1.2,0.5) node (X2) {}
	+(1.2,-0.5) node (X1) {};
	\draw (X) -- (copy0) .. controls (0.75,0.4) .. (X1.west);
	\draw (copy0) .. controls (0.75,-0.4) .. (X2.west);
	\end{tikzpicture}
\label{eq:ccom3}
\end{align}

The discard map $\stopper{0.2}$ can ``fall through'' any Markov kernel:

\begin{align}
\begin{tikzpicture}
\path (0,0) node (X) {}
++(0.7,0) node[kernel] (A) {$\kernel{A}$}
++(0.7,0) node (S) {};
\draw (X) -- (A);
\draw[-{Rays [n=8]}] (A) -- (S);
\end{tikzpicture}
= 
\begin{tikzpicture}
\path (0,0) node (X) {}
++(0.7,0) node (S) {};
\draw[-{Rays [n=8]}] (X) -- (S);
\end{tikzpicture}\label{eq:termobj1}
\end{align}

Combining \ref{eq:ccom2} and \ref{eq:termobj1} we can derive the following: integrating $\kernel{A}:X\to \Delta(\mathcal{Y})$ with respect to $\mu\in\Delta(\mathcal{X})$ and then discarding the output of $\kernel{A}$ leaves us with $\mu$:

\begin{align}
\begin{tikzpicture}
\path (0,0) node[dist] (mu) {$\mu$}
++ (1,0) coordinate (copy0)
+ (1.4,0.5) node (X) {}
++ (0.7,-0.5) node[kernel] (A) {$\kernel{A}$}
++(0.7,0) node (Y) {};
\draw (mu)--(copy0);
\draw (copy0) to [bend left] (X);
\draw[-{Rays [n=8]}] (copy0) to [bend right] (A) (A) -- (Y);
\end{tikzpicture}
= 
\begin{tikzpicture}
\path (0,0) node[dist] (mu) {$\mu$}
++ (1,0) coordinate (copy0)
+ (1.2,0.5) node (X) {}
++ (0.4,-0.3) coordinate (A)
++(0.1,0) node (Y) {};
\draw (mu)--(copy0);
\draw (copy0) to [bend left] (X);
\draw[-{Rays [n=8]}] (copy0) to [bend right] (A) (A) -- (Y);
\end{tikzpicture}
=
\begin{tikzpicture}
\path (0,0) node[dist] (mu) {$\mu$}
++ (1,0) node (X) {};
\draw (mu)--(X);
\end{tikzpicture}
\end{align}

In elementary notation, this is equivalent to the fact that, for all $B\in \mathcal{X}$, $\int_B \kernel{A}(x;B)d\mu(x) = \mu(B)$.

The following additional properties hold for $\stopper{0.2}$ and $\splitter{0.1}$:

\begin{align}
\begin{tikzpicture}
\path (0,0) node (XY) {$X\times Y$}
++ (1.5,0) node (Z) {};
\draw[-{Rays [n=8]}] (XY) -- (Z);
\end{tikzpicture} &=
\begin{tikzpicture}
\path (0,0) node (X) {$X$} 
++ (1,0) node (X1) {}
(0,-0.3) node (Y) {$Y$}
++ (1,0) node (Y1) {};
\draw[-{Rays [n=8]}] (X) -- (X1);
\draw[-{Rays [n=8]}] (Y) -- (Y1);
\end{tikzpicture}
\end{align}
\begin{align}
\begin{tikzpicture}
\path (0,0) node (XY) {$X\times Y$}
++ (1.2,0) coordinate (copy0)
++(1.2,0.3) node (XY1) {$X \times Y$}
++(0,-0.6) node (XY2) {$X\times Y$};
\draw (XY) -- (copy0) to [bend left] (XY1);
\draw (XY) -- (copy0) to [bend right] (XY2);
\end{tikzpicture} &=
\begin{tikzpicture}
\path (0,0) node (XY) {$X$}
++ (1.,0) coordinate (copy0)
++(1.,0.5) node (XY1) {$X$}
++(0,-1) node (XY2) {$X$}
(0,-0.3) node (F) {$Y$}
++(1.,0) coordinate (copy1)
++(1.,0.5) node (F1) {$Y$}
++(0,-1) node (F2) {$Y$};
\draw (XY) -- (copy0) to [bend left] (XY1);
\draw (copy0) to [bend right] (XY2);
\draw (F) -- (copy1) to [bend left] (F1);
\draw (copy1) to [bend right] (F2);
\end{tikzpicture}
\end{align}

A key fact that \emph{does not} hold in general is

\begin{align}
 \begin{tikzpicture}
\path (0,0) node (E) {}
++ (0.7,0) node[kernel] (A) {$\kernel{A}$}
++(0.7,0) coordinate (copy0)
++(0.5,0.3) node (F1) {}
+(0,-0.6) node (F2) {};
\draw (E) -- (A) -- (copy0) to [bend left] (F1);
\draw (copy0) to [bend right] (F2);
\end{tikzpicture} 
=
\begin{tikzpicture}
\path (0,0) node (E) {}
++(0.5,0) coordinate (copy0)
++(0.7,0.3) node[kernel] (A1) {$\kernel{A}$}
+(0,-0.6) node[kernel] (A2) {$\kernel{A}$}
++(0.75,0) node (F1) {}
+(0,-0.6) node (F2) {};
\draw (E) -- (copy0) to [bend left] (A1) (A1) -- (F1);
\draw (copy0) to [bend right] (A2) (A2) -- (F2);
\end{tikzpicture}
\label{eq:copy_commutes}
\end{align}

In fact, it holds only when $\kernel{A}$ is a \emph{deterministic} kernel.

\begin{definition}[Deterministic Markov kernel]
A \emph{deterministic} Markov kernel $\kernel{A}:E\to \Delta(\mathcal{F})$ is a kernel such that $\kernel{A}_x(B)\in\{0,1\}$ for all $x\in E$, $B\in\mathcal{F}$.
\end{definition}

\begin{theorem}[Copy map commutes for deterministic kernels \citep{fong_causal_2013}]
Equation \ref{eq:copy_commutes} holds iff $\kernel{A}$ is deterministic.
\end{theorem}

\subsubsection{Examples}

Given $\mu\in\Delta(X),\kernel{K}:X\to \Delta(Y)$, $A\in \mathcal{X}$ and $B\in\mathcal{Y}$:

\begin{align}
&A\times B\mapsto \int_A \kernel{K}(x;B)d\mu(x)\\ &\equiv \\\mu 
&\splitter{0.1}(\textbf{Id}_X\otimes \kernel{K})\\ &\equiv \\
&\begin{tikzpicture}
\path (0,0) node[dist] (mu) {$\mu$}
++ (1,0) coordinate (copy0)
+ (1.2,0.5) node (X) {$X$}
++ (0.5,-0.5) node[kernel] (A) {$\kernel{K}$}
++(0.7,0) node (Y) {$Y$};
\draw (mu)--(copy0);
\draw (copy0) to [bend left] (X);
\draw (copy0) to [bend right] (A) (A) -- (Y);
\end{tikzpicture}\label{eq:joint_measure}
\end{align}

\citet{cho_disintegration_2019} calls this operation ``integrating $\kernel{K}$ with respect to $\mu$''.

Given $\nu\in \Delta(\sigalg{X}\otimes\sigalg{Y})$, define the marginal $\nu^{\RV{Y}}\in \Delta(\mathcal{Y}):B\mapsto \mu(X\times B)$ for $B\in \mathcal{Y}$. Say that $\nu^{\RV{Y}}$ is obtained by marginalising over ``$X$'' (a notion that can be made more precise by assigning names to wires). Then

\begin{align}
	\nu(\stopper{0.25}\otimes \mathrm{Id}^Y) &= \begin{tikzpicture}
		\path (0,0) node[dist] (nu) {$\nu$}
		++ (0.7,-0.15) node (X) {$Y$}
		+(0,0.3) node (Y) {};
		\draw ($(nu.east)+(0,-0.15)$) -- (X);
		\draw[-{Rays[n=8]}] ($(nu.east)+(0,0.15)$) -- (Y);
	\end{tikzpicture}\\
	\nu(\stopper{0.25}\otimes \mathrm{Id}^Y)(B) &:= \nu(\stopper{0.25}\otimes \mathrm{Id}^Y)(B\times\{*\})\\
												&=\int_{X\times Y} \mathrm{Id}^Y_y(B) \stopper{0.2}_x(\{*\}) d\nu(x,y)\\
	&= \int_{X\times Y} \delta_y(B) \delta_*(\{*\}) d\nu(x,y)\\
	&= \int_{X\times B} d\nu(x,y)\\
	&= \nu(X\times B)\\
	&= \nu^{\RV{Y}}(B)
\end{align}

Thus the action of the erasing wire ``$X$'' is equivalent to marginalising over ``$X$''.

Consider the result of marginalising \ref{eq:joint_measure} over ``$X$'':
\begin{align}
  \nu^Y (B) &= \begin{tikzpicture}
\path (0,0) node[dist] (mu) {$\mu$}
++ (1,0) coordinate (copy0)
+ (1.2,0.5) node (X) {}
++ (0.5,-0.5) node[kernel] (A) {$\kernel{A}$}
++(0.7,0) node (Y) {$Y$};
\draw (mu)--(copy0);
\draw[-{Rays [n=8]}] (copy0) to [bend left] (X);
\draw (copy0) to [bend right] (A) (A) -- (Y);
\end{tikzpicture} \\
 &= \begin{tikzpicture}
\path (0,0) node[dist] (mu) {$\mu$} ++ (1,0) node[kernel] (A) {$\kernel{A}$} ++ (0.7,0) node (Y) {$Y$}; \draw (mu) -- (A) -- (Y);
\end{tikzpicture} \label{eq:marginalisation_graph}
\end{align}

\subsection{Random Variables}\label{ssec:random_variables}

The summary of this section is:
\begin{itemize}
\item Random variables are usually defined as measurable functions on a \emph{probability space}
\item It's possible to define them as measurable functions on a \emph{Markov kernel space} instead
\item It is useful to label wires with random variable names instead of names of spaces
\end{itemize}

Probability theory is primarily concerned with the behaviour of \emph{random variables}. This behaviour can be analysed via a collection of probability measures and Markov kernels representing joint, marginal and conditional distributions of random variables of interest. In the framework developed by Kolmogorov, this collection of joint, marginal and conditional distributions is modeled by a single underlying \emph{probability space}, and random variables by measurable functions on the probability space. 

We use the same approach here, with a couple of additions. We are interested in variables whose outcomes depend both on random processes and decisions. Suppose that given a particular distribution over decision variables, a probability distribution over the decision variables and random variables is obtained. Such a model is described by a Markov kernel rather than a probability distribution. We therefore investigate \emph{Markov kernel spaces}.

In the graphical notation that we are using, random variables can be thought of as a means of assigning unambiguous names to each wire in a set of diagrams. In order to do this, it is necessary to suppose that all diagrams in the set describe properties of an \emph{ambient Markov kernel} or \emph{ambient probability measure}. Consider the following example with the ambient probability measure $\mu\in\Delta(\mathcal{X}\otimes\mathcal{X})$. Suppose we have a Markov kernel $\kernel{K}:X\to \Delta(\mathcal{X})$ such that the following holds:

\begin{align}
\begin{tikzpicture}
\path (0,0) node[dist] (m) {$\mu$}
++ (0.7,0.15) node (E) {$X$}
++ (0,-0.3) node (F) {$X$};
\draw ($(m.east) + (0,0.15)$) -- (E);
\draw ($(m.east) + (0,-0.15)$) -- (F);
\end{tikzpicture} = \begin{tikzpicture}
\path (0,0) node[dist] (m) {$\mu$}
++ (0.7,0.15) coordinate (copy0)
+(0,-0.3) node (Fs) {}
++ (1.2,0) node (E) {$X$}
++(-0.7,-0.3) node[kernel] (K) {$\kernel{K}$}
++(0.7,0) node (F) {$X$};
\draw ($(m.east) + (0,0.15)$) -- (E);
\draw (copy0) to [bend right] (K) (K) -- (F);
\draw[-{Rays [n=8]}] ($(m.east) + (0,-0.15)$) -- (Fs);
\end{tikzpicture}\label{eq:disint_example}
\end{align}

Suppose that we also assign the names $\RV{X}_1$ to the upper output wire and $\RV{X}_2$ to the lower output wire in the diagram of $\mu$:

\begin{align}
\begin{tikzpicture}
\path (0,0) node[dist] (m) {$\mu$}
++ (0.7,0.15) node (E) {$\RV{X}_1$}
++ (0,-0.3) node (F) {$\RV{X}_2$};
\draw ($(m.east) + (0,0.15)$) -- (E);
\draw ($(m.east) + (0,-0.15)$) -- (F);
\end{tikzpicture}
\end{align}

Then it seems sensible to call $\kernel{K}$ ``the probability of $\RV{X}_2$ given $\RV{X}_1$''. We will make this precise, and it will match the usual notion of the probability of one variable given another (see \citet{cinlar_probability_2011} for a definition of this usual notion). 

\begin{definition}[Probability space, Markov kernel space]\label{def:kernel_space}
A \emph{Markov kernel space} $(\kernel{K},(D,\mathcal{D}),(\Omega,\mathcal{F}))$ is a Markov kernel $\kernel{K}:D\to \Delta(\mathcal{D}\otimes\mathcal{F})$, called the \emph{ambient kernel}, along with the sample space $(\Omega,\mathcal{F})$ and the domain $(D,\mathcal{D})$. We suppose that $\kernel{K}$ is such that there exists a \emph{fundamental kernel} $\kernel{K}_0$ satisfying

\begin{align}
\prob{K} := \begin{tikzpicture}
\path (0,0) node (O) {}
++(0.5,0) coordinate (copy0)
++ (0.5,0) node[kernel] (m) {$\kernel{K}_0$}
++ (0.7,0.) node (E) {}
++(0,-0.45) node (G) {};
\draw (O) -- (m) -- (E);
\draw (copy0) to [bend right] (G);
\end{tikzpicture}
\end{align}

For brevity, we will omit the $\sigma$-algebras in further definitions of Markov kernel spaces: $(\kernel{K},D,\Omega)$.

A \emph{probability space} $(\prob{P},\Omega,\mathcal{F})$ is a probability measure $\prob{P}:\Delta(\Omega)$, which we call the \emph{ambient measure}, along with the \emph{sample space} $\Omega$ and the \emph{events} $\mathcal{F}$. A probability space is equivalent to a Markov kernel space with domain $D=\{*\}$ - note that $\Omega\times \{*\}\cong \Omega$.
\end{definition}

\begin{definition}[Random variable]\label{def:random_variable}
Given a Markov kernel space $(\kernel{K},D,\Omega)$, a random variable $\RV{X}$ is a measurable function $\Omega\times D\to E$ for arbitrary measurable $E$.
\end{definition}

\begin{definition}[Domain variable]\label{def:domain_variable}
Given a Markov kernel space $(\kernel{K},D,\Omega)$, the \emph{domain variable} $\RV{D}:\Omega\times D\to D$ is the distinguished random variable $\RV{D}:(x,d)\mapsto d$.
\end{definition}

Unlike random variables on probability spaces, random variables on Markov kernel spaces do not generally have unique marginal distributions. An analogous operation of \emph{marginalisation} can be defined, but the result is generally a Markov kernel. We will define marginalisation via coupled tensor products.

\begin{definition}[Coupled tensor product $\utimes$]\label{def:ctensor}
Given two Markov kernels $\kernel{M}$ and $\kernel{N}$ or functions $f$ and $g$ with shared domain $E$, let $\kernel{M}\utimes\kernel{N}:=\splitter{0.1}(\kernel{M}\otimes\kernel{N})$ and $f\utimes g:=\splitter{0.1}(f\otimes g)$ where these expressions are interpreted using standard product notation. Graphically:

\begin{align}
\kernel{M}\utimes\kernel{N}&:=\begin{tikzpicture}
\path (0,0) node (E) {$E$}
++(0.5,0) coordinate (copy0)
+ (0.5,0.3) node[kernel] (M) {$\kernel{M}$}
+(1.2,0.3) node (X) {$\RV{X}$}
+ (0.5,-0.3) node[kernel] (N) {$\kernel{N}$}
+(1.2,-0.3) node (Y) {$\RV{Y}$};
\draw (E) -- (copy0) to [bend left] (M) (copy0) to [bend right] (N);
\draw (M) -- (X) (N) -- (Y);
\end{tikzpicture}\\
f\utimes g&:= \begin{tikzpicture}[scale=1.2]\path (0,0) node (E) {$E$}
++(0.5,0) coordinate (copy0)
+ (0.5,0.3) node[expectation] (M) {$f$}
+ (0.5,-0.3) node[expectation] (N) {$g$};
\draw (E) -- (copy0) to [bend left] (M) (copy0) to [bend right] (N);
\end{tikzpicture}
\end{align}
The operation denoted by $\utimes$ is associative (Lemma \ref{lem:utimes_assoc}), so we can without ambiguity write $f\utimes g\utimes h=(f\utimes g)\utimes h = f\utimes(g\utimes h)$ for finite groups of functions or Markov kernels sharing a domain. 

The notation $\utimes_{i\in [N]} f_i$ is taken to mean $f_1\utimes f_2\utimes ...\utimes f_N$.
\end{definition}

\begin{lemma}[$\utimes$ is associative]\label{lem:utimes_assoc}
For Markov kernels $\kernel{L}:E\to \delta(\mathcal{F})$, $\kernel{M}:E\to \delta(\mathcal{G})$ and $\kernel{N}:E\to \delta(\mathcal{H})$, $(\kernel{L}\utimes\kernel{M})\utimes\kernel{N}=\kernel{L}\utimes(\kernel{M}\utimes\kernel{N})$.
\end{lemma}

\begin{proof}

\begin{align}
	\kernel{L}\utimes(\kernel{M}\utimes\kernel{N}) &= 
	\begin{tikzpicture}[scale=0.8]
	\path (0,0) node (X) {$E$} 
	++ (0.8,0) coordinate (copy0)
	+ (1.5,0.5) node[kernel] (X1) {$\kernel{L}$} + (2.5,0.5) node (F) {$F$}
	++ (0.5,-0.5) coordinate (copy1)
	+(1,0.3) node[kernel] (X2) {$\kernel{M}$} + (2,0.3) node (G) {$G$}
	+(1,-0.5) node[kernel] (X3) {$\kernel{N}$} + (2,-0.5) node (H) {$H$};
	\draw (X) -- (copy0) to [bend left] (X1) (copy0) to [bend right] (copy1) to [bend left] (X2) (copy1) to [bend right] (X3);
	\draw (X1) -- (F) (X2) -- (G) (X3) -- (H);
	\end{tikzpicture}\\
	&=
	\begin{tikzpicture}[scale=0.8]
	\path (0,0) node (X) {$E$} 
	++ (0.8,0) coordinate (copy0)
	+ (1.5,0.7) node[kernel] (X1) {$\kernel{L}$} + (2.5,0.7) node (F) {$F$}
	+ (0.5,0.3) coordinate (copy1)
	++ (0.5,-0.5) coordinate (next)
	+(1,0.5) node[kernel] (X2) {$\kernel{M}$} + (2,0.5) node (G) {$G$}
	+(1,-0.5) node[kernel] (X3) {$\kernel{N}$} + (2,-0.5) node (H) {$H$};
	\draw (X) -- (copy0) to [bend left] (copy1) (copy0) to [bend right] (X3);
	\draw (copy1) to [bend left] (X1) (copy1) to [bend right] (X2);
	\draw (X1) -- (F) (X2) -- (G) (X3) -- (H);
	\end{tikzpicture}\\
	&= (\kernel{L}\utimes\kernel{M})\utimes\kernel{N}
\end{align}
This follows directly from Equation \ref{eq:ccom1}.
\end{proof}

\begin{definition}[Marginal distribution, marginal kernel]\label{def:marginal_distribution}
Given a probability space $(\prob{P},\Omega,\mathcal{F})$ and the random variable $\RV{X}:\Omega\to G$ the \emph{marginal distribution} of $\RV{X}$ is the probability measure $\prob{P}^{\RV{X}}:= \prob{P}\kernel{F}^{\RV{X}}$.

See Lemma \ref{lem:pushf_funk} for the proof that this matches the usual definition of marginal distribution.

Given a Markov kernel space $(\kernel{K},\Omega,\mathcal{F},D,\mathcal{D})$ and the random variable $\RV{X}:\Omega\to G$, the \emph{marginal kernel} is $\kernel{K}^{\RV{X}|\RV{D}}:=\kernel{K}\kernel{F}^{\RV{X}}$.
\end{definition}

\begin{definition}[Joint distribution, joint kernel]\label{def:joint_distribution}
Given a probability space $(\prob{P},\Omega,\mathcal{F})$ and the random variables $\RV{X}:\Omega\to G$ and $\RV{Y}:\Omega\to H$, the \emph{joint distribution} of $\RV{X}$ and $\RV{Y}$, $\prob{P}^{\RV{X}\RV{Y}}\in \Delta(\mathcal{G}\otimes\mathcal{H})$, is the marginal distribution of $\RV{X}\utimes\RV{Y}$. That is, $\prob{P}^{\RV{X}\RV{Y}}:=\prob{P} \kernel{F}^{\RV{X}\utimes\RV{Y}}$

This is identical to the definition in \citet{cinlar_probability_2011} if we note that the random variable $(\RV{X},\RV{Y}):\omega\mapsto (\RV{X}(\omega),\RV{Y}(\omega))$ (\c{C}inlar's definition) is precisely the same thing as $\RV{X}\utimes\RV{Y}$.

Analogously, the joint kernel $\kernel{K}^{\RV{X}\RV{Y}|\RV{D}}$ is the product $\kernel{K}\kernel{F}^{\RV{X}\utimes\RV{Y}}$.
\end{definition}

Joint distributions and kernels have a nice visual representation, as a result of Lemma \ref{lem:jdist_cprod} which follows.

\begin{lemma}[Product marginalisation interchange]\label{lem:jdist_cprod}
Given two functions, the kernel associated with their coupled product is equal to the coupled product of the kernels associated with each function.

Given $\RV{X}:\Omega\to G$ and $\RV{Y}:\Omega\to H$, $\kernel{F}^{\RV{X}\utimes\RV{Y}}=\kernel{F}^\RV{X}\utimes\kernel{F}^\RV{Y}$
\end{lemma}

\begin{proof}
For $a\in \Omega$, $B\in \mathcal{G}$, $C\in \mathcal{H}$,
\begin{align}
\kernel{F}^{\RV{X}\utimes\RV{Y}} (a;B\times C) &= \delta_{\RV{X}(a),\RV{Y}(a)}(B\times C)\\
									   &= \delta_{\RV{X}(a)}(B)\delta_{\RV{Y}(a)}(C)\\
									   &= (\delta_{\RV{X}(a)}\otimes\delta_{\RV{Y}(a)})(B\times C)\\
									   &= \kernel{F}^{\RV{X}}\utimes\kernel{F}^{\RV{Y}}
\end{align}
Equality follows from the monotone class theorem.
\end{proof}

\begin{corollary}\label{corr:rewrite_joint_dist}
Given a Markov kernel space $(\kernel{K}, \Omega, D)$ and random variables $\RV{X}:\Omega\times D\to X$, $\RV{Y}:\Omega\times D\to Y$, the following holds:

\begin{align}
\begin{tikzpicture}
\path (0,0) node (O) {$D$}
++(1,0) node[kernel] (K) {$\kernel{K}^{\RV{X}\RV{Y}|\RV{D}}$}
++ (1,0.15) node (X) {$X$}
+(0,-0.3) node (Y) {$Y$};
\draw (O) -- (K);
\draw ($(K.east) + (0,0.15)$) -- (X);
\draw ($(K.east) + (0,-0.15)$) -- (Y);
\end{tikzpicture}=
\begin{tikzpicture}
\path (0,0) node (O) {$D$}
++ (0.7, 0) node[kernel] (K) {$\kernel{K}$}
++ (0.6,0) coordinate (copy0)
++ (0.4,0.25) node[kernel] (X) {$\kernel{F}^{\RV{X}}$}
+(0,-0.5) node[kernel] (Y) {$\kernel{F}^{\RV{Y}}$}
++(0.7,0) node (Xo) {$X$}
+(0,-0.5) node (Yo) {$Y$};
\draw (O) -- (K) -- (copy0);
\draw (copy0) to [bend left] (X) (X) -- (Xo);
\draw (copy0) to [bend right] (Y) (Y) -- (Yo);
\end{tikzpicture}
\end{align}
\end{corollary}

We will now define wire labels for ``output'' wires.

\begin{definition}[Wire labels - joint kernels]\label{def:wl_jprob}
Suppose we have a Markov kernel space $(\kernel{K},D,\Omega)$, random variables $\RV{X}:\Omega\times D\to X$, $\RV{Y}:\Omega\times D\to Y$ and a Markov kernel $\kernel{L}:D\to \Delta(\mathcal{X}\times\mathcal{Y})$. The following \emph{output labelling} of $\mathbf{L}$:

\begin{align}
\begin{tikzpicture}
\path (0,0) node (A) {$D$}
++ (0.7,0) node[kernel] (m) {$\kernel{L}$}
++ (0.7,0.15) node (E) {\color{blue}$\RV{X}$}
++ (0,-0.3) node (F) {\color{blue}$\RV{Y}$};
\draw (A) -- (m);
\draw ($(m.east) + (0,0.15)$) -- (E);
\draw ($(m.east) + (0,-0.15)$) -- (F);
\end{tikzpicture}
\end{align}

is \emph{valid} iff

\begin{align}
\kernel{L} = \kernel{K}_{\RV{X}\RV{Y}|\RV{D}}\label{eq:labels_express_joint}
\end{align}

and

\begin{align}
\begin{tikzpicture}
\path (0,0) node (A) {$D$}
++ (1,0) node[kernel] (m) {$\kernel{L}$}
++ (1,0.15) node (E) {\color{blue}$\RV{X}$}
++ (0,-0.3) node (F) {};
\draw (A) -- (m) ($(m.east) + (0,0.15)$) -- (E);
\draw[-{Rays [n=8]}] ($(m.east) + (0,-0.15)$) -- (F);
\end{tikzpicture} = \kernel{K}^{\RV{X}|\RV{D}}\label{eq:labels_express_marginal_upper}
\end{align}

and

\begin{align}
\begin{tikzpicture}
\path (0,0) node (A) {$D$}
++ (1,0) node[kernel] (m) {$\kernel{L}$}
++ (1,0.15) node (E) {}
++ (0,-0.3) node (F) {\color{blue}$\RV{Y}$};
\draw (A) -- (m);
\draw[-{Rays [n=8]}] ($(m.east) + (0,0.15)$) -- (E);
\draw ($(m.east) + (0,-0.15)$) -- (F);
\end{tikzpicture} = \kernel{K}^{\RV{Y}|\RV{D}}\label{eq:labels_express_marginal_lower}
\end{align}

The second and third conditions are nontrivial: suppose $\RV{X}$ takes values in some product space $Range(\RV{X}) = W\times Z$, and $\RV{Y}$ takes values in $Y$. Then we could have $\kernel{L}=\kernel{K}^{\RV{X}\RV{Y}|\RV{D}}$ and draw the diagram

\begin{align}
\begin{tikzpicture}
\path (0,0) node (A) {$D$}
++ (0.7,0) node[kernel] (m) {$\kernel{L}$}
++ (1,0.15) node (E) {$W$}
++ (0.3,-0.3) node (F) {$Z\times Y$};
\draw (A) -- (m);
\draw ($(m.east) + (0,0.15)$) -- (E);
\draw ($(m.east) + (0,-0.15)$) -- (F);
\end{tikzpicture}\label{eq:cannot_marginalise}
\end{align}

For \emph{this} diagram, properties \ref{eq:labels_express_marginal_upper} and \ref{eq:labels_express_marginal_lower} do not hold, even though \ref{eq:labels_express_joint} does.

\end{definition}

\begin{lemma}[Output label assignments exist]
Given Markov kernel space $(\kernel{K},D,\Omega)$, random variables $\RV{X}:\Omega\times D\to X$ and $\RV{Y}:\Omega\times D\to Y$ then there exists a diagram of $\kernel{L}:=\kernel{K}^{\RV{X}\RV{Y}|\RV{D}}$ with a valid output labelling assigning ${\color{blue}\RV{X}}$ and ${\color{blue}\RV{Y}}$ to the output wires.
\end{lemma}

\begin{proof}
By definition, $\kernel{L}$ has signature $D\to \Delta(\mathcal{X}\otimes\mathcal{Y})$. Thus, by the rule that tensor product spaces can be represented by parallel wires, we can draw

\begin{align}
\begin{tikzpicture}
\path (0,0) node (A) {$D$}
++ (0.7,0) node[kernel] (m) {$\kernel{L}$}
++ (0.7,0.15) node (E) {$X$}
++ (0,-0.3) node (F) {$Y$};
\draw (A) -- (m);
\draw ($(m.east) + (0,0.15)$) -- (E);
\draw ($(m.east) + (0,-0.15)$) -- (F);
\end{tikzpicture}
\end{align}

By Corollary \ref{corr:rewrite_joint_dist}, we have

\begin{align}
\begin{tikzpicture}
\path (0,0) node (A) {$D$}
++ (0.7,0) node[kernel] (m) {$\kernel{L}$}
++ (0.7,0.15) node (E) {$X$}
++ (0,-0.3) node (F) {$Y$};
\draw (A) -- (m);
\draw ($(m.east) + (0,0.15)$) -- (E);
\draw ($(m.east) + (0,-0.15)$) -- (F);
\end{tikzpicture} = \begin{tikzpicture}
\path (0,0) node (O) {$D$}
++ (0.7, 0) node[kernel] (K) {$\kernel{K}$}
++ (0.6,0) coordinate (copy0)
++ (0.4,0.25) node[kernel] (X) {$\kernel{F}^{\RV{X}}$}
+(0,-0.5) node[kernel] (Y) {$\kernel{F}^{\RV{Y}}$}
++(0.7,0) node (Xo) {$X$}
+(0,-0.5) node (Yo) {$Y$};
\draw (O) -- (K) -- (copy0);
\draw (copy0) to [bend left] (X) (X) -- (Xo);
\draw (copy0) to [bend right] (Y) (Y) -- (Yo);
\end{tikzpicture}
\end{align}

Therefore 

\begin{align}
\begin{tikzpicture}
\path (0,0) node (O) {$D$}
++ (0.7, 0) node[kernel] (K) {$\kernel{K}$}
++ (0.6,0) coordinate (copy0)
++ (0.4,0.25) node[kernel] (X) {$\kernel{F}^{\RV{X}}$}
+(0,-0.5) node[kernel] (Y) {$\kernel{F}^{\RV{Y}}$}
++(0.7,0) node (Xo) {$X$}
+(0,-0.5) node (Yo) {};
\draw (O) -- (K) -- (copy0);
\draw (copy0) to [bend left] (X) (X) -- (Xo);
\draw[-{Rays[n=8]}] (copy0) to [bend right] (Y) (Y) -- (Yo);
\end{tikzpicture} &= \kernel{K}\kernel{F}^{\RV{X}}\\
				 &= \kernel{K}^{\RV{X}|\RV{D}}
\end{align}

\begin{align}
\begin{tikzpicture}
\path (0,0) node (O) {$D$}
++ (0.7, 0) node[kernel] (K) {$\kernel{K}$}
++ (0.6,0) coordinate (copy0)
++ (0.4,0.25) node[kernel] (X) {$\kernel{F}^{\RV{X}}$}
+(0,-0.5) node[kernel] (Y) {$\kernel{F}^{\RV{Y}}$}
++(0.7,0) node (Xo) {}
+(0,-0.5) node (Yo) {$Y$};
\draw (O) -- (K) -- (copy0);
\draw[-{Rays[n=8]}] (copy0) to [bend left] (X) (X) -- (Xo);
\draw (copy0) to [bend right] (Y) (Y) -- (Yo);
\end{tikzpicture} &= \kernel{K}\kernel{F}^{\RV{Y}}\\
				 &= \kernel{K}^{\RV{Y}|\RV{D}}
\end{align}
\end{proof}

In all further work, wire labels will be used without special colouring.

\begin{definition}[Disintegration]\label{def:disintegration}
Given a probability space $(\prob{P},\Omega,\mathcal{F})$, and random variables $\RV{X}$ and $\RV{Y}$, we say that $\kernel{M}:E\to \Delta(\mathcal{F})$ is a \emph{$\RV{Y}$ given $\RV{X}$ disintegration} of $\prob{P}$ iff
\begin{align}
\begin{tikzpicture}
\path (0,0) node[dist] (m) {$\prob{P}^{\RV{X}\RV{Y}}$}
++ (1,0.15) node (E) {$\RV{X}$}
++ (0,-0.3) node (F) {$\RV{Y}$};
\draw ($(m.east) + (0,0.15)$) -- (E);
\draw ($(m.east) + (0,-0.15)$) -- (F);
\end{tikzpicture} = \begin{tikzpicture}
\path (0,0) node[dist] (m) {$\prob{P}^{\RV{X}}$}
++ (0.7,0.15) coordinate (copy0)
+(0.2,-0.3) node (T) {}
++ (1.2,0) node (E) {$\RV{X}$}
++(-0.7,-0.3) node[kernel] (K) {$\kernel{M}$}
++(0.7,0) node (F) {$\RV{Y}$};
\draw ($(m.east) + (0,0.15)$) -- (E);
\draw (copy0) to [bend right] (K) (K) -- (F);
\draw[-{Rays [n=8]}] ($(m.east) + (0,-0.15)$) -- (T);
\end{tikzpicture}\label{eq:ordinary_disint}
\end{align}
$\kernel{M}$ is a version of $\prob{P}^{\RV{Y}|\RV{X}}$, ``the probability of $\RV{Y}$ given $\RV{X}$''. Let $\prob{P}^{\{\RV{Y}|\RV{X}\}}$ be the set of all kernels that satisfy \ref{eq:ordinary_disint} and $\prob{P}^{\RV{Y}|\RV{X}}$ an arbitrary member of $\prob{P}^{\RV{Y}|\RV{X}}$.

Given a Markov kernel space $(\kernel{K},D,\Omega)$ and random variables $\RV{X}:\Omega\times D\to X$, $\RV{Y}:\Omega\times D\to Y$, $\kernel{M}:D\times E\to \Delta(\mathcal{F})$ is a \emph{$\RV{Y}$ given $\RV{DX}$ disintegration} of $\kernel{K}^{\RV{YX}|\RV{D}}$ iff

\begin{align}
\begin{tikzpicture}
\path (0,0) node (O) {}
++ (1,0) node[kernel] (m) {$\kernel{K}^{\RV{YX}|\RV{D}}$}
++ (1,0.15) node (E) {$\RV{X}$}
++ (0,-0.3) node (F) {$\RV{Y}$};
\draw (O) -- (m) ($(m.east) + (0,0.15)$) -- (E);
\draw ($(m.east) + (0,-0.15)$) -- (F);
\end{tikzpicture} = \begin{tikzpicture}
\path (0,0) node (O) {}
++ (0.3,0) coordinate (copy1)
++ (1,0) node[kernel] (m) {$\kernel{K}^{\RV{YX}|\RV{D}}$}
++ (1,0.15) coordinate (copy0)
+(0.2,-0.3) node (T) {}
++ (1.2,0) node (E) {$\RV{X}$}
++(-0.7,-0.3) node[kernel] (K) {$\kernel{M}$}
++(0.7,0) node (F) {$\RV{Y}$};
\draw (O) -- (m) ($(m.east) + (0,0.15)$) -- (E);
\draw (copy0) to [bend right] (K) (K) -- (F);
\draw (copy1) to [out=290,in=180] ($(K.west) + (0,-0.15)$);
\draw[-{Rays [n=8]}] ($(m.east) + (0,-0.15)$) -- (T);
\end{tikzpicture}\label{eq:def_k_disint}
\end{align}

Write $\kernel{K}^{\{\RV{Y}|\RV{XD}\}}$ for the set of kernels satisfying \ref{eq:def_k_disint} and $\kernel{K}^{\RV{Y}|\RV{XD}}$ for an arbitrary member of $\kernel{K}^{\{\RV{Y}|\RV{XD}\}}$.
\end{definition}

\begin{definition}[Wire labels -- input]\label{def:wl_disint}

An input wire is \emph{connected} to an output wire if it is possible to trace a path from the start of the input wire to the end of the output wire without passing through any boxes, erase maps or right facing triangles.

If an input wire is connected to an output wire and that output wire has a valid label $\RV{X}$, then it is valid to label the input wire with $\RV{X}$.

For example, if the following are valid output labels with respect to $(\prob{P},\Omega)$:

\begin{align}
\begin{tikzpicture}
\path (0,0) node (A) {}
++ (0.7,0) coordinate (copy0)
++ (0.7,0) node[kernel] (m) {$\kernel{L}$}
++ (0.7,0) node (E) {\color{blue}$\RV{X}$}
++ (0,-0.3) node (F) {\color{blue}$\RV{Y}$};
\draw (A) -- (m) -- (E);
\draw (copy0) to [out=-60,in=180] (F);
\end{tikzpicture}\label{dia:kernel_l}
\end{align}

i.e. if $\kernel{L}\in \prob{P}^{\{\RV{X}\RV{Y}|\RV{Y}\}}$, then the following is a valid input label:


\begin{align}
\begin{tikzpicture}
\path (0,0) node (A) {\color{blue}$\RV{Y}$}
++ (0.7,0) coordinate (copy0)
++ (0.7,0) node[kernel] (m) {$\kernel{L}$}
++ (0.7,0) node (E) {\color{blue}$\RV{X}$}
++ (0,-0.3) node (F) {\color{blue}$\RV{Y}$};
\draw (A) -- (m) -- (E);
\draw (copy0) to [out=-60,in=180] (F);
\end{tikzpicture}
\end{align}

An input wire in a diagram for $\kernel{M}$ may be labeled $\RV{X}$ \emph{if and only if} copy and identity maps can be inserted to yield a diagram in which the input wire labeled $\RV{X}$ is connected to an output wire with valid label $\RV{X}$.

So, if $\kernel{M}\in \prob{P}^{\{\RV{X}|\RV{Y}\}}$, then it is straightforward to show that

\begin{align}
\begin{tikzpicture}
\path (0,0) node (A) {}
++ (0.7,0) coordinate (copy0)
++ (0.7,0) node[kernel] (m) {$\kernel{M}$}
++ (0.7,0) node (E) {\color{blue}$\RV{X}$}
++ (0,-0.3) node (F) {\color{blue}$\RV{Y}$};
\draw (A) -- (m) -- (E);
\draw (copy0) to [out=-60,in=180] (F);
\end{tikzpicture} \in \prob{P}^{\{\RV{X}\RV{Y}|\RV{Y}\}} \label{eq:const_from_m}
\end{align}

and hence the output labels are valid. Diagram \ref{eq:const_from_m} is constructed by taking the product of the copy map with $\kernel{M}\otimes\textbf{Id}$. Thus it is valid to label $\kernel{M}$ with

\begin{align}
\begin{tikzpicture}
\path (0,0) node (A) {\color{blue}$\RV{Y}$}
++ (0.7,0) node[kernel] (m) {$\kernel{M}$}
++ (0.7,0) node (E) {\color{blue}$\RV{X}$};
\draw (A) -- (m) -- (E);
\end{tikzpicture}
\end{align}
\end{definition}

\begin{lemma}[Labeling of disintegrations]
Given a kernel space $(\kernel{K},D,\Omega)$, random variables $\RV{X}$ and $\RV{Y}$, domain variable $\RV{D}$ and disintegration $\kernel{L}\in \kernel{K}^{\{\RV{Y}|\RV{X}\RV{D}\}}$, there is a diagram of $\kernel{L}$ with valid input labels ${\color{blue} \RV{X}}$ and ${\color{blue} \RV{D}}$ and valid output label ${\color{blue} \RV{Y}}$.
\end{lemma}

\begin{proof}
Note that for any variable $\RV{W}:\Omega\times D\to W$ and the domain variable $\RV{D}:\Omega\times D\to D$ we have by definition of $\kernel{K}$:
\begin{align}
\begin{tikzpicture}
\path (0,0) node (O) {}
++ (1,0) node[kernel] (m) {$\kernel{K}^{\RV{WD}|\RV{D}}$}
++ (1,0.15) node (E) {$\RV{W}$}
++ (0,-0.3) node (F) {$\RV{D}$};
\draw (O) -- (m) ($(m.east) + (0,0.15)$) -- (E);
\draw ($(m.east) + (0,-0.15)$) -- (F);
\end{tikzpicture} &= \begin{tikzpicture}
\path (0,0) node (O) {}
++ (0.3,0) coordinate (copy1)
++ (1,0) node[kernel] (m) {$\kernel{K}_{0}$}
++ (0.7,0) coordinate (copy0)
+ (0,-0.5) coordinate (copy2)
++ (0.7,0.3) node[kernel] (Fx) {$\kernel{F}^\RV{W}$}
++(0,-0.8) node[kernel] (Fd) {$\kernel{F}^{\RV{D}}$}
++(0.7,0) node (D) {$\RV{D}$}
++ (0,0.8) node (X) {$\RV{W}$};
\draw (O) -- (m) -- (copy0);
\draw (copy0) to [bend left] ($(Fx.west)+(0,0.1)$) (copy0) to [bend right] ($(Fd.west)+(0,0.1)$);
\draw (copy1) to [out=290,in=180] (copy2);
\draw (copy2) to [bend left] ($(Fx.west)+(0,-0.1)$) (copy2) to [bend right] ($(Fd.west)+(0,-0.1)$);
\draw (Fx) -- (X) (Fd) -- (D);
\end{tikzpicture}\\
&= \begin{tikzpicture}
\path (0,0) node (O) {}
++ (0.3,0) coordinate (copy1)
++ (1,0) node[kernel] (m) {$\kernel{K}_{0}$}
++ (0.7,0) coordinate (copy0)
+ (0,-0.5) coordinate (copy2)
++ (0.7,0.3) node[kernel] (Fx) {$\kernel{F}^\RV{W}$}
++(0,-0.8) coordinate (Fd)
++(0.7,0) node (D) {$\RV{D}$}
++ (0,0.8) node (X) {$\RV{W}$};
\draw (O) -- (m) -- (copy0);
\draw (copy0) to [bend left] ($(Fx.west)+(0,0.1)$);
\draw (copy1) to [out=290,in=180] (copy2) -- (D);
\draw (copy2) to [bend left] ($(Fx.west)+(0,-0.1)$);
\draw (Fx) -- (X) (Fd) -- (D);
\end{tikzpicture}\\
&= \begin{tikzpicture}
\path (0,0) node (O) {}
++ (0.3,0) coordinate (copy1)
+ (0.2,0) coordinate (copy3)
++ (1,0) node[kernel] (m) {$\kernel{K}_{0}$}
++ (0.7,0) coordinate (copy0)
+ (0,-0.5) coordinate (copy2)
++ (0.7,-0.1) node[kernel] (Fx) {$\kernel{F}^\RV{W}$}
++(0,-0.5) coordinate (Fd)
++(0.7,0) node (D) {$\RV{D}$}
++ (0,0.5) node (X) {$\RV{W}$};
\draw (O) -- (m);
\draw (m) to [out=0,in=180]  ($(Fx.west)+(0,0.1)$);
\draw (copy1) to [out=290,in=180] (D);
\draw (copy3) to [out=290,in=180] ($(Fx.west)+(0,-0.1)$);
\draw (Fx) -- (X);
\end{tikzpicture}\\
&= \begin{tikzpicture}
\path (0,0) node (O) {}
++ (0.3,0) coordinate (copy1)
++ (1,0) node[kernel] (m) {$\kernel{K}$}
++ (0.7,0) coordinate (copy0)
+ (0,-0.5) coordinate (copy2)
++ (0.7,-0.) node[kernel] (Fx) {$\kernel{F}^\RV{W}$}
++(0,-0.5) coordinate (Fd)
++(0.7,0) node (D) {$\RV{D}$}
++ (0,0.5) node (X) {$\RV{W}$};
\draw (O) -- (m);
\draw (m) to [out=0,in=180]  ($(Fx.west)+(0,0.0)$);
\draw (copy1) to [out=290,in=180] (D);
\draw (Fx) -- (X);
\end{tikzpicture}\\
&=\begin{tikzpicture}
\path (0,0) node (O) {}
++ (0.3,0) coordinate (copy1)
++ (1,0) node[kernel] (m) {$\kernel{K}^{\RV{W}|\RV{D}}$}
++(0,-0.5) coordinate (Fd)
++(1,0) node (D) {$\RV{D}$}
++ (0,0.5) node (X) {$\RV{W}$};
\draw (O) -- (m) -- (X);
\draw (copy1) to [out=290,in=180] (D);
\end{tikzpicture}
\end{align}
\end{proof}

We use the informal convention of labelling wires in quote marks $``\RV{X}''$ if that wire is ``supposed to'' carry the label $\RV{X}$ but the label may not be valid.

\begin{theorem}[Iterated disintegration]\label{th:iterated_disint}
Given a kernel space $(\kernel{K},D,\Omega)$, random variables $\RV{X}$, $\RV{Y}$ and $\RV{Z}$ and domain variable $\RV{D}$,
\begin{align}
\begin{tikzpicture}
	\path (0,0.15) node (D) {$``\RV{D}''$}
	+ (0,-0.3) node (X) {$``\RV{X}''$}
	++ (.7,-0.15) coordinate (copy0)
	++ (.7,0) node[kernel] (Yxd) {$\kernel{K}^{\RV{Y}|\RV{XD}}$}
	++ (0.7,0) coordinate (copy1)
	++(1.5,0) node[kernel] (Zxyd) {$\kernel{K}^{\RV{Z}|\RV{XYD}}$}
	++(1.5,0) node (Z) {$``\RV{Z}''$}
	+(0,-0.4) node (Y) {$``\RV{Y}''$};
	\draw (D) -- ($(Yxd.west) + (0,0.15)$) (X) -- ($(Yxd.west) + (0,-0.15)$);
	\draw (Yxd) -- (Zxyd);
	\draw ($(copy0) + (0,0.15)$) to [out=90,in=180] ($(Zxyd.west)+(0,0.15)$);
	\draw ($(copy0) + (0,-0.15)$) to [out=-90,in = 180] ($(Zxyd.west)+(0,-0.15)$);
	\draw (copy1) to [out=-90,in=180] (Y) (Zxyd) -- (Z);
\end{tikzpicture}\in \kernel{K}^{\{\RV{ZY}|\RV{XD}\}}
\end{align}

Equivalently, for $d\in D$ and $x\in X$, $A\in \sigalg{Y}$, $B\in\sigalg{Z}$,

\begin{align}
	(d,x;A,B)\mapsto \int_A \kernel{K}^{\RV{Z}|\RV{XYD}}_{(x,y,d)}(B) d\kernel{K}^{\RV{Y}|\RV{XD}}_{(x,d)}(y) \in \kernel{K}^{\{\RV{ZY}|\RV{XD}\}}
\end{align}
\end{theorem}

\begin{proof}
\todo[inline]{write this up}

\end{proof}

The existence of disintegrations of standard measurable probability spaces is well known.

\begin{theorem}[Disintegration existence - probability space]\label{th:disintegration_exist}
Given a probability measure $\prob{P}\in \Delta(\mathcal{X}\otimes \mathcal{Y})$, if $(F,\mathcal{F})$ is standard then a disintegration $\prob{P}^{\RV{Y}|\RV{X}}:X\to \Delta(\mathcal{Y})$ exists \citep{cinlar_probability_2011}.
\end{theorem}

In particular, if for all $x\in X$, $\prob{P}^{\RV{X}}(\RV{X}\in\{x\})>0$, then $\prob{P}^{\RV{Y}|\RV{X}}_x(A) = \frac{\prob{P}^{\RV{X}\RV{Y}}(\{x\}\times A)}{\prob{P}^{\RV{X}}(\{x\})}$.

For Markov kernel spaces, standard measurability is not known to guarantee that a disintegration exists. Given a kernel space $(\kernel{K},(D,\sigalg{D}),(\Omega,\sigalg{E}))$ with $D=[0,1]$ and $\Omega=[0,1]^2$, both Borel, let $\RV{X},\RV{Y},\RV{D}:\Omega\times D\to [0,1]$ project the first, second and third dimensions of $\Omega\times D$ respectively. Let $\kernel{K}_d(A) = \lambda(A)$, the Lebesgue measure of $A\in \sigalg{E}$ on $[0,1]^2$ for all $d\in D$. By Theorem \ref{th:disintegration_exist}, we have for each $d\in D$ a disintegration $Q(d):=(\kernel{K}_d)^{\RV{Y}|\RV{X}}$ of $(\kernel{K}_d)^{\RV{X}\RV{Y}}$, and it is fairly straightforward to show it must be the case that $Qd_x(A)=\lambda(A)$ for all $A\in \sigalg{B}([0,1])$ and $\lambda$-almost all $x\in [0,1]$. $Q(d)_x$ is a probability measure for every $(d,x)\in [0,1]^2$ because it is a disintegration, but $Q:D\times X\to \Delta(\sigalg{Y})$ given by $(d,x,A)\mapsto Q(d)_x(A)$ may fail to be a Markov kernel. Let $I:[0,1]\to \{0,1\}$ be the indicator function on a non-measurable set $C$, and define

\begin{align}
	Q(d)_{x}(A) = (1-I(d)\delta_{d}(\{x\}))\lambda(A) + I(d)\delta_{d}(\{x\})\delta_{0}(A)\label{eq:non_measurable_disint}
\end{align}

That is, $Q$ is the measure $\lambda{A}$ for all points $(x,d)$ except where $x=d$ and $d\in C$. Note that for each value of $d$, $Q$ differs from $\lambda(A)$ on at most a single point $x\in[0,1]$, which has measure $0$ under the Lebesgue measure $\lambda$. Thus $Q(d)$ so defined is indeed a disintegration in (\kernel{K}_d)^{\{\RV{Y}|\RV{X}\}}. Consider the function

\begin{align}
	Q^{\{0\}}:(d,x)&\mapsto Q(d)_x(\{0\})\\
	Q^{\{0\}-1}(\{1\})&=\{(d,x):Q(d)_x(\{0\})=1\}\\
	&= \{(d,x):d=x\And d\in C\}
\end{align}

Thus $Q^{\{0\}}$ is not measurable and consequently $Q$ fails to be a Markov kernel. The problem comes from the fact that $Q$ is defined by an uncountable collection of disintegrations $Q(d)$, each of which is individually measurable. In this case, the problem can be easily solved by defining $Q'$ without the non-measurable component in \ref{eq:non_measurable_disint}. What we would like are general conditions under which we know that we can choose an appropriate set of disintegrations $Q(d)$ in order for the resulting $Q$ to be a Markov kernel.

The following two theorems establish two separate sufficient conditions for the existence of disintegrations in a Markov kernel space. In the work that follows I will exclusively assume the first condition -- that all probability measures in the range of any kernel space under consideration are combinations of point measures and measures absolutely continuous with respect to the Lebesgue measure -- because I think it is unproblematic in this context to exclude continuous singular measures like the Cantor distribution. The second condition is included for completeness.

\begin{theorem}[Existence of disintegrations on kernel spaces: point and lebesgue continuous measures]
Given a kernel space $(\kernel{K},(D,\sigalg{D}),(\Omega,\sigalg{E}))$ with $D$, $\Omega$ standard measurable, let $f:\Omega\times D\to [0,1]$ be an isomorphism (such an isomorphism exists for all standard measurable spaces \citep{cinlar_probability_2011}. If for all $d\in D$, $\kernel{K}_d\kernel{F}_F = \mu_d^\lambda+\mu_d^p$ where $\mu_d^\lambda$ is absolutely continuous with respect to the Lebesgue measure and $\mu_d^p$ is a point measure, then for any $\RV{X},\RV{Y}\in \sigalg{E}\otimes\sigalg{D}$ and domain variable $\RV{D}:\Omega\times D\mapsto D$ a disintegration $\kernel{K}^{\RV{Y}|\RV{X}\RV{D}}$ exists.
\end{theorem}

\begin{proof}
We have a kernel space $(\kernel{K},(D),(\Omega),\sigalg{D}\otimes\sigalg{E}))$ with $D$, $\Omega$ standard measurable, an isomorphism $f:\Omega\times D\to [0,1]$ and for all $d\in D$, $\kernel{K}_d\kernel{F}_f = \mu_d^\lambda+\mu_d^p$ where $\mu_d^\lambda$ is absolutely continuous with respect to the Lebesgue measure and $\mu_d^p$ is a point measure. Let $\RV{D}:\Omega\times D \to D$ be a domain variable and $\RV{X}:\Omega\times D\to \mathbb{R}$ be a random variable. We will work in the image space $(\kernel{K}\kernel{F}_f,f(D),f(\Omega),\sigma(f))$ with random variables $\RV{D}f:=\RV{D}\circ f^{-1}:[0,1]\to D$ and $\RV{X}f:=\RV{X}\circ f^{-1}:[0,1]\to \mathbb{R}$. Let $\sigma(\RV{X}f):=\sigalg{F}$, and let $\RV{Y}:=f^{-1}$.

$\sigalg{F}$ is a sub-$\sigma$-algebra of $\sigalg{B}([0,1])$. 

\begin{align}
	\sigalg{F}_n &= \sigma(H_1,...,H_n)\\
	\sigalg{F} = \cup_{n\in\mathbb{N}}} \sigalg{F}_n
\end{align}

Each $\sigalg{F}_n$ has a finite partition $\sigalg{G}_n$. 

Therefore there exists 
\end{proof}

\begin{theorem}[Existence of disintegrations on kernel spaces: denumerable range]
Given a kernel space $(\kernel{K},(D,\sigalg{D}),(\Omega,\sigalg{E}))$ with $D$ denumerable and $\Omega$ standard measurable, then for any $\RV{X},\RV{Y}\in \sigalg{E}\otimes\sigalg{D}$ and domain variable $\RV{D}:\Omega\times D\mapsto D$ a disintegration $\kernel{K}^{\RV{Y}|\RV{X}\RV{D}}$ exists.
\end{theorem}


\begin{definition}[Relative probability space]

\todo[inline]{better name}

Given a Markov kernel space $(\kernel{K},D,\Omega)$ and a positive definite measure $\mu\in \Delta(\mathcal{D})$, $(\mu\kernel{K},\Omega\times D)$ is a \emph{relative} probability space.

For any random variable $\RV{X}:\Omega\times D\to X$ on $(\kernel{K},D,\Omega)$, its relative on $(\mu\kernel{K},\Omega\times D)$ is given by the same measurable function, and we give it the same name $\RV{X}$.
\end{definition}


\begin{lemma}[Agreement of disintegrations]\label{lem:agree_disint}
Given a Markov kernel space $(\kernel{K},D,\Omega)$, any relative probability space $(\mu\prob{K},\Omega\times D)$ and any random variables $\RV{X}:\Omega\times D\to X$, $\RV{Y}:\Omega\times D\to Y$, $\kernel{K}^{\{\RV{Y}|\RV{X}\RV{D}\}}=(\mu\prob{K})^{\{\RV{Y}|\RV{X}\RV{D}\}}$ (note that this set equality).
\end{lemma}

\begin{proof}
Define $\prob{P}:=\mu\kernel{K}$ and let $\kernel{M}$ be an arbitrary version of $\kernel{K}^{\{\RV{Y}|\RV{X}\RV{D}\}}$. Then
\begin{align}
\begin{tikzpicture}
\path (0,0) node[dist,inner sep=0 pt] (m) {$\prob{P}^{\RV{X}\RV{Y}\RV{D}}$}
++ (1,0.3) node (E) {$\RV{X}$}
++ (0,-0.3) node (F) {$\RV{Y}$}
++ (0,-0.3) node (D) {$\RV{D}$};
\draw ($(m.east) + (0,0.3)$) -- (E);
\draw ($(m.east) + (0,0)$) -- (F);
\draw ($(m.east) + (0,-0.3)$) -- (D);
\end{tikzpicture} &= \begin{tikzpicture}
\path (0,0) node[dist] (O) {$\mu$}
+ (0.75,0) coordinate (copy0)
++ (1.5,0) node[kernel] (m) {$\kernel{K}^{\RV{XY}|\RV{D}}$}
++ (1,0.15) node (E) {$\RV{X}$}
++ (0,-0.3) node (F) {$\RV{Y}$}
++ (0,-0.3) node (D) {$\RV{D}$};
\draw (O) -- (m) ($(m.east) + (0,0.15)$) -- (E);
\draw ($(m.east) + (0,-0.15)$) -- (F);
\draw (copy0) to [out=-60,in=180] (D);
\end{tikzpicture}\\
 &= \begin{tikzpicture}\path (0,0) node[dist] (O) {$\mu$}
++ (0.3,0) coordinate (copy1)
++ (1,0) node[kernel] (m) {$\kernel{K}^{\RV{X}|\RV{D}}$}
++ (1,0.15) coordinate (copy0)
++ (1.2,0) node (E) {$\RV{X}$}
++(-0.7,-0.3) node[kernel] (K) {$\kernel{M}$}
++(0.7,0) node (F) {$\RV{Y}$}
++(0,-0.3) node (D) {$\RV{D}$};
\draw (O) -- (m) ($(m.east) + (0,0.15)$) -- (E);
\draw (copy0) to [bend right] ($(K.west) + (0,0.1)$) (K) -- (F);
\draw (copy1) to [out=-45,in=180] ($(K.west) + (0,-0.1)$);
\draw (copy1) to [out=-90,in=180] (D);
\end{tikzpicture}\\
 &= \begin{tikzpicture}
\path (0,0) node[dist] (m) {$\prob{P}^{\RV{X}\RV{D}}$}
++ (0.7,0.15) coordinate (copy0)
+ (0,-0.3) coordinate (copy1)
+(0.2,-0.3) node (T) {}
++ (1.2,0) node (E) {$\RV{X}$}
++(-0.7,-0.3) node[kernel] (K) {$\kernel{M}$}
++(0.7,0) node (F) {$\RV{Y}$}
++ (0,-0.3) node (D) {$\RV{D}$};
\draw ($(m.east) + (0,0.15)$) -- (E);
\draw (copy0) to [bend right] ($(K.west) + (0,0.1)$) (K) -- (F);
\draw ($(m.east) + (0,-0.15)$) -- (copy1) -- ($(K.west) + (0,0)$);
\draw (copy1) to [out = -60, in=180] (D);
\end{tikzpicture}
\end{align}

Thus $\kernel{M}\in \prob{P}^{\{\RV{Y}|\RV{X}\RV{D}\}}$.

Let $\kernel{N}$ be an arbitrary version of $\prob{P}^{\{\RV{Y}|\RV{X}\RV{D}\}}$. To show that $\kernel{N}\in \kernel{K}^{\{\RV{Y}|\RV{X}\RV{D}\}}$, we will show for all $d\in D$

\begin{align}
	\prob{Q} &:= \begin{tikzpicture}
\path (0,0) node[dist] (D) {$\delta_{d}$}
++ (0.7,0) coordinate (copy0)
++(0.7,0) node[kernel] (K) {$\kernel{K}^{\RV{X}|\RV{D}}$}
++(0.5,0) coordinate (copy1)
++(0.8,0) node[kernel] (N) {$\kernel{N}$}
++(1,0) node (Y) {$\RV{Y}$}
++(0,-0.3) node (X) {$\RV{X}$}
++(0,-0.3) node (Do) {$\RV{D}$};
\draw (D) -- (K) -- (N) -- (Y);
\draw (copy0) to [out=-90,in=180] (Do);
\draw (copy1) to [out=-45,in=180] (X);
\draw (copy0) to [out=90,in=180] ($(N.west)+(0,0.15)$);
\end{tikzpicture}\\
 &= \kernel{K}^{\RV{X}\RV{Y}\RV{D}|\RV{D}}_d\label{eq:prob_disint_in_kernel_disint}
\end{align}



For $A\in\sigalg{X}$,$B\in\sigalg{Y}$, $d\in D$, we have $\prob{Q}(A\times B\times \emptyset)=0=\kernel{K}^{\RV{X}\RV{Y}\RV{D}|\RV{D}}_d(A\times B\times \emptyset$, and for $\{d\}\in\sigalg{D}$ we have $\mu(\{d\})>0$ so:

\begin{align}
\prob{Q}(A\times B\times \{d\}) &= \int_{X^2} \int_X \int_{D^3} \kernel{N}_{d'',x'}(A) \textbf{Id}_{x''}(B) \textbf{Id}_{d'''} (\{d\}) d\splitter{0.1}_d(d',d'',d''') d\kernel{K}^{\RV{X}|\RV{D}}_{d'}(x)d\splitter{0.1}_x(x',x'')\\
							&= \delta_d(\{d\}) \int_X \kernel{N}_{d,x}(A) \delta_x(B) d\kernel{K}^{\RV{X}|\RV{D}}_d(x)\\
							&= \frac{1}{\mu(\{d\})} \int_{\{d\}} d\mu(d') \int_X \kernel{N}_{d,x}(A) \delta_x(B) d\kernel{K}^{\RV{X}|\RV{D}}_d(x)\\
							&= \frac{1}{\mu(\{d\})} \int_D\int_X \kernel{N}_{d,x}(A) \delta_{d'}(\{d\}) \delta_x(B) d\kernel{K}^{\RV{X}|\RV{D}}_d(a) d\mu(d')\\
							&= \frac{1}{\mu(\{d\})} \int_D\int_X \kernel{N}_{d,x}(A) \delta_{d'}(\{d\}) \delta_x(B) d\kernel{K}^{\RV{X}|\RV{D}}_{d'}(a) d\mu(d')\\
							&= \frac{1}{\mu(\{d\})} \prob{P}^{\RV{X}\RV{Y}\RV{D}}(A\times B\times \{d\})\\
							&= \frac{1}{\mu(\{d\})} \int_D \kernel{K}_{d'}^{\RV{X}\RV{Y}\RV{D}|\RV{D}}(A\times B\times \{d\})d\mu(d')\\
							&= \frac{1}{\mu(\{d\})} \int_D \kernel{K}_{d'}{\RV{X}\RV{Y}|\RV{D}}(A\times B) \delta_{d'}(\{d\})d\mu(d')\\
							&= \kernel{K}_{d}^{\RV{X}\RV{Y}|\RV{D}}(A\times B)\\
							&= \kernel{K}_d^{\RV{X}\RV{Y}|\RV{D}}(A\times B) \delta_d(\{d\})\\
							&= \int_D \kernel{K}_{d'}^{\RV{X}\RV{Y}} (A\times B) \delta_{d''}(\{d\}) d\splitter{0.1}_d(d',d'')\\
							&= \kernel{K}_d^{\RV{X}\RV{Y}\RV{D}|\RV{D}}(A\times B\times \{d\})
\end{align}


Equality follows from the monotone class theorem. Thus $\kernel{N}\in \kernel{K}^{\{\RV{Y}|\RV{X}\RV{D}\}}$.
\end{proof}

Thus any kernel conditional probability $\kernel{K}^{\RV{Y}|\RV{X}\RV{D}}$ can equally well be considered a regular conditional probability $\prob{P}^{\RV{Y}|\RV{X}\RV{D}}$ for a related probability space $(\prob{P},\Omega\times D)$ under the obvious identification of random variables, provided $D$ is countable. Note that any conditional probability $\prob{P}^{\RV{Y}|\RV{X}}$ that is \emph{not} conditioned on $\RV{D}$ is undefined in the kernel space $(\kernel{K},D,\Omega)$.

\subsubsection{Conditional Independence}

\begin{definition}[Kernels constant in an argument]
	Given a kernel $(\kernel{K},D,\Omega)$ and random variables $\RV{Y}$ and $\RV{X}$, we say a verstion of the disintegration $\kernel{K}^{\RV{Y}|\RV{X}\RV{D}}$ is constant in $\RV{D}$ if for all $x\in X$, $d,d'\in D$, $\kernel{K}^{\RV{Y}|\RV{X}\RV{D}}_{(x,d)} = \kernel{K}^{\RV{Y}|\RV{X}\RV{D}}_{(x,d')}$.

\end{definition}

\begin{definition}[Domain Conditional Independence]
Given a kernel space $(\kernel{K},D,\Omega)$, relative probability space $(\prob{P},\Omega\times D)$, variables $\RV{X}$,$\RV{Y}$ and domain variable $\RV{D}$, $\RV{X}$ is \emph{conditionally independent} of $\RV{D}$ given $\RV{Y}$, written $\RV{X}\CI_{\kernel{K}} \RV{D}|\RV{Y}$ if any of the following equivalent conditions hold:

\todo[inline]{Almost sure equality}

\begin{enumerate}
	\item $\prob{P}^{\RV{X}\RV{D}|\RV{Y}} \sim \prob{P}^{\RV{X}|\RV{Y}}\utimes \prob{P}^{\RV{D}|\RV{Y}}$
	\item For any version of $\prob{P}^{\{\RV{X}|\RV{Y}\}}$, $\prob{P}^{\RV{X}|\RV{Y}}\otimes\stopper{0.1}_D$ is a version of  $\kernel{K}^{\{\RV{X}|\RV{Y}\RV{D}\}}$
	\item There exists a version of $\kernel{K}^{\{\RV{X}|\RV{Y}\RV{D}\}}\text{ constant in }\RV{D}$
\end{enumerate}
\end{definition}

\begin{theorem}[Definitions are equivalent]\label{th:ci_equivalence}
(1)$\implies$(2):
By Lemma \ref{lem:agree_disint}, $\prob{P}^{\{\RV{Y}|\RV{X}\RV{D}\}}=\kernel{K}^{\{\RV{Y}|\RV{X}\RV{D}\}}$. Thus it is sufficient to show that $\prob{P}^{\RV{X}|\RV{Y}}\otimes\stopper{0.1}$ is a version of $\prob{P}^{\{\RV{X}|\RV{Y}\RV{D}\}}$.

\begin{align}
\begin{tikzpicture}
	\path (0,0) node[dist] (Pxd) {$\prob{P}^{\RV{Y}\RV{D}}$}
	+ (0.7,0.1) coordinate (copy0)
	+ (0.7,-0.1) coordinate (copy1)
	++ (1.5,0) node[kernel] (Pyxd) {$\prob{P}^{\RV{X}|\RV{Y}}$}
	++(1,0) node (Y) {$``\RV{X}''$}
	+(0,0.3) node (D) {$``\RV{D}''$}
	+(0,0.6) node (X) {$``\RV{Y}''$};
	\draw ($(Pxd.east) + (0,0.1)$) -- ($(Pyxd.west)+(0,0.1)$);
	\draw ($(Pxd.east) + (0,-0.1)$) -- (copy1);
	\draw[-{Rays[n=8]}] (copy1) to [out=-80,in=180] ($(Pyxd.south)+(0,-0.3)$);
	\draw (copy0) to [out=80,in=180] (X);
	\draw (copy1) to [out=80,in=180] (D);
	\draw (Pyxd) -- (Y);
\end{tikzpicture} &= \begin{tikzpicture}
	\path (0,0) node[dist] (Pxd) {$\prob{P}^{\RV{Y}\RV{D}}$}
	+ (0.7,0.1) coordinate (copy0)
	+ (0.7,-0.1) coordinate (copy1)
	++ (1.5,0) node[kernel] (Pyxd) {$\prob{P}^{\RV{X}|\RV{Y}}$}
	++(1,0) node (Y) {$``\RV{X}''$}
	+(0,0.3) node (D) {$``\RV{D}''$}
	+(0,0.6) node (X) {$``\RV{Y}''$};
	\draw ($(Pxd.east) + (0,0.1)$) -- ($(Pyxd.west)+(0,0.1)$);
	\draw ($(Pxd.east) + (0,-0.1)$) -- (copy1);
	\draw (copy0) to [out=80,in=180] (X);
	\draw (copy1) to [out=80,in=180] (D);
	\draw (Pyxd) -- (Y);
\end{tikzpicture} \\
 &= \begin{tikzpicture}
	\path (0,0) node[dist] (Pxd) {$\prob{P}^{\RV{Y}}$}
	+ (0.7,-0.2) coordinate (copy1)
	++ (1.5,-0.2) node[kernel] (Pyxd) {$\prob{P}^{\RV{X}|\RV{Y}}$}
	+ (0,0.5) node[kernel] (Pdx) {$\prob{P}^{\RV{D}|\RV{Y}}$}
	++(1,0) node (Y) {$``\RV{X}''$}
	+(0,0.5) node (D) {$``\RV{D}''$}
	+(0,1.2) node (X) {$``\RV{Y}''$};
	\draw ($(Pxd.east) + (0,-0.2)$) -- ($(Pyxd.west)+(0,0)$);
	\draw (copy1) to [out=90,in=180] (X);
	\draw (copy1) to [out=80,in=180] (Pdx);
	\draw (Pdx) -- (D);
	\draw (Pyxd) -- (Y);
\end{tikzpicture} \\
&\overset{condition (1)}{=} \begin{tikzpicture}
	\path (0,0) node[dist] (Pxd) {$\prob{P}^{\RV{Y}}$}
	+ (0.7,0) coordinate (copy1)
	++ (1.5,0) node[kernel] (Pyxd) {$\prob{P}^{\RV{X}\RV{D}|\RV{Y}}$}
	++(1,-0.15) node (Y) {$``\RV{X}''$}
	+(0,0.3) node (D) {$``\RV{D}''$}
	+(0,0.6) node (X) {$``\RV{Y}''$};
	\draw (Pxd) -- (Pyxd);
	\draw (copy1) to [out=90,in=180] (X);
	\draw ($(Pyxd.east)+(0,0.15)$) -- (D);
	\draw ($(Pyxd.east)+(0,-0.15)$) -- (Y);
\end{tikzpicture}\\
&= \begin{tikzpicture}
	\path (0,0) node[dist] (Pxd) {$\prob{P}^{\RV{YDX}}$}
	++(1,-0.3) node (Y) {$\RV{X}$}
	+(0,0.3) node (D) {$\RV{D}$}
	+(0,0.6) node (X) {$\RV{Y}$};
	\draw ($(Pxd.east) + (0,0.3)$) -- (X) ($(Pxd.east) + (0,-0.3)$) -- (Y) (Pxd) -- (D);
\end{tikzpicture}
\end{align}

(2)$\implies$ (3)

$\prob{P}^{\RV{X}|\RV{Y}}\otimes\stopper{0.1}_D$ is a version of $\kernel{K}^{\{\RV{X}|\RV{Y}\RV{D}\}}$ by assumption, and is clearly constant in $\RV{D}$.

(3)$\implies$ (1)

By lemma \ref{lem:agree_disint}, there also exists a version of $\prob{P}^{\{\RV{X}|\RV{Y}\RV{D}\}}$ constant in $\RV{D}$. Let $\kernel{M}:Y\times D\to \Delta(\sigalg{X})$ be such a version. For arbitrary $d_0\in D$, let $\kernel{N}:=\kernel{M}_{(\cdot,d_0)}:Y\to \Delta(\sigalg{X})$ be the map $x\mapsto \kernel{M}_{(x,d_0)}$. By constancy in $\RV{D}$, $\kernel{M} = \stopper{0.1}\otimes \kernel{N}$. We wish to show $\prob{P}^{\RV{X}|\RV{Y}}\utimes \prob{P}^{\RV{D}|\RV{Y}}\in \prob{P}^{\{\RV{XD}|\RV{Y}\}}$. By Theorem \ref{th:iterated_disint}, we have 

\begin{align}
\begin{tikzpicture}
	\path (0,0) node[dist] (Pxd) {$\prob{P}^{\RV{Y}\RV{D}}$}
	+ (0.7,0.1) coordinate (copy0)
	+ (0.7,-0.1) coordinate (copy1)
	++ (1.5,0) node[kernel] (Pyxd) {$\kernel{N}$}
	++(1,0) node (Y) {$\RV{X}$}
	+(0,0.3) node (D) {$\RV{D}$}
	+(0,0.6) node (X) {$\RV{Y}$};
	\draw ($(Pxd.east) + (0,0.1)$) -- ($(Pyxd.west)+(0,0.1)$);
	\draw ($(Pxd.east) + (0,-0.1)$) -- (copy1);
	\draw[-{Rays[n=8]}] (copy1) to [out=-80,in=180] ($(Pyxd.south)+(0,-0.3)$);
	\draw (copy0) to [out=80,in=180] (X);
	\draw (copy1) to [out=80,in=180] (D);
	\draw (Pyxd) -- (Y);
\end{tikzpicture} &= \begin{tikzpicture}
	\path (0,0) node[dist] (Pxd) {$\prob{P}^{\RV{Y}}$}
	+ (0.7,0) coordinate (copy0)
	++ (1.5,0) node[kernel] (Dy) {$\prob{P}^{\RV{D}|\RV{Y}}$}
	++ (1.5,0) node[kernel] (Pyxd) {$\kernel{N}$}
	++(1,0) node (Y) {$\RV{X}$}
	+(0,0.3) node (D) {$\RV{D}$}
	+(0,0.6) node (X) {$\RV{Y}$};
	\draw ($(Pxd.east) + (0,0.1)$) -- ($(Pyxd.west)+(0,0.1)$);
	\draw ($(Pxd.east) + (0,-0.1)$) -- (copy1);
	\draw[-{Rays[n=8]}] (copy1) to [out=-80,in=180] ($(Pyxd.south)+(0,-0.3)$);
	\draw (copy0) to [out=80,in=180] (X);
	\draw (copy1) to [out=80,in=180] (D);
	\draw (Pyxd) -- (Y);
\end{tikzpicture}
\end{align}
\end{theorem}

\begin{definition}[Conditional probability existence]\label{def:conditional_probability_existence}
Given a kernel space $(\kernel{K},D,\Omega)$ and random variables $\RV{X}$, $\RV{Y}$, we say $\kernel{K}^{\{\RV{Y}|\RV{X}\}}$ \emph{exists} if $\RV{Y}\CI_{\kernel{K}} \RV{D}|\RV{X}$. If $\kernel{K}^{\{\RV{Y}|\RV{X}\}}$ exists then it is by definition equal to $\prob{P}^{\{\RV{Y}|\RV{X}\}}$ for any related probability space $(\prob{P},\Omega\times D)$.
\end{definition}

Note that $\kernel{K}^{\{\RV{Y}|\RV{X}\RV{D}\}}$ always exists.

\begin{definition}[Conditional Independence]\label{def:conditional_independence}
Given a kernel space $(\kernel{K},D,\Omega)$, some relative probability space $(\prob{P},\Omega\times D)$, variables $\RV{X}$,$\RV{Y}$ and $\RV{Z}$, $\RV{X}$ is \emph{conditionally independent} of $\RV{Z}$ given $\RV{Y}$, written $\RV{X}\CI_{\kernel{K}} \RV{Z}|\RV{Y}$ if $\kernel{K}^{\{\RV{XY}|\RV{Z}\}}$ exists and any of the following equivalent conditions hold:

\todo[inline]{Almost sure equality}

\begin{itemize}
	\item $\prob{P}^{\RV{X}\RV{Z}|\RV{Y}} \sim \prob{P}^{\RV{X}|\RV{Y}}\utimes \prob{P}^{\RV{Z}|\RV{Y}}$
	\item For any version of $\prob{P}^{\{\RV{X}|\RV{Y}\}}$, $\prob{P}^{\RV{X}|\RV{Y}}\otimes\stopper{0.1}_Z$ is a version of  $\kernel{K}^{\{\RV{X}|\RV{Y}\RV{Z}\}}$
	\item There exists a version of $\kernel{K}^{\{\RV{X}|\RV{Y}\RV{Z}\}}\text{ constant in }\RV{Z}$
\end{itemize}
\end{definition}

\begin{lemma}[Diagrammatic consequences of labels]

In general, diagram labels are ``well behaved'' with regard to the application of any of the special Markov kernels: identities \ref{eq:identity}, swaps \ref{eq:swap}, discards \ref{eq:discard} and copies \ref{eq:copy} as well as with respect to the coherence theorem of the CD category. They are not ``well behaved'' with respect to composition.

Fix some Markov kernel space $(\kernel{K},D,\Omega)$ and random variables $\RV{X}$, $\RV{Y}$, $\RV{Z}$ taking values in $X,Y,Z$ respectively. $\mathrm{Sat:}$ indicates that a labeled diagram satisfies definitions \ref{def:wl_jprob} and \ref{def:wl_disint} with respect to $(\mathscr{K},D,\Omega)$ and $\RV{X}$, $\RV{Y}$, $\RV{Z}$.  The following always holds:

\begin{align}
\mathrm{Sat:}
\begin{tikzpicture}
\path (0,0) node (A) {$\RV{X}$}
++(0.8,0) node (X) {$\RV{X}$};
\draw (A) -- (X);
\end{tikzpicture}
\end{align}

and the following implications hold:
\begin{align}
\mathrm{Sat:}\;\begin{tikzpicture}
\path (0,0) node (Z) {$\RV{Z}$} 
++ (0.7,0) node[kernel] (M) {$\kernel{K}$}
++ (0.7,0.15) node (X) {$\RV{X}$}
++(0,-0.3) node (Y) {$\RV{Y}$};
\draw (Z) -- (M) ($(M.east) + (0,0.15)$) -- (X);
\draw ($(M.east) + (0,-0.15)$) -- (Y);
\end{tikzpicture} &\implies \mathrm{Sat:}\; \begin{tikzpicture}
\path (0,0) node (Z) {$\RV{Z}$} 
++ (0.7,0) node[kernel] (M) {$\kernel{K}$}
++ (0.7,0.15) node (X) {$\RV{X}$}
++(0,-0.3) node (Y) {};
\draw (Z) -- (M) ($(M.east) + (0,0.15)$) -- (X);
\draw[-{Rays [n=8]}] ($(M.east) + (0,-0.15)$) -- (Y);
\end{tikzpicture}\\
\mathrm{Sat:}\;\begin{tikzpicture}
\path (0,0) node (Z) {$\RV{Z}$} 
++ (0.7,0) node[kernel] (M) {$\kernel{K}$}
++ (0.7,0.15) node (X) {$\RV{X}$}
++(0,-0.3) node (Y) {$\RV{Y}$};
\draw (Z) -- (M) ($(M.east) + (0,0.15)$) -- (X);
\draw ($(M.east) + (0,-0.15)$) -- (Y);
\end{tikzpicture} &\implies \mathrm{Sat:}\; \begin{tikzpicture}
\path (0,0) node (Z) {$\RV{Z}$} 
++ (0.7,0) node[kernel] (M) {$\kernel{K}$}
++ (0.7,0.15) node (X) {$\RV{Y}$}
++(0,-0.3) node (Y) {$\RV{X}$};
\draw (Z) -- (M) ($(M.east) + (0,0.15)$) to [out = 0, in = 180] (Y);
\draw ($(M.east) + (0,-0.15)$) to [out = 0, in = 180] (X);
\end{tikzpicture}\\
\mathrm{Sat:}\begin{tikzpicture}
\path (0,0) node (Z) {$\RV{Z}$} 
++ (0.7,0) node[kernel] (M) {$\mathrm{L}$}
++(0.6,0) node (X1) {$\RV{X}$};
\draw (Z) -- (M) (M)--(X1);
\end{tikzpicture}
&\implies \mathrm{Sat:}\begin{tikzpicture}
\path (0,0) node (Z) {$\RV{Z}$} 
++ (0.7,0) node[kernel] (M) {$\mathrm{L}$}
++ (0.7,0) coordinate (copy0)
++(0.5,0.2) node (X1) {$\RV{X}$}
++(0,-0.4) node (X2) {$\RV{X}$};
\draw (Z) -- (M) (M) -- (copy0) to [bend left] (X1);
\draw (copy0) to [bend right] (X2);
\end{tikzpicture}\\
\mathrm{Sat:}\begin{tikzpicture}
\path (0,0) node (X) {$\RV{Z}$}
++ (0.7,0) node[kernel] (K) {$\kernel{K}$}
++(0.7,0) node (Y) {$\RV{Y}$};
\draw (X) -- (K) -- (Y);
\end{tikzpicture} &\implies \mathrm{Sat:}
\begin{tikzpicture}
\path (0,0) node (A) {$\RV{Z}$}
++(0.5,0) coordinate (copy0)
+(1.2,0.3) node (X) {$\RV{Z}$}
++(0.5,-0.3) node[kernel] (K) {$\kernel{K}$}
+(0.7,0) node (Y) {$\RV{Y}$};
\draw (A) -- (copy0) to [bend left] (X);
\draw (copy0) to [bend right] (K) (K) -- (Y);
\end{tikzpicture}\label{eq:splitter_preserves_name}
\end{align}
\end{lemma}


\begin{proof}
\begin{itemize}
	\item $\mathrm{Id}_X$ is a version of $\prob{P}_{\RV{X}|\RV{X}}$ for all $\prob{P}$; $\prob{P}_{\RV{X}}\mathrm{Id}_X = \prob{P}_{\RV{X}}$
	\item $\kernel{K}\mathrm{Id}\otimes \stopper{0.2})(w;A) = \int_{X\times Y} \delta_x(A) \mathds{1}_Y(y) d\kernel{K}_w(x,y) = \kernel{K}_w(A\times Y) = \prob{P}_{\RV{X}|\RV{Z}}(w;A)$
	\item $\int_{X\times Y} \delta_{\mathrm{swap(x,y)}}(A\times B)d\kernel{K}_w(x,y) = \prob{P}_{\RV{Y}\RV{X}|\RV{Z}}(w;A\times B)$
	\item $\kernel{K}\splitter{0.1} (w;A\times B) = \int_{X} \delta_{x,x}(A\times B) d\kernel{K}_w(x) = \prob{P}_{\RV{X}\RV{X}|\RV{Z}} (w;A\times B)$
\end{itemize}
\ref{eq:splitter_preserves_name}: Suppose $\kernel{K}$ is a version of $\prob{P}_{\RV{Y}|\RV{Z}}$. Then
\begin{align}
\prob{P}_{\RV{Z}\RV{Y}} &= \begin{tikzpicture}
\path (0,0) node[dist] (m) {$\prob{P}_{\RV{Z}}$}
++ (0.7,0.15) coordinate (copy0)
++ (1.2,0) node (E) {$\RV{Z}$}
++(-0.7,-0.3) node[kernel] (K) {$\kernel{K}$}
++(0.7,0) node (F) {$\RV{Y}$};
\draw ($(m.east) + (0,0.15)$) -- (E);
\draw (copy0) to [bend right] (K) (K) -- (F);
\end{tikzpicture}\\
\prob{P}_{\RV{Z}\RV{Z}\RV{Y}} &= \begin{tikzpicture}
\path (0,0) node[dist] (m) {$\prob{P}_{\RV{Z}}$}
++ (0.7,0.15) coordinate (copy0)
+ (0.5,0) coordinate (copy1)
+ (1.2,0.3) node (Xm) {$\RV{Z}$}
++ (1.2,0) node (E) {$\RV{Z}$}
++(-0.7,-0.3) node[kernel] (K) {$\kernel{K}$}
++(0.7,0) node (F) {$\RV{Y}$};
\draw ($(m.east) + (0,0.15)$) -- (E);
\draw (copy0) to [bend right] (K) (K) -- (F);
\draw (copy1) to [bend left] (Xm);
\end{tikzpicture}\\
&= \begin{tikzpicture}
\path (0,0) node[dist] (m) {$\prob{P}_{\RV{Z}}$}
+ (0.5,0.15) coordinate (copy1)
++ (0.7,0.15) coordinate (copy0)
+ (1.2,0.3) node (Xm) {$\RV{Z}$}
++ (1.2,0) node (E) {$\RV{Z}$}
++(-0.7,-0.3) node[kernel] (K) {$\kernel{K}$}
++(0.7,0) node (F) {$\RV{Y}$};
\draw ($(m.east) + (0,0.15)$) -- (E);
\draw (copy0) to [bend right] (K) (K) -- (F);
\draw (copy1) to [bend left] (Xm);
\end{tikzpicture}
\end{align}
Therefore $\splitter{0.1}(\mathrm{Id}_X\otimes\kernel{K})$ is a version of $\prob{P}_{\RV{Z}\RV{Y}|\RV{Z}}$ by \ref{def:labeled_disint} 
\end{proof}

The following property, on the other hand, does \emph{not} generally hold:
\begin{align}
\mathrm{Sat:}\begin{tikzpicture}
\path (0,0) node (X) {$\RV{Z}$}
++ (0.7,0) node[kernel] (K) {$\kernel{K}$}
++(0.7,0) node (Y) {$\RV{Y}$};
\draw (X) -- (K) -- (Y);
\end{tikzpicture},
\begin{tikzpicture}
\path (0,0) node (X) {$\RV{Y}$}
++ (0.7,0) node[kernel] (K) {$\kernel{L}$}
++(0.7,0) node (Y) {$\RV{X}$};
\draw (X) -- (K) -- (Y);
\end{tikzpicture}
 &\implies \mathrm{Sat:}
\begin{tikzpicture}
\path (0,0) node (X) {$\RV{Z}$}
++ (0.7,0) node[kernel] (K) {$\kernel{K}$}
++(0.7,0) node[kernel] (L) {$\kernel{L}$}
++(0.7,0) node (X1) {$\RV{X}$};
\draw (X) -- (K) -- (Y) -- (L) -- (X1);
\end{tikzpicture}\label{eq:composition}
\end{align}

Consider some ambient measure $\prob{P}$ with $\RV{Z}=\RV{X}$ and $\prob{P}_{\RV{Y}|\RV{X}}=x\mapsto \mathrm{Bernoulli}(0.5)$ for all $z\in Z$. Then $\prob{P}_{\RV{Z}|\RV{Y}}=y\mapsto \prob{P}_{\RV{Z}}$, $\forall y\in Y$ and therefore $\prob{P}_{\RV{Y}|\RV{Z}}\prob{P}_{\RV{Z}|\RV{Y}}=x\mapsto \prob{P}_{\RV{Z}}$ but $\prob{P}_{\RV{Z}|\RV{X}} = x\mapsto \delta_x\neq \kernel{\prob{P}_{\RV{Y}|\RV{Z}}\prob{P}_{\RV{Z}|\RV{Y}}}$.






%!TEX root = main.tex

\chapter{Chapter 3: See-do models}

Consider the following problem: you are presented with a collection $\RV{H}$ of hypotheses about how the world might function and a vector $\mathbf{x}$ of observational data which you know could have taken values in some space $X$. You want to determine which hypothesis $\RV{H}\in \RV{H}$ best describes the world. However you ultimately solve the problem, the next step you take will probably be to determine for each $\RV{H}\in \RV{H}$ a probability distribution $\kernel{P}_\RV{H}\in \Delta(\sigalg{X})$ that indicates how likely you would be to observe the various elements of $X$ were $\RV{H}$ in fact the case. This is a \emph{statistical model} -- an indexed set of probability distributions $\{\prob{P}_\RV{H}|\RV{H}\in \RV{H}\}$. Statistical models are ubiquitous in the field of statistics -- they are found in statistical decision theory where the elements of $\RV{H}$ are typically called ``states''\citep{wald_statistical_1950}, in Bayesian inference where the elements of $\RV{H}$ may be called ``parameters'' \citep{freedman_asymptotic_1963} and in frequentist inference where elements of $\RV{H}$ they may be called ``hypotheses'' \citep{fisher_statistical_1992}. 

These different approaches to statistics may have different notions of what the ``best hypothesis'' $\RV{H}$ is, may employ different estimation methods and may not even agree about what ``distributed according to $\kernel{P}_\RV{H}$'' means. Nonetheless, the interpretation of the statistical model in each case is roughly the same: supposing $\RV{H}\in\RV{H}$ is true, the data will be distributed according to $\kernel{P}_\RV{H}$. A statistical model takes a hypothesis and tells you what you are likely to \emph{see}.

Sometimes we are interested in modelling situations where we can also make some choices that also affect the eventual consequences. For example, I might hypothesise $\RV{H}_1$: the switch on the wall controls my light, $\RV{H}_2$: the switch on the wall does not control my light. Then, given $\RV{H}_1$ I can choose to toggle the switch, and I will see my light turn on, or I can choose not to toggle the switch and I will not see my light turn on. Given $\RV{H}_2$, neither choice will result in a light turned on. Choices are clearly different to hypotheses: the choice I make depends on what I want to happen, while whether or not a hypothesis is true has no regard for my ambitions.

A ``statistical model with choices'' is simply a map $\prob{T}:D\times \RV{H}\to \Delta(\sigalg{E})$ for some set of choices $D$, hypotheses $\RV{H}$ and outcome space $(E,\sigalg{E})$. We can also distinguish two types of outcomes: \emph{observations} which are given prior to a choice being made and \emph{consequences} which happen after a choice is made. Observations cannot be affected by the choices made, while consequences are not subject to this restriction. That is, observations are what we might \emph{see} before making a choice, which depends on the hypothesis alone, and if we are lucky we may be able to invert this dependence to learn something about the hypothesis from observations. On the other hand, the consequences of what we \emph{do} depends jointly on the hypothesis and the choice we make and we judge which choices are more desirable on the basis of which consequences we expect them to produce. 

What we are studying is a family of models that generalises of statistical models to include hypotheses, choices, observations and consequences. These models are referred to as \emph{see-do models}. Hypotheses, observations, consequences and choices are not individually new ideas. \emph{Statistical decision problems} \citep{wald_statistical_1950,savage_foundations_1972} extend statistical models with decisions and \emph{losses}. Like consequences, losses depend on which choices are made. However, unlike consequences, losses must be ordered and reflect the preferences of a decision maker. \emph{Influence diagrams} are directed graphs created to represent decision problems that feature ``choice nodes'', ``chance nodes'' and ``utility nodes''. An influence diagram may be associated with a particular probability distribution \cite{nilsson_evaluating_2013} or with a set of probability distributions \cite{dawid_influence_2002}.

See-do models have deep roots in decision theory. Decision theory asks, out of a set of available acts, which ones ought to be chosen. See-do models answer an intermediate question: out of a set of available acts, what are the consequences of each? This question is described by \citet{pearl_causality:_2009} as an ``interventional'' question. To model questions described by Pearl as ``counterfactual'', we can use a special kind of see-do model with \emph{parallel choices}. Parallel choices model sequences of experiments where taking the same action repeatedly deterministically results in the same outcome, and where the result of a sequence of actions doesn't depend on the order in which the actions are taken. Parallel choices can arguably model a kind of commonsense counterfactuals as expressed in the question ``what would have happened if I chose $b$ instead of $a$?''. Whether or not this interpretation is compelling, we show that \emph{potential outcomes} \citep{rubin_causal_2005} exist in a see-do model if and only if it is a model of parallel choices.

\todo[inline]{Interestingly, it seems to be possible to construct a see-do model where the ``hypothesis'' is a quantum state, and quantum mechanics + locality seems to rule out parallel choices in such models in a manner similar to Bell's theorem. ``Seems to'' because I haven't actually proven any of these things.}

\section{Definition}

\todo[inline]{Terminology question: The variables $\RV{H}$ and $\RV{D}$ aren't \emph{random} in the commonsense understanding of the word. They're also defined on a \emph{kernel space} rather than a \emph{probability space}. They're currently called random variables simply by virtue of being measurable functions on the outcome space. I'm not a huge fan of ``quasirandom variables'', but it does capture the idea that these things are very similar to random variables but not exactly the same.}

\begin{definition}[See-Do model]\label{def:seedo}
A \emph{see-do model} $\langle\kernel{T},\RV{H},\RV{D},\RV{X},\RV{Y}\rangle$ is a kernel space (Definition \ref{def:kernel_space}) $(\kernel{T},H\times D,X\times Y)$ along with four random variables: the \emph{hypothesis} $\RV{H}:H\times D\times X\times Y\to H$, the \emph{choice} $\RV{D}:H\times D\times X\times Y\to D$, the \emph{observations} $\RV{X}:H\times D\times X\times Y\to X$ and the \emph{consequences} $\RV{Y}:H\times D\times X\times Y\to Y$, all given by the obvious projection maps. 

The spaces $H$, $D$, $X$ and $Y$ are the hypothesis, choice, observation and consequence spaces respectively.

A see-do model has the additional property that, holding the hypothesis fixed, the observations are independent of the choices - i.e. $\RV{X}\CI_{\kernel{T}} \RV{D}|\RV{H}$. We require that $H\times D$ is countable.
\end{definition}


\begin{theorem}[Observation and Consequence maps]\label{th:obs_cmaps}
Any see-do model $(\kernel{T},\RV{H},\RV{D},\RV{X},\RV{Y})$ can be uniquely represented by the following pair of Markov kernels:
\begin{itemize}
    \item The \emph{observation map} $\kernel{T}^{\RV{X}|\RV{H}}$
    \item The \emph{consequence map} $\kernel{T}^{\RV{Y}|\RV{X}\RV{H}\RV{D}}$
\end{itemize}

Furthermore
\begin{align}
\kernel{T} = \begin{tikzpicture} \path (0,0) node (T) {$\RV{H}$}
        + (0,-1.15) node (D) {$\RV{D}$}
        ++ (0.5,0) node[copymap] (copy0) {}
        + (0.,-1.15) node[copymap] (copy2) {}
        ++ (0.7,0) node[kernel] (O) {$\kernel{T}^{\RV{X}|\RV{H}}$}
        ++ (0.7,0) node[copymap] (copy1) {}
        +  (0.9,-1) node[kernel] (C) {$\kernel{T}^{\RV{Y}|\RV{X}\RV{H}\RV{D}}$}
        ++ (1.9,0) node (X) {$\RV{X}$}
        +  (0,-1) node (Y) {$\RV{Y}$}
        + (0,0.5) node (H) {$\RV{H}$}
        + (0,-1.5) node (D2) {$\RV{D}$};
        \draw (T) -- (O) -- (X);
        \draw (copy0) to [out=-90,in=180] ($(C.west) + (0,0)$);
        \draw (D) to [out=0,in=180] ($(C.west) + (0,-0.15)$);
        \draw (copy1) to [out=-60,in=180] ($(C.west)+ (0,0.15)$);
        \draw (C) -- (Y);
        \draw (copy0) to [out = 65, in = 180] (H);
        \draw (copy2) to [out = -65, in = 180] (D2);
    \end{tikzpicture}
\end{align}
\end{theorem}

\todo[inline]{Maybe moves proofs out of main text}

\begin{proof}
By \ref{th:representaiton}, 

\begin{align}
\kernel{T} = \begin{tikzpicture} \path (0,0) node (T) {$\RV{H}$}
        + (0,-1.15) node (D) {$\RV{D}$}
        ++ (0.5,0) node[copymap] (copy0) {}
        + (0.,-1.15) node[copymap] (copy2) {}
        ++ (0.7,0) node[kernel] (O) {$\kernel{T}^{\RV{X}|\RV{H}\RV{D}}$}
        ++ (0.7,0) node[copymap] (copy1) {}
        +  (0.9,-1) node[kernel] (C) {$\kernel{T}^{\RV{Y}|\RV{X}\RV{H}\RV{D}}$}
        ++ (1.9,0) node (X) {$\RV{X}$}
        +  (0,-1) node (Y) {$\RV{Y}$}
        + (0,0.5) node (H) {$\RV{H}$}
        + (0,-1.5) node (D2) {$\RV{D}$};
        \draw (T) -- (O) -- (X);
        \draw[name path=P1] (copy0) to [out=-90,in=180] ($(C.west) + (0,0)$);
        \draw (D) to [out=0,in=180] ($(C.west) + (0,-0.15)$);
        \draw (copy1) to [out=-60,in=180] ($(C.west)+ (0,0.15)$);
        \draw (C) -- (Y);
        \draw (copy0) to [out = 65, in = 180] (H);
        \draw (copy2) to [out = -65, in = 180] (D2);
        \draw[name path=P2] (copy2) to [out = 65, in = 180] ($(O.west)+(0,-0.15)$);
    \end{tikzpicture}
\end{align}


By the assumption $\RV{X}\CI_{\kernel{T}} \RV{D}|\RV{H}$ and version 2 of conditional independence from Theorem \ref{th:ci_equivalence},

\begin{align}
\kernel{T} &= \begin{tikzpicture} \path (0,0) node (T) {$\RV{H}$}
        + (0,-1.15) node (D) {$\RV{D}$}
        ++ (0.5,0) node[copymap] (copy0) {}
        + (0.,-1.15) node[copymap] (copy2) {}
        ++ (0.7,0) node[kernel] (O) {$\kernel{T}^{\RV{X}|\RV{H}}$}
        ++ (0.7,0) node[copymap] (copy1) {}
        +  (0.9,-1) node[kernel] (C) {$\kernel{T}^{\RV{Y}|\RV{X}\RV{H}\RV{D}}$}
        ++ (1.9,0) node (X) {$\RV{X}$}
        +  (0,-1) node (Y) {$\RV{Y}$}
        + (0,0.5) node (H) {$\RV{H}$}
        + (0,-1.5) node (D2) {$\RV{D}$};
        \draw (T) -- (O) -- (X);
        \draw (copy0) to [out=-90,in=180] ($(C.west) + (0,0)$);
        \draw (D) to [out=0,in=180] ($(C.west) + (0,-0.15)$);
        \draw (copy1) to [out=-60,in=180] ($(C.west)+ (0,0.15)$);
        \draw (C) -- (Y);
        \draw (copy0) to [out = 65, in = 180] (H);
        \draw (copy2) to [out = -65, in = 180] (D2);
        \draw[-{Rays[n=8]}] (copy2) to [out = 65, in = 180] ($(O.west)+(-0.2,-0.5)$);
    \end{tikzpicture}\\
    &= \begin{tikzpicture} \path (0,0) node (T) {$\RV{H}$}
        + (0,-1.15) node (D) {$\RV{D}$}
        ++ (0.5,0) node[copymap] (copy0) {}
        + (0.,-1.15) node[copymap] (copy2) {}
        ++ (0.7,0) node[kernel] (O) {$\kernel{T}^{\RV{X}|\RV{H}}$}
        ++ (0.7,0) node[copymap] (copy1) {}
        +  (0.9,-1) node[kernel] (C) {$\kernel{T}^{\RV{Y}|\RV{X}\RV{H}\RV{D}}$}
        ++ (1.9,0) node (X) {$\RV{X}$}
        +  (0,-1) node (Y) {$\RV{Y}$}
        + (0,0.5) node (H) {$\RV{H}$}
        + (0,-1.5) node (D2) {$\RV{D}$};
        \draw (T) -- (O) -- (X);
        \draw (copy0) to [out=-90,in=180] ($(C.west) + (0,0)$);
        \draw (D) to [out=0,in=180] ($(C.west) + (0,-0.15)$);
        \draw (copy1) to [out=-60,in=180] ($(C.west)+ (0,0.15)$);
        \draw (C) -- (Y);
        \draw (copy0) to [out = 65, in = 180] (H);
        \draw (copy2) to [out = -65, in = 180] (D2);
    \end{tikzpicture}
\end{align}

\end{proof}

\todo[inline]{Not quite sure if this is the right place for the following definition}

The independence of observations and choices is preserved when we take the product of a see-do model and a \emph{prior} over hypotheses. Such a product produces a \emph{Bayesian see-do model}:

\begin{definition}[Bayesian See-Do Model]
A Bayesian See-Do Model $\langle\kernel{U},\RV{D},\RV{X},\RV{Y}\rangle$ is a Markov kernel space $(\kernel{U},D,X\times Y)$ with the property $\RV{X}\CI_{\kernel{U}}\RV{D}$, along with choices $\RV{D}$, observations $\RV{X}$ and consequences $\RV{Y}$, defined as before.
\end{definition}

\begin{theorem}[A see-do model with a prior is a Bayesian see-do model]
The product of a see-do model $\kernel{T}$ and a prior $\gamma\in \Delta(\sigalg{H})$
\begin{align}
    \kernel{U} &:= (\gamma\otimes \mathrm{Id}^D)\kernel{T}
\end{align}
Is a Bayesian see-do model.
\end{theorem}

\todo[inline]{Maybe moves proofs out of main text}

\begin{proof}

It nees to be shown that $\RV{X}\CI_{\kernel{U}}\RV{D}$.

By definition
\begin{align}
\kernel{U}^{\RV{X}|\RV{D}} &= \kernel{U}\kernel{F}^{\RV{X}}\\
                            &= (\gamma\otimes \mathrm{Id}^D)\kernel{T}\kernel{F}^{\RV{X}}\\
                            &= \begin{tikzpicture} \path (0,0) node[dist] (T) {$\gamma$}
                                    + (0,-1.15) node (D) {$\RV{D}$}
                                    ++ (0.5,0) node[copymap] (copy0) {}
                                    + (0.,-1.15) node[copymap] (copy2) {}
                                    ++ (0.7,0) node[kernel] (O) {$\kernel{T}^{\RV{X}|\RV{H}}$}
                                    ++ (0.7,0) node[copymap] (copy1) {}
                                    +  (0.9,-1) node[kernel] (C) {$\kernel{T}^{\RV{Y}|\RV{X}\RV{H}\RV{D}}$}
                                    ++ (1.9,0) node (X) {$\RV{X}$}
                                    +  (0,-1) node (Y) {};
                                    \draw (T) -- (O) -- (X);
                                    \draw (copy0) to [out=-90,in=180] ($(C.west) + (0,0)$);
                                    \draw (D) to [out=0,in=180] ($(C.west) + (0,-0.15)$);
                                    \draw (copy1) to [out=-60,in=180] ($(C.west)+ (0,0.15)$);
                                    \draw[-{Rays[n=8]}] (C) -- (Y);
                                \end{tikzpicture}\\
                            &= \begin{tikzpicture} \path (0,0) node[dist] (T) {$\gamma$}
                                    + (0,-1.15) node (D) {$\RV{D}$}
                                    ++ (0.5,0) coordinate (copy0)
                                    ++ (0.7,0) node[kernel] (O) {$\kernel{T}^{\RV{X}|\RV{H}}$}
                                    ++ (0.7,0) coordinate (copy1)
                                    ++ (1.9,0) node (X) {$\RV{X}$}
                                    +  (0,-1) node (Y) {};
                                    \draw (T) -- (O) -- (X);
                                    \draw[-{Rays[n=8]}] (D) -- (Y);
                                \end{tikzpicture}
\end{align}

Which implies $\RV{X}\CI_{\kernel{U}} \RV{D}$ by version (2) of conditional indpendence (Theorem \ref{th:ci_equivalence}).
\end{proof}


\begin{definition}[Hypothesis sufficiency]
The hypothesis $\RV{H}$ is \emph{sufficient} for a see-do model if the consequence map has no dependence on observations $\RV{X}$ conditional on $\RV{H}$. That is, $\RV{Y}\CI_\kernel{T} \RV{X}|\RV{D}\RV{H}$. 

A hypothesis sufficient see-do model can be specified with:

\begin{itemize}
    \item Hypothesis space $\RV{H}$, choices $D$, observations $X$ and consequences $Y$
    \item Observation map $\kernel{T}^{\RV{X}|\RV{H}}$
    \item Reduced consequence map $\kernel{T}^{\RV{Y}|\RV{H}\RV{D}}$
\end{itemize}
\end{definition}

Given observations $\RV{X}$, assumed to be an IID sequence $\RV{X}_1,\RV{X}_2,...$ conditional on $\RV{H}$, a common ``causal inference problem'' is to estimate the ``true'' distribution of observations $\kernel{T}^{\RV{X}_i|\RV{H}}_{h^*}$ and from this to estimate the consequence map $\kernel{T}^{\RV{Y}|\RV{H}\RV{D}}_{h^*\cdot}$, if this is possible. This problem only makes sense if hypothesis sufficiency is assumed -- once $h^*$ is given, the consequence map of interest has no further dependence on $\RV{X}$. We show that all decision problems can be modeled by a hypothesis sufficient see-do model.


\subsubsection{Examples of hypothesis sufficient and insufficient see-do models}

Suppose we are betting on the outcome of the flip of a possibly biased coin with payout 1 for a correct guess and 0 for an incorrect guess, and we are given $N$ previous flips of the coin to inspect. This situation can be modeled by a hypothesis sufficient see-do model. Define $\kernel{B}:(0,1)\to \Delta(\{0,1\})$ by $\kernel{B}:\RV{H}\mapsto \mathrm{Bernoulli}(\RV{H})$. Then define $\prescript{1}{}{\kernel{T}}$ by:

\begin{itemize}
    \item $D=\{0,1\}$
    \item $X=\{0,1\}^N$
    \item $Y=\{0,1\}$
    \item $H=(0,1)$
    \item $\prescript{1}{}{\kernel{T}}^{\RV{X}|\RV{H}}:\splitter{0.1}^N\kernel{B}$
    \item $\prescript{1}{}{\kernel{T}}^{\RV{Y}|\RV{D}\RV{H}}:(\RV{H},d)\mapsto \mathrm{Bernoulli}(1-|d-\RV{H}|)$
\end{itemize}

In this model, the chance $\RV{H}$ of the coin landing on heads is as much as we can hope to know about how our bet will work out.

Suppose instead that in addition to the $N$ prior flips, we manage to look at the outcome of the flip on which we will bet. In this case, the situation can be modeled by the following hypothesis insufficient see-do model $\prescript{2}{}{\kernel{T}}$:

\begin{itemize}
    \item $D=\{0,1\}$
    \item $X=\{0,1\}^{N+1}$
    \item $Y=\{0,1\}$
    \item $H=(0,1)$
    \item $\prescript{2}{}{\kernel{T}}^{\RV{X}|\RV{H}}:\splitter{0.1}^{N+1}\kernel{B}$
    \item $\prescript{2}{}{\kernel{T}}^{\RV{Y}|\RV{X}\RV{H}\RV{D}}:(\RV{H},\mathbf{x},d)\mapsto \delta_{1-|d-x_{N+1}|}$
\end{itemize}

In this case, even if we are told the value of $\RV{H}$, we still benefit from using the observed data when making our decision.

It appears that it might be possible to model the second situation with a hypothesis sufficient model by including the result of the $N+1$th flip in the hypothesis. Define the new hypothesis space $H'=(0,1)\times\{0,1\}$ and define $\prescript{3}{}{\kernel{T}}$ by:

\begin{itemize}
    \item $D=\{0,1\}$
    \item $X=\{0,1\}^{N+1}$
    \item $Y=\{0,1\}$
    \item $H'=(0,1)\times\{0,1\}$
    \item $\prescript{3}{}{\kernel{T}}^{\RV{X}|\RV{H}'}:(\splitter{0.1}^N\kernel{B}\otimes \delta_{x_{N+1}}$
    \item $\prescript{3}{}{\kernel{T}}^{\RV{Y}|\RV{H}'\RV{D}}:(h,x_{N+1},d)\mapsto \delta_{1-|d-x_{N+1}|}$
\end{itemize}

However, $\RV{X}_{N+1}$ is related to the previous flips $\vecRV{X}_{<N}$ and $\prescript{3}{}{\kernel{T}}$ ignores this fact.  In particular, given any $\RV{H}'=(h,\_)$, $\RV{X}_{N+1}$ as well as $\RV{X}_{i}$, $i\leq N$ should all distributed according to Bernoulli($h$). Thus $\prescript{2}{}{\kernel{T}}$ is preferable to $\prescript{3}{}{\kernel{T}}$ because it represents more of the knowledge we have about the problem.

If a see-do model is employed in a \emph{decision problem} -- defined in the next section -- there is an alternative way to avoid hypothesis insufficiency that does not require throwing out some of the model structure.

\todo[inline]{The importance of this is that counterfactual questions are usually \emph{not} decision problems and so they do not have the possibility of avoiding insufficiency available; also \emph{in practice} counterfactual problems are usually hypothesis insufficient while decision problems are usually not.}

\subsection{Causal questions and decision functions}

\citet{pearl_book_2018} has proposed three types of causal question:
\begin{enumerate}
    \item Association: How are $\RV{W}$ and $\RV{Z}$ related? How would observing $\RV{W}$ change my beliefs about $\RV{Z}$?
    \item Intervention: What would happen if I do ... ? How can I make ... happen?
    \item Counterfactual: What if I had done ... instead of what I actually did?
\end{enumerate}

\emph{Causal decision problems} are, roughly speaking, ``interventional'' problems. In English, a causal decision problem roughly asks

\begin{quote}
    Given that I have data $\RV{X}$ and I know which values of $\RV{Y}$ I would like to see and some knowledge about how the world works, which of my available choices $D$ should I select?
\end{quote}

Note that this presupposes somewhat more than Pearl's prototypical interventional questions. First, it supposes that we have \emph{preferences} over the values that $\RV{Y}$ might take, which we need not have to answer the question ``What would happen if I do ...?''. Secondly, and crucially to our theory, causal decision problem suppose that we are given data and a set of choices. 
\end{definition}

We will return to the question of preferences. For now, we will focus on the idea that a causal decision problem is about selecting a choice given data. That is, however the selection is made, the answer to a causal decision problem is always a \emph{decision function} $\kernel{D}:X\to \Delta(\sigalg{D})$.

\subsubsection{Avoiding insufficiency with decision functions}

Given that 

\subsubsection{Decision rules}

See-do models encode the relationship between observed data and consequences of decisions. In order to actually make decisions, we also require preferences over consequences. We suppose that a \emph{utility function} is given, and evaluate the desirability of consequences using \emph{expected utility}. A see-do model along with a utility allows us to evaluate the desirability of \emph{decisions rules} according to each hypothesis.

\begin{definition}[Utility function]
Given a See-Do Model $\kernel{T}:\RV{H}\times D\to \Delta(\sigalg{X}\otimes\sigalg{Y})$, a \emph{utility function} $u$ is a measurable function $Y\to \mathbb{R}$. 
\end{definition}

\begin{definition}[Expected utility]
Given a utility function $u:Y\to \mathbb{R}$ and probability measures $\mu,\nu\in \Delta(\sigalg{Y})$, the \emph{expected utility} of $\mu$ is $\mathbb{E}_{\mu}[u]$.

$\mu$ is \emph{preferred} to $\nu$ if $\mathbb{E}_{\mu}[u]\geq \mathbb{E}_{\nu}[u]$, and \emph{strictly preferred} if $\mathbb{E}_{\mu}[u]>\mathbb{E}_{\nu}[u]$.
\end{definition}

\begin{definition}[Decision rule]
Given a see-to map $\kernel{T}:\RV{H}\times D\to \Delta(\sigalg{X}\otimes\sigalg{Y})$, a \emph{decision rule} is a Markov kernel $X\to \Delta(\sigalg{D})$. A \emph{deterministic decision rule} is a decision rule that is deterministic.

\todo[inline]{Define deterministic Markov kernels}
\end{definition}

Expected utility together with a decision rule gives rise to the definition of \emph{risk}, which connects CSDT to classical statistical decision theory (SDT). For historical reasons, risks are minimised while utilities are maximised.

\begin{definition}[Risk]
Given a see-to map $\kernel{T}:\RV{H}\times D\to \Delta(\sigalg{X}\otimes\sigalg{Y})$, a utility $u:Y\to \mathbb{R}$ and the set of decision rules $\mathscr{U}$, the \emph{risk} is a function $l:\RV{H}\times \mathscr{U}\to \mathbb{R}$ given by

\begin{align}
    R(\RV{H},\kernel{U}) := - \int_X  \kernel{U}_x \kernel{T}^{\RV{Y}|\RV{D}\RV{X}\RV{H}}_{\cdot,x,\RV{H}} u d\kernel{T}^{\RV{X}|\RV{H}}_\RV{H}(x)
\end{align}

for $\RV{H}\in \RV{H}$, $\kernel{U}\in \mathscr{U}$. Here $\kernel{U}_x \kernel{T}^{\RV{Y}|\RV{D}\RV{X}\RV{H}}_{\cdot,x,\RV{H}} u$ is the product of the measure $\kernel{U}_x$, the kernel $\kernel{T}^{\RV{Y}|\RV{D}\RV{X}\RV{H}}_{\cdot,x,\RV{H}}:D\to \Delta(\sigalg{Y})$ and the function $u$.
\end{definition}

The loss induces a partial order on decision rules. If for all $\RV{H}$, $l(\RV{H},\kernel{U})\leq l(\RV{H},\kernel{U}')$ then $\kernel{U}$ is at least as good as $\kernel{U}'$. If, furthermore, there is some $\RV{H}_0$ such that $l(\RV{H}_0,\kernel{U})<l(\RV{H}_0,\kernel{U}')$ then $\kernel{U}$ is preferred to $\kernel{U}'$.

\begin{definition}[Induced statistical decision problem]
A see-do model $\kernel{T}:\RV{H}\times D\to \Delta(\sigalg{X}\otimes\sigalg{Y})$ along with a utility $u$ induces the \emph{statistical decision problem} $(\RV{H},\mathscr{U},R)$ with states $\RV{H}$, decisions $\mathscr{U}$ and risks $R$.

\todo[inline]{Statistical decision problems usually define the risk via the loss, but it is only possible to define a loss with a hypothesis sufficient model. We don't actually need a loss, though: the complete class theorem still holds via the induced risk and Bayes risk}

\end{definition}


A key difference between CSDT and other approaches to causal inference is that diagrams in CSDT feature two coupled maps $\kernel{O}$ and $\kernel{C}$, while most other approaches to causal inference represent both $\kernel{O}$ and $\kernel{C}$ in one diagram. \citet{lattimore_replacing_2019} is the only other example I am aware of that represents both $\kernel{O}$ and $\kernel{C}$. Nevertheless, ``one-picture'' causal models such as Causal Bayesian Networks, Single World Intervention Graphs \emph{do} represent observational distributions and interventional maps, and the two differ (see Section \ref{sec:single_double_representation})

A causal hypothesis class $\RV{H}$ induces a binary relation between observed probability distributions $\prob{O}_\RV{H}$ and consequence maps $\prob{C}_\RV{H}$. This approach is very agnostic about the actual relation induced -- we do not even insist that the range of the observed data $X$ is the same as the range of possible consequences $Y$ (though we will generally limit our attention to cases where the two coincide). 

In common with \citet{heckerman_decision-theoretic_1995}, decisions (or ``acts'') are primitive elements of See-Do Models. In contrast to our work, \citet{heckerman_decision-theoretic_1995} only discuss deterministic \emph{consequence maps}, while See-Do Models represent relations between consequence maps and observed probability.

Decisions are similar to the ``regime indicators'' found in \citet{dawid_decision-theoretic_2020}. They coincide precisely if we suppose that the observation and consequence spaces coincide ($X=Y$) and there exists an ``idle'' decision $d^*\in D$ such that $\kernel{C}_{(\cdot,d^*)} = \kernel{O}_{\cdot}$. However, in general we don't require that $\kernel{O}$ and $\kernel{C}$ are related in this manner. This assumption will be revisited in \todo[inline]{A section I haven't written yet}.

\subsection{D-causation}

While we take $D$ to be a primitive element of causal decision problems, and therefore a primitive of See-Do Models. Causes are not primitive, but we can offer a secondary notion of causation. We call this $D$-causation to stress the fact that it arises in a theory of causal inference in which the set $D$ of available decisions is primitive. A similar idea is discussed extensively in \citet{heckerman_decision-theoretic_1995}. The main differences are that what we call ``consequence maps'' map decisions to probability distributions over possible consequences while Heckerman and Shachter work with ``states'' that map decisions deterministically to consequences. In addition, while we define $D$-causation relative to a particular consequence map $\kernel{C}_\RV{H}$, Heckerman and Shachter define it with respect to a \emph{set} of states.

Section \ref{sec:cbns_without_d} explores the difficulty of defining ``objective causation'' without reference to a set of basic decisions, acts or operations. $D$ need not be interpreted as the set of decisions an agent may make, but whatever interpretation it is assigned, all existing examples of causal models seem to require a ``domain set''.

See Section \ref{ssec:random_variables} for the definition of random variables.

\todo[inline]{Add definition of conditional independence, revise wire label definitions}

One way to motivate the notion of $D$-causation is to observe that for many decision problems, the full set $D$ may be extremely large. Suppose I aim to have my light switched on, and there is a switch that controls the light. Often, the relevant choice of acts for such a problem would appear to be $D_0=\{\text{flip the switch},\text{don't flip the switch}\}$. However, in principle I have a much larger range of options to choose from. For simplicity's sake, suppose I have instead the following set of options:

\begin{align*}
D_1:=&\{``\text{walk to the switch and press it with my thumb}'', \\
    &``\text{trip over the lego on the floor, hop to the light switch and stab my finger at it}'',\\
    &``\text{stay in bed}''\}
\end{align*}

If having the light turned on is all that matters, I could consider any acts in $D_1$ to be equivalent if they have the same ultimate impact on the position of the light switch. $D_0$ is a quotient over $D_1$ under this equivalence relation. 

If I hypothesize that, relative to $D_1$, the ultimate state of the light switch is all that matters to determine the ultimate state of the light, I can say that the light switch $D_1$-causes the state of the light. Given this $D_1$-causation, the $D_1$ decision problem can (subject to my hypothesis) be reduced to a $D_0$ decision between states of the light switch.

If I consider an even larger set of possible acts $D_2$, I might not accept the hypothesis of $D_2$-causation. Let $D_2$ be the following acts:

\begin{align*}
D_2:=&\{``\text{walk to the switch and press it with my thumb}'', \\
    &``\text{trip over the lego on the floor, hop to the light switch and stab my finger at it}'',\\
    &``\text{stay in bed}'',
    &``\text{toggle the mains power, then flip the light switch}''\}
\end{align*}

In this case, it would be unreasonable to hypothesize that all acts that left the light switch in the ``on'' position would also result in the light being ``on''. Thus the switch does not $D_2$-cause the light to be on.

Formally, $D$-causation is defined in terms of conditional independence:

\begin{definition}[$D$-causation]\label{def:d_cause}
Given a consequence map $\kernel{C}_\RV{H}:D\to \Delta(\mathcal{Y})$, random variables $\RV{Y}_1:Y\times D\to Y_1$, $\RV{Y}_2:Y\times D\to Y_2$ and domain variable $\RV{D}:Y\times D\to D$ (Definition \ref{def:domain_variable}), $\RV{Y}_1$ $D$-causes $\RV{Y}_2$ iff $\RV{Y}_2\CI_{\kernel{C}_\RV{H}} \RV{D}|\RV{Y}_1$.
\end{definition}

\subsection{D-causation vs Heckerman and Shachter}

Heckerman and Shachter study deterministic ``consequence maps''. Furthermore, what we call hypotheses $\RV{H}\in\RV{H}$, Heckerman and Schachter call states $s\in S$. One could consider a state to be a hypothesis that is specific enough to yield a deterministic map from decisions to outcomes. Heckerman and Shachter's notion of causation is defined by \emph{limited unresponsiveness} rather than \emph{conditional independence}, which depends on a partition of states rather than a particular hypothesis.

\begin{definition}[Limited unresponsiveness]
    Given states $S$, deterministic consequence maps $\kernel{C}_s:D\to \Delta(F)$ for each $s\in A$ and a random variables $\RV{X}:F\to X$, $\RV{Y}:F\to Y$, $\RV{Y}$ is unresponsive to $\RV{D}$ in states limited by $\RV{X}$ if $\kernel{C}_{(s,d)}^{\RV{X}|\RV{D}}=\kernel{C}_{(s,d')}^{\RV{X}|\RV{D}\RV{S}}\implies \kernel{C}_{(s,d)}^{\RV{Y}|\RV{D}\RV{S}}=\kernel{C}_{(s,d')}^{\RV{Y}|\RV{D}\RV{S}}$ for all $d,d'\in D$, $s\in S$. Write $\RV{Y}\not\hookleftarrow_{\RV{X}} \RV{D}$
\end{definition}

\begin{lemma}[Limited unresponsiveness implies $D$-causation]
For deterministic consequence maps, $\RV{Y}\not\hookleftarrow_{\RV{X}} \RV{D} $ implies $\RV{X}$ $D$-causes $\RV{Y}$ in every state $s\in S$.
\end{lemma}

\begin{proof}
By the assumption of determinism, for each $s\in S$ and $d\in D$ there exists $x(s,d)$ and $y(s,d)$ such that $\kernel{C}^{\RV{X}\RV{Y}|\RV{D}\RV{S}}_{d,s} = \delta_{x(s,d)}\otimes\delta_{y(s,d)}$.

By the assumption of limited unresponsiveness, for all $d,d'$ such that $x(s,d)=x(s,d')$, $y(s,d)=y(s,d')$ also. Define $f:X\times S\to Y$ by $(s,x)\mapsto y(s,[x(s,\cdot)]^{-1}(x(s,d)))$ where $[x(s,\cdot)]^{-1}(a)$ is an arbitrary element of $\{d|x(s,d)=a\}$. For all $s,d$, $f(x(s,d),s)=y(s,d)$. Define $\kernel{M}:X\times D\times S\to \Delta(\mathcal{Y})$ by $(x,d,s)\mapsto \delta_{f(x,s)}$. $\kernel{M}$ is a version of $\kernel{C}^{\RV{Y}|\RV{X},\RV{D},\RV{S}}$ because, for all $A\in \mathcal{X}$, $B\in \mathcal{Y}$, $s\in S$, $d\in D$:

\begin{align}
    \kernel{C}^{\RV{X}|\RV{D}\RV{S}}_{(d,s)}\splitter{0.1}(\kernel{M}\otimes\mathrm{Id}) &= \int_A \kernel{M}(x',d,s;B) d\delta_{x(s,d)}(x') \\
                                                                                        &= \int_A \delta_{f(x',s)}(B) d\delta_{x(s,d)}(x') \\
                                                                                        &= \delta_{f(x(s,d),s)}(B)\delta_{x(s,d)}(A) \\
                                                                                        &= \delta_{y(s,d)}(B)\delta_{x(s,d)}(A)\\
                                                                                        &= \delta_{x(s,d)}\otimes\delta_{y(s,d)}(A\times B)
\end{align}

$\kernel{M}$ is also independent of $\RV{D}$, given the obvious labeling of inputs. Therefore $\RV{Y}\CI_{\kernel{C}_s}\RV{D}|\RV{X}$.
\end{proof}

However, despite limited unresponsiveness implying $D$-causation within every state, it does not imply $D$-causation in mixtures of states. Suppose $D=\{0,1\}$ where $1$ stands for ``toggle light switch'' and $0$ stands for ``do nothing''. Suppose $S=\{[0,0],[0,1],[1,0],[1,1]\}$ where $[0,0]$ represents ``switch initially off, mains off'' the other states generalise this in the obvious way. Finally, $\RV{F}\in\{0,1\}$ is the final position of the switch and $\RV{L}\in\{0,1\}$ is the final state of the light. We have

\begin{align}
    \kernel{C}^{\RV{L}\RV{F}|\RV{D}\RV{S}}_{d,[i,m]} = \delta_{(d\text{ XOR }i)\text{ AND }m}\otimes \delta_{(d\text{ XOR }i)\text{ AND }m}
\end{align}

Within states $[0,0]$ and $[1,0]$, the light is always off, so $\RV{F}=a\implies \RV{L}=0$ for any $a$. In states $[0,1]$ and $[1,1]$, $\RV{F}=1\implies \RV{L}=1$ and $\RV{F}=0\implies \RV{L}=0$. Thus $\RV{L}\not\hookleftarrow_{\RV{F}} \RV{D}$. However, suppose we take a mixture of consequence maps:
\begin{align}
    \kernel{C}_\gamma &= \frac{1}{4}\kernel{C}_{\cdot,[0,0]} + \frac{1}{4}\kernel{C}_{\cdot,[0,1]} + \frac{1}{2}\kernel{C}_{\cdot,[1,1]}\\
    \kernel{C}^{\RV{F}\RV{L}|\RV{D}}_\gamma &= \frac{1}{4} \left[\begin{matrix}
                        1 & 0\\ 0 & 1
                      \end{matrix}\right]\otimes \left[\begin{matrix}
                        1 & 0\\ 1 & 0
                      \end{matrix}\right] + \frac{1}{4} \left[\begin{matrix}
                        1 & 0\\ 0 & 1
                      \end{matrix}\right]\otimes \left[\begin{matrix}
                        1 & 0\\ 0 & 1
                      \end{matrix}\right] + \frac{1}{2}\left[\begin{matrix}
                        0 & 1\\ 1 & 0
                      \end{matrix}\right]\otimes \left[\begin{matrix}
                        0 & 1\\ 1 & 0
                      \end{matrix}\right]
\end{align}

Then

\begin{align}
    [1,0]\kernel{C}^{\RV{F}\RV{L}|\RV{D}}_{\gamma} &= \frac{1}{4}[0,1]\otimes[1,0]+\frac{1}{4}[0,1]\otimes[0,1]+\frac{1}{2}[1,0]\otimes[1,0]\\
    [1,0]\splitter{0.1}(\kernel{C}^{\RV{F}|\RV{D}}_\gamma\otimes \kernel{C}^{\RV{L}|\RV{D}}_\gamma) &= (\frac{1}{2}[0,1]+\frac{1}{2}[1,0])\otimes(\frac{1}{4}[0,1]+\frac{3}{4}[1,0])\\
    \implies [1,0]\kernel{C}^{\RV{F}\RV{L}|\RV{D}}_{\gamma} &\neq [1,0] \splitter{0.1} (\kernel{C}^{\RV{F}|\RV{D}}_\gamma\otimes \kernel{C}^{\RV{L}|\RV{D}}_\gamma)
\end{align}

Thus under hypothesis mixture $\gamma$, $\RV{F}$ does not $D$-cause $\RV{L}$ even though $\RV{F}$ $D$-causes $\RV{L}$ in all states $S$. The definition of $D$-causation was motivated by the idea that we could reduce a difficult decision problem with a large set $D$ to a simpler problem with a smaller ``effective'' set of decisions by exploiting conditional independence. Even if $\RV{X}$ $D$-causes $\RV{Y}$ in every $\RV{H}\in S$, $\RV{X}$ does not necessarily $D$-cause $\RV{Y}$ in mixtures of states in $S$. For this reason, we do not say that $\RV{X}$ $D$-causes $\RV{Y}$ in $S$ if $\RV{X}$ $D$-causes $\RV{Y}$ in every $\RV{H}\in S$, and in this way we differ substantially from \citet{heckerman_decision-theoretic_1995}.

Instead, we simply extend the definition of $D$-causation to mixtures of hypotheses: if $\gamma\in \Delta(\RV{H})$ is a mixture of hypotheses, define $\kernel{C}_\gamma:= (\gamma\otimes\textbf{Id})\kernel{C}$. Then $\RV{X}$ $D$-causes $\RV{Y}$ relative to $\gamma$ iff $\RV{Y}\CI_{\kernel{C}_\gamma} \RV{D}|\RV{X}$.

Theorem \ref{th:univ_d_causation} shows that under some conditions, $D$-causation can hold for arbitrary mixtures over subsets of the hypothesis class $\RV{H}$.

\begin{theorem}[Universal $D$-causation]\label{th:univ_d_causation}
If $\kernel{C}^{\RV{X}|\RV{D}}_{\RV{H}} = \kernel{C}^{\RV{X}|\RV{D}}_{\RV{H}'}$ for all $\RV{H},\RV{H}'\in S\subset \RV{H}$ and $\RV{X}$ $D$-causes $\RV{Y}$ in all $\RV{H}\in S$, then $\RV{X}$ $D$-causes $\RV{Y}$ with respect to all mixed consequence maps $\kernel{C}_\gamma$ for all $\gamma\in \Delta(\RV{H})$ with $\gamma(S)=1$.
\end{theorem}

\begin{proof}

For $\gamma\in \Delta(\RV{H})$, define the mixture

\begin{align}
\kernel{C}_\gamma := \begin{tikzpicture}
    \path (0,0) node[dist] (g) {$\gamma$}
    + (0,-0.45) node (D) {$\RV{D}$}
    ++ (1,-0.3) node[kernel] (C) {$\kernel{C}$}
    ++ (1,0) node (F) {$\RV{F}$};
    \draw (g) to [out=0,in=180] ($(C.west) + (0,0.15)$) (D) -- ($(C.west) + (0,-0.15)$) (C) -- (F);
\end{tikzpicture}
\end{align}

Because $\kernel{C}_\RV{H}^{\RV{X}|\RV{D}} = \kernel{C}_{\RV{H}'}^{\RV{X}|\RV{D}}$ for all $\RV{H},\RV{H}'\in \RV{H}$, we have

\begin{align}
\begin{tikzpicture}
    \path (0,0) node[dist] (g) {$\gamma$}
    + (0.7,-0.15) node[copymap] (copy0) {}
    + (0,-0.45) node (D) {$\RV{D}$}
    ++ (1.5,-0.3) node[kernel] (C) {$\kernel{C}^{\RV{X}|\RV{D}\RV{H}}$}
    ++ (1,0) node (X) {$\RV{X}$}
    + (0,0.5) node (T) {$\RV{H}$};
    \draw (g) to [out=0,in=180] (copy0) -- ($(C.west) + (0,0.15)$) (D) -- ($(C.west) + (0,-0.15)$);
    \draw (C) -- (X);
    \draw (copy0) to [out=90,in=180] (T);
\end{tikzpicture} &= \begin{tikzpicture}
    \path (0,0) node[dist] (g) {$\gamma$}
    + (0,0.5) node[dist] (g2) {$\gamma$}
    + (0.7,-0.15) node[copymap] (copy0) {}
    + (0,-0.45) node (D) {$\RV{D}$}
    ++ (1.5,-0.3) node[kernel] (C) {$\kernel{C}^{\RV{X}|\RV{D}\RV{H}}$}
    ++ (1,0) node (X) {$\RV{X}$}
    + (0,0.3) node (T) {$\RV{H}$};
    \draw (g) to [out=0,in=180] (copy0) -- ($(C.west) + (0,0.15)$) (D) -- ($(C.west) + (0,-0.15)$);
    \draw (C) -- (X);
    \draw (g2) to [out=0,in=180] (T);
\end{tikzpicture} \label{eq:decompose_condi_x}
\end{align}

Also

\begin{align}
    \kernel{C}_\gamma^{\RV{XY}|\RV{D}} &= \begin{tikzpicture}
    \path (0,0) node[dist] (g) {$\gamma$}
    + (0,-0.45) node (D) {$\RV{D}$}
    ++ (1,-0.3) node[kernel] (C) {$\kernel{C}$}
    ++ (1,0) node[kernel] (F) {$\kernel{F}^{\RV{X}\utimes\RV{Y}}$}
    ++ (1,0.15) node (X) {$\RV{X}$}
    + (0,-0.3) node (Y) {$\RV{Y}$};
    \draw (g) to [out=0,in=180] ($(C.west) + (0,0.15)$) (D) -- ($(C.west) + (0,-0.15)$) (C) -- (F);
    \draw ($(F.east) + (0,0.15)$) -- (X) ($(F.east) + (0,-0.15)$) -- (Y);
\end{tikzpicture}\\
    &= \begin{tikzpicture}
    \path (0,0) node[dist] (g) {$\gamma$}
    + (0,-0.45) node (D) {$\RV{D}$}
    ++ (1,-0.3) node[kernel] (C) {$\kernel{C}^{\RV{XY}|\RV{D}\RV{H}}$}
    ++ (1,0.15) node (X) {$\RV{X}$}
    + (0,-0.3) node (Y) {$\RV{Y}$};
    \draw (g) to [out=0,in=180] ($(C.west) + (0,0.15)$) (D) -- ($(C.west) + (0,-0.15)$);
    \draw ($(C.east) + (0,0.15)$) -- (X) ($(C.east) + (0,-0.15)$) -- (Y);
\end{tikzpicture}\\
 &= \begin{tikzpicture}
    \path (0,0) node[dist] (g) {$\gamma$}
    + (0,-0.45) node (D) {$\RV{D}$}
    + (0.7,-0.45) node[copymap] (copy0) {}
    + (0.7,-0.15) node[copymap] (copy1) {}
    ++ (1.4,-0.3) node[kernel] (C) {$\kernel{C}^{\RV{X}|\RV{D}\RV{H}}$}
    + (0,0.6) coordinate (via0)
    + (0,-0.6) coordinate (via1)
    ++ (0.9,0) node[copymap] (copy2) {}
    ++ (0.7,0) node[kernel] (Yx) {$\kernel{C}^{\RV{Y}|\RV{X}\RV{D}\RV{H}}$}
    ++ (1.2,0.15) node (X) {$\RV{Y}$}
    + (0,-0.5) node (Y) {$\RV{X}$};
    \draw (g) to [out=0,in=180] (copy1) -- ($(C.west) + (0,0.15)$) (D) -- ($(C.west) + (0,-0.15)$) (C)--(Yx);
    \draw (copy0) to [out=-90,in=180] (via1) to [out=0,in=180] ($(Yx.west) + (0,-0.15)$) (copy1) to [out=90,in=180] (via0) to [out=0,in=180] ($(Yx.west) + (0,0.15)$);
    \draw ($(Yx.east) + (0,0.15)$) -- (X) (copy2) to [out=-90,in=180] (Y);
 \end{tikzpicture}\\
 &\overset{\RV{Y}\CI \RV{D}|\RV{X}\RV{H}}{=} \begin{tikzpicture}
    \path (0,0) node[dist] (g) {$\gamma$}
    + (0,-0.45) node (D) {$\RV{D}$}
    + (0.7,-0.15) node[copymap] (copy1) {}
    ++ (1.4,-0.3) node[kernel] (C) {$\kernel{C}^{\RV{X}|\RV{D}\RV{H}}$}
    ++ (0.9,0.1) node[copymap] (copy2) {}
    ++ (0.7,0.3) node[kernel] (Yx) {$\kernel{C}^{\RV{Y}|\RV{X}\RV{H}}$}
    ++ (1.2,0.15) node (X) {$\RV{Y}$}
    + (0,-0.5) node (Y) {$\RV{X}$};
    \draw (g) to [out=0,in=180] (copy1) -- ($(C.west) + (0,0.15)$) (D) -- ($(C.west) + (0,-0.15)$) (C) to [out=0,in=180] (copy2) to [out=0,in=180] (Yx);
    \draw (copy1) to [out=90,in=180] ($(Yx.west) + (0,0.15)$);
    \draw ($(Yx.east) + (0,0.15)$) -- (X) (copy2) to [out=-90,in=180] (Y);
 \end{tikzpicture} \\
 &\overset{\ref{eq:decompose_condi_x}}{=} \begin{tikzpicture}
    \path (0,0) node[dist] (g) {$\gamma$}
    + (0,-0.45) node (D) {$\RV{D}$}
    + (0.7,-0.15) node[copymap] (copy1) {}
    ++ (1.4,-0.3) node[kernel] (C) {$\kernel{C}^{\RV{X}|\RV{D}\RV{H}}$}
    + (1,0.6) node[dist] (g2) {$\gamma$}
    ++ (0.9,0.1) node[copymap] (copy2) {}
    ++ (1,0.3) node[kernel] (Yx) {$\kernel{C}^{\RV{Y}|\RV{X}\RV{H}}$}
    ++ (1.2,0.15) node (X) {$\RV{Y}$}
    + (0,-0.5) node (Y) {$\RV{X}$};
    \draw (g) to [out=0,in=180] (copy1) -- ($(C.west) + (0,0.15)$) (D) -- ($(C.west) + (0,-0.15)$) (C) to [out=0,in=180] (copy2) to [out=0,in=180] (Yx);
    \draw (g2) to [out=0,in=180] ($(Yx.west) + (0,0.15)$);
    \draw ($(Yx.east) + (0,0.15)$) -- (X) (copy2) to [out=-90,in=180] (Y);
 \end{tikzpicture}\\
 &= \overset{\ref{eq:decompose_condi_x}}{=} \begin{tikzpicture}
    \path (0,0) node (g) {}
    + (0,-0.45) node (D) {$\RV{D}$}
    + (0.7,-0.45) node[copymap] (copy1) {}
    ++ (1.4,-0.3) node[kernel] (C) {$\kernel{C}_\gamma^{\RV{X}|\RV{D}\RV{H}}$}
    + (1,0.6) node[dist] (g2) {$\gamma$}
    ++ (0.9,0.1) node[copymap] (copy2) {}
    ++ (1,0.3) node[kernel] (Yx) {$\kernel{C}^{\RV{Y}|\RV{X}\RV{H}}$}
    + (-0.5,0.6) coordinate (stop0)
    ++ (1.2,0.15) node (X) {$\RV{Y}$}
    + (0,-0.5) node (Y) {$\RV{X}$};
    \draw (D) -- ($(C.west) + (0,-0.15)$) (C) to [out=0,in=180] (copy2) to [out=0,in=180] (Yx);
    \draw (g2) to [out=0,in=180] ($(Yx.west) + (0,0.15)$);
    \draw ($(Yx.east) + (0,0.15)$) -- (X) (copy2) to [out=-90,in=180] (Y);
    \draw[-{Rays[n=8]}] (copy1) to [out=90,in=180] (stop0);
 \end{tikzpicture}\label{eq:is_conditional}
\end{align}
Equation \ref{eq:is_conditional} establishes that $(\gamma\otimes\textbf{Id}_X\otimes\stopper{0.3}_D)\kernel{C}^{\RV{Y}|\RV{X}\RV{H}}$ is a version of $\kernel{C}_\gamma^{\RV{Y}|\RV{X}\RV{D}}$, and thus $\RV{Y}\CI_{\kernel{C}_\gamma} \RV{D}|\RV{X}$.

This can also be derived from the semi-graphoid rules:

\begin{align}
    \RV{H}\CI \RV{D} \land \RV{H}\CI \RV{X} | \RV{D} &\implies \RV{H}\CI \RV{XD}\\
    &\implies \RV{H}\CI \RV{D}|\RV{X}\\
    \RV{D} \CI \RV{H}|\RV{X} \land \RV{D}\CI \RV{Y}|\RV{X}\RV{H} &\implies \RV{D}\CI \RV{Y}|\RV{X}\\
    &\implies \RV{Y}\CI\RV{D}|\RV{X}
\end{align}
\end{proof}

\subsection{Properties of D-causation}

If $\RV{X}$ D-causes $\RV{Y}$ relative to $\kernel{C}_\RV{H}$, then the following holds:

\begin{align}
    \kernel{C}_{\RV{H}}^{\RV{X}|\RV{D}} &= \begin{tikzpicture}
    \path (0,0) node (D) {$\RV{D}$}
    ++ (0.9,0) node[kernel] (Xd) {$\kernel{C}^{\RV{X}|\RV{D}}$}
    ++ (1.3,0) node[kernel] (Yd) {$\kernel{C}^{\RV{Y}|\RV{X}}$}
    ++ (0.9,0) node (Y) {$\RV{Y}$};
    \draw (D) -- (Xd) -- (Yd) -- (Y); 
    \end{tikzpicture}
\end{align}

This follows from version (2) of Definition \ref{def:conditional_independence}:

\begin{align}
    \kernel{C}_\RV{H}^{\RV{X}|\RV{D}} &= \begin{tikzpicture}
    \path (0,0) node (D) {$\RV{D}$}
    ++ (0.7,0) node[copymap] (copy0) {}
    ++ (0.7,0) node[kernel] (Xd) {$\kernel{C}^{\RV{X}|\RV{D}}$}
    + (0,0.5) coordinate (via1)
    ++ (1.3,0) node[kernel] (Yd) {$\kernel{C}^{\RV{Y}|\RV{X}\RV{D}}$}
    ++ (0.9,0) node (Y) {$\RV{Y}$};
    \draw (D) -- (Xd) -- (Yd) -- (Y);
    \draw (copy0) to [out=90,in=180] (via1) to [out=0,in=180] ($(Yd.west)+(0,0.15)$); 
    \end{tikzpicture}\\
     &= \begin{tikzpicture}
    \path (0,0) node (D) {$\RV{D}$}
    ++ (0.7,0) node[copymap] (copy0) {}
    ++ (0.7,0) node[kernel] (Xd) {$\kernel{C}^{\RV{X}|\RV{D}}$}
    + (1.3,0.5) coordinate (via1)
    ++ (1.3,0) node[kernel] (Yd) {$\kernel{C}^{\RV{Y}|\RV{X}}$}
    ++ (0.9,0) node (Y) {$\RV{Y}$};
    \draw (D) -- (Xd) -- (Yd) -- (Y);
    \draw[-{Rays[n=8]}] (copy0) to [out=90,in=180] (via1); 
    \end{tikzpicture}\\
    &= \begin{tikzpicture}
    \path (0,0) node (D) {$\RV{D}$}
    ++ (0.9,0) node[kernel] (Xd) {$\kernel{C}^{\RV{X}|\RV{D}}$}
    ++ (1.3,0) node[kernel] (Yd) {$\kernel{C}^{\RV{Y}|\RV{X}}$}
    ++ (0.9,0) node (Y) {$\RV{Y}$};
    \draw (D) -- (Xd) -- (Yd) -- (Y); 
    \end{tikzpicture}
\end{align}

D-causation is not transitive: if $\RV{X}$ D-causes $\RV{Y}$ and $\RV{Y}$ D-causes $\RV{Z}$ then $\RV{X}$ doesn't necessarily D-cause $\RV{Z}$.

%!TEX root = main.tex

\chapter{Chapter 4: See-do models compared to causal graphical models and potential outcomes}\label{ch:4}
\input{chapter_5_inferenceprinciples}

\bibliographystyle{plainnat}
\bibliography{references}

\appendix
\newpage
\section*{Appendix:}

% \input{appendix_AIstats}

\end{document}
